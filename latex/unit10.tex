\documentclass[10pt]{scrartcl}

    % included at the top of the generated TeX file

\usepackage{tgcursor}

\usepackage[utf8]{inputenc}

\KOMAoptions{parskip=half*}
\KOMAoptions{paper=a4,twoside=false}
\KOMAoptions{numbers=noendperiod}

\addtokomafont{disposition}{\rmfamily}
\setcounter{tocdepth}{\subsectiontocdepth}

    \usepackage[breakable]{tcolorbox}
    \usepackage{parskip} % Stop auto-indenting (to mimic markdown behaviour)
    
    \usepackage{iftex}
    \ifPDFTeX
    	\usepackage[T1]{fontenc}
    	\usepackage{mathpazo}
    \else
    	\usepackage{fontspec}
    \fi

    % Basic figure setup, for now with no caption control since it's done
    % automatically by Pandoc (which extracts ![](path) syntax from Markdown).
    \usepackage{graphicx}
    % Maintain compatibility with old templates. Remove in nbconvert 6.0
    \let\Oldincludegraphics\includegraphics
    % Ensure that by default, figures have no caption (until we provide a
    % proper Figure object with a Caption API and a way to capture that
    % in the conversion process - todo).
    \usepackage{caption}
    \DeclareCaptionFormat{nocaption}{}
    \captionsetup{format=nocaption,aboveskip=0pt,belowskip=0pt}

    \usepackage{float}
    \floatplacement{figure}{H} % forces figures to be placed at the correct location
    \usepackage{xcolor} % Allow colors to be defined
    \usepackage{enumerate} % Needed for markdown enumerations to work
    \usepackage{geometry} % Used to adjust the document margins
    \usepackage{amsmath} % Equations
    \usepackage{amssymb} % Equations
    \usepackage{textcomp} % defines textquotesingle
    % Hack from http://tex.stackexchange.com/a/47451/13684:
    \AtBeginDocument{%
        \def\PYZsq{\textquotesingle}% Upright quotes in Pygmentized code
    }
    \usepackage{upquote} % Upright quotes for verbatim code
    \usepackage{eurosym} % defines \euro
    \usepackage[mathletters]{ucs} % Extended unicode (utf-8) support
    \usepackage{fancyvrb} % verbatim replacement that allows latex
    \usepackage{grffile} % extends the file name processing of package graphics 
                         % to support a larger range
    \makeatletter % fix for old versions of grffile with XeLaTeX
    \@ifpackagelater{grffile}{2019/11/01}
    {
      % Do nothing on new versions
    }
    {
      \def\Gread@@xetex#1{%
        \IfFileExists{"\Gin@base".bb}%
        {\Gread@eps{\Gin@base.bb}}%
        {\Gread@@xetex@aux#1}%
      }
    }
    \makeatother
    \usepackage[Export]{adjustbox} % Used to constrain images to a maximum size
    \adjustboxset{max size={0.9\linewidth}{0.9\paperheight}}

    % The hyperref package gives us a pdf with properly built
    % internal navigation ('pdf bookmarks' for the table of contents,
    % internal cross-reference links, web links for URLs, etc.)
    \usepackage{hyperref}
    % The default LaTeX title has an obnoxious amount of whitespace. By default,
    % titling removes some of it. It also provides customization options.
    \usepackage{titling}
    \usepackage{longtable} % longtable support required by pandoc >1.10
    \usepackage{booktabs}  % table support for pandoc > 1.12.2
    \usepackage[inline]{enumitem} % IRkernel/repr support (it uses the enumerate* environment)
    \usepackage[normalem]{ulem} % ulem is needed to support strikethroughs (\sout)
                                % normalem makes italics be italics, not underlines
    \usepackage{mathrsfs}
    

    
    % Colors for the hyperref package
    \definecolor{urlcolor}{rgb}{0,.145,.698}
    \definecolor{linkcolor}{rgb}{.71,0.21,0.01}
    \definecolor{citecolor}{rgb}{.12,.54,.11}

    % ANSI colors
    \definecolor{ansi-black}{HTML}{3E424D}
    \definecolor{ansi-black-intense}{HTML}{282C36}
    \definecolor{ansi-red}{HTML}{E75C58}
    \definecolor{ansi-red-intense}{HTML}{B22B31}
    \definecolor{ansi-green}{HTML}{00A250}
    \definecolor{ansi-green-intense}{HTML}{007427}
    \definecolor{ansi-yellow}{HTML}{DDB62B}
    \definecolor{ansi-yellow-intense}{HTML}{B27D12}
    \definecolor{ansi-blue}{HTML}{208FFB}
    \definecolor{ansi-blue-intense}{HTML}{0065CA}
    \definecolor{ansi-magenta}{HTML}{D160C4}
    \definecolor{ansi-magenta-intense}{HTML}{A03196}
    \definecolor{ansi-cyan}{HTML}{60C6C8}
    \definecolor{ansi-cyan-intense}{HTML}{258F8F}
    \definecolor{ansi-white}{HTML}{C5C1B4}
    \definecolor{ansi-white-intense}{HTML}{A1A6B2}
    \definecolor{ansi-default-inverse-fg}{HTML}{FFFFFF}
    \definecolor{ansi-default-inverse-bg}{HTML}{000000}

    % common color for the border for error outputs.
    \definecolor{outerrorbackground}{HTML}{FFDFDF}

    % commands and environments needed by pandoc snippets
    % extracted from the output of `pandoc -s`
    \providecommand{\tightlist}{%
      \setlength{\itemsep}{0pt}\setlength{\parskip}{0pt}}
    \DefineVerbatimEnvironment{Highlighting}{Verbatim}{commandchars=\\\{\}}
    % Add ',fontsize=\small' for more characters per line
    \newenvironment{Shaded}{}{}
    \newcommand{\KeywordTok}[1]{\textcolor[rgb]{0.00,0.44,0.13}{\textbf{{#1}}}}
    \newcommand{\DataTypeTok}[1]{\textcolor[rgb]{0.56,0.13,0.00}{{#1}}}
    \newcommand{\DecValTok}[1]{\textcolor[rgb]{0.25,0.63,0.44}{{#1}}}
    \newcommand{\BaseNTok}[1]{\textcolor[rgb]{0.25,0.63,0.44}{{#1}}}
    \newcommand{\FloatTok}[1]{\textcolor[rgb]{0.25,0.63,0.44}{{#1}}}
    \newcommand{\CharTok}[1]{\textcolor[rgb]{0.25,0.44,0.63}{{#1}}}
    \newcommand{\StringTok}[1]{\textcolor[rgb]{0.25,0.44,0.63}{{#1}}}
    \newcommand{\CommentTok}[1]{\textcolor[rgb]{0.38,0.63,0.69}{\textit{{#1}}}}
    \newcommand{\OtherTok}[1]{\textcolor[rgb]{0.00,0.44,0.13}{{#1}}}
    \newcommand{\AlertTok}[1]{\textcolor[rgb]{1.00,0.00,0.00}{\textbf{{#1}}}}
    \newcommand{\FunctionTok}[1]{\textcolor[rgb]{0.02,0.16,0.49}{{#1}}}
    \newcommand{\RegionMarkerTok}[1]{{#1}}
    \newcommand{\ErrorTok}[1]{\textcolor[rgb]{1.00,0.00,0.00}{\textbf{{#1}}}}
    \newcommand{\NormalTok}[1]{{#1}}
    
    % Additional commands for more recent versions of Pandoc
    \newcommand{\ConstantTok}[1]{\textcolor[rgb]{0.53,0.00,0.00}{{#1}}}
    \newcommand{\SpecialCharTok}[1]{\textcolor[rgb]{0.25,0.44,0.63}{{#1}}}
    \newcommand{\VerbatimStringTok}[1]{\textcolor[rgb]{0.25,0.44,0.63}{{#1}}}
    \newcommand{\SpecialStringTok}[1]{\textcolor[rgb]{0.73,0.40,0.53}{{#1}}}
    \newcommand{\ImportTok}[1]{{#1}}
    \newcommand{\DocumentationTok}[1]{\textcolor[rgb]{0.73,0.13,0.13}{\textit{{#1}}}}
    \newcommand{\AnnotationTok}[1]{\textcolor[rgb]{0.38,0.63,0.69}{\textbf{\textit{{#1}}}}}
    \newcommand{\CommentVarTok}[1]{\textcolor[rgb]{0.38,0.63,0.69}{\textbf{\textit{{#1}}}}}
    \newcommand{\VariableTok}[1]{\textcolor[rgb]{0.10,0.09,0.49}{{#1}}}
    \newcommand{\ControlFlowTok}[1]{\textcolor[rgb]{0.00,0.44,0.13}{\textbf{{#1}}}}
    \newcommand{\OperatorTok}[1]{\textcolor[rgb]{0.40,0.40,0.40}{{#1}}}
    \newcommand{\BuiltInTok}[1]{{#1}}
    \newcommand{\ExtensionTok}[1]{{#1}}
    \newcommand{\PreprocessorTok}[1]{\textcolor[rgb]{0.74,0.48,0.00}{{#1}}}
    \newcommand{\AttributeTok}[1]{\textcolor[rgb]{0.49,0.56,0.16}{{#1}}}
    \newcommand{\InformationTok}[1]{\textcolor[rgb]{0.38,0.63,0.69}{\textbf{\textit{{#1}}}}}
    \newcommand{\WarningTok}[1]{\textcolor[rgb]{0.38,0.63,0.69}{\textbf{\textit{{#1}}}}}
    
    
    % Define a nice break command that doesn't care if a line doesn't already
    % exist.
    \def\br{\hspace*{\fill} \\* }
    % Math Jax compatibility definitions
    \def\gt{>}
    \def\lt{<}
    \let\Oldtex\TeX
    \let\Oldlatex\LaTeX
    \renewcommand{\TeX}{\textrm{\Oldtex}}
    \renewcommand{\LaTeX}{\textrm{\Oldlatex}}
    % Document parameters
    % Document title
    \newcommand*{\mytitle}{Unit 10: Error handling}
    
    % Included at the bottom of the preamble

\usepackage{microtype}


\title{\mytitle}
\author{Richard Foltyn}
\hypersetup{pdfauthor={Richard Foltyn}, pdftitle={\mytitle}}

% Remove horizontal rules in a very hackish way
\renewcommand{\rule}[2]{}

\RequirePackage{xspace}


\newcommand*{\eg}{e.g.\@\xspace}
\newcommand*{\Eg}{E.g.\@\xspace}
\newcommand*{\etc}{etc.\@\xspace}
\newcommand*{\ie}{i.e.\@\xspace}
\newcommand*{\vs}{vs.\@\xspace}
\newcommand*{\viz}{viz.\@\xspace}
\newcommand*{\US}{U.S.\@\xspace}


    
    
    
% Pygments definitions
\makeatletter
\def\PY@reset{\let\PY@it=\relax \let\PY@bf=\relax%
    \let\PY@ul=\relax \let\PY@tc=\relax%
    \let\PY@bc=\relax \let\PY@ff=\relax}
\def\PY@tok#1{\csname PY@tok@#1\endcsname}
\def\PY@toks#1+{\ifx\relax#1\empty\else%
    \PY@tok{#1}\expandafter\PY@toks\fi}
\def\PY@do#1{\PY@bc{\PY@tc{\PY@ul{%
    \PY@it{\PY@bf{\PY@ff{#1}}}}}}}
\def\PY#1#2{\PY@reset\PY@toks#1+\relax+\PY@do{#2}}

\expandafter\def\csname PY@tok@w\endcsname{\def\PY@tc##1{\textcolor[rgb]{0.73,0.73,0.73}{##1}}}
\expandafter\def\csname PY@tok@c\endcsname{\let\PY@it=\textit\def\PY@tc##1{\textcolor[rgb]{0.25,0.50,0.50}{##1}}}
\expandafter\def\csname PY@tok@cp\endcsname{\def\PY@tc##1{\textcolor[rgb]{0.74,0.48,0.00}{##1}}}
\expandafter\def\csname PY@tok@k\endcsname{\let\PY@bf=\textbf\def\PY@tc##1{\textcolor[rgb]{0.00,0.50,0.00}{##1}}}
\expandafter\def\csname PY@tok@kp\endcsname{\def\PY@tc##1{\textcolor[rgb]{0.00,0.50,0.00}{##1}}}
\expandafter\def\csname PY@tok@kt\endcsname{\def\PY@tc##1{\textcolor[rgb]{0.69,0.00,0.25}{##1}}}
\expandafter\def\csname PY@tok@o\endcsname{\def\PY@tc##1{\textcolor[rgb]{0.40,0.40,0.40}{##1}}}
\expandafter\def\csname PY@tok@ow\endcsname{\let\PY@bf=\textbf\def\PY@tc##1{\textcolor[rgb]{0.67,0.13,1.00}{##1}}}
\expandafter\def\csname PY@tok@nb\endcsname{\def\PY@tc##1{\textcolor[rgb]{0.00,0.50,0.00}{##1}}}
\expandafter\def\csname PY@tok@nf\endcsname{\def\PY@tc##1{\textcolor[rgb]{0.00,0.00,1.00}{##1}}}
\expandafter\def\csname PY@tok@nc\endcsname{\let\PY@bf=\textbf\def\PY@tc##1{\textcolor[rgb]{0.00,0.00,1.00}{##1}}}
\expandafter\def\csname PY@tok@nn\endcsname{\let\PY@bf=\textbf\def\PY@tc##1{\textcolor[rgb]{0.00,0.00,1.00}{##1}}}
\expandafter\def\csname PY@tok@ne\endcsname{\let\PY@bf=\textbf\def\PY@tc##1{\textcolor[rgb]{0.82,0.25,0.23}{##1}}}
\expandafter\def\csname PY@tok@nv\endcsname{\def\PY@tc##1{\textcolor[rgb]{0.10,0.09,0.49}{##1}}}
\expandafter\def\csname PY@tok@no\endcsname{\def\PY@tc##1{\textcolor[rgb]{0.53,0.00,0.00}{##1}}}
\expandafter\def\csname PY@tok@nl\endcsname{\def\PY@tc##1{\textcolor[rgb]{0.63,0.63,0.00}{##1}}}
\expandafter\def\csname PY@tok@ni\endcsname{\let\PY@bf=\textbf\def\PY@tc##1{\textcolor[rgb]{0.60,0.60,0.60}{##1}}}
\expandafter\def\csname PY@tok@na\endcsname{\def\PY@tc##1{\textcolor[rgb]{0.49,0.56,0.16}{##1}}}
\expandafter\def\csname PY@tok@nt\endcsname{\let\PY@bf=\textbf\def\PY@tc##1{\textcolor[rgb]{0.00,0.50,0.00}{##1}}}
\expandafter\def\csname PY@tok@nd\endcsname{\def\PY@tc##1{\textcolor[rgb]{0.67,0.13,1.00}{##1}}}
\expandafter\def\csname PY@tok@s\endcsname{\def\PY@tc##1{\textcolor[rgb]{0.73,0.13,0.13}{##1}}}
\expandafter\def\csname PY@tok@sd\endcsname{\let\PY@it=\textit\def\PY@tc##1{\textcolor[rgb]{0.73,0.13,0.13}{##1}}}
\expandafter\def\csname PY@tok@si\endcsname{\let\PY@bf=\textbf\def\PY@tc##1{\textcolor[rgb]{0.73,0.40,0.53}{##1}}}
\expandafter\def\csname PY@tok@se\endcsname{\let\PY@bf=\textbf\def\PY@tc##1{\textcolor[rgb]{0.73,0.40,0.13}{##1}}}
\expandafter\def\csname PY@tok@sr\endcsname{\def\PY@tc##1{\textcolor[rgb]{0.73,0.40,0.53}{##1}}}
\expandafter\def\csname PY@tok@ss\endcsname{\def\PY@tc##1{\textcolor[rgb]{0.10,0.09,0.49}{##1}}}
\expandafter\def\csname PY@tok@sx\endcsname{\def\PY@tc##1{\textcolor[rgb]{0.00,0.50,0.00}{##1}}}
\expandafter\def\csname PY@tok@m\endcsname{\def\PY@tc##1{\textcolor[rgb]{0.40,0.40,0.40}{##1}}}
\expandafter\def\csname PY@tok@gh\endcsname{\let\PY@bf=\textbf\def\PY@tc##1{\textcolor[rgb]{0.00,0.00,0.50}{##1}}}
\expandafter\def\csname PY@tok@gu\endcsname{\let\PY@bf=\textbf\def\PY@tc##1{\textcolor[rgb]{0.50,0.00,0.50}{##1}}}
\expandafter\def\csname PY@tok@gd\endcsname{\def\PY@tc##1{\textcolor[rgb]{0.63,0.00,0.00}{##1}}}
\expandafter\def\csname PY@tok@gi\endcsname{\def\PY@tc##1{\textcolor[rgb]{0.00,0.63,0.00}{##1}}}
\expandafter\def\csname PY@tok@gr\endcsname{\def\PY@tc##1{\textcolor[rgb]{1.00,0.00,0.00}{##1}}}
\expandafter\def\csname PY@tok@ge\endcsname{\let\PY@it=\textit}
\expandafter\def\csname PY@tok@gs\endcsname{\let\PY@bf=\textbf}
\expandafter\def\csname PY@tok@gp\endcsname{\let\PY@bf=\textbf\def\PY@tc##1{\textcolor[rgb]{0.00,0.00,0.50}{##1}}}
\expandafter\def\csname PY@tok@go\endcsname{\def\PY@tc##1{\textcolor[rgb]{0.53,0.53,0.53}{##1}}}
\expandafter\def\csname PY@tok@gt\endcsname{\def\PY@tc##1{\textcolor[rgb]{0.00,0.27,0.87}{##1}}}
\expandafter\def\csname PY@tok@err\endcsname{\def\PY@bc##1{\setlength{\fboxsep}{0pt}\fcolorbox[rgb]{1.00,0.00,0.00}{1,1,1}{\strut ##1}}}
\expandafter\def\csname PY@tok@kc\endcsname{\let\PY@bf=\textbf\def\PY@tc##1{\textcolor[rgb]{0.00,0.50,0.00}{##1}}}
\expandafter\def\csname PY@tok@kd\endcsname{\let\PY@bf=\textbf\def\PY@tc##1{\textcolor[rgb]{0.00,0.50,0.00}{##1}}}
\expandafter\def\csname PY@tok@kn\endcsname{\let\PY@bf=\textbf\def\PY@tc##1{\textcolor[rgb]{0.00,0.50,0.00}{##1}}}
\expandafter\def\csname PY@tok@kr\endcsname{\let\PY@bf=\textbf\def\PY@tc##1{\textcolor[rgb]{0.00,0.50,0.00}{##1}}}
\expandafter\def\csname PY@tok@bp\endcsname{\def\PY@tc##1{\textcolor[rgb]{0.00,0.50,0.00}{##1}}}
\expandafter\def\csname PY@tok@fm\endcsname{\def\PY@tc##1{\textcolor[rgb]{0.00,0.00,1.00}{##1}}}
\expandafter\def\csname PY@tok@vc\endcsname{\def\PY@tc##1{\textcolor[rgb]{0.10,0.09,0.49}{##1}}}
\expandafter\def\csname PY@tok@vg\endcsname{\def\PY@tc##1{\textcolor[rgb]{0.10,0.09,0.49}{##1}}}
\expandafter\def\csname PY@tok@vi\endcsname{\def\PY@tc##1{\textcolor[rgb]{0.10,0.09,0.49}{##1}}}
\expandafter\def\csname PY@tok@vm\endcsname{\def\PY@tc##1{\textcolor[rgb]{0.10,0.09,0.49}{##1}}}
\expandafter\def\csname PY@tok@sa\endcsname{\def\PY@tc##1{\textcolor[rgb]{0.73,0.13,0.13}{##1}}}
\expandafter\def\csname PY@tok@sb\endcsname{\def\PY@tc##1{\textcolor[rgb]{0.73,0.13,0.13}{##1}}}
\expandafter\def\csname PY@tok@sc\endcsname{\def\PY@tc##1{\textcolor[rgb]{0.73,0.13,0.13}{##1}}}
\expandafter\def\csname PY@tok@dl\endcsname{\def\PY@tc##1{\textcolor[rgb]{0.73,0.13,0.13}{##1}}}
\expandafter\def\csname PY@tok@s2\endcsname{\def\PY@tc##1{\textcolor[rgb]{0.73,0.13,0.13}{##1}}}
\expandafter\def\csname PY@tok@sh\endcsname{\def\PY@tc##1{\textcolor[rgb]{0.73,0.13,0.13}{##1}}}
\expandafter\def\csname PY@tok@s1\endcsname{\def\PY@tc##1{\textcolor[rgb]{0.73,0.13,0.13}{##1}}}
\expandafter\def\csname PY@tok@mb\endcsname{\def\PY@tc##1{\textcolor[rgb]{0.40,0.40,0.40}{##1}}}
\expandafter\def\csname PY@tok@mf\endcsname{\def\PY@tc##1{\textcolor[rgb]{0.40,0.40,0.40}{##1}}}
\expandafter\def\csname PY@tok@mh\endcsname{\def\PY@tc##1{\textcolor[rgb]{0.40,0.40,0.40}{##1}}}
\expandafter\def\csname PY@tok@mi\endcsname{\def\PY@tc##1{\textcolor[rgb]{0.40,0.40,0.40}{##1}}}
\expandafter\def\csname PY@tok@il\endcsname{\def\PY@tc##1{\textcolor[rgb]{0.40,0.40,0.40}{##1}}}
\expandafter\def\csname PY@tok@mo\endcsname{\def\PY@tc##1{\textcolor[rgb]{0.40,0.40,0.40}{##1}}}
\expandafter\def\csname PY@tok@ch\endcsname{\let\PY@it=\textit\def\PY@tc##1{\textcolor[rgb]{0.25,0.50,0.50}{##1}}}
\expandafter\def\csname PY@tok@cm\endcsname{\let\PY@it=\textit\def\PY@tc##1{\textcolor[rgb]{0.25,0.50,0.50}{##1}}}
\expandafter\def\csname PY@tok@cpf\endcsname{\let\PY@it=\textit\def\PY@tc##1{\textcolor[rgb]{0.25,0.50,0.50}{##1}}}
\expandafter\def\csname PY@tok@c1\endcsname{\let\PY@it=\textit\def\PY@tc##1{\textcolor[rgb]{0.25,0.50,0.50}{##1}}}
\expandafter\def\csname PY@tok@cs\endcsname{\let\PY@it=\textit\def\PY@tc##1{\textcolor[rgb]{0.25,0.50,0.50}{##1}}}

\def\PYZbs{\char`\\}
\def\PYZus{\char`\_}
\def\PYZob{\char`\{}
\def\PYZcb{\char`\}}
\def\PYZca{\char`\^}
\def\PYZam{\char`\&}
\def\PYZlt{\char`\<}
\def\PYZgt{\char`\>}
\def\PYZsh{\char`\#}
\def\PYZpc{\char`\%}
\def\PYZdl{\char`\$}
\def\PYZhy{\char`\-}
\def\PYZsq{\char`\'}
\def\PYZdq{\char`\"}
\def\PYZti{\char`\~}
% for compatibility with earlier versions
\def\PYZat{@}
\def\PYZlb{[}
\def\PYZrb{]}
\makeatother


    % For linebreaks inside Verbatim environment from package fancyvrb. 
    \makeatletter
        \newbox\Wrappedcontinuationbox 
        \newbox\Wrappedvisiblespacebox 
        \newcommand*\Wrappedvisiblespace {\textcolor{red}{\textvisiblespace}} 
        \newcommand*\Wrappedcontinuationsymbol {\textcolor{red}{\llap{\tiny$\m@th\hookrightarrow$}}} 
        \newcommand*\Wrappedcontinuationindent {3ex } 
        \newcommand*\Wrappedafterbreak {\kern\Wrappedcontinuationindent\copy\Wrappedcontinuationbox} 
        % Take advantage of the already applied Pygments mark-up to insert 
        % potential linebreaks for TeX processing. 
        %        {, <, #, %, $, ' and ": go to next line. 
        %        _, }, ^, &, >, - and ~: stay at end of broken line. 
        % Use of \textquotesingle for straight quote. 
        \newcommand*\Wrappedbreaksatspecials {% 
            \def\PYGZus{\discretionary{\char`\_}{\Wrappedafterbreak}{\char`\_}}% 
            \def\PYGZob{\discretionary{}{\Wrappedafterbreak\char`\{}{\char`\{}}% 
            \def\PYGZcb{\discretionary{\char`\}}{\Wrappedafterbreak}{\char`\}}}% 
            \def\PYGZca{\discretionary{\char`\^}{\Wrappedafterbreak}{\char`\^}}% 
            \def\PYGZam{\discretionary{\char`\&}{\Wrappedafterbreak}{\char`\&}}% 
            \def\PYGZlt{\discretionary{}{\Wrappedafterbreak\char`\<}{\char`\<}}% 
            \def\PYGZgt{\discretionary{\char`\>}{\Wrappedafterbreak}{\char`\>}}% 
            \def\PYGZsh{\discretionary{}{\Wrappedafterbreak\char`\#}{\char`\#}}% 
            \def\PYGZpc{\discretionary{}{\Wrappedafterbreak\char`\%}{\char`\%}}% 
            \def\PYGZdl{\discretionary{}{\Wrappedafterbreak\char`\$}{\char`\$}}% 
            \def\PYGZhy{\discretionary{\char`\-}{\Wrappedafterbreak}{\char`\-}}% 
            \def\PYGZsq{\discretionary{}{\Wrappedafterbreak\textquotesingle}{\textquotesingle}}% 
            \def\PYGZdq{\discretionary{}{\Wrappedafterbreak\char`\"}{\char`\"}}% 
            \def\PYGZti{\discretionary{\char`\~}{\Wrappedafterbreak}{\char`\~}}% 
        } 
        % Some characters . , ; ? ! / are not pygmentized. 
        % This macro makes them "active" and they will insert potential linebreaks 
        \newcommand*\Wrappedbreaksatpunct {% 
            \lccode`\~`\.\lowercase{\def~}{\discretionary{\hbox{\char`\.}}{\Wrappedafterbreak}{\hbox{\char`\.}}}% 
            \lccode`\~`\,\lowercase{\def~}{\discretionary{\hbox{\char`\,}}{\Wrappedafterbreak}{\hbox{\char`\,}}}% 
            \lccode`\~`\;\lowercase{\def~}{\discretionary{\hbox{\char`\;}}{\Wrappedafterbreak}{\hbox{\char`\;}}}% 
            \lccode`\~`\:\lowercase{\def~}{\discretionary{\hbox{\char`\:}}{\Wrappedafterbreak}{\hbox{\char`\:}}}% 
            \lccode`\~`\?\lowercase{\def~}{\discretionary{\hbox{\char`\?}}{\Wrappedafterbreak}{\hbox{\char`\?}}}% 
            \lccode`\~`\!\lowercase{\def~}{\discretionary{\hbox{\char`\!}}{\Wrappedafterbreak}{\hbox{\char`\!}}}% 
            \lccode`\~`\/\lowercase{\def~}{\discretionary{\hbox{\char`\/}}{\Wrappedafterbreak}{\hbox{\char`\/}}}% 
            \catcode`\.\active
            \catcode`\,\active 
            \catcode`\;\active
            \catcode`\:\active
            \catcode`\?\active
            \catcode`\!\active
            \catcode`\/\active 
            \lccode`\~`\~ 	
        }
    \makeatother

    \let\OriginalVerbatim=\Verbatim
    \makeatletter
    \renewcommand{\Verbatim}[1][1]{%
        %\parskip\z@skip
        \sbox\Wrappedcontinuationbox {\Wrappedcontinuationsymbol}%
        \sbox\Wrappedvisiblespacebox {\FV@SetupFont\Wrappedvisiblespace}%
        \def\FancyVerbFormatLine ##1{\hsize\linewidth
            \vtop{\raggedright\hyphenpenalty\z@\exhyphenpenalty\z@
                \doublehyphendemerits\z@\finalhyphendemerits\z@
                \strut ##1\strut}%
        }%
        % If the linebreak is at a space, the latter will be displayed as visible
        % space at end of first line, and a continuation symbol starts next line.
        % Stretch/shrink are however usually zero for typewriter font.
        \def\FV@Space {%
            \nobreak\hskip\z@ plus\fontdimen3\font minus\fontdimen4\font
            \discretionary{\copy\Wrappedvisiblespacebox}{\Wrappedafterbreak}
            {\kern\fontdimen2\font}%
        }%
        
        % Allow breaks at special characters using \PYG... macros.
        \Wrappedbreaksatspecials
        % Breaks at punctuation characters . , ; ? ! and / need catcode=\active 	
        \OriginalVerbatim[#1,fontsize=\small,codes*=\Wrappedbreaksatpunct]%
    }
    \makeatother

    % Exact colors from NB
    \definecolor{incolor}{HTML}{303F9F}
    \definecolor{outcolor}{HTML}{D84315}
    \definecolor{cellborder}{HTML}{CFCFCF}
    \definecolor{cellbackground}{HTML}{F7F7F7}
    
    % prompt
    \makeatletter
    \newcommand{\boxspacing}{\kern\kvtcb@left@rule\kern\kvtcb@boxsep}
    \makeatother
    \newcommand{\prompt}[4]{
        {\ttfamily\llap{{\color{#2}[#3]:\hspace{3pt}#4}}\vspace{-\baselineskip}}
    }
    

    
    % Prevent overflowing lines due to hard-to-break entities
    \sloppy 
    % Setup hyperref package
    \hypersetup{
      breaklinks=true,  % so long urls are correctly broken across lines
      colorlinks=true,
      urlcolor=urlcolor,
      linkcolor=linkcolor,
      citecolor=citecolor,
      }
    % Slightly bigger margins than the latex defaults
    
    \geometry{verbose,tmargin=1in,bmargin=1in,lmargin=1in,rmargin=1in}
    
    

\begin{document}
    
    \maketitle
    \tableofcontents

    

    
    \hypertarget{error-handling}{%
\section{Error handling}\label{error-handling}}

In this unit we will briefly look at error handling in Python. The
Python approach to error handling is ``to ask for forgiveness rather
than for permission.'' This means that when writing Python code, we
frequently don't check whether some data satisfies certain requirements,
but we instead attempt to clean up once something does not work as
expected.

    \hypertarget{exceptions}{%
\subsection{Exceptions}\label{exceptions}}

If something goes wrong in a function, we in principle have two options
to communicate the error to the caller:

\begin{enumerate}
\def\labelenumi{\arabic{enumi}.}
\item
  We can return some special value (a status code or error flag) that
  signals when something fails.

  This approach is quite inelegant, since error codes can overlap with
  the actual result a function would return in the absence of error. For
  this reasons, functions need to implement two different return values
  and reserve one for the status code.

  In Python, this could look like this:

\begin{verbatim}
def func(x):
    # process x
    # Two return values: actual result and error flag
    return result, flag
\end{verbatim}
\item
  We can use so-called exceptions for error handling. This is the
  approach taken by almost all modern languages such as Java, C++ and
  also Python (see here for the
  \href{https://docs.python.org/3/tutorial/errors.html}{official
  documentation} on error and exception handling).

  Exceptions provide means to communicate errors that are completely
  independent of regular return values.

  Furthermore, exceptions propagate along the entire call stack: If we
  call \texttt{func1()}, which in turn calls \texttt{func2()}, and an
  error occurs in \texttt{func2()}, there is no need to handle this
  error in \texttt{func1()}: the exception will automatically be
  propagated to the caller of \texttt{func1()}.
\end{enumerate}

    \hypertarget{common-exceptions}{%
\subsubsection{Common exceptions}\label{common-exceptions}}

We have already encountered numerous exceptions throughout this course,
but so far we did not know how to handle them other than fixing the code
that produced the exception.

There are numerous exceptions in Python, see
\href{https://docs.python.org/3/library/exceptions.html}{here} for a
list of built-in ones. We provide a few examples of exceptions that you
are most likely to encounter below.

\emph{Examples:}

Trying to access an element in a collection outside of the permissible
ranger produces an \texttt{IndexError}.

    \begin{tcolorbox}[breakable, size=fbox, boxrule=1pt, pad at break*=1mm,colback=cellbackground, colframe=cellborder]
\prompt{In}{incolor}{1}{\boxspacing}
\begin{Verbatim}[commandchars=\\\{\}]
\PY{c+c1}{\PYZsh{} access to out\PYZhy{}of\PYZhy{}bounds index in a collection}
\PY{n}{items} \PY{o}{=} \PY{l+m+mi}{1}\PY{p}{,} \PY{l+m+mi}{2}\PY{p}{,} \PY{l+m+mi}{3}
\PY{n}{items}\PY{p}{[}\PY{l+m+mi}{5}\PY{p}{]}
\end{Verbatim}
\end{tcolorbox}

    \begin{Verbatim}[commandchars=\\\{\}, frame=single, framerule=2mm, rulecolor=\color{outerrorbackground}]
\textcolor{ansi-red}{---------------------------------------------------------------------------}
\textcolor{ansi-red}{IndexError}                                Traceback (most recent call last)
\textcolor{ansi-green}{<ipython-input-1-75d6658ee705>} in \textcolor{ansi-cyan}{<module>}
\textcolor{ansi-green-intense}{\textbf{      1}} \textcolor{ansi-red}{\# access to out-of-bounds index in a collection}
\textcolor{ansi-green-intense}{\textbf{      2}} items \textcolor{ansi-blue}{=} \textcolor{ansi-cyan}{1}\textcolor{ansi-blue}{,} \textcolor{ansi-cyan}{2}\textcolor{ansi-blue}{,} \textcolor{ansi-cyan}{3}
\textcolor{ansi-green}{----> 3}\textcolor{ansi-red}{ }items\textcolor{ansi-blue}{[}\textcolor{ansi-cyan}{5}\textcolor{ansi-blue}{]}

\textcolor{ansi-red}{IndexError}: tuple index out of range
    \end{Verbatim}

    Retrieving a non-existent key in a dictionary raises another type of
exception, a \texttt{KeyError}.

    \begin{tcolorbox}[breakable, size=fbox, boxrule=1pt, pad at break*=1mm,colback=cellbackground, colframe=cellborder]
\prompt{In}{incolor}{2}{\boxspacing}
\begin{Verbatim}[commandchars=\\\{\}]
\PY{c+c1}{\PYZsh{} Access non\PYZhy{}existant dictionary key}
\PY{n}{dct} \PY{o}{=} \PY{p}{\PYZob{}}\PY{l+s+s1}{\PYZsq{}}\PY{l+s+s1}{language}\PY{l+s+s1}{\PYZsq{}}\PY{p}{:} \PY{l+s+s1}{\PYZsq{}}\PY{l+s+s1}{Python}\PY{l+s+s1}{\PYZsq{}}\PY{p}{,} \PY{l+s+s1}{\PYZsq{}}\PY{l+s+s1}{version}\PY{l+s+s1}{\PYZsq{}}\PY{p}{:} \PY{l+m+mf}{3.8}\PY{p}{\PYZcb{}}
\PY{n}{dct}\PY{p}{[}\PY{l+s+s1}{\PYZsq{}}\PY{l+s+s1}{course}\PY{l+s+s1}{\PYZsq{}}\PY{p}{]}
\end{Verbatim}
\end{tcolorbox}

    \begin{Verbatim}[commandchars=\\\{\}, frame=single, framerule=2mm, rulecolor=\color{outerrorbackground}]
\textcolor{ansi-red}{---------------------------------------------------------------------------}
\textcolor{ansi-red}{KeyError}                                  Traceback (most recent call last)
\textcolor{ansi-green}{<ipython-input-1-84b072dc0e31>} in \textcolor{ansi-cyan}{<module>}
\textcolor{ansi-green-intense}{\textbf{      1}} \textcolor{ansi-red}{\# Access non-existant dictionary key}
\textcolor{ansi-green-intense}{\textbf{      2}} dct \textcolor{ansi-blue}{=} \textcolor{ansi-blue}{\{}\textcolor{ansi-blue}{'language'}\textcolor{ansi-blue}{:} \textcolor{ansi-blue}{'Python'}\textcolor{ansi-blue}{,} \textcolor{ansi-blue}{'version'}\textcolor{ansi-blue}{:} \textcolor{ansi-cyan}{3.8}\textcolor{ansi-blue}{\}}
\textcolor{ansi-green}{----> 3}\textcolor{ansi-red}{ }dct\textcolor{ansi-blue}{[}\textcolor{ansi-blue}{'course'}\textcolor{ansi-blue}{]}

\textcolor{ansi-red}{KeyError}: 'course'
    \end{Verbatim}

    Mistakenly trying to access a non-existent attribute will trigger an
\texttt{AttributeError}:

    \begin{tcolorbox}[breakable, size=fbox, boxrule=1pt, pad at break*=1mm,colback=cellbackground, colframe=cellborder]
\prompt{In}{incolor}{3}{\boxspacing}
\begin{Verbatim}[commandchars=\\\{\}]
\PY{n}{value} \PY{o}{=} \PY{l+m+mf}{1.0}
\PY{n}{value}\PY{o}{.}\PY{n}{shape} 
\end{Verbatim}
\end{tcolorbox}

    \begin{Verbatim}[commandchars=\\\{\}, frame=single, framerule=2mm, rulecolor=\color{outerrorbackground}]
\textcolor{ansi-red}{---------------------------------------------------------------------------}
\textcolor{ansi-red}{AttributeError}                            Traceback (most recent call last)
\textcolor{ansi-green}{<ipython-input-1-7bb7836aaf0a>} in \textcolor{ansi-cyan}{<module>}
\textcolor{ansi-green-intense}{\textbf{      1}} value \textcolor{ansi-blue}{=} \textcolor{ansi-cyan}{1.0}
\textcolor{ansi-green}{----> 2}\textcolor{ansi-red}{ }value\textcolor{ansi-blue}{.}shape

\textcolor{ansi-red}{AttributeError}: 'float' object has no attribute 'shape'
    \end{Verbatim}

    When we try to apply an operation to data that does not support that
particular operation, we get a \texttt{TypeError}:

    \begin{tcolorbox}[breakable, size=fbox, boxrule=1pt, pad at break*=1mm,colback=cellbackground, colframe=cellborder]
\prompt{In}{incolor}{4}{\boxspacing}
\begin{Verbatim}[commandchars=\\\{\}]
\PY{n}{items} \PY{o}{=} \PY{l+m+mi}{1}\PY{p}{,} \PY{l+m+mi}{2}\PY{p}{,} \PY{l+m+mi}{3}
\PY{n}{items} \PY{o}{+} \PY{l+m+mi}{1}
\end{Verbatim}
\end{tcolorbox}

    \begin{Verbatim}[commandchars=\\\{\}, frame=single, framerule=2mm, rulecolor=\color{outerrorbackground}]
\textcolor{ansi-red}{---------------------------------------------------------------------------}
\textcolor{ansi-red}{TypeError}                                 Traceback (most recent call last)
\textcolor{ansi-green}{<ipython-input-1-ac055c45dd3e>} in \textcolor{ansi-cyan}{<module>}
\textcolor{ansi-green-intense}{\textbf{      1}} items \textcolor{ansi-blue}{=} \textcolor{ansi-cyan}{1}\textcolor{ansi-blue}{,} \textcolor{ansi-cyan}{2}\textcolor{ansi-blue}{,} \textcolor{ansi-cyan}{3}
\textcolor{ansi-green}{----> 2}\textcolor{ansi-red}{ }items \textcolor{ansi-blue}{+} \textcolor{ansi-cyan}{1}

\textcolor{ansi-red}{TypeError}: can only concatenate tuple (not "int") to tuple
    \end{Verbatim}

    Division by zero also triggers an exception of type
\texttt{ZeroDivisionError}:

    \begin{tcolorbox}[breakable, size=fbox, boxrule=1pt, pad at break*=1mm,colback=cellbackground, colframe=cellborder]
\prompt{In}{incolor}{5}{\boxspacing}
\begin{Verbatim}[commandchars=\\\{\}]
\PY{l+m+mi}{1}\PY{o}{/}\PY{l+m+mi}{0}
\end{Verbatim}
\end{tcolorbox}

    \begin{Verbatim}[commandchars=\\\{\}, frame=single, framerule=2mm, rulecolor=\color{outerrorbackground}]
\textcolor{ansi-red}{---------------------------------------------------------------------------}
\textcolor{ansi-red}{ZeroDivisionError}                         Traceback (most recent call last)
\textcolor{ansi-green}{<ipython-input-1-9e1622b385b6>} in \textcolor{ansi-cyan}{<module>}
\textcolor{ansi-green}{----> 1}\textcolor{ansi-red}{ }\textcolor{ansi-cyan}{1}\textcolor{ansi-blue}{/}\textcolor{ansi-cyan}{0}

\textcolor{ansi-red}{ZeroDivisionError}: division by zero
    \end{Verbatim}

    Attempting to import a module or symbol from within a module that does
not exist raises an \texttt{ImportError}:

    \begin{tcolorbox}[breakable, size=fbox, boxrule=1pt, pad at break*=1mm,colback=cellbackground, colframe=cellborder]
\prompt{In}{incolor}{6}{\boxspacing}
\begin{Verbatim}[commandchars=\\\{\}]
\PY{k+kn}{from} \PY{n+nn}{numpy} \PY{k+kn}{import} \PY{n}{function\PYZus{}that\PYZus{}does\PYZus{}not\PYZus{}exist}
\end{Verbatim}
\end{tcolorbox}

    \begin{Verbatim}[commandchars=\\\{\}, frame=single, framerule=2mm, rulecolor=\color{outerrorbackground}]
\textcolor{ansi-red}{---------------------------------------------------------------------------}
\textcolor{ansi-red}{ImportError}                               Traceback (most recent call last)
\textcolor{ansi-green}{<ipython-input-1-d7647cfd2f6a>} in \textcolor{ansi-cyan}{<module>}
\textcolor{ansi-green}{----> 1}\textcolor{ansi-red}{ }\textcolor{ansi-green}{from} numpy \textcolor{ansi-green}{import} function\_that\_does\_not\_exist

\textcolor{ansi-red}{ImportError}: cannot import name 'function\_that\_does\_not\_exist' from 'numpy' (/home/richard/.conda/envs/py3-default/lib/python3.8/site-packages/numpy/\_\_init\_\_.py)
    \end{Verbatim}

    Performing an operation on arrays of non-conforming shape produces a
\texttt{ValueError}:

    \begin{tcolorbox}[breakable, size=fbox, boxrule=1pt, pad at break*=1mm,colback=cellbackground, colframe=cellborder]
\prompt{In}{incolor}{7}{\boxspacing}
\begin{Verbatim}[commandchars=\\\{\}]
\PY{k+kn}{import} \PY{n+nn}{numpy} \PY{k}{as} \PY{n+nn}{np}

\PY{n}{a} \PY{o}{=} \PY{n}{np}\PY{o}{.}\PY{n}{arange}\PY{p}{(}\PY{l+m+mi}{3}\PY{p}{)}
\PY{n}{b} \PY{o}{=} \PY{n}{np}\PY{o}{.}\PY{n}{arange}\PY{p}{(}\PY{l+m+mi}{2}\PY{p}{)}
\PY{n}{a} \PY{o}{+} \PY{n}{b}
\end{Verbatim}
\end{tcolorbox}

    \begin{Verbatim}[commandchars=\\\{\}, frame=single, framerule=2mm, rulecolor=\color{outerrorbackground}]
\textcolor{ansi-red}{---------------------------------------------------------------------------}
\textcolor{ansi-red}{ValueError}                                Traceback (most recent call last)
\textcolor{ansi-green}{<ipython-input-1-c4d121264809>} in \textcolor{ansi-cyan}{<module>}
\textcolor{ansi-green-intense}{\textbf{      3}} a \textcolor{ansi-blue}{=} np\textcolor{ansi-blue}{.}arange\textcolor{ansi-blue}{(}\textcolor{ansi-cyan}{3}\textcolor{ansi-blue}{)}
\textcolor{ansi-green-intense}{\textbf{      4}} b \textcolor{ansi-blue}{=} np\textcolor{ansi-blue}{.}arange\textcolor{ansi-blue}{(}\textcolor{ansi-cyan}{2}\textcolor{ansi-blue}{)}
\textcolor{ansi-green}{----> 5}\textcolor{ansi-red}{ }a \textcolor{ansi-blue}{+} b

\textcolor{ansi-red}{ValueError}: operands could not be broadcast together with shapes (3,) (2,) 
    \end{Verbatim}

    Trying to open a non-existing file will raise an \texttt{OSError}.

    \begin{tcolorbox}[breakable, size=fbox, boxrule=1pt, pad at break*=1mm,colback=cellbackground, colframe=cellborder]
\prompt{In}{incolor}{8}{\boxspacing}
\begin{Verbatim}[commandchars=\\\{\}]
\PY{k+kn}{import} \PY{n+nn}{numpy} \PY{k}{as} \PY{n+nn}{np}

\PY{n}{data} \PY{o}{=} \PY{n}{np}\PY{o}{.}\PY{n}{loadtxt}\PY{p}{(}\PY{l+s+s1}{\PYZsq{}}\PY{l+s+s1}{path/to/nonexisting/file.txt}\PY{l+s+s1}{\PYZsq{}}\PY{p}{)}
\end{Verbatim}
\end{tcolorbox}

    \begin{Verbatim}[commandchars=\\\{\}, frame=single, framerule=2mm, rulecolor=\color{outerrorbackground}]
\textcolor{ansi-red}{---------------------------------------------------------------------------}
\textcolor{ansi-red}{OSError}                                   Traceback (most recent call last)
\textcolor{ansi-green}{<ipython-input-1-ed79c4c059ee>} in \textcolor{ansi-cyan}{<module>}
\textcolor{ansi-green-intense}{\textbf{      1}} \textcolor{ansi-green}{import} numpy \textcolor{ansi-green}{as} np
\textcolor{ansi-green-intense}{\textbf{      2}} 
\textcolor{ansi-green}{----> 3}\textcolor{ansi-red}{ }data \textcolor{ansi-blue}{=} np\textcolor{ansi-blue}{.}loadtxt\textcolor{ansi-blue}{(}\textcolor{ansi-blue}{'path/to/nonexisting/file.txt'}\textcolor{ansi-blue}{)}

\textcolor{ansi-green}{\textasciitilde{}/.conda/envs/py3-default/lib/python3.8/site-packages/numpy/lib/npyio.py} in \textcolor{ansi-cyan}{loadtxt}\textcolor{ansi-blue}{(fname, dtype, comments, delimiter, converters, skiprows, usecols, unpack, ndmin, encoding, max\_rows)}
\textcolor{ansi-green-intense}{\textbf{    959}}             fname \textcolor{ansi-blue}{=} os\_fspath\textcolor{ansi-blue}{(}fname\textcolor{ansi-blue}{)}
\textcolor{ansi-green-intense}{\textbf{    960}}         \textcolor{ansi-green}{if} \_is\_string\_like\textcolor{ansi-blue}{(}fname\textcolor{ansi-blue}{)}\textcolor{ansi-blue}{:}
\textcolor{ansi-green}{--> 961}\textcolor{ansi-red}{             }fh \textcolor{ansi-blue}{=} np\textcolor{ansi-blue}{.}lib\textcolor{ansi-blue}{.}\_datasource\textcolor{ansi-blue}{.}open\textcolor{ansi-blue}{(}fname\textcolor{ansi-blue}{,} \textcolor{ansi-blue}{'rt'}\textcolor{ansi-blue}{,} encoding\textcolor{ansi-blue}{=}encoding\textcolor{ansi-blue}{)}
\textcolor{ansi-green-intense}{\textbf{    962}}             fencoding \textcolor{ansi-blue}{=} getattr\textcolor{ansi-blue}{(}fh\textcolor{ansi-blue}{,} \textcolor{ansi-blue}{'encoding'}\textcolor{ansi-blue}{,} \textcolor{ansi-blue}{'latin1'}\textcolor{ansi-blue}{)}
\textcolor{ansi-green-intense}{\textbf{    963}}             fh \textcolor{ansi-blue}{=} iter\textcolor{ansi-blue}{(}fh\textcolor{ansi-blue}{)}

\textcolor{ansi-green}{\textasciitilde{}/.conda/envs/py3-default/lib/python3.8/site-packages/numpy/lib/\_datasource.py} in \textcolor{ansi-cyan}{open}\textcolor{ansi-blue}{(path, mode, destpath, encoding, newline)}
\textcolor{ansi-green-intense}{\textbf{    193}} 
\textcolor{ansi-green-intense}{\textbf{    194}}     ds \textcolor{ansi-blue}{=} DataSource\textcolor{ansi-blue}{(}destpath\textcolor{ansi-blue}{)}
\textcolor{ansi-green}{--> 195}\textcolor{ansi-red}{     }\textcolor{ansi-green}{return} ds\textcolor{ansi-blue}{.}open\textcolor{ansi-blue}{(}path\textcolor{ansi-blue}{,} mode\textcolor{ansi-blue}{,} encoding\textcolor{ansi-blue}{=}encoding\textcolor{ansi-blue}{,} newline\textcolor{ansi-blue}{=}newline\textcolor{ansi-blue}{)}
\textcolor{ansi-green-intense}{\textbf{    196}} 
\textcolor{ansi-green-intense}{\textbf{    197}} 

\textcolor{ansi-green}{\textasciitilde{}/.conda/envs/py3-default/lib/python3.8/site-packages/numpy/lib/\_datasource.py} in \textcolor{ansi-cyan}{open}\textcolor{ansi-blue}{(self, path, mode, encoding, newline)}
\textcolor{ansi-green-intense}{\textbf{    533}}                                       encoding=encoding, newline=newline)
\textcolor{ansi-green-intense}{\textbf{    534}}         \textcolor{ansi-green}{else}\textcolor{ansi-blue}{:}
\textcolor{ansi-green}{--> 535}\textcolor{ansi-red}{             }\textcolor{ansi-green}{raise} IOError\textcolor{ansi-blue}{(}\textcolor{ansi-blue}{"\%s not found."} \textcolor{ansi-blue}{\%} path\textcolor{ansi-blue}{)}
\textcolor{ansi-green-intense}{\textbf{    536}} 
\textcolor{ansi-green-intense}{\textbf{    537}} 

\textcolor{ansi-red}{OSError}: path/to/nonexisting/file.txt not found.
    \end{Verbatim}

    \hypertarget{handling-errors}{%
\subsection{Handling errors}\label{handling-errors}}

As you just saw, there are numerous types of exceptions raised by Python
libraries we use every day. We can handle these in two ways:

\begin{enumerate}
\def\labelenumi{\arabic{enumi}.}
\tightlist
\item
  Avoid errors before they arise.
\item
  Catch exceptions once they arise in special exception-handling blocks.
\end{enumerate}

    \hypertarget{avoiding-errors}{%
\subsubsection{Avoiding errors}\label{avoiding-errors}}

We could have avoided almost all of the above exception if we had
surrounded them with \texttt{if} statements and checked whether an
operation could actually be performed.

This, however, is usually not the way we write Python code, unless we
are implementing library functions that need to work in situations over
which we have little control. We certainly don't want to clutter
``regular'' code with \texttt{if} statements everywhere.

There are other ways to avoid errors.

\emph{Examples:}

    Returning to the dictionary example, we could write something like this:

    \begin{tcolorbox}[breakable, size=fbox, boxrule=1pt, pad at break*=1mm,colback=cellbackground, colframe=cellborder]
\prompt{In}{incolor}{9}{\boxspacing}
\begin{Verbatim}[commandchars=\\\{\}]
\PY{c+c1}{\PYZsh{} Access non\PYZhy{}existant dictionary key}
\PY{n}{dct} \PY{o}{=} \PY{p}{\PYZob{}}\PY{l+s+s1}{\PYZsq{}}\PY{l+s+s1}{language}\PY{l+s+s1}{\PYZsq{}}\PY{p}{:} \PY{l+s+s1}{\PYZsq{}}\PY{l+s+s1}{Python}\PY{l+s+s1}{\PYZsq{}}\PY{p}{,} \PY{l+s+s1}{\PYZsq{}}\PY{l+s+s1}{version}\PY{l+s+s1}{\PYZsq{}}\PY{p}{:} \PY{l+m+mf}{3.8}\PY{p}{\PYZcb{}}
\PY{k}{if} \PY{l+s+s1}{\PYZsq{}}\PY{l+s+s1}{course}\PY{l+s+s1}{\PYZsq{}} \PY{o+ow}{in} \PY{n}{dct}\PY{p}{:}
    \PY{n+nb}{print}\PY{p}{(}\PY{n}{dct}\PY{p}{[}\PY{l+s+s1}{\PYZsq{}}\PY{l+s+s1}{course}\PY{l+s+s1}{\PYZsq{}}\PY{p}{]}\PY{p}{)}
\end{Verbatim}
\end{tcolorbox}

    However, if we have a default value that should be used whenever a key
is not present, we can more elegently use the \texttt{get()} method
which accepts a default value. No \texttt{if}'s needed:

    \begin{tcolorbox}[breakable, size=fbox, boxrule=1pt, pad at break*=1mm,colback=cellbackground, colframe=cellborder]
\prompt{In}{incolor}{10}{\boxspacing}
\begin{Verbatim}[commandchars=\\\{\}]
\PY{c+c1}{\PYZsh{} access non\PYZhy{}existing key}
\PY{n}{dct}\PY{o}{.}\PY{n}{get}\PY{p}{(}\PY{l+s+s1}{\PYZsq{}}\PY{l+s+s1}{course}\PY{l+s+s1}{\PYZsq{}}\PY{p}{,} \PY{l+s+s1}{\PYZsq{}}\PY{l+s+s1}{The Python course}\PY{l+s+s1}{\PYZsq{}}\PY{p}{)}
\end{Verbatim}
\end{tcolorbox}

            \begin{tcolorbox}[breakable, size=fbox, boxrule=.5pt, pad at break*=1mm, opacityfill=0]
\prompt{Out}{outcolor}{10}{\boxspacing}
\begin{Verbatim}[commandchars=\\\{\}]
'The Python course'
\end{Verbatim}
\end{tcolorbox}
        
    If a key does exist, the default will of course be ignored:

    \begin{tcolorbox}[breakable, size=fbox, boxrule=1pt, pad at break*=1mm,colback=cellbackground, colframe=cellborder]
\prompt{In}{incolor}{11}{\boxspacing}
\begin{Verbatim}[commandchars=\\\{\}]
\PY{c+c1}{\PYZsh{} access existing key}
\PY{n}{dct}\PY{o}{.}\PY{n}{get}\PY{p}{(}\PY{l+s+s1}{\PYZsq{}}\PY{l+s+s1}{language}\PY{l+s+s1}{\PYZsq{}}\PY{p}{,} \PY{l+s+s1}{\PYZsq{}}\PY{l+s+s1}{The Python course}\PY{l+s+s1}{\PYZsq{}}\PY{p}{)}
\end{Verbatim}
\end{tcolorbox}

            \begin{tcolorbox}[breakable, size=fbox, boxrule=.5pt, pad at break*=1mm, opacityfill=0]
\prompt{Out}{outcolor}{11}{\boxspacing}
\begin{Verbatim}[commandchars=\\\{\}]
'Python'
\end{Verbatim}
\end{tcolorbox}
        
    Another easily avoidable exception is the \texttt{IndexError}. There is
hardly ever a reason to attempt retrieving elements at arbitrary
indices. Usually, we first check the size of a collection:

    \begin{tcolorbox}[breakable, size=fbox, boxrule=1pt, pad at break*=1mm,colback=cellbackground, colframe=cellborder]
\prompt{In}{incolor}{12}{\boxspacing}
\begin{Verbatim}[commandchars=\\\{\}]
\PY{n}{items} \PY{o}{=} \PY{l+m+mi}{1}\PY{p}{,} \PY{l+m+mi}{2}\PY{p}{,} \PY{l+m+mi}{3}
\PY{c+c1}{\PYZsh{} Assume we want to access the 1001\PYZhy{}st element.}
\PY{c+c1}{\PYZsh{} Enforce valid upper bound in case the index is}
\PY{c+c1}{\PYZsh{} out of bounds.}
\PY{n}{items}\PY{p}{[}\PY{n+nb}{min}\PY{p}{(}\PY{l+m+mi}{1000}\PY{p}{,} \PY{n+nb}{len}\PY{p}{(}\PY{n}{items}\PY{p}{)} \PY{o}{\PYZhy{}} \PY{l+m+mi}{1}\PY{p}{)}\PY{p}{]}
\end{Verbatim}
\end{tcolorbox}

            \begin{tcolorbox}[breakable, size=fbox, boxrule=.5pt, pad at break*=1mm, opacityfill=0]
\prompt{Out}{outcolor}{12}{\boxspacing}
\begin{Verbatim}[commandchars=\\\{\}]
3
\end{Verbatim}
\end{tcolorbox}
        
    When operating on NumPy arrays, we frequently have to retrieve their
dimensions first, so there is no risk of accessing an invalid position:

    \begin{tcolorbox}[breakable, size=fbox, boxrule=1pt, pad at break*=1mm,colback=cellbackground, colframe=cellborder]
\prompt{In}{incolor}{13}{\boxspacing}
\begin{Verbatim}[commandchars=\\\{\}]
\PY{k+kn}{import} \PY{n+nn}{numpy} \PY{k}{as} \PY{n+nn}{np} 

\PY{n}{mat} \PY{o}{=} \PY{n}{np}\PY{o}{.}\PY{n}{arange}\PY{p}{(}\PY{l+m+mi}{6}\PY{p}{)}\PY{o}{.}\PY{n}{reshape}\PY{p}{(}\PY{l+m+mi}{2}\PY{p}{,} \PY{l+m+mi}{3}\PY{p}{)}

\PY{c+c1}{\PYZsh{} Retrieve array dimensions}
\PY{n}{nrow}\PY{p}{,} \PY{n}{ncol} \PY{o}{=} \PY{n}{mat}\PY{o}{.}\PY{n}{shape}

\PY{c+c1}{\PYZsh{} Loop makes sure to never step out of bounds}
\PY{k}{for} \PY{n}{i} \PY{o+ow}{in} \PY{n+nb}{range}\PY{p}{(}\PY{n}{nrow}\PY{p}{)}\PY{p}{:}
    \PY{k}{for} \PY{n}{j} \PY{o+ow}{in} \PY{n+nb}{range}\PY{p}{(}\PY{n}{ncol}\PY{p}{)}\PY{p}{:}
        \PY{n+nb}{print}\PY{p}{(}\PY{n}{mat}\PY{p}{[}\PY{n}{i}\PY{p}{,} \PY{n}{j}\PY{p}{]}\PY{p}{)}
\end{Verbatim}
\end{tcolorbox}

    \begin{Verbatim}[commandchars=\\\{\}]
0
1
2
3
4
5
    \end{Verbatim}

    There are also many helper routines that allow for ``robust''
programming. Imagine we want a function that returns the element at
position \texttt{{[}0,0{]}}:

    \begin{tcolorbox}[breakable, size=fbox, boxrule=1pt, pad at break*=1mm,colback=cellbackground, colframe=cellborder]
\prompt{In}{incolor}{14}{\boxspacing}
\begin{Verbatim}[commandchars=\\\{\}]
\PY{k}{def} \PY{n+nf}{get\PYZus{}elem}\PY{p}{(}\PY{n}{x}\PY{p}{)}\PY{p}{:}
    \PY{k}{return} \PY{n}{x}\PY{p}{[}\PY{l+m+mi}{0}\PY{p}{,}\PY{l+m+mi}{0}\PY{p}{]}
\end{Verbatim}
\end{tcolorbox}

    Calling this on a matrix works as intended:

    \begin{tcolorbox}[breakable, size=fbox, boxrule=1pt, pad at break*=1mm,colback=cellbackground, colframe=cellborder]
\prompt{In}{incolor}{15}{\boxspacing}
\begin{Verbatim}[commandchars=\\\{\}]
\PY{n}{get\PYZus{}elem}\PY{p}{(}\PY{n}{np}\PY{o}{.}\PY{n}{ones}\PY{p}{(}\PY{p}{(}\PY{l+m+mi}{2}\PY{p}{,}\PY{l+m+mi}{2}\PY{p}{)}\PY{p}{)}\PY{p}{)}
\end{Verbatim}
\end{tcolorbox}

            \begin{tcolorbox}[breakable, size=fbox, boxrule=.5pt, pad at break*=1mm, opacityfill=0]
\prompt{Out}{outcolor}{15}{\boxspacing}
\begin{Verbatim}[commandchars=\\\{\}]
1.0
\end{Verbatim}
\end{tcolorbox}
        
    But what if we pass a nested list or tuple?

    \begin{tcolorbox}[breakable, size=fbox, boxrule=1pt, pad at break*=1mm,colback=cellbackground, colframe=cellborder]
\prompt{In}{incolor}{16}{\boxspacing}
\begin{Verbatim}[commandchars=\\\{\}]
\PY{n}{get\PYZus{}elem}\PY{p}{(}\PY{p}{[}\PY{p}{[}\PY{l+m+mi}{1}\PY{p}{,}\PY{l+m+mi}{2}\PY{p}{]}\PY{p}{,} \PY{p}{[}\PY{l+m+mi}{3}\PY{p}{,}\PY{l+m+mi}{4}\PY{p}{]}\PY{p}{]}\PY{p}{)}
\end{Verbatim}
\end{tcolorbox}

    \begin{Verbatim}[commandchars=\\\{\}, frame=single, framerule=2mm, rulecolor=\color{outerrorbackground}]
\textcolor{ansi-red}{---------------------------------------------------------------------------}
\textcolor{ansi-red}{TypeError}                                 Traceback (most recent call last)
\textcolor{ansi-green}{<ipython-input-1-65cf63d51c44>} in \textcolor{ansi-cyan}{<module>}
\textcolor{ansi-green}{----> 1}\textcolor{ansi-red}{ }get\_elem\textcolor{ansi-blue}{(}\textcolor{ansi-blue}{[}\textcolor{ansi-blue}{[}\textcolor{ansi-cyan}{1}\textcolor{ansi-blue}{,}\textcolor{ansi-cyan}{2}\textcolor{ansi-blue}{]}\textcolor{ansi-blue}{,} \textcolor{ansi-blue}{[}\textcolor{ansi-cyan}{3}\textcolor{ansi-blue}{,}\textcolor{ansi-cyan}{4}\textcolor{ansi-blue}{]}\textcolor{ansi-blue}{]}\textcolor{ansi-blue}{)}

\textcolor{ansi-green}{<ipython-input-1-f3152bb37c6a>} in \textcolor{ansi-cyan}{get\_elem}\textcolor{ansi-blue}{(x)}
\textcolor{ansi-green-intense}{\textbf{      1}} \textcolor{ansi-green}{def} get\_elem\textcolor{ansi-blue}{(}x\textcolor{ansi-blue}{)}\textcolor{ansi-blue}{:}
\textcolor{ansi-green}{----> 2}\textcolor{ansi-red}{     }\textcolor{ansi-green}{return} x\textcolor{ansi-blue}{[}\textcolor{ansi-cyan}{0}\textcolor{ansi-blue}{,}\textcolor{ansi-cyan}{0}\textcolor{ansi-blue}{]}

\textcolor{ansi-red}{TypeError}: list indices must be integers or slices, not tuple
    \end{Verbatim}

    With very little effort, we can make this function more robust by using
\href{https://numpy.org/doc/stable/reference/generated/numpy.atleast_2d.html}{\texttt{np.atleast\_2d()}}
which ensures that its result is at least a 2-dimensional NumPy array
(it returns higher-dimensional arrays unmodified):

    \begin{tcolorbox}[breakable, size=fbox, boxrule=1pt, pad at break*=1mm,colback=cellbackground, colframe=cellborder]
\prompt{In}{incolor}{17}{\boxspacing}
\begin{Verbatim}[commandchars=\\\{\}]
\PY{k}{def} \PY{n+nf}{get\PYZus{}elem}\PY{p}{(}\PY{n}{x}\PY{p}{)}\PY{p}{:}
    \PY{n}{x} \PY{o}{=} \PY{n}{np}\PY{o}{.}\PY{n}{atleast\PYZus{}2d}\PY{p}{(}\PY{n}{x}\PY{p}{)}
    \PY{k}{return} \PY{n}{x}\PY{p}{[}\PY{l+m+mi}{0}\PY{p}{,}\PY{l+m+mi}{0}\PY{p}{]}
\end{Verbatim}
\end{tcolorbox}

    \begin{tcolorbox}[breakable, size=fbox, boxrule=1pt, pad at break*=1mm,colback=cellbackground, colframe=cellborder]
\prompt{In}{incolor}{18}{\boxspacing}
\begin{Verbatim}[commandchars=\\\{\}]
\PY{n}{get\PYZus{}elem}\PY{p}{(}\PY{p}{[}\PY{p}{[}\PY{l+m+mi}{1}\PY{p}{,}\PY{l+m+mi}{2}\PY{p}{]}\PY{p}{,} \PY{p}{[}\PY{l+m+mi}{3}\PY{p}{,}\PY{l+m+mi}{4}\PY{p}{]}\PY{p}{]}\PY{p}{)}        \PY{c+c1}{\PYZsh{} Now works on nested lists}
\end{Verbatim}
\end{tcolorbox}

            \begin{tcolorbox}[breakable, size=fbox, boxrule=.5pt, pad at break*=1mm, opacityfill=0]
\prompt{Out}{outcolor}{18}{\boxspacing}
\begin{Verbatim}[commandchars=\\\{\}]
1
\end{Verbatim}
\end{tcolorbox}
        
    This function suddenly becomes much more flexible, maybe too flexible
since it works on all sorts of arguments:

    \begin{tcolorbox}[breakable, size=fbox, boxrule=1pt, pad at break*=1mm,colback=cellbackground, colframe=cellborder]
\prompt{In}{incolor}{19}{\boxspacing}
\begin{Verbatim}[commandchars=\\\{\}]
\PY{n}{get\PYZus{}elem}\PY{p}{(}\PY{p}{[}\PY{l+m+mi}{1}\PY{p}{,} \PY{l+m+mi}{2}\PY{p}{]}\PY{p}{)}        \PY{c+c1}{\PYZsh{} simple list}
\PY{n}{get\PYZus{}elem}\PY{p}{(}\PY{l+m+mf}{1.0}\PY{p}{)}           \PY{c+c1}{\PYZsh{} scalar}
\end{Verbatim}
\end{tcolorbox}

            \begin{tcolorbox}[breakable, size=fbox, boxrule=.5pt, pad at break*=1mm, opacityfill=0]
\prompt{Out}{outcolor}{19}{\boxspacing}
\begin{Verbatim}[commandchars=\\\{\}]
1.0
\end{Verbatim}
\end{tcolorbox}
        
    NumPy also implements \texttt{np.atleast\_1d()} and
\texttt{np.atleast\_3d()} which serve the same purpose, but return
1-dimensional and 3-dimensional arrays instead.

    \hypertarget{raising-exceptions}{%
\subsubsection{Raising exceptions}\label{raising-exceptions}}

There are situations when we explicitly want to ensure that some
condition is met, instead of letting the code fail somewhere down the
line. This is particularly important when we write library functions
that might be called from many different contexts or by many different
users. Raising an exception with a clear error message is beneficial in
such situations.

To illustrate the benefit of clear error messages, consider the
following (highly artificial) example:

    \begin{tcolorbox}[breakable, size=fbox, boxrule=1pt, pad at break*=1mm,colback=cellbackground, colframe=cellborder]
\prompt{In}{incolor}{20}{\boxspacing}
\begin{Verbatim}[commandchars=\\\{\}]
\PY{k}{def} \PY{n+nf}{get\PYZus{}row}\PY{p}{(}\PY{n}{mat}\PY{p}{,} \PY{n}{i}\PY{p}{)}\PY{p}{:}
    \PY{c+c1}{\PYZsh{} restrict to valid row indices}
    \PY{n}{irow} \PY{o}{=} \PY{n+nb}{min}\PY{p}{(}\PY{n}{mat}\PY{o}{.}\PY{n}{shape}\PY{p}{[}\PY{l+m+mi}{0}\PY{p}{]} \PY{o}{\PYZhy{}} \PY{l+m+mi}{1}\PY{p}{,} \PY{n+nb}{max}\PY{p}{(}\PY{l+m+mi}{0}\PY{p}{,} \PY{n}{i}\PY{p}{)}\PY{p}{)}

    \PY{c+c1}{\PYZsh{} return row}
    \PY{n}{row} \PY{o}{=} \PY{n}{mat}\PY{p}{[}\PY{n}{irow}\PY{p}{]}
    \PY{k}{return} \PY{n}{row}
\end{Verbatim}
\end{tcolorbox}

    We define the function \texttt{get\_row} that returns the \texttt{i}-th
row of a matrix. The function ensures that the row index is within the
admissible range for the given array.

Let's call this function as follows:

    \begin{tcolorbox}[breakable, size=fbox, boxrule=1pt, pad at break*=1mm,colback=cellbackground, colframe=cellborder]
\prompt{In}{incolor}{21}{\boxspacing}
\begin{Verbatim}[commandchars=\\\{\}]
\PY{k+kn}{import} \PY{n+nn}{numpy} \PY{k}{as} \PY{n+nn}{np}
\PY{n}{mat} \PY{o}{=} \PY{n}{np}\PY{o}{.}\PY{n}{arange}\PY{p}{(}\PY{l+m+mi}{6}\PY{p}{)}\PY{o}{.}\PY{n}{reshape}\PY{p}{(}\PY{p}{(}\PY{l+m+mi}{3}\PY{p}{,} \PY{l+m+mi}{2}\PY{p}{)}\PY{p}{)}
\PY{n}{get\PYZus{}row}\PY{p}{(}\PY{n}{mat}\PY{p}{,} \PY{l+m+mf}{1.0}\PY{p}{)}
\end{Verbatim}
\end{tcolorbox}

    \begin{Verbatim}[commandchars=\\\{\}, frame=single, framerule=2mm, rulecolor=\color{outerrorbackground}]
\textcolor{ansi-red}{---------------------------------------------------------------------------}
\textcolor{ansi-red}{IndexError}                                Traceback (most recent call last)
\textcolor{ansi-green}{<ipython-input-1-ea74baa9d300>} in \textcolor{ansi-cyan}{<module>}
\textcolor{ansi-green-intense}{\textbf{      1}} \textcolor{ansi-green}{import} numpy \textcolor{ansi-green}{as} np
\textcolor{ansi-green-intense}{\textbf{      2}} mat \textcolor{ansi-blue}{=} np\textcolor{ansi-blue}{.}arange\textcolor{ansi-blue}{(}\textcolor{ansi-cyan}{6}\textcolor{ansi-blue}{)}\textcolor{ansi-blue}{.}reshape\textcolor{ansi-blue}{(}\textcolor{ansi-blue}{(}\textcolor{ansi-cyan}{3}\textcolor{ansi-blue}{,} \textcolor{ansi-cyan}{2}\textcolor{ansi-blue}{)}\textcolor{ansi-blue}{)}
\textcolor{ansi-green}{----> 3}\textcolor{ansi-red}{ }get\_row\textcolor{ansi-blue}{(}mat\textcolor{ansi-blue}{,} \textcolor{ansi-cyan}{1.0}\textcolor{ansi-blue}{)}

\textcolor{ansi-green}{<ipython-input-1-789c9658e65f>} in \textcolor{ansi-cyan}{get\_row}\textcolor{ansi-blue}{(mat, i)}
\textcolor{ansi-green-intense}{\textbf{      4}} 
\textcolor{ansi-green-intense}{\textbf{      5}}     \textcolor{ansi-red}{\# return row}
\textcolor{ansi-green}{----> 6}\textcolor{ansi-red}{     }row \textcolor{ansi-blue}{=} mat\textcolor{ansi-blue}{[}irow\textcolor{ansi-blue}{]}
\textcolor{ansi-green-intense}{\textbf{      7}}     \textcolor{ansi-green}{return} row

\textcolor{ansi-red}{IndexError}: only integers, slices (`:`), ellipsis (`{\ldots}`), numpy.newaxis (`None`) and integer or boolean arrays are valid indices
    \end{Verbatim}

    This raises an \texttt{IndexError}, notifying the user that the
statement \texttt{row\ =\ mat{[}irow{]}} was problematic. However, the
caller does not know what \texttt{irow} is since this is not the name of
the original argument. In the worst case, the user would have to inspect
the implementation of \texttt{get\_row()} to figure out what is wrong.

How can we rectify this situation? We cannot prevent someone from
calling this function with an inadmissible value, but we can raise an
exception once such a value is encountered.

We raise exceptions using the \texttt{raise} statement which is followed
by an exception:

    \begin{tcolorbox}[breakable, size=fbox, boxrule=1pt, pad at break*=1mm,colback=cellbackground, colframe=cellborder]
\prompt{In}{incolor}{22}{\boxspacing}
\begin{Verbatim}[commandchars=\\\{\}]
\PY{k}{def} \PY{n+nf}{get\PYZus{}row}\PY{p}{(}\PY{n}{mat}\PY{p}{,} \PY{n}{i}\PY{p}{)}\PY{p}{:}
    \PY{c+c1}{\PYZsh{} Check whether i is an integer}
    \PY{k}{if} \PY{o+ow}{not} \PY{n+nb}{isinstance}\PY{p}{(}\PY{n}{i}\PY{p}{,} \PY{n+nb}{int}\PY{p}{)}\PY{p}{:}
        \PY{n}{msg} \PY{o}{=} \PY{l+s+sa}{f}\PY{l+s+s1}{\PYZsq{}}\PY{l+s+s1}{Integer argument required, received }\PY{l+s+si}{\PYZob{}}\PY{n}{i}\PY{l+s+si}{\PYZcb{}}\PY{l+s+s1}{\PYZsq{}}
        \PY{k}{raise} \PY{n+ne}{ValueError}\PY{p}{(}\PY{n}{msg}\PY{p}{)}
    \PY{c+c1}{\PYZsh{} restrict to valid row indices}
    \PY{n}{irow} \PY{o}{=} \PY{n+nb}{min}\PY{p}{(}\PY{n}{mat}\PY{o}{.}\PY{n}{shape}\PY{p}{[}\PY{l+m+mi}{0}\PY{p}{]} \PY{o}{\PYZhy{}} \PY{l+m+mi}{1}\PY{p}{,} \PY{n+nb}{max}\PY{p}{(}\PY{l+m+mi}{0}\PY{p}{,} \PY{n}{i}\PY{p}{)}\PY{p}{)}

    \PY{c+c1}{\PYZsh{} return row}
    \PY{n}{row} \PY{o}{=} \PY{n}{mat}\PY{p}{[}\PY{n}{irow}\PY{p}{]}
    \PY{k}{return} \PY{n}{row}
\end{Verbatim}
\end{tcolorbox}

    To check whether \texttt{i} is of integer type, we use the
\texttt{isinstance()} function.

The convention is to raise a \texttt{ValueError} when a function
argument does not satisfy some requirement. We can optionally pass an
error message, as in the example above. There is no need or possibility
to add an explicit \texttt{return} statement: as soon as an exception is
raised, all execution of the remaining code is halted. We will examine
the details below.

    \begin{tcolorbox}[breakable, size=fbox, boxrule=1pt, pad at break*=1mm,colback=cellbackground, colframe=cellborder]
\prompt{In}{incolor}{23}{\boxspacing}
\begin{Verbatim}[commandchars=\\\{\}]
\PY{n}{get\PYZus{}row}\PY{p}{(}\PY{n}{mat}\PY{p}{,} \PY{l+m+mi}{1}\PY{p}{)}     \PY{c+c1}{\PYZsh{} Call with integer argument; nothing happens.}
\end{Verbatim}
\end{tcolorbox}

            \begin{tcolorbox}[breakable, size=fbox, boxrule=.5pt, pad at break*=1mm, opacityfill=0]
\prompt{Out}{outcolor}{23}{\boxspacing}
\begin{Verbatim}[commandchars=\\\{\}]
array([2, 3])
\end{Verbatim}
\end{tcolorbox}
        
    \begin{tcolorbox}[breakable, size=fbox, boxrule=1pt, pad at break*=1mm,colback=cellbackground, colframe=cellborder]
\prompt{In}{incolor}{24}{\boxspacing}
\begin{Verbatim}[commandchars=\\\{\}]
\PY{n}{get\PYZus{}row}\PY{p}{(}\PY{n}{mat}\PY{p}{,} \PY{l+m+mf}{1.0}\PY{p}{)}   \PY{c+c1}{\PYZsh{} Call with float argument; raises exception}
\end{Verbatim}
\end{tcolorbox}

    \begin{Verbatim}[commandchars=\\\{\}, frame=single, framerule=2mm, rulecolor=\color{outerrorbackground}]
\textcolor{ansi-red}{---------------------------------------------------------------------------}
\textcolor{ansi-red}{ValueError}                                Traceback (most recent call last)
\textcolor{ansi-green}{<ipython-input-1-2bbbaad8f23b>} in \textcolor{ansi-cyan}{<module>}
\textcolor{ansi-green}{----> 1}\textcolor{ansi-red}{ }get\_row\textcolor{ansi-blue}{(}mat\textcolor{ansi-blue}{,} \textcolor{ansi-cyan}{1.0}\textcolor{ansi-blue}{)}   \textcolor{ansi-red}{\# Call with float argument; raises exception}

\textcolor{ansi-green}{<ipython-input-1-ab11029fc5ed>} in \textcolor{ansi-cyan}{get\_row}\textcolor{ansi-blue}{(mat, i)}
\textcolor{ansi-green-intense}{\textbf{      3}}     \textcolor{ansi-green}{if} \textcolor{ansi-green}{not} isinstance\textcolor{ansi-blue}{(}i\textcolor{ansi-blue}{,} int\textcolor{ansi-blue}{)}\textcolor{ansi-blue}{:}
\textcolor{ansi-green-intense}{\textbf{      4}}         msg \textcolor{ansi-blue}{=} \textcolor{ansi-blue}{f'Integer argument required, received \{i\}'}
\textcolor{ansi-green}{----> 5}\textcolor{ansi-red}{         }\textcolor{ansi-green}{raise} ValueError\textcolor{ansi-blue}{(}msg\textcolor{ansi-blue}{)}
\textcolor{ansi-green-intense}{\textbf{      6}}     \textcolor{ansi-red}{\# restrict to valid row indices}
\textcolor{ansi-green-intense}{\textbf{      7}}     irow \textcolor{ansi-blue}{=} min\textcolor{ansi-blue}{(}mat\textcolor{ansi-blue}{.}shape\textcolor{ansi-blue}{[}\textcolor{ansi-cyan}{0}\textcolor{ansi-blue}{]} \textcolor{ansi-blue}{-} \textcolor{ansi-cyan}{1}\textcolor{ansi-blue}{,} max\textcolor{ansi-blue}{(}\textcolor{ansi-cyan}{0}\textcolor{ansi-blue}{,} i\textcolor{ansi-blue}{)}\textcolor{ansi-blue}{)}

\textcolor{ansi-red}{ValueError}: Integer argument required, received 1.0
    \end{Verbatim}

    As you see, an exception is raised and a clear error message is returned
to the caller.

    \hypertarget{catching-exceptions}{%
\subsubsection{Catching exceptions}\label{catching-exceptions}}

    If we are unable or unwilling to take measures to avoid an error, we
have to deal with the resulting exception, should one occur. If we fail
to do so, the entire program will be terminated.

We handle exceptions using the
\href{https://docs.python.org/3/tutorial/errors.html\#handling-exceptions}{\texttt{try}
statement} (we sometimes say we ``catch'' exceptions, which is the
keyword used in some other programming languages):

\begin{itemize}
\tightlist
\item
  The code that potentially raises an exception is placed in the
  \texttt{try} clause.
\item
  If an error occurs, control is immediately passed on to\\
  the \texttt{except} clause and any remaining statements in the
  \texttt{try} clause are skipped.
\item
  The \texttt{except} clause takes care of handling the exception,
  should one occur. If no exception is raised, the \texttt{except}
  clause is never executed.
\end{itemize}

\emph{Examples:}

Say we need to process an integer value but are unsure about the data
type of the input; calling \texttt{int()} might therefore work, or it
might not:

    \begin{tcolorbox}[breakable, size=fbox, boxrule=1pt, pad at break*=1mm,colback=cellbackground, colframe=cellborder]
\prompt{In}{incolor}{25}{\boxspacing}
\begin{Verbatim}[commandchars=\\\{\}]
\PY{n}{x} \PY{o}{=} \PY{l+m+mf}{1.2345}
\PY{n+nb}{int}\PY{p}{(}\PY{n}{x}\PY{p}{)}          \PY{c+c1}{\PYZsh{} Works, float is truncated to integer}
\end{Verbatim}
\end{tcolorbox}

            \begin{tcolorbox}[breakable, size=fbox, boxrule=.5pt, pad at break*=1mm, opacityfill=0]
\prompt{Out}{outcolor}{25}{\boxspacing}
\begin{Verbatim}[commandchars=\\\{\}]
1
\end{Verbatim}
\end{tcolorbox}
        
    \begin{tcolorbox}[breakable, size=fbox, boxrule=1pt, pad at break*=1mm,colback=cellbackground, colframe=cellborder]
\prompt{In}{incolor}{26}{\boxspacing}
\begin{Verbatim}[commandchars=\\\{\}]
\PY{n}{x} \PY{o}{=} \PY{l+s+s1}{\PYZsq{}}\PY{l+s+s1}{abc}\PY{l+s+s1}{\PYZsq{}}
\PY{n+nb}{int}\PY{p}{(}\PY{n}{x}\PY{p}{)}          \PY{c+c1}{\PYZsh{} Does not work}
\end{Verbatim}
\end{tcolorbox}

    \begin{Verbatim}[commandchars=\\\{\}, frame=single, framerule=2mm, rulecolor=\color{outerrorbackground}]
\textcolor{ansi-red}{---------------------------------------------------------------------------}
\textcolor{ansi-red}{ValueError}                                Traceback (most recent call last)
\textcolor{ansi-green}{<ipython-input-1-c233272b1fb2>} in \textcolor{ansi-cyan}{<module>}
\textcolor{ansi-green-intense}{\textbf{      1}} x \textcolor{ansi-blue}{=} \textcolor{ansi-blue}{'abc'}
\textcolor{ansi-green}{----> 2}\textcolor{ansi-red}{ }int\textcolor{ansi-blue}{(}x\textcolor{ansi-blue}{)}          \textcolor{ansi-red}{\# Does not work}

\textcolor{ansi-red}{ValueError}: invalid literal for int() with base 10: 'abc'
    \end{Verbatim}

    Calling \texttt{int()} with a string such as
\texttt{\textquotesingle{}abc\textquotesingle{}} which cannot be
interpreted as an integer will raise a \texttt{ValueError}. We could
handle such a situation as follows:

    \begin{tcolorbox}[breakable, size=fbox, boxrule=1pt, pad at break*=1mm,colback=cellbackground, colframe=cellborder]
\prompt{In}{incolor}{27}{\boxspacing}
\begin{Verbatim}[commandchars=\\\{\}]
\PY{n}{x} \PY{o}{=} \PY{l+s+s1}{\PYZsq{}}\PY{l+s+s1}{abc}\PY{l+s+s1}{\PYZsq{}}

\PY{k}{try}\PY{p}{:}
    \PY{n}{i} \PY{o}{=} \PY{n+nb}{int}\PY{p}{(}\PY{n}{x}\PY{p}{)}
    \PY{n+nb}{print}\PY{p}{(}\PY{l+s+s1}{\PYZsq{}}\PY{l+s+s1}{Conversion to integer works!}\PY{l+s+s1}{\PYZsq{}}\PY{p}{)}
\PY{k}{except} \PY{n+ne}{ValueError}\PY{p}{:}
    \PY{n+nb}{print}\PY{p}{(}\PY{l+s+sa}{f}\PY{l+s+s1}{\PYZsq{}}\PY{l+s+si}{\PYZob{}}\PY{n}{x}\PY{l+s+si}{\PYZcb{}}\PY{l+s+s1}{ cannot be converted to an integer}\PY{l+s+s1}{\PYZsq{}}\PY{p}{)}
\end{Verbatim}
\end{tcolorbox}

    \begin{Verbatim}[commandchars=\\\{\}]
abc cannot be converted to an integer
    \end{Verbatim}

    We see that the execution of the \texttt{try} clause terminates as soon
as the exception is raised, so the \texttt{print()} function is never
called. Instead, execution is passed on to the \texttt{except} clause
which matches the exception type.

We can have multiple \texttt{except} clauses covering all sorts of
exceptions:

    \begin{tcolorbox}[breakable, size=fbox, boxrule=1pt, pad at break*=1mm,colback=cellbackground, colframe=cellborder]
\prompt{In}{incolor}{28}{\boxspacing}
\begin{Verbatim}[commandchars=\\\{\}]
\PY{k}{def} \PY{n+nf}{func}\PY{p}{(}\PY{n}{x}\PY{p}{)}\PY{p}{:}
    \PY{k}{try}\PY{p}{:}
        \PY{n}{i} \PY{o}{=} \PY{n+nb}{int}\PY{p}{(}\PY{n}{x}\PY{p}{)}
        \PY{n+nb}{print}\PY{p}{(}\PY{l+s+s1}{\PYZsq{}}\PY{l+s+s1}{Conversion to integer works!}\PY{l+s+s1}{\PYZsq{}}\PY{p}{)}
        \PY{c+c1}{\PYZsh{} Return some value}
        \PY{k}{return} \PY{l+m+mi}{10}\PY{o}{/}\PY{n}{i}
    \PY{k}{except} \PY{n+ne}{ValueError}\PY{p}{:}
        \PY{n+nb}{print}\PY{p}{(}\PY{l+s+sa}{f}\PY{l+s+s1}{\PYZsq{}}\PY{l+s+si}{\PYZob{}}\PY{n}{x}\PY{l+s+si}{\PYZcb{}}\PY{l+s+s1}{ cannot be converted to an integer}\PY{l+s+s1}{\PYZsq{}}\PY{p}{)}
    \PY{k}{except} \PY{n+ne}{ZeroDivisionError}\PY{p}{:}
        \PY{n+nb}{print}\PY{p}{(}\PY{l+s+s1}{\PYZsq{}}\PY{l+s+s1}{Division by zero}\PY{l+s+s1}{\PYZsq{}}\PY{p}{)}
    \PY{k}{except}\PY{p}{:}
        \PY{n+nb}{print}\PY{p}{(}\PY{l+s+s1}{\PYZsq{}}\PY{l+s+s1}{Other exception type occured}\PY{l+s+s1}{\PYZsq{}}\PY{p}{)}
\end{Verbatim}
\end{tcolorbox}

    \begin{tcolorbox}[breakable, size=fbox, boxrule=1pt, pad at break*=1mm,colback=cellbackground, colframe=cellborder]
\prompt{In}{incolor}{29}{\boxspacing}
\begin{Verbatim}[commandchars=\\\{\}]
\PY{n}{func}\PY{p}{(}\PY{l+s+s1}{\PYZsq{}}\PY{l+s+s1}{abc}\PY{l+s+s1}{\PYZsq{}}\PY{p}{)}         \PY{c+c1}{\PYZsh{} ValueError: cannot convert integer}
\end{Verbatim}
\end{tcolorbox}

    \begin{Verbatim}[commandchars=\\\{\}]
abc cannot be converted to an integer
    \end{Verbatim}

    \begin{tcolorbox}[breakable, size=fbox, boxrule=1pt, pad at break*=1mm,colback=cellbackground, colframe=cellborder]
\prompt{In}{incolor}{30}{\boxspacing}
\begin{Verbatim}[commandchars=\\\{\}]
\PY{n}{func}\PY{p}{(}\PY{l+m+mi}{0}\PY{p}{)}             \PY{c+c1}{\PYZsh{} ZeroDivisionError}
\end{Verbatim}
\end{tcolorbox}

    \begin{Verbatim}[commandchars=\\\{\}]
Conversion to integer works!
Division by zero
    \end{Verbatim}

    An \texttt{except} clause without an exception type catches any
exceptions which do not match any preceding \texttt{except} clause. For
example, this code raises a \texttt{TypeError} which is not specifically
handled:

    \begin{tcolorbox}[breakable, size=fbox, boxrule=1pt, pad at break*=1mm,colback=cellbackground, colframe=cellborder]
\prompt{In}{incolor}{31}{\boxspacing}
\begin{Verbatim}[commandchars=\\\{\}]
\PY{n}{func}\PY{p}{(}\PY{p}{[}\PY{l+m+mi}{1}\PY{p}{,} \PY{l+m+mi}{2}\PY{p}{,} \PY{l+m+mi}{3}\PY{p}{]}\PY{p}{)}     \PY{c+c1}{\PYZsh{} TypeError, caught by default clause}
\end{Verbatim}
\end{tcolorbox}

    \begin{Verbatim}[commandchars=\\\{\}]
Other exception type occured
    \end{Verbatim}

    If there is no default \texttt{except} clause and an unhandled exception
occurs, it will be propagated back to the caller as if no error handling
was present at all:

    \begin{tcolorbox}[breakable, size=fbox, boxrule=1pt, pad at break*=1mm,colback=cellbackground, colframe=cellborder]
\prompt{In}{incolor}{32}{\boxspacing}
\begin{Verbatim}[commandchars=\\\{\}]
\PY{c+c1}{\PYZsh{} Define func to only handle ValueError}
\PY{k}{def} \PY{n+nf}{func}\PY{p}{(}\PY{n}{x}\PY{p}{)}\PY{p}{:}
    \PY{k}{try}\PY{p}{:}
        \PY{n}{i} \PY{o}{=} \PY{n+nb}{int}\PY{p}{(}\PY{n}{x}\PY{p}{)}
        \PY{n+nb}{print}\PY{p}{(}\PY{l+s+s1}{\PYZsq{}}\PY{l+s+s1}{Conversion to integer works!}\PY{l+s+s1}{\PYZsq{}}\PY{p}{)}
        \PY{c+c1}{\PYZsh{} Return some value}
        \PY{k}{return} \PY{l+m+mi}{10}\PY{o}{/}\PY{n}{i}
    \PY{k}{except} \PY{n+ne}{ValueError}\PY{p}{:}
        \PY{n+nb}{print}\PY{p}{(}\PY{l+s+sa}{f}\PY{l+s+s1}{\PYZsq{}}\PY{l+s+si}{\PYZob{}}\PY{n}{x}\PY{l+s+si}{\PYZcb{}}\PY{l+s+s1}{ cannot be converted to an integer}\PY{l+s+s1}{\PYZsq{}}\PY{p}{)}
\end{Verbatim}
\end{tcolorbox}

    \begin{tcolorbox}[breakable, size=fbox, boxrule=1pt, pad at break*=1mm,colback=cellbackground, colframe=cellborder]
\prompt{In}{incolor}{33}{\boxspacing}
\begin{Verbatim}[commandchars=\\\{\}]
\PY{n}{func}\PY{p}{(}\PY{l+m+mi}{0}\PY{p}{)}         \PY{c+c1}{\PYZsh{} Raises ZeroDivisionError, which is passed to caller}
\end{Verbatim}
\end{tcolorbox}

    \begin{Verbatim}[commandchars=\\\{\}]
Conversion to integer works!
    \end{Verbatim}

    \begin{Verbatim}[commandchars=\\\{\}, frame=single, framerule=2mm, rulecolor=\color{outerrorbackground}]
\textcolor{ansi-red}{---------------------------------------------------------------------------}
\textcolor{ansi-red}{ZeroDivisionError}                         Traceback (most recent call last)
\textcolor{ansi-green}{<ipython-input-1-5f9bca549f91>} in \textcolor{ansi-cyan}{<module>}
\textcolor{ansi-green}{----> 1}\textcolor{ansi-red}{ }func\textcolor{ansi-blue}{(}\textcolor{ansi-cyan}{0}\textcolor{ansi-blue}{)}         \textcolor{ansi-red}{\# Raises ZeroDivisionError, which is passed to caller}

\textcolor{ansi-green}{<ipython-input-1-70d22ee64a1d>} in \textcolor{ansi-cyan}{func}\textcolor{ansi-blue}{(x)}
\textcolor{ansi-green-intense}{\textbf{      5}}         print\textcolor{ansi-blue}{(}\textcolor{ansi-blue}{'Conversion to integer works!'}\textcolor{ansi-blue}{)}
\textcolor{ansi-green-intense}{\textbf{      6}}         \textcolor{ansi-red}{\# Return some value}
\textcolor{ansi-green}{----> 7}\textcolor{ansi-red}{         }\textcolor{ansi-green}{return} \textcolor{ansi-cyan}{10}\textcolor{ansi-blue}{/}i
\textcolor{ansi-green-intense}{\textbf{      8}}     \textcolor{ansi-green}{except} ValueError\textcolor{ansi-blue}{:}
\textcolor{ansi-green-intense}{\textbf{      9}}         print\textcolor{ansi-blue}{(}\textcolor{ansi-blue}{f'\{x\} cannot be converted to an integer'}\textcolor{ansi-blue}{)}

\textcolor{ansi-red}{ZeroDivisionError}: division by zero
    \end{Verbatim}

    This even works across multiple levels of the call stack:

    \begin{tcolorbox}[breakable, size=fbox, boxrule=1pt, pad at break*=1mm,colback=cellbackground, colframe=cellborder]
\prompt{In}{incolor}{34}{\boxspacing}
\begin{Verbatim}[commandchars=\\\{\}]
\PY{c+c1}{\PYZsh{} inner function converts to integer}
\PY{k}{def} \PY{n+nf}{inner}\PY{p}{(}\PY{n}{x}\PY{p}{)}\PY{p}{:}
    \PY{n}{i} \PY{o}{=} \PY{n+nb}{int}\PY{p}{(}\PY{n}{x}\PY{p}{)}
    \PY{k}{return} \PY{n}{i}

\PY{c+c1}{\PYZsh{} outer function divides by integer value}
\PY{k}{def} \PY{n+nf}{outer}\PY{p}{(}\PY{n}{x}\PY{p}{)}\PY{p}{:}
    \PY{n}{i} \PY{o}{=} \PY{n}{inner}\PY{p}{(}\PY{n}{x}\PY{p}{)}
    \PY{k}{return} \PY{l+m+mi}{10} \PY{o}{/}\PY{n}{i} 
\end{Verbatim}
\end{tcolorbox}

    \begin{tcolorbox}[breakable, size=fbox, boxrule=1pt, pad at break*=1mm,colback=cellbackground, colframe=cellborder]
\prompt{In}{incolor}{35}{\boxspacing}
\begin{Verbatim}[commandchars=\\\{\}]
\PY{n}{outer}\PY{p}{(}\PY{l+s+s1}{\PYZsq{}}\PY{l+s+s1}{abc}\PY{l+s+s1}{\PYZsq{}}\PY{p}{)}            \PY{c+c1}{\PYZsh{} ValueError raised in inner()}
\end{Verbatim}
\end{tcolorbox}

    \begin{Verbatim}[commandchars=\\\{\}, frame=single, framerule=2mm, rulecolor=\color{outerrorbackground}]
\textcolor{ansi-red}{---------------------------------------------------------------------------}
\textcolor{ansi-red}{ValueError}                                Traceback (most recent call last)
\textcolor{ansi-green}{<ipython-input-1-7c25c3decd65>} in \textcolor{ansi-cyan}{<module>}
\textcolor{ansi-green}{----> 1}\textcolor{ansi-red}{ }outer\textcolor{ansi-blue}{(}\textcolor{ansi-blue}{'abc'}\textcolor{ansi-blue}{)}            \textcolor{ansi-red}{\# ValueError raised in inner()}

\textcolor{ansi-green}{<ipython-input-1-70e0ddbb84ed>} in \textcolor{ansi-cyan}{outer}\textcolor{ansi-blue}{(x)}
\textcolor{ansi-green-intense}{\textbf{      6}} \textcolor{ansi-red}{\# outer function divides by integer value}
\textcolor{ansi-green-intense}{\textbf{      7}} \textcolor{ansi-green}{def} outer\textcolor{ansi-blue}{(}x\textcolor{ansi-blue}{)}\textcolor{ansi-blue}{:}
\textcolor{ansi-green}{----> 8}\textcolor{ansi-red}{     }i \textcolor{ansi-blue}{=} inner\textcolor{ansi-blue}{(}x\textcolor{ansi-blue}{)}
\textcolor{ansi-green-intense}{\textbf{      9}}     \textcolor{ansi-green}{return} \textcolor{ansi-cyan}{10} \textcolor{ansi-blue}{/}i

\textcolor{ansi-green}{<ipython-input-1-70e0ddbb84ed>} in \textcolor{ansi-cyan}{inner}\textcolor{ansi-blue}{(x)}
\textcolor{ansi-green-intense}{\textbf{      1}} \textcolor{ansi-red}{\# inner function converts to integer}
\textcolor{ansi-green-intense}{\textbf{      2}} \textcolor{ansi-green}{def} inner\textcolor{ansi-blue}{(}x\textcolor{ansi-blue}{)}\textcolor{ansi-blue}{:}
\textcolor{ansi-green}{----> 3}\textcolor{ansi-red}{     }i \textcolor{ansi-blue}{=} int\textcolor{ansi-blue}{(}x\textcolor{ansi-blue}{)}
\textcolor{ansi-green-intense}{\textbf{      4}}     \textcolor{ansi-green}{return} i
\textcolor{ansi-green-intense}{\textbf{      5}} 

\textcolor{ansi-red}{ValueError}: invalid literal for int() with base 10: 'abc'
    \end{Verbatim}

    Here we call \texttt{outer()}, which in turn calls \texttt{inner()},
passing on its argument. Conversion to an integer fails in
\texttt{inner()}, but since \texttt{outer()} does not handle this
exception, it is automatically passed on the the original call site.

    \hypertarget{exercises}{%
\section{Exercises}\label{exercises}}

    \hypertarget{exercise-1-sign-function}{%
\subsection{Exercise 1: Sign function}\label{exercise-1-sign-function}}

Revisit the sign function you implemented in Unit 4, Exercise 1. To
refresh your memory, the suggested solution looks as follows:

    \begin{tcolorbox}[breakable, size=fbox, boxrule=1pt, pad at break*=1mm,colback=cellbackground, colframe=cellborder]
\prompt{In}{incolor}{36}{\boxspacing}
\begin{Verbatim}[commandchars=\\\{\}]
\PY{k+kn}{import} \PY{n+nn}{numpy} \PY{k}{as} \PY{n+nn}{np}

\PY{k}{def} \PY{n+nf}{sign}\PY{p}{(}\PY{n}{x}\PY{p}{)}\PY{p}{:}
    \PY{k}{if} \PY{n}{x} \PY{o}{\PYZlt{}} \PY{l+m+mf}{0.0}\PY{p}{:}
        \PY{k}{return} \PY{o}{\PYZhy{}}\PY{l+m+mf}{1.0}
    \PY{k}{elif} \PY{n}{x} \PY{o}{==} \PY{l+m+mf}{0.0}\PY{p}{:}
        \PY{k}{return} \PY{l+m+mf}{0.0}
    \PY{k}{elif} \PY{n}{x} \PY{o}{\PYZgt{}} \PY{l+m+mf}{0.0}\PY{p}{:}
        \PY{k}{return} \PY{l+m+mf}{1.0}
    \PY{k}{else}\PY{p}{:}
        \PY{c+c1}{\PYZsh{} Argument is not a proper numerical value, return NaN}
        \PY{c+c1}{\PYZsh{} (NaN = Not a Number)}
        \PY{k}{return} \PY{n}{np}\PY{o}{.}\PY{n}{nan}
\end{Verbatim}
\end{tcolorbox}

    This implementation is not very robust, as it returns all sorts of
exceptions when passed non-numeric arguments:

    \begin{tcolorbox}[breakable, size=fbox, boxrule=1pt, pad at break*=1mm,colback=cellbackground, colframe=cellborder]
\prompt{In}{incolor}{37}{\boxspacing}
\begin{Verbatim}[commandchars=\\\{\}]
\PY{n}{sign}\PY{p}{(}\PY{l+s+s1}{\PYZsq{}}\PY{l+s+s1}{abc}\PY{l+s+s1}{\PYZsq{}}\PY{p}{)}             \PY{c+c1}{\PYZsh{} pass in string}
\end{Verbatim}
\end{tcolorbox}

    \begin{Verbatim}[commandchars=\\\{\}, frame=single, framerule=2mm, rulecolor=\color{outerrorbackground}]
\textcolor{ansi-red}{---------------------------------------------------------------------------}
\textcolor{ansi-red}{TypeError}                                 Traceback (most recent call last)
\textcolor{ansi-green}{<ipython-input-1-ec900cf825a1>} in \textcolor{ansi-cyan}{<module>}
\textcolor{ansi-green}{----> 1}\textcolor{ansi-red}{ }sign\textcolor{ansi-blue}{(}\textcolor{ansi-blue}{'abc'}\textcolor{ansi-blue}{)}             \textcolor{ansi-red}{\# pass in string}

\textcolor{ansi-green}{<ipython-input-1-d7e0a9e92dc2>} in \textcolor{ansi-cyan}{sign}\textcolor{ansi-blue}{(x)}
\textcolor{ansi-green-intense}{\textbf{      2}} 
\textcolor{ansi-green-intense}{\textbf{      3}} \textcolor{ansi-green}{def} sign\textcolor{ansi-blue}{(}x\textcolor{ansi-blue}{)}\textcolor{ansi-blue}{:}
\textcolor{ansi-green}{----> 4}\textcolor{ansi-red}{     }\textcolor{ansi-green}{if} x \textcolor{ansi-blue}{<} \textcolor{ansi-cyan}{0.0}\textcolor{ansi-blue}{:}
\textcolor{ansi-green-intense}{\textbf{      5}}         \textcolor{ansi-green}{return} \textcolor{ansi-blue}{-}\textcolor{ansi-cyan}{1.0}
\textcolor{ansi-green-intense}{\textbf{      6}}     \textcolor{ansi-green}{elif} x \textcolor{ansi-blue}{==} \textcolor{ansi-cyan}{0.0}\textcolor{ansi-blue}{:}

\textcolor{ansi-red}{TypeError}: '<' not supported between instances of 'str' and 'float'
    \end{Verbatim}

    \begin{tcolorbox}[breakable, size=fbox, boxrule=1pt, pad at break*=1mm,colback=cellbackground, colframe=cellborder]
\prompt{In}{incolor}{38}{\boxspacing}
\begin{Verbatim}[commandchars=\\\{\}]
\PY{n}{sign}\PY{p}{(}\PY{n}{np}\PY{o}{.}\PY{n}{array}\PY{p}{(}\PY{p}{[}\PY{l+m+mi}{1}\PY{p}{,} \PY{l+m+mi}{2}\PY{p}{,} \PY{l+m+mi}{3}\PY{p}{]}\PY{p}{)}\PY{p}{)}           \PY{c+c1}{\PYZsh{} Pass in NumPy array}
\end{Verbatim}
\end{tcolorbox}

    \begin{Verbatim}[commandchars=\\\{\}, frame=single, framerule=2mm, rulecolor=\color{outerrorbackground}]
\textcolor{ansi-red}{---------------------------------------------------------------------------}
\textcolor{ansi-red}{ValueError}                                Traceback (most recent call last)
\textcolor{ansi-green}{<ipython-input-1-7d3a4ed6e60f>} in \textcolor{ansi-cyan}{<module>}
\textcolor{ansi-green}{----> 1}\textcolor{ansi-red}{ }sign\textcolor{ansi-blue}{(}np\textcolor{ansi-blue}{.}array\textcolor{ansi-blue}{(}\textcolor{ansi-blue}{[}\textcolor{ansi-cyan}{1}\textcolor{ansi-blue}{,} \textcolor{ansi-cyan}{2}\textcolor{ansi-blue}{,} \textcolor{ansi-cyan}{3}\textcolor{ansi-blue}{]}\textcolor{ansi-blue}{)}\textcolor{ansi-blue}{)}           \textcolor{ansi-red}{\# Pass in NumPy array}

\textcolor{ansi-green}{<ipython-input-1-d7e0a9e92dc2>} in \textcolor{ansi-cyan}{sign}\textcolor{ansi-blue}{(x)}
\textcolor{ansi-green-intense}{\textbf{      2}} 
\textcolor{ansi-green-intense}{\textbf{      3}} \textcolor{ansi-green}{def} sign\textcolor{ansi-blue}{(}x\textcolor{ansi-blue}{)}\textcolor{ansi-blue}{:}
\textcolor{ansi-green}{----> 4}\textcolor{ansi-red}{     }\textcolor{ansi-green}{if} x \textcolor{ansi-blue}{<} \textcolor{ansi-cyan}{0.0}\textcolor{ansi-blue}{:}
\textcolor{ansi-green-intense}{\textbf{      5}}         \textcolor{ansi-green}{return} \textcolor{ansi-blue}{-}\textcolor{ansi-cyan}{1.0}
\textcolor{ansi-green-intense}{\textbf{      6}}     \textcolor{ansi-green}{elif} x \textcolor{ansi-blue}{==} \textcolor{ansi-cyan}{0.0}\textcolor{ansi-blue}{:}

\textcolor{ansi-red}{ValueError}: The truth value of an array with more than one element is ambiguous. Use a.any() or a.all()
    \end{Verbatim}

    Modify the \texttt{sign()} function such that it only accepts built-in
numerical Python types (integers, floats) and raises a
\texttt{ValueError} in all other cases

    \hypertarget{exercise-2-factorials}{%
\subsection{Exercise 2: Factorials}\label{exercise-2-factorials}}

Consider the \texttt{factorial()} function you wrote in Unit 4, Exercise
4:

    \begin{tcolorbox}[breakable, size=fbox, boxrule=1pt, pad at break*=1mm,colback=cellbackground, colframe=cellborder]
\prompt{In}{incolor}{39}{\boxspacing}
\begin{Verbatim}[commandchars=\\\{\}]
\PY{k}{def} \PY{n+nf}{factorial}\PY{p}{(}\PY{n}{n}\PY{p}{)}\PY{p}{:}
    \PY{k}{if} \PY{n}{n} \PY{o}{==} \PY{l+m+mi}{0}\PY{p}{:}
        \PY{k}{return} \PY{l+m+mi}{1}
    \PY{k}{else}\PY{p}{:}
        \PY{c+c1}{\PYZsh{} Use recursion to compute factorial}
        \PY{k}{return} \PY{n}{n} \PY{o}{*} \PY{n}{factorial}\PY{p}{(}\PY{n}{n}\PY{o}{\PYZhy{}}\PY{l+m+mi}{1}\PY{p}{)}
\end{Verbatim}
\end{tcolorbox}

    This implementation is also not very robust to nonsensical arguments,
for example:

    \begin{tcolorbox}[breakable, size=fbox, boxrule=1pt, pad at break*=1mm,colback=cellbackground, colframe=cellborder]
\prompt{In}{incolor}{40}{\boxspacing}
\begin{Verbatim}[commandchars=\\\{\}]
\PY{n}{factorial}\PY{p}{(}\PY{l+m+mf}{1.123}\PY{p}{)}
\end{Verbatim}
\end{tcolorbox}

    \begin{Verbatim}[commandchars=\\\{\}, frame=single, framerule=2mm, rulecolor=\color{outerrorbackground}]
\textcolor{ansi-red}{---------------------------------------------------------------------------}
\textcolor{ansi-red}{RecursionError}                            Traceback (most recent call last)
\textcolor{ansi-green}{<ipython-input-1-bdb880c86adf>} in \textcolor{ansi-cyan}{<module>}
\textcolor{ansi-green}{----> 1}\textcolor{ansi-red}{ }factorial\textcolor{ansi-blue}{(}\textcolor{ansi-cyan}{1.123}\textcolor{ansi-blue}{)}

\textcolor{ansi-green}{<ipython-input-1-d831dc1a43ca>} in \textcolor{ansi-cyan}{factorial}\textcolor{ansi-blue}{(n)}
\textcolor{ansi-green-intense}{\textbf{      4}}     \textcolor{ansi-green}{else}\textcolor{ansi-blue}{:}
\textcolor{ansi-green-intense}{\textbf{      5}}         \textcolor{ansi-red}{\# Use recursion to compute factorial}
\textcolor{ansi-green}{----> 6}\textcolor{ansi-red}{         }\textcolor{ansi-green}{return} n \textcolor{ansi-blue}{*} factorial\textcolor{ansi-blue}{(}n\textcolor{ansi-blue}{-}\textcolor{ansi-cyan}{1}\textcolor{ansi-blue}{)}

{\ldots} last 1 frames repeated, from the frame below {\ldots}

\textcolor{ansi-green}{<ipython-input-1-d831dc1a43ca>} in \textcolor{ansi-cyan}{factorial}\textcolor{ansi-blue}{(n)}
\textcolor{ansi-green-intense}{\textbf{      4}}     \textcolor{ansi-green}{else}\textcolor{ansi-blue}{:}
\textcolor{ansi-green-intense}{\textbf{      5}}         \textcolor{ansi-red}{\# Use recursion to compute factorial}
\textcolor{ansi-green}{----> 6}\textcolor{ansi-red}{         }\textcolor{ansi-green}{return} n \textcolor{ansi-blue}{*} factorial\textcolor{ansi-blue}{(}n\textcolor{ansi-blue}{-}\textcolor{ansi-cyan}{1}\textcolor{ansi-blue}{)}

\textcolor{ansi-red}{RecursionError}: maximum recursion depth exceeded in comparison
    \end{Verbatim}

    Modify this function such that it only accepts \emph{numerical}
arguments that are either integers, or can be interpreted as integers
without loss of data, such as a float \texttt{1.0} or a scalar array
\texttt{np.array(1.0)}.

The function should raise a \texttt{ValueError} for all other inputs.

    \hypertarget{exercise-3-bisection}{%
\subsection{Exercise 3: Bisection}\label{exercise-3-bisection}}

Recall the \texttt{bisect()} function from Unit 4, Exercise 5:

    \begin{tcolorbox}[breakable, size=fbox, boxrule=1pt, pad at break*=1mm,colback=cellbackground, colframe=cellborder]
\prompt{In}{incolor}{41}{\boxspacing}
\begin{Verbatim}[commandchars=\\\{\}]
\PY{k}{def} \PY{n+nf}{bisect}\PY{p}{(}\PY{n}{f}\PY{p}{,} \PY{n}{a}\PY{p}{,} \PY{n}{b}\PY{p}{,} \PY{n}{tol}\PY{o}{=}\PY{l+m+mf}{1.0e\PYZhy{}6}\PY{p}{,} \PY{n}{xtol}\PY{o}{=}\PY{l+m+mf}{1.0e\PYZhy{}6}\PY{p}{,} \PY{n}{maxiter}\PY{o}{=}\PY{l+m+mi}{100}\PY{p}{)}\PY{p}{:}

    \PY{k}{for} \PY{n}{iteration} \PY{o+ow}{in} \PY{n+nb}{range}\PY{p}{(}\PY{n}{maxiter}\PY{p}{)}\PY{p}{:}
        \PY{c+c1}{\PYZsh{} Compute candidate value as midpoint between a and b}
        \PY{n}{mid} \PY{o}{=} \PY{p}{(}\PY{n}{a} \PY{o}{+} \PY{n}{b}\PY{p}{)} \PY{o}{/} \PY{l+m+mf}{2.0}
        \PY{k}{if} \PY{n+nb}{abs}\PY{p}{(}\PY{n}{b}\PY{o}{\PYZhy{}}\PY{n}{a}\PY{p}{)} \PY{o}{\PYZlt{}} \PY{n}{xtol}\PY{p}{:}
            \PY{c+c1}{\PYZsh{} Remaining interval is too small}
            \PY{k}{break}

        \PY{n}{fmid} \PY{o}{=} \PY{n}{f}\PY{p}{(}\PY{n}{mid}\PY{p}{)}

        \PY{k}{if} \PY{n+nb}{abs}\PY{p}{(}\PY{n}{fmid}\PY{p}{)} \PY{o}{\PYZlt{}} \PY{n}{tol}\PY{p}{:}
            \PY{c+c1}{\PYZsh{} function value is close enough to zero}
            \PY{k}{break}

        \PY{n+nb}{print}\PY{p}{(}\PY{l+s+s1}{\PYZsq{}}\PY{l+s+s1}{Iteration }\PY{l+s+si}{\PYZob{}\PYZcb{}}\PY{l+s+s1}{: f(mid) = }\PY{l+s+si}{\PYZob{}:.4e\PYZcb{}}\PY{l+s+s1}{\PYZsq{}}\PY{o}{.}\PY{n}{format}\PY{p}{(}\PY{n}{iteration}\PY{p}{,} \PY{n}{fmid}\PY{p}{)}\PY{p}{)}
        \PY{k}{if} \PY{n}{fmid}\PY{o}{*}\PY{n}{f}\PY{p}{(}\PY{n}{b}\PY{p}{)} \PY{o}{\PYZgt{}} \PY{l+m+mf}{0.0}\PY{p}{:}
            \PY{c+c1}{\PYZsh{} f(mid) and f(b) have the same sign, update upper bound b}
            \PY{n+nb}{print}\PY{p}{(}\PY{l+s+s1}{\PYZsq{}}\PY{l+s+s1}{  Updating upper bound to }\PY{l+s+si}{\PYZob{}:.8f\PYZcb{}}\PY{l+s+s1}{\PYZsq{}}\PY{o}{.}\PY{n}{format}\PY{p}{(}\PY{n}{mid}\PY{p}{)}\PY{p}{)}
            \PY{n}{b} \PY{o}{=} \PY{n}{mid}
        \PY{k}{else}\PY{p}{:}
            \PY{c+c1}{\PYZsh{} f(mid) and f(a) have the same sign, or at least one of}
            \PY{c+c1}{\PYZsh{} them is zero.}
            \PY{n+nb}{print}\PY{p}{(}\PY{l+s+s1}{\PYZsq{}}\PY{l+s+s1}{  Updating lower bound to }\PY{l+s+si}{\PYZob{}:.8f\PYZcb{}}\PY{l+s+s1}{\PYZsq{}}\PY{o}{.}\PY{n}{format}\PY{p}{(}\PY{n}{mid}\PY{p}{)}\PY{p}{)}
            \PY{n}{a} \PY{o}{=} \PY{n}{mid}

    \PY{k}{return} \PY{n}{mid}
\end{Verbatim}
\end{tcolorbox}

    This function accepts quite a few arguments, but we never implemented
any input validation. Add the following input checks at the top of the
function and raise a \texttt{ValueError} if any of them is violated:

\begin{enumerate}
\def\labelenumi{\arabic{enumi}.}
\tightlist
\item
  Check that \texttt{f(a)} and \texttt{f(b)} are of opposite sign, a
  precondition for the bisection algorithm to work.
\item
  Check that \texttt{tol} and \texttt{xtol} are positive and can be
  interpreted as floating-point numbers.
\item
  Check that \texttt{maxiter} is positive and can be interpreted as an
  integer.
\end{enumerate}

    \hypertarget{solutions}{%
\section{Solutions}\label{solutions}}

    \hypertarget{solution-for-exercise-1}{%
\subsection{Solution for exercise 1}\label{solution-for-exercise-1}}

    We can use the built-in \texttt{float()} function to determine whether
something can be represented as a floating-point number.

We use only the default \texttt{except} clause without any type
specification as the code in the \texttt{try} clause raises several
types of exceptions, depending on the input argument.

    \begin{tcolorbox}[breakable, size=fbox, boxrule=1pt, pad at break*=1mm,colback=cellbackground, colframe=cellborder]
\prompt{In}{incolor}{42}{\boxspacing}
\begin{Verbatim}[commandchars=\\\{\}]
\PY{k+kn}{import} \PY{n+nn}{numpy} \PY{k}{as} \PY{n+nn}{np}

\PY{k}{def} \PY{n+nf}{sign}\PY{p}{(}\PY{n}{x}\PY{p}{)}\PY{p}{:}
    \PY{k}{try}\PY{p}{:}
        \PY{c+c1}{\PYZsh{} Convert to float, which is more generic than int}
        \PY{n}{x} \PY{o}{=} \PY{n+nb}{float}\PY{p}{(}\PY{n}{x}\PY{p}{)}
    \PY{k}{except}\PY{p}{:}
        \PY{c+c1}{\PYZsh{} The above statement raises at least two types}
        \PY{c+c1}{\PYZsh{} of exceptions: ValueError and TypeError}
        \PY{k}{raise} \PY{n+ne}{ValueError}\PY{p}{(}\PY{l+s+s1}{\PYZsq{}}\PY{l+s+s1}{Numerical argument required!}\PY{l+s+s1}{\PYZsq{}}\PY{p}{)}

    \PY{k}{if} \PY{n}{x} \PY{o}{\PYZlt{}} \PY{l+m+mf}{0.0}\PY{p}{:}
        \PY{k}{return} \PY{o}{\PYZhy{}}\PY{l+m+mf}{1.0}
    \PY{k}{elif} \PY{n}{x} \PY{o}{==} \PY{l+m+mf}{0.0}\PY{p}{:}
        \PY{k}{return} \PY{l+m+mf}{0.0}
    \PY{k}{elif} \PY{n}{x} \PY{o}{\PYZgt{}} \PY{l+m+mf}{0.0}\PY{p}{:}
        \PY{k}{return} \PY{l+m+mf}{1.0}
    \PY{k}{else}\PY{p}{:}
        \PY{c+c1}{\PYZsh{} Argument is not a proper numerical value, return NaN}
        \PY{c+c1}{\PYZsh{} (NaN = Not a Number)}
        \PY{k}{return} \PY{n}{np}\PY{o}{.}\PY{n}{nan}
\end{Verbatim}
\end{tcolorbox}

    \begin{tcolorbox}[breakable, size=fbox, boxrule=1pt, pad at break*=1mm,colback=cellbackground, colframe=cellborder]
\prompt{In}{incolor}{43}{\boxspacing}
\begin{Verbatim}[commandchars=\\\{\}]
\PY{n}{sign}\PY{p}{(}\PY{l+m+mi}{123}\PY{p}{)}           \PY{c+c1}{\PYZsh{} integer argument}
\end{Verbatim}
\end{tcolorbox}

            \begin{tcolorbox}[breakable, size=fbox, boxrule=.5pt, pad at break*=1mm, opacityfill=0]
\prompt{Out}{outcolor}{43}{\boxspacing}
\begin{Verbatim}[commandchars=\\\{\}]
1.0
\end{Verbatim}
\end{tcolorbox}
        
    \begin{tcolorbox}[breakable, size=fbox, boxrule=1pt, pad at break*=1mm,colback=cellbackground, colframe=cellborder]
\prompt{In}{incolor}{44}{\boxspacing}
\begin{Verbatim}[commandchars=\\\{\}]
\PY{n}{sign}\PY{p}{(}\PY{l+s+s1}{\PYZsq{}}\PY{l+s+s1}{abc}\PY{l+s+s1}{\PYZsq{}}\PY{p}{)}         \PY{c+c1}{\PYZsh{} string argument}
\end{Verbatim}
\end{tcolorbox}

    \begin{Verbatim}[commandchars=\\\{\}, frame=single, framerule=2mm, rulecolor=\color{outerrorbackground}]
\textcolor{ansi-red}{---------------------------------------------------------------------------}
\textcolor{ansi-red}{ValueError}                                Traceback (most recent call last)
\textcolor{ansi-green}{<ipython-input-1-5613754af8ce>} in \textcolor{ansi-cyan}{sign}\textcolor{ansi-blue}{(x)}
\textcolor{ansi-green-intense}{\textbf{      5}}         \textcolor{ansi-red}{\# Convert to float, which is more generic than int}
\textcolor{ansi-green}{----> 6}\textcolor{ansi-red}{         }x \textcolor{ansi-blue}{=} float\textcolor{ansi-blue}{(}x\textcolor{ansi-blue}{)}
\textcolor{ansi-green-intense}{\textbf{      7}}     \textcolor{ansi-green}{except}\textcolor{ansi-blue}{:}

\textcolor{ansi-red}{ValueError}: could not convert string to float: 'abc'

During handling of the above exception, another exception occurred:

\textcolor{ansi-red}{ValueError}                                Traceback (most recent call last)
\textcolor{ansi-green}{<ipython-input-1-5c677b34ea37>} in \textcolor{ansi-cyan}{<module>}
\textcolor{ansi-green}{----> 1}\textcolor{ansi-red}{ }sign\textcolor{ansi-blue}{(}\textcolor{ansi-blue}{'abc'}\textcolor{ansi-blue}{)}         \textcolor{ansi-red}{\# string argument}

\textcolor{ansi-green}{<ipython-input-1-5613754af8ce>} in \textcolor{ansi-cyan}{sign}\textcolor{ansi-blue}{(x)}
\textcolor{ansi-green-intense}{\textbf{      8}}         \textcolor{ansi-red}{\# The above statement raises at least two types}
\textcolor{ansi-green-intense}{\textbf{      9}}         \textcolor{ansi-red}{\# of exceptions: ValueError and TypeError}
\textcolor{ansi-green}{---> 10}\textcolor{ansi-red}{         }\textcolor{ansi-green}{raise} ValueError\textcolor{ansi-blue}{(}\textcolor{ansi-blue}{'Numerical argument required!'}\textcolor{ansi-blue}{)}
\textcolor{ansi-green-intense}{\textbf{     11}} 
\textcolor{ansi-green-intense}{\textbf{     12}}     \textcolor{ansi-green}{if} x \textcolor{ansi-blue}{<} \textcolor{ansi-cyan}{0.0}\textcolor{ansi-blue}{:}

\textcolor{ansi-red}{ValueError}: Numerical argument required!
    \end{Verbatim}

    \begin{tcolorbox}[breakable, size=fbox, boxrule=1pt, pad at break*=1mm,colback=cellbackground, colframe=cellborder]
\prompt{In}{incolor}{45}{\boxspacing}
\begin{Verbatim}[commandchars=\\\{\}]
\PY{n}{sign}\PY{p}{(}\PY{n}{np}\PY{o}{.}\PY{n}{array}\PY{p}{(}\PY{p}{[}\PY{l+m+mi}{1}\PY{p}{,} \PY{l+m+mi}{2}\PY{p}{,} \PY{l+m+mi}{3}\PY{p}{]}\PY{p}{)}\PY{p}{)}       \PY{c+c1}{\PYZsh{} NumPy array argument}
\end{Verbatim}
\end{tcolorbox}

    \begin{Verbatim}[commandchars=\\\{\}, frame=single, framerule=2mm, rulecolor=\color{outerrorbackground}]
\textcolor{ansi-red}{---------------------------------------------------------------------------}
\textcolor{ansi-red}{TypeError}                                 Traceback (most recent call last)
\textcolor{ansi-green}{<ipython-input-1-5613754af8ce>} in \textcolor{ansi-cyan}{sign}\textcolor{ansi-blue}{(x)}
\textcolor{ansi-green-intense}{\textbf{      5}}         \textcolor{ansi-red}{\# Convert to float, which is more generic than int}
\textcolor{ansi-green}{----> 6}\textcolor{ansi-red}{         }x \textcolor{ansi-blue}{=} float\textcolor{ansi-blue}{(}x\textcolor{ansi-blue}{)}
\textcolor{ansi-green-intense}{\textbf{      7}}     \textcolor{ansi-green}{except}\textcolor{ansi-blue}{:}

\textcolor{ansi-red}{TypeError}: only size-1 arrays can be converted to Python scalars

During handling of the above exception, another exception occurred:

\textcolor{ansi-red}{ValueError}                                Traceback (most recent call last)
\textcolor{ansi-green}{<ipython-input-1-120b08b4eb20>} in \textcolor{ansi-cyan}{<module>}
\textcolor{ansi-green}{----> 1}\textcolor{ansi-red}{ }sign\textcolor{ansi-blue}{(}np\textcolor{ansi-blue}{.}array\textcolor{ansi-blue}{(}\textcolor{ansi-blue}{[}\textcolor{ansi-cyan}{1}\textcolor{ansi-blue}{,} \textcolor{ansi-cyan}{2}\textcolor{ansi-blue}{,} \textcolor{ansi-cyan}{3}\textcolor{ansi-blue}{]}\textcolor{ansi-blue}{)}\textcolor{ansi-blue}{)}       \textcolor{ansi-red}{\# NumPy array argument}

\textcolor{ansi-green}{<ipython-input-1-5613754af8ce>} in \textcolor{ansi-cyan}{sign}\textcolor{ansi-blue}{(x)}
\textcolor{ansi-green-intense}{\textbf{      8}}         \textcolor{ansi-red}{\# The above statement raises at least two types}
\textcolor{ansi-green-intense}{\textbf{      9}}         \textcolor{ansi-red}{\# of exceptions: ValueError and TypeError}
\textcolor{ansi-green}{---> 10}\textcolor{ansi-red}{         }\textcolor{ansi-green}{raise} ValueError\textcolor{ansi-blue}{(}\textcolor{ansi-blue}{'Numerical argument required!'}\textcolor{ansi-blue}{)}
\textcolor{ansi-green-intense}{\textbf{     11}} 
\textcolor{ansi-green-intense}{\textbf{     12}}     \textcolor{ansi-green}{if} x \textcolor{ansi-blue}{<} \textcolor{ansi-cyan}{0.0}\textcolor{ansi-blue}{:}

\textcolor{ansi-red}{ValueError}: Numerical argument required!
    \end{Verbatim}

    \hypertarget{solution-for-exercise-2}{%
\subsection{Solution for exercise 2}\label{solution-for-exercise-2}}

One possible solution looks as follows:

    \begin{tcolorbox}[breakable, size=fbox, boxrule=1pt, pad at break*=1mm,colback=cellbackground, colframe=cellborder]
\prompt{In}{incolor}{46}{\boxspacing}
\begin{Verbatim}[commandchars=\\\{\}]
\PY{k}{def} \PY{n+nf}{factorial}\PY{p}{(}\PY{n}{n}\PY{p}{)}\PY{p}{:}
    \PY{k}{try}\PY{p}{:}
        \PY{n}{i} \PY{o}{=} \PY{n+nb}{int}\PY{p}{(}\PY{n}{n}\PY{p}{)}
        \PY{k}{assert} \PY{n}{i} \PY{o}{==} \PY{n}{n}
    \PY{k}{except}\PY{p}{:}
        \PY{k}{raise} \PY{n+ne}{ValueError}\PY{p}{(}\PY{l+s+sa}{f}\PY{l+s+s1}{\PYZsq{}}\PY{l+s+s1}{Not an integer argument: }\PY{l+s+si}{\PYZob{}}\PY{n}{n}\PY{l+s+si}{\PYZcb{}}\PY{l+s+s1}{\PYZsq{}}\PY{p}{)}

    \PY{k}{if} \PY{n}{i} \PY{o}{==} \PY{l+m+mi}{0}\PY{p}{:}
        \PY{k}{return} \PY{l+m+mi}{1}
    \PY{k}{else}\PY{p}{:}
        \PY{c+c1}{\PYZsh{} Use recursion to compute factorial}
        \PY{k}{return} \PY{n}{i} \PY{o}{*} \PY{n}{factorial}\PY{p}{(}\PY{n}{i}\PY{o}{\PYZhy{}}\PY{l+m+mi}{1}\PY{p}{)}
\end{Verbatim}
\end{tcolorbox}

    We perform input validation in two steps:

\begin{enumerate}
\def\labelenumi{\arabic{enumi}.}
\item
  We use the \texttt{int()} function to convert the input to an integer.
  This will eliminate some inadmissible arguments such as
  \texttt{\textquotesingle{}abc\textquotesingle{}} or
  \texttt{{[}1,\ 2,\ 3{]}} but will accept others such as
  \texttt{\textquotesingle{}1\textquotesingle{}} or \texttt{1.1}. We
  want to eliminate these as well, since
  \texttt{\textquotesingle{}1\textquotesingle{}} is not numeric and
  \texttt{1.1} cannot be represented as an integer without loss of data.
\item
  We achieve this with the \texttt{assert} statement where we check
  whether \texttt{i\ ==\ n}: this will only be true if \texttt{n} is
  numerical and does not have a fractional part.

  The \texttt{assert} statement will raise an \texttt{AssertionError}
  whenever a condition is not \texttt{True}, which will also be handled
  by the \texttt{except} clause.
\end{enumerate}

    \begin{tcolorbox}[breakable, size=fbox, boxrule=1pt, pad at break*=1mm,colback=cellbackground, colframe=cellborder]
\prompt{In}{incolor}{47}{\boxspacing}
\begin{Verbatim}[commandchars=\\\{\}]
\PY{n}{factorial}\PY{p}{(}\PY{l+m+mi}{1}\PY{p}{)}            \PY{c+c1}{\PYZsh{} integer argument}
\end{Verbatim}
\end{tcolorbox}

            \begin{tcolorbox}[breakable, size=fbox, boxrule=.5pt, pad at break*=1mm, opacityfill=0]
\prompt{Out}{outcolor}{47}{\boxspacing}
\begin{Verbatim}[commandchars=\\\{\}]
1
\end{Verbatim}
\end{tcolorbox}
        
    \begin{tcolorbox}[breakable, size=fbox, boxrule=1pt, pad at break*=1mm,colback=cellbackground, colframe=cellborder]
\prompt{In}{incolor}{48}{\boxspacing}
\begin{Verbatim}[commandchars=\\\{\}]
\PY{n}{factorial}\PY{p}{(}\PY{l+m+mf}{1.0}\PY{p}{)}          \PY{c+c1}{\PYZsh{} not an integer argument, but can be}
                        \PY{c+c1}{\PYZsh{} represented as an integer.}
\end{Verbatim}
\end{tcolorbox}

            \begin{tcolorbox}[breakable, size=fbox, boxrule=.5pt, pad at break*=1mm, opacityfill=0]
\prompt{Out}{outcolor}{48}{\boxspacing}
\begin{Verbatim}[commandchars=\\\{\}]
1
\end{Verbatim}
\end{tcolorbox}
        
    \begin{tcolorbox}[breakable, size=fbox, boxrule=1pt, pad at break*=1mm,colback=cellbackground, colframe=cellborder]
\prompt{In}{incolor}{49}{\boxspacing}
\begin{Verbatim}[commandchars=\\\{\}]
\PY{n}{factorial}\PY{p}{(}\PY{l+m+mf}{1.1}\PY{p}{)}          \PY{c+c1}{\PYZsh{} Floating\PYZhy{}point argument that }
                        \PY{c+c1}{\PYZsh{} cannot be represented as an integer}
\end{Verbatim}
\end{tcolorbox}

    \begin{Verbatim}[commandchars=\\\{\}, frame=single, framerule=2mm, rulecolor=\color{outerrorbackground}]
\textcolor{ansi-red}{---------------------------------------------------------------------------}
\textcolor{ansi-red}{AssertionError}                            Traceback (most recent call last)
\textcolor{ansi-green}{<ipython-input-1-a354663ec7e9>} in \textcolor{ansi-cyan}{factorial}\textcolor{ansi-blue}{(n)}
\textcolor{ansi-green-intense}{\textbf{      3}}         i \textcolor{ansi-blue}{=} int\textcolor{ansi-blue}{(}n\textcolor{ansi-blue}{)}
\textcolor{ansi-green}{----> 4}\textcolor{ansi-red}{         }\textcolor{ansi-green}{assert} i \textcolor{ansi-blue}{==} n
\textcolor{ansi-green-intense}{\textbf{      5}}     \textcolor{ansi-green}{except}\textcolor{ansi-blue}{:}

\textcolor{ansi-red}{AssertionError}: 

During handling of the above exception, another exception occurred:

\textcolor{ansi-red}{ValueError}                                Traceback (most recent call last)
\textcolor{ansi-green}{<ipython-input-1-27f79227a9de>} in \textcolor{ansi-cyan}{<module>}
\textcolor{ansi-green}{----> 1}\textcolor{ansi-red}{ }factorial\textcolor{ansi-blue}{(}\textcolor{ansi-cyan}{1.1}\textcolor{ansi-blue}{)}          \textcolor{ansi-red}{\# Floating-point argument that}
\textcolor{ansi-green-intense}{\textbf{      2}}                         \textcolor{ansi-red}{\# cannot be represented as an integer}

\textcolor{ansi-green}{<ipython-input-1-a354663ec7e9>} in \textcolor{ansi-cyan}{factorial}\textcolor{ansi-blue}{(n)}
\textcolor{ansi-green-intense}{\textbf{      4}}         \textcolor{ansi-green}{assert} i \textcolor{ansi-blue}{==} n
\textcolor{ansi-green-intense}{\textbf{      5}}     \textcolor{ansi-green}{except}\textcolor{ansi-blue}{:}
\textcolor{ansi-green}{----> 6}\textcolor{ansi-red}{         }\textcolor{ansi-green}{raise} ValueError\textcolor{ansi-blue}{(}\textcolor{ansi-blue}{f'Not an integer argument: \{n\}'}\textcolor{ansi-blue}{)}
\textcolor{ansi-green-intense}{\textbf{      7}} 
\textcolor{ansi-green-intense}{\textbf{      8}}     \textcolor{ansi-green}{if} i \textcolor{ansi-blue}{==} \textcolor{ansi-cyan}{0}\textcolor{ansi-blue}{:}

\textcolor{ansi-red}{ValueError}: Not an integer argument: 1.1
    \end{Verbatim}

    \begin{tcolorbox}[breakable, size=fbox, boxrule=1pt, pad at break*=1mm,colback=cellbackground, colframe=cellborder]
\prompt{In}{incolor}{50}{\boxspacing}
\begin{Verbatim}[commandchars=\\\{\}]
\PY{n}{factorial}\PY{p}{(}\PY{l+s+s1}{\PYZsq{}}\PY{l+s+s1}{1}\PY{l+s+s1}{\PYZsq{}}\PY{p}{)}          \PY{c+c1}{\PYZsh{} String argument}
\end{Verbatim}
\end{tcolorbox}

    \begin{Verbatim}[commandchars=\\\{\}, frame=single, framerule=2mm, rulecolor=\color{outerrorbackground}]
\textcolor{ansi-red}{---------------------------------------------------------------------------}
\textcolor{ansi-red}{AssertionError}                            Traceback (most recent call last)
\textcolor{ansi-green}{<ipython-input-1-a354663ec7e9>} in \textcolor{ansi-cyan}{factorial}\textcolor{ansi-blue}{(n)}
\textcolor{ansi-green-intense}{\textbf{      3}}         i \textcolor{ansi-blue}{=} int\textcolor{ansi-blue}{(}n\textcolor{ansi-blue}{)}
\textcolor{ansi-green}{----> 4}\textcolor{ansi-red}{         }\textcolor{ansi-green}{assert} i \textcolor{ansi-blue}{==} n
\textcolor{ansi-green-intense}{\textbf{      5}}     \textcolor{ansi-green}{except}\textcolor{ansi-blue}{:}

\textcolor{ansi-red}{AssertionError}: 

During handling of the above exception, another exception occurred:

\textcolor{ansi-red}{ValueError}                                Traceback (most recent call last)
\textcolor{ansi-green}{<ipython-input-1-b68ba63f8ab8>} in \textcolor{ansi-cyan}{<module>}
\textcolor{ansi-green}{----> 1}\textcolor{ansi-red}{ }factorial\textcolor{ansi-blue}{(}\textcolor{ansi-blue}{'1'}\textcolor{ansi-blue}{)}          \textcolor{ansi-red}{\# String argument}

\textcolor{ansi-green}{<ipython-input-1-a354663ec7e9>} in \textcolor{ansi-cyan}{factorial}\textcolor{ansi-blue}{(n)}
\textcolor{ansi-green-intense}{\textbf{      4}}         \textcolor{ansi-green}{assert} i \textcolor{ansi-blue}{==} n
\textcolor{ansi-green-intense}{\textbf{      5}}     \textcolor{ansi-green}{except}\textcolor{ansi-blue}{:}
\textcolor{ansi-green}{----> 6}\textcolor{ansi-red}{         }\textcolor{ansi-green}{raise} ValueError\textcolor{ansi-blue}{(}\textcolor{ansi-blue}{f'Not an integer argument: \{n\}'}\textcolor{ansi-blue}{)}
\textcolor{ansi-green-intense}{\textbf{      7}} 
\textcolor{ansi-green-intense}{\textbf{      8}}     \textcolor{ansi-green}{if} i \textcolor{ansi-blue}{==} \textcolor{ansi-cyan}{0}\textcolor{ansi-blue}{:}

\textcolor{ansi-red}{ValueError}: Not an integer argument: 1
    \end{Verbatim}

    \begin{tcolorbox}[breakable, size=fbox, boxrule=1pt, pad at break*=1mm,colback=cellbackground, colframe=cellborder]
\prompt{In}{incolor}{51}{\boxspacing}
\begin{Verbatim}[commandchars=\\\{\}]
\PY{n}{factorial}\PY{p}{(}\PY{n}{np}\PY{o}{.}\PY{n}{array}\PY{p}{(}\PY{l+m+mf}{10.0}\PY{p}{)}\PY{p}{)}       \PY{c+c1}{\PYZsh{} Scalar array argument}
\end{Verbatim}
\end{tcolorbox}

            \begin{tcolorbox}[breakable, size=fbox, boxrule=.5pt, pad at break*=1mm, opacityfill=0]
\prompt{Out}{outcolor}{51}{\boxspacing}
\begin{Verbatim}[commandchars=\\\{\}]
3628800
\end{Verbatim}
\end{tcolorbox}
        
    \hypertarget{solution-for-exercise-3}{%
\subsection{Solution for exercise 3}\label{solution-for-exercise-3}}

    We modify the function as follows:

    \begin{tcolorbox}[breakable, size=fbox, boxrule=1pt, pad at break*=1mm,colback=cellbackground, colframe=cellborder]
\prompt{In}{incolor}{52}{\boxspacing}
\begin{Verbatim}[commandchars=\\\{\}]
\PY{k}{def} \PY{n+nf}{bisect}\PY{p}{(}\PY{n}{f}\PY{p}{,} \PY{n}{a}\PY{p}{,} \PY{n}{b}\PY{p}{,} \PY{n}{tol}\PY{o}{=}\PY{l+m+mf}{1.0e\PYZhy{}6}\PY{p}{,} \PY{n}{xtol}\PY{o}{=}\PY{l+m+mf}{1.0e\PYZhy{}6}\PY{p}{,} \PY{n}{maxiter}\PY{o}{=}\PY{l+m+mi}{100}\PY{p}{)}\PY{p}{:}

    \PY{n}{fa} \PY{o}{=} \PY{n}{f}\PY{p}{(}\PY{n}{a}\PY{p}{)}
    \PY{n}{fb} \PY{o}{=} \PY{n}{f}\PY{p}{(}\PY{n}{b}\PY{p}{)}

    \PY{k}{if} \PY{n}{fa}\PY{o}{*}\PY{n}{fb} \PY{o}{\PYZgt{}} \PY{l+m+mf}{0.0}\PY{p}{:}
        \PY{k}{raise} \PY{n+ne}{ValueError}\PY{p}{(}\PY{l+s+sa}{f}\PY{l+s+s1}{\PYZsq{}}\PY{l+s+s1}{Not a bracketing interval [}\PY{l+s+si}{\PYZob{}}\PY{n}{a}\PY{l+s+si}{\PYZcb{}}\PY{l+s+s1}{, }\PY{l+s+si}{\PYZob{}}\PY{n}{b}\PY{l+s+si}{\PYZcb{}}\PY{l+s+s1}{]}\PY{l+s+s1}{\PYZsq{}}\PY{p}{)}
    \PY{k}{try}\PY{p}{:}
        \PY{n}{tol} \PY{o}{=} \PY{n+nb}{float}\PY{p}{(}\PY{n}{tol}\PY{p}{)}
        \PY{k}{assert} \PY{n}{tol} \PY{o}{\PYZgt{}} \PY{l+m+mf}{0.0}
    \PY{k}{except}\PY{p}{:}
        \PY{k}{raise} \PY{n+ne}{ValueError}\PY{p}{(}\PY{l+s+s1}{\PYZsq{}}\PY{l+s+s1}{Argument tol must be a positive number!}\PY{l+s+s1}{\PYZsq{}}\PY{p}{)}

    \PY{k}{try}\PY{p}{:}
        \PY{n}{xtol} \PY{o}{=} \PY{n+nb}{float}\PY{p}{(}\PY{n}{xtol}\PY{p}{)}
        \PY{k}{assert} \PY{n}{xtol} \PY{o}{\PYZgt{}} \PY{l+m+mf}{0.0}
    \PY{k}{except}\PY{p}{:}
        \PY{k}{raise} \PY{n+ne}{ValueError}\PY{p}{(}\PY{l+s+s1}{\PYZsq{}}\PY{l+s+s1}{Argument xtol must be a positive number!}\PY{l+s+s1}{\PYZsq{}}\PY{p}{)}

    \PY{k}{try}\PY{p}{:}
        \PY{n}{maxiter} \PY{o}{=} \PY{n+nb}{int}\PY{p}{(}\PY{n}{maxiter}\PY{p}{)}
        \PY{k}{assert} \PY{n}{maxiter} \PY{o}{\PYZgt{}} \PY{l+m+mi}{0}
    \PY{k}{except}\PY{p}{:}
        \PY{k}{raise} \PY{n+ne}{ValueError}\PY{p}{(}\PY{l+s+s1}{\PYZsq{}}\PY{l+s+s1}{Argument maxiter must be a positive integer!}\PY{l+s+s1}{\PYZsq{}}\PY{p}{)}


    \PY{k}{for} \PY{n}{iteration} \PY{o+ow}{in} \PY{n+nb}{range}\PY{p}{(}\PY{n}{maxiter}\PY{p}{)}\PY{p}{:}
        \PY{c+c1}{\PYZsh{} Compute candidate value as midpoint between a and b}
        \PY{n}{mid} \PY{o}{=} \PY{p}{(}\PY{n}{a} \PY{o}{+} \PY{n}{b}\PY{p}{)} \PY{o}{/} \PY{l+m+mf}{2.0}
        \PY{k}{if} \PY{n+nb}{abs}\PY{p}{(}\PY{n}{b}\PY{o}{\PYZhy{}}\PY{n}{a}\PY{p}{)} \PY{o}{\PYZlt{}} \PY{n}{xtol}\PY{p}{:}
            \PY{c+c1}{\PYZsh{} Remaining interval is too small}
            \PY{k}{break}

        \PY{n}{fmid} \PY{o}{=} \PY{n}{f}\PY{p}{(}\PY{n}{mid}\PY{p}{)}

        \PY{k}{if} \PY{n+nb}{abs}\PY{p}{(}\PY{n}{fmid}\PY{p}{)} \PY{o}{\PYZlt{}} \PY{n}{tol}\PY{p}{:}
            \PY{c+c1}{\PYZsh{} function value is close enough to zero}
            \PY{k}{break}

        \PY{n+nb}{print}\PY{p}{(}\PY{l+s+s1}{\PYZsq{}}\PY{l+s+s1}{Iteration }\PY{l+s+si}{\PYZob{}\PYZcb{}}\PY{l+s+s1}{: f(mid) = }\PY{l+s+si}{\PYZob{}:.4e\PYZcb{}}\PY{l+s+s1}{\PYZsq{}}\PY{o}{.}\PY{n}{format}\PY{p}{(}\PY{n}{iteration}\PY{p}{,} \PY{n}{fmid}\PY{p}{)}\PY{p}{)}
        \PY{k}{if} \PY{n}{fmid}\PY{o}{*}\PY{n}{f}\PY{p}{(}\PY{n}{b}\PY{p}{)} \PY{o}{\PYZgt{}} \PY{l+m+mf}{0.0}\PY{p}{:}
            \PY{c+c1}{\PYZsh{} f(mid) and f(b) have the same sign, update upper bound b}
            \PY{n+nb}{print}\PY{p}{(}\PY{l+s+s1}{\PYZsq{}}\PY{l+s+s1}{  Updating upper bound to }\PY{l+s+si}{\PYZob{}:.8f\PYZcb{}}\PY{l+s+s1}{\PYZsq{}}\PY{o}{.}\PY{n}{format}\PY{p}{(}\PY{n}{mid}\PY{p}{)}\PY{p}{)}
            \PY{n}{b} \PY{o}{=} \PY{n}{mid}
        \PY{k}{else}\PY{p}{:}
            \PY{c+c1}{\PYZsh{} f(mid) and f(a) have the same sign, or at least one of}
            \PY{c+c1}{\PYZsh{} them is zero.}
            \PY{n+nb}{print}\PY{p}{(}\PY{l+s+s1}{\PYZsq{}}\PY{l+s+s1}{  Updating lower bound to }\PY{l+s+si}{\PYZob{}:.8f\PYZcb{}}\PY{l+s+s1}{\PYZsq{}}\PY{o}{.}\PY{n}{format}\PY{p}{(}\PY{n}{mid}\PY{p}{)}\PY{p}{)}
            \PY{n}{a} \PY{o}{=} \PY{n}{mid}

    \PY{k}{return} \PY{n}{mid}
\end{Verbatim}
\end{tcolorbox}

    As in the main loop of the function, we check whether two values are
non-zero and have the same sign using the condition
\texttt{fa\ *\ fb\ \textgreater{}\ 0}, in which case we have no
bracketing interval and need to raise a \texttt{ValueError}.

The remaining checks are performed using the same code as in earlier
exercises.

    \begin{tcolorbox}[breakable, size=fbox, boxrule=1pt, pad at break*=1mm,colback=cellbackground, colframe=cellborder]
\prompt{In}{incolor}{53}{\boxspacing}
\begin{Verbatim}[commandchars=\\\{\}]
\PY{c+c1}{\PYZsh{} Function call with valid argument}
\PY{n}{x0} \PY{o}{=} \PY{n}{bisect}\PY{p}{(}\PY{k}{lambda} \PY{n}{x}\PY{p}{:} \PY{n}{x}\PY{o}{*}\PY{o}{*}\PY{l+m+mf}{2.0} \PY{o}{\PYZhy{}} \PY{l+m+mf}{4.0}\PY{p}{,} \PY{o}{\PYZhy{}}\PY{l+m+mf}{3.0}\PY{p}{,} \PY{l+m+mf}{0.0}\PY{p}{,} \PY{n}{tol}\PY{o}{=}\PY{l+m+mf}{1.0e\PYZhy{}3}\PY{p}{)}
\end{Verbatim}
\end{tcolorbox}

    \begin{Verbatim}[commandchars=\\\{\}]
Iteration 0: f(mid) = -1.7500e+00
  Updating upper bound to -1.50000000
Iteration 1: f(mid) = 1.0625e+00
  Updating lower bound to -2.25000000
Iteration 2: f(mid) = -4.8438e-01
  Updating upper bound to -1.87500000
Iteration 3: f(mid) = 2.5391e-01
  Updating lower bound to -2.06250000
Iteration 4: f(mid) = -1.2402e-01
  Updating upper bound to -1.96875000
Iteration 5: f(mid) = 6.2744e-02
  Updating lower bound to -2.01562500
Iteration 6: f(mid) = -3.1189e-02
  Updating upper bound to -1.99218750
Iteration 7: f(mid) = 1.5640e-02
  Updating lower bound to -2.00390625
Iteration 8: f(mid) = -7.8087e-03
  Updating upper bound to -1.99804688
Iteration 9: f(mid) = 3.9072e-03
  Updating lower bound to -2.00097656
Iteration 10: f(mid) = -1.9529e-03
  Updating upper bound to -1.99951172
    \end{Verbatim}

    \begin{tcolorbox}[breakable, size=fbox, boxrule=1pt, pad at break*=1mm,colback=cellbackground, colframe=cellborder]
\prompt{In}{incolor}{54}{\boxspacing}
\begin{Verbatim}[commandchars=\\\{\}]
\PY{c+c1}{\PYZsh{} Function call with f(a) and f(b) both positive}
\PY{n}{x0} \PY{o}{=} \PY{n}{bisect}\PY{p}{(}\PY{k}{lambda} \PY{n}{x}\PY{p}{:} \PY{n}{x}\PY{o}{*}\PY{o}{*}\PY{l+m+mf}{2.0} \PY{o}{\PYZhy{}} \PY{l+m+mf}{4.0}\PY{p}{,} \PY{l+m+mf}{10.0}\PY{p}{,} \PY{l+m+mf}{20.0}\PY{p}{)}
\end{Verbatim}
\end{tcolorbox}

    \begin{Verbatim}[commandchars=\\\{\}, frame=single, framerule=2mm, rulecolor=\color{outerrorbackground}]
\textcolor{ansi-red}{---------------------------------------------------------------------------}
\textcolor{ansi-red}{ValueError}                                Traceback (most recent call last)
\textcolor{ansi-green}{<ipython-input-1-6b31dc691c41>} in \textcolor{ansi-cyan}{<module>}
\textcolor{ansi-green-intense}{\textbf{      1}} \textcolor{ansi-red}{\# Function call with f(a) and f(b) both positive}
\textcolor{ansi-green}{----> 2}\textcolor{ansi-red}{ }x0 \textcolor{ansi-blue}{=} bisect\textcolor{ansi-blue}{(}\textcolor{ansi-green}{lambda} x\textcolor{ansi-blue}{:} x\textcolor{ansi-blue}{**}\textcolor{ansi-cyan}{2.0} \textcolor{ansi-blue}{-} \textcolor{ansi-cyan}{4.0}\textcolor{ansi-blue}{,} \textcolor{ansi-cyan}{10.0}\textcolor{ansi-blue}{,} \textcolor{ansi-cyan}{20.0}\textcolor{ansi-blue}{)}

\textcolor{ansi-green}{<ipython-input-1-af11735b41f7>} in \textcolor{ansi-cyan}{bisect}\textcolor{ansi-blue}{(f, a, b, tol, xtol, maxiter)}
\textcolor{ansi-green-intense}{\textbf{      5}} 
\textcolor{ansi-green-intense}{\textbf{      6}}     \textcolor{ansi-green}{if} fa\textcolor{ansi-blue}{*}fb \textcolor{ansi-blue}{>} \textcolor{ansi-cyan}{0.0}\textcolor{ansi-blue}{:}
\textcolor{ansi-green}{----> 7}\textcolor{ansi-red}{         }\textcolor{ansi-green}{raise} ValueError\textcolor{ansi-blue}{(}\textcolor{ansi-blue}{f'Not a bracketing interval [\{a\}, \{b\}]'}\textcolor{ansi-blue}{)}
\textcolor{ansi-green-intense}{\textbf{      8}}     \textcolor{ansi-green}{try}\textcolor{ansi-blue}{:}
\textcolor{ansi-green-intense}{\textbf{      9}}         tol \textcolor{ansi-blue}{=} float\textcolor{ansi-blue}{(}tol\textcolor{ansi-blue}{)}

\textcolor{ansi-red}{ValueError}: Not a bracketing interval [10.0, 20.0]
    \end{Verbatim}

    \begin{tcolorbox}[breakable, size=fbox, boxrule=1pt, pad at break*=1mm,colback=cellbackground, colframe=cellborder]
\prompt{In}{incolor}{55}{\boxspacing}
\begin{Verbatim}[commandchars=\\\{\}]
\PY{c+c1}{\PYZsh{} Function call with invalid tolerance criterion}
\PY{n}{x0} \PY{o}{=} \PY{n}{bisect}\PY{p}{(}\PY{k}{lambda} \PY{n}{x}\PY{p}{:} \PY{n}{x}\PY{o}{*}\PY{o}{*}\PY{l+m+mf}{2.0} \PY{o}{\PYZhy{}} \PY{l+m+mf}{4.0}\PY{p}{,} \PY{o}{\PYZhy{}}\PY{l+m+mf}{3.0}\PY{p}{,} \PY{l+m+mf}{0.0}\PY{p}{,} \PY{n}{tol}\PY{o}{=}\PY{l+m+mf}{0.0}\PY{p}{)}
\end{Verbatim}
\end{tcolorbox}

    \begin{Verbatim}[commandchars=\\\{\}, frame=single, framerule=2mm, rulecolor=\color{outerrorbackground}]
\textcolor{ansi-red}{---------------------------------------------------------------------------}
\textcolor{ansi-red}{AssertionError}                            Traceback (most recent call last)
\textcolor{ansi-green}{<ipython-input-1-af11735b41f7>} in \textcolor{ansi-cyan}{bisect}\textcolor{ansi-blue}{(f, a, b, tol, xtol, maxiter)}
\textcolor{ansi-green-intense}{\textbf{      9}}         tol \textcolor{ansi-blue}{=} float\textcolor{ansi-blue}{(}tol\textcolor{ansi-blue}{)}
\textcolor{ansi-green}{---> 10}\textcolor{ansi-red}{         }\textcolor{ansi-green}{assert} tol \textcolor{ansi-blue}{>} \textcolor{ansi-cyan}{0.0}
\textcolor{ansi-green-intense}{\textbf{     11}}     \textcolor{ansi-green}{except}\textcolor{ansi-blue}{:}

\textcolor{ansi-red}{AssertionError}: 

During handling of the above exception, another exception occurred:

\textcolor{ansi-red}{ValueError}                                Traceback (most recent call last)
\textcolor{ansi-green}{<ipython-input-1-505fd0b61d10>} in \textcolor{ansi-cyan}{<module>}
\textcolor{ansi-green-intense}{\textbf{      1}} \textcolor{ansi-red}{\# Function call with invalid tolerance criterion}
\textcolor{ansi-green}{----> 2}\textcolor{ansi-red}{ }x0 \textcolor{ansi-blue}{=} bisect\textcolor{ansi-blue}{(}\textcolor{ansi-green}{lambda} x\textcolor{ansi-blue}{:} x\textcolor{ansi-blue}{**}\textcolor{ansi-cyan}{2.0} \textcolor{ansi-blue}{-} \textcolor{ansi-cyan}{4.0}\textcolor{ansi-blue}{,} \textcolor{ansi-blue}{-}\textcolor{ansi-cyan}{3.0}\textcolor{ansi-blue}{,} \textcolor{ansi-cyan}{0.0}\textcolor{ansi-blue}{,} tol\textcolor{ansi-blue}{=}\textcolor{ansi-cyan}{0.0}\textcolor{ansi-blue}{)}

\textcolor{ansi-green}{<ipython-input-1-af11735b41f7>} in \textcolor{ansi-cyan}{bisect}\textcolor{ansi-blue}{(f, a, b, tol, xtol, maxiter)}
\textcolor{ansi-green-intense}{\textbf{     10}}         \textcolor{ansi-green}{assert} tol \textcolor{ansi-blue}{>} \textcolor{ansi-cyan}{0.0}
\textcolor{ansi-green-intense}{\textbf{     11}}     \textcolor{ansi-green}{except}\textcolor{ansi-blue}{:}
\textcolor{ansi-green}{---> 12}\textcolor{ansi-red}{         }\textcolor{ansi-green}{raise} ValueError\textcolor{ansi-blue}{(}\textcolor{ansi-blue}{'Argument tol must be a positive number!'}\textcolor{ansi-blue}{)}
\textcolor{ansi-green-intense}{\textbf{     13}} 
\textcolor{ansi-green-intense}{\textbf{     14}}     \textcolor{ansi-green}{try}\textcolor{ansi-blue}{:}

\textcolor{ansi-red}{ValueError}: Argument tol must be a positive number!
    \end{Verbatim}


    % Add a bibliography block to the postdoc
    
    
    
\end{document}
