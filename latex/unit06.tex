\documentclass[10pt]{scrartcl}

    % included at the top of the generated TeX file

\usepackage{tgcursor}

\usepackage[utf8]{inputenc}

\KOMAoptions{parskip=half*}
\KOMAoptions{paper=a4,twoside=false}
\KOMAoptions{numbers=noendperiod}

\addtokomafont{disposition}{\rmfamily}
\setcounter{tocdepth}{\subsectiontocdepth}

    \usepackage[breakable]{tcolorbox}
    \usepackage{parskip} % Stop auto-indenting (to mimic markdown behaviour)
    
    \usepackage{iftex}
    \ifPDFTeX
    	\usepackage[T1]{fontenc}
    	\usepackage{mathpazo}
    \else
    	\usepackage{fontspec}
    \fi

    % Basic figure setup, for now with no caption control since it's done
    % automatically by Pandoc (which extracts ![](path) syntax from Markdown).
    \usepackage{graphicx}
    % Maintain compatibility with old templates. Remove in nbconvert 6.0
    \let\Oldincludegraphics\includegraphics
    % Ensure that by default, figures have no caption (until we provide a
    % proper Figure object with a Caption API and a way to capture that
    % in the conversion process - todo).
    \usepackage{caption}
    \DeclareCaptionFormat{nocaption}{}
    \captionsetup{format=nocaption,aboveskip=0pt,belowskip=0pt}

    \usepackage{float}
    \floatplacement{figure}{H} % forces figures to be placed at the correct location
    \usepackage{xcolor} % Allow colors to be defined
    \usepackage{enumerate} % Needed for markdown enumerations to work
    \usepackage{geometry} % Used to adjust the document margins
    \usepackage{amsmath} % Equations
    \usepackage{amssymb} % Equations
    \usepackage{textcomp} % defines textquotesingle
    % Hack from http://tex.stackexchange.com/a/47451/13684:
    \AtBeginDocument{%
        \def\PYZsq{\textquotesingle}% Upright quotes in Pygmentized code
    }
    \usepackage{upquote} % Upright quotes for verbatim code
    \usepackage{eurosym} % defines \euro
    \usepackage[mathletters]{ucs} % Extended unicode (utf-8) support
    \usepackage{fancyvrb} % verbatim replacement that allows latex
    \usepackage{grffile} % extends the file name processing of package graphics 
                         % to support a larger range
    \makeatletter % fix for old versions of grffile with XeLaTeX
    \@ifpackagelater{grffile}{2019/11/01}
    {
      % Do nothing on new versions
    }
    {
      \def\Gread@@xetex#1{%
        \IfFileExists{"\Gin@base".bb}%
        {\Gread@eps{\Gin@base.bb}}%
        {\Gread@@xetex@aux#1}%
      }
    }
    \makeatother
    \usepackage[Export]{adjustbox} % Used to constrain images to a maximum size
    \adjustboxset{max size={0.9\linewidth}{0.9\paperheight}}

    % The hyperref package gives us a pdf with properly built
    % internal navigation ('pdf bookmarks' for the table of contents,
    % internal cross-reference links, web links for URLs, etc.)
    \usepackage{hyperref}
    % The default LaTeX title has an obnoxious amount of whitespace. By default,
    % titling removes some of it. It also provides customization options.
    \usepackage{titling}
    \usepackage{longtable} % longtable support required by pandoc >1.10
    \usepackage{booktabs}  % table support for pandoc > 1.12.2
    \usepackage[inline]{enumitem} % IRkernel/repr support (it uses the enumerate* environment)
    \usepackage[normalem]{ulem} % ulem is needed to support strikethroughs (\sout)
                                % normalem makes italics be italics, not underlines
    \usepackage{mathrsfs}
    

    
    % Colors for the hyperref package
    \definecolor{urlcolor}{rgb}{0,.145,.698}
    \definecolor{linkcolor}{rgb}{.71,0.21,0.01}
    \definecolor{citecolor}{rgb}{.12,.54,.11}

    % ANSI colors
    \definecolor{ansi-black}{HTML}{3E424D}
    \definecolor{ansi-black-intense}{HTML}{282C36}
    \definecolor{ansi-red}{HTML}{E75C58}
    \definecolor{ansi-red-intense}{HTML}{B22B31}
    \definecolor{ansi-green}{HTML}{00A250}
    \definecolor{ansi-green-intense}{HTML}{007427}
    \definecolor{ansi-yellow}{HTML}{DDB62B}
    \definecolor{ansi-yellow-intense}{HTML}{B27D12}
    \definecolor{ansi-blue}{HTML}{208FFB}
    \definecolor{ansi-blue-intense}{HTML}{0065CA}
    \definecolor{ansi-magenta}{HTML}{D160C4}
    \definecolor{ansi-magenta-intense}{HTML}{A03196}
    \definecolor{ansi-cyan}{HTML}{60C6C8}
    \definecolor{ansi-cyan-intense}{HTML}{258F8F}
    \definecolor{ansi-white}{HTML}{C5C1B4}
    \definecolor{ansi-white-intense}{HTML}{A1A6B2}
    \definecolor{ansi-default-inverse-fg}{HTML}{FFFFFF}
    \definecolor{ansi-default-inverse-bg}{HTML}{000000}

    % common color for the border for error outputs.
    \definecolor{outerrorbackground}{HTML}{FFDFDF}

    % commands and environments needed by pandoc snippets
    % extracted from the output of `pandoc -s`
    \providecommand{\tightlist}{%
      \setlength{\itemsep}{0pt}\setlength{\parskip}{0pt}}
    \DefineVerbatimEnvironment{Highlighting}{Verbatim}{commandchars=\\\{\}}
    % Add ',fontsize=\small' for more characters per line
    \newenvironment{Shaded}{}{}
    \newcommand{\KeywordTok}[1]{\textcolor[rgb]{0.00,0.44,0.13}{\textbf{{#1}}}}
    \newcommand{\DataTypeTok}[1]{\textcolor[rgb]{0.56,0.13,0.00}{{#1}}}
    \newcommand{\DecValTok}[1]{\textcolor[rgb]{0.25,0.63,0.44}{{#1}}}
    \newcommand{\BaseNTok}[1]{\textcolor[rgb]{0.25,0.63,0.44}{{#1}}}
    \newcommand{\FloatTok}[1]{\textcolor[rgb]{0.25,0.63,0.44}{{#1}}}
    \newcommand{\CharTok}[1]{\textcolor[rgb]{0.25,0.44,0.63}{{#1}}}
    \newcommand{\StringTok}[1]{\textcolor[rgb]{0.25,0.44,0.63}{{#1}}}
    \newcommand{\CommentTok}[1]{\textcolor[rgb]{0.38,0.63,0.69}{\textit{{#1}}}}
    \newcommand{\OtherTok}[1]{\textcolor[rgb]{0.00,0.44,0.13}{{#1}}}
    \newcommand{\AlertTok}[1]{\textcolor[rgb]{1.00,0.00,0.00}{\textbf{{#1}}}}
    \newcommand{\FunctionTok}[1]{\textcolor[rgb]{0.02,0.16,0.49}{{#1}}}
    \newcommand{\RegionMarkerTok}[1]{{#1}}
    \newcommand{\ErrorTok}[1]{\textcolor[rgb]{1.00,0.00,0.00}{\textbf{{#1}}}}
    \newcommand{\NormalTok}[1]{{#1}}
    
    % Additional commands for more recent versions of Pandoc
    \newcommand{\ConstantTok}[1]{\textcolor[rgb]{0.53,0.00,0.00}{{#1}}}
    \newcommand{\SpecialCharTok}[1]{\textcolor[rgb]{0.25,0.44,0.63}{{#1}}}
    \newcommand{\VerbatimStringTok}[1]{\textcolor[rgb]{0.25,0.44,0.63}{{#1}}}
    \newcommand{\SpecialStringTok}[1]{\textcolor[rgb]{0.73,0.40,0.53}{{#1}}}
    \newcommand{\ImportTok}[1]{{#1}}
    \newcommand{\DocumentationTok}[1]{\textcolor[rgb]{0.73,0.13,0.13}{\textit{{#1}}}}
    \newcommand{\AnnotationTok}[1]{\textcolor[rgb]{0.38,0.63,0.69}{\textbf{\textit{{#1}}}}}
    \newcommand{\CommentVarTok}[1]{\textcolor[rgb]{0.38,0.63,0.69}{\textbf{\textit{{#1}}}}}
    \newcommand{\VariableTok}[1]{\textcolor[rgb]{0.10,0.09,0.49}{{#1}}}
    \newcommand{\ControlFlowTok}[1]{\textcolor[rgb]{0.00,0.44,0.13}{\textbf{{#1}}}}
    \newcommand{\OperatorTok}[1]{\textcolor[rgb]{0.40,0.40,0.40}{{#1}}}
    \newcommand{\BuiltInTok}[1]{{#1}}
    \newcommand{\ExtensionTok}[1]{{#1}}
    \newcommand{\PreprocessorTok}[1]{\textcolor[rgb]{0.74,0.48,0.00}{{#1}}}
    \newcommand{\AttributeTok}[1]{\textcolor[rgb]{0.49,0.56,0.16}{{#1}}}
    \newcommand{\InformationTok}[1]{\textcolor[rgb]{0.38,0.63,0.69}{\textbf{\textit{{#1}}}}}
    \newcommand{\WarningTok}[1]{\textcolor[rgb]{0.38,0.63,0.69}{\textbf{\textit{{#1}}}}}
    
    
    % Define a nice break command that doesn't care if a line doesn't already
    % exist.
    \def\br{\hspace*{\fill} \\* }
    % Math Jax compatibility definitions
    \def\gt{>}
    \def\lt{<}
    \let\Oldtex\TeX
    \let\Oldlatex\LaTeX
    \renewcommand{\TeX}{\textrm{\Oldtex}}
    \renewcommand{\LaTeX}{\textrm{\Oldlatex}}
    % Document parameters
    % Document title
    \newcommand*{\mytitle}{Unit 6: Advanced NumPy}

    % Included at the bottom of the preamble

\usepackage{microtype}


\title{\mytitle}
\author{Richard Foltyn}
\hypersetup{pdfauthor={Richard Foltyn}, pdftitle={\mytitle}}

% Remove horizontal rules in a very hackish way
\renewcommand{\rule}[2]{}

\RequirePackage{xspace}


\newcommand*{\eg}{e.g.\@\xspace}
\newcommand*{\Eg}{E.g.\@\xspace}
\newcommand*{\etc}{etc.\@\xspace}
\newcommand*{\ie}{i.e.\@\xspace}
\newcommand*{\vs}{vs.\@\xspace}
\newcommand*{\viz}{viz.\@\xspace}
\newcommand*{\US}{U.S.\@\xspace}


    
    
    
    
    
% Pygments definitions
\makeatletter
\def\PY@reset{\let\PY@it=\relax \let\PY@bf=\relax%
    \let\PY@ul=\relax \let\PY@tc=\relax%
    \let\PY@bc=\relax \let\PY@ff=\relax}
\def\PY@tok#1{\csname PY@tok@#1\endcsname}
\def\PY@toks#1+{\ifx\relax#1\empty\else%
    \PY@tok{#1}\expandafter\PY@toks\fi}
\def\PY@do#1{\PY@bc{\PY@tc{\PY@ul{%
    \PY@it{\PY@bf{\PY@ff{#1}}}}}}}
\def\PY#1#2{\PY@reset\PY@toks#1+\relax+\PY@do{#2}}

\expandafter\def\csname PY@tok@w\endcsname{\def\PY@tc##1{\textcolor[rgb]{0.73,0.73,0.73}{##1}}}
\expandafter\def\csname PY@tok@c\endcsname{\let\PY@it=\textit\def\PY@tc##1{\textcolor[rgb]{0.25,0.50,0.50}{##1}}}
\expandafter\def\csname PY@tok@cp\endcsname{\def\PY@tc##1{\textcolor[rgb]{0.74,0.48,0.00}{##1}}}
\expandafter\def\csname PY@tok@k\endcsname{\let\PY@bf=\textbf\def\PY@tc##1{\textcolor[rgb]{0.00,0.50,0.00}{##1}}}
\expandafter\def\csname PY@tok@kp\endcsname{\def\PY@tc##1{\textcolor[rgb]{0.00,0.50,0.00}{##1}}}
\expandafter\def\csname PY@tok@kt\endcsname{\def\PY@tc##1{\textcolor[rgb]{0.69,0.00,0.25}{##1}}}
\expandafter\def\csname PY@tok@o\endcsname{\def\PY@tc##1{\textcolor[rgb]{0.40,0.40,0.40}{##1}}}
\expandafter\def\csname PY@tok@ow\endcsname{\let\PY@bf=\textbf\def\PY@tc##1{\textcolor[rgb]{0.67,0.13,1.00}{##1}}}
\expandafter\def\csname PY@tok@nb\endcsname{\def\PY@tc##1{\textcolor[rgb]{0.00,0.50,0.00}{##1}}}
\expandafter\def\csname PY@tok@nf\endcsname{\def\PY@tc##1{\textcolor[rgb]{0.00,0.00,1.00}{##1}}}
\expandafter\def\csname PY@tok@nc\endcsname{\let\PY@bf=\textbf\def\PY@tc##1{\textcolor[rgb]{0.00,0.00,1.00}{##1}}}
\expandafter\def\csname PY@tok@nn\endcsname{\let\PY@bf=\textbf\def\PY@tc##1{\textcolor[rgb]{0.00,0.00,1.00}{##1}}}
\expandafter\def\csname PY@tok@ne\endcsname{\let\PY@bf=\textbf\def\PY@tc##1{\textcolor[rgb]{0.82,0.25,0.23}{##1}}}
\expandafter\def\csname PY@tok@nv\endcsname{\def\PY@tc##1{\textcolor[rgb]{0.10,0.09,0.49}{##1}}}
\expandafter\def\csname PY@tok@no\endcsname{\def\PY@tc##1{\textcolor[rgb]{0.53,0.00,0.00}{##1}}}
\expandafter\def\csname PY@tok@nl\endcsname{\def\PY@tc##1{\textcolor[rgb]{0.63,0.63,0.00}{##1}}}
\expandafter\def\csname PY@tok@ni\endcsname{\let\PY@bf=\textbf\def\PY@tc##1{\textcolor[rgb]{0.60,0.60,0.60}{##1}}}
\expandafter\def\csname PY@tok@na\endcsname{\def\PY@tc##1{\textcolor[rgb]{0.49,0.56,0.16}{##1}}}
\expandafter\def\csname PY@tok@nt\endcsname{\let\PY@bf=\textbf\def\PY@tc##1{\textcolor[rgb]{0.00,0.50,0.00}{##1}}}
\expandafter\def\csname PY@tok@nd\endcsname{\def\PY@tc##1{\textcolor[rgb]{0.67,0.13,1.00}{##1}}}
\expandafter\def\csname PY@tok@s\endcsname{\def\PY@tc##1{\textcolor[rgb]{0.73,0.13,0.13}{##1}}}
\expandafter\def\csname PY@tok@sd\endcsname{\let\PY@it=\textit\def\PY@tc##1{\textcolor[rgb]{0.73,0.13,0.13}{##1}}}
\expandafter\def\csname PY@tok@si\endcsname{\let\PY@bf=\textbf\def\PY@tc##1{\textcolor[rgb]{0.73,0.40,0.53}{##1}}}
\expandafter\def\csname PY@tok@se\endcsname{\let\PY@bf=\textbf\def\PY@tc##1{\textcolor[rgb]{0.73,0.40,0.13}{##1}}}
\expandafter\def\csname PY@tok@sr\endcsname{\def\PY@tc##1{\textcolor[rgb]{0.73,0.40,0.53}{##1}}}
\expandafter\def\csname PY@tok@ss\endcsname{\def\PY@tc##1{\textcolor[rgb]{0.10,0.09,0.49}{##1}}}
\expandafter\def\csname PY@tok@sx\endcsname{\def\PY@tc##1{\textcolor[rgb]{0.00,0.50,0.00}{##1}}}
\expandafter\def\csname PY@tok@m\endcsname{\def\PY@tc##1{\textcolor[rgb]{0.40,0.40,0.40}{##1}}}
\expandafter\def\csname PY@tok@gh\endcsname{\let\PY@bf=\textbf\def\PY@tc##1{\textcolor[rgb]{0.00,0.00,0.50}{##1}}}
\expandafter\def\csname PY@tok@gu\endcsname{\let\PY@bf=\textbf\def\PY@tc##1{\textcolor[rgb]{0.50,0.00,0.50}{##1}}}
\expandafter\def\csname PY@tok@gd\endcsname{\def\PY@tc##1{\textcolor[rgb]{0.63,0.00,0.00}{##1}}}
\expandafter\def\csname PY@tok@gi\endcsname{\def\PY@tc##1{\textcolor[rgb]{0.00,0.63,0.00}{##1}}}
\expandafter\def\csname PY@tok@gr\endcsname{\def\PY@tc##1{\textcolor[rgb]{1.00,0.00,0.00}{##1}}}
\expandafter\def\csname PY@tok@ge\endcsname{\let\PY@it=\textit}
\expandafter\def\csname PY@tok@gs\endcsname{\let\PY@bf=\textbf}
\expandafter\def\csname PY@tok@gp\endcsname{\let\PY@bf=\textbf\def\PY@tc##1{\textcolor[rgb]{0.00,0.00,0.50}{##1}}}
\expandafter\def\csname PY@tok@go\endcsname{\def\PY@tc##1{\textcolor[rgb]{0.53,0.53,0.53}{##1}}}
\expandafter\def\csname PY@tok@gt\endcsname{\def\PY@tc##1{\textcolor[rgb]{0.00,0.27,0.87}{##1}}}
\expandafter\def\csname PY@tok@err\endcsname{\def\PY@bc##1{\setlength{\fboxsep}{0pt}\fcolorbox[rgb]{1.00,0.00,0.00}{1,1,1}{\strut ##1}}}
\expandafter\def\csname PY@tok@kc\endcsname{\let\PY@bf=\textbf\def\PY@tc##1{\textcolor[rgb]{0.00,0.50,0.00}{##1}}}
\expandafter\def\csname PY@tok@kd\endcsname{\let\PY@bf=\textbf\def\PY@tc##1{\textcolor[rgb]{0.00,0.50,0.00}{##1}}}
\expandafter\def\csname PY@tok@kn\endcsname{\let\PY@bf=\textbf\def\PY@tc##1{\textcolor[rgb]{0.00,0.50,0.00}{##1}}}
\expandafter\def\csname PY@tok@kr\endcsname{\let\PY@bf=\textbf\def\PY@tc##1{\textcolor[rgb]{0.00,0.50,0.00}{##1}}}
\expandafter\def\csname PY@tok@bp\endcsname{\def\PY@tc##1{\textcolor[rgb]{0.00,0.50,0.00}{##1}}}
\expandafter\def\csname PY@tok@fm\endcsname{\def\PY@tc##1{\textcolor[rgb]{0.00,0.00,1.00}{##1}}}
\expandafter\def\csname PY@tok@vc\endcsname{\def\PY@tc##1{\textcolor[rgb]{0.10,0.09,0.49}{##1}}}
\expandafter\def\csname PY@tok@vg\endcsname{\def\PY@tc##1{\textcolor[rgb]{0.10,0.09,0.49}{##1}}}
\expandafter\def\csname PY@tok@vi\endcsname{\def\PY@tc##1{\textcolor[rgb]{0.10,0.09,0.49}{##1}}}
\expandafter\def\csname PY@tok@vm\endcsname{\def\PY@tc##1{\textcolor[rgb]{0.10,0.09,0.49}{##1}}}
\expandafter\def\csname PY@tok@sa\endcsname{\def\PY@tc##1{\textcolor[rgb]{0.73,0.13,0.13}{##1}}}
\expandafter\def\csname PY@tok@sb\endcsname{\def\PY@tc##1{\textcolor[rgb]{0.73,0.13,0.13}{##1}}}
\expandafter\def\csname PY@tok@sc\endcsname{\def\PY@tc##1{\textcolor[rgb]{0.73,0.13,0.13}{##1}}}
\expandafter\def\csname PY@tok@dl\endcsname{\def\PY@tc##1{\textcolor[rgb]{0.73,0.13,0.13}{##1}}}
\expandafter\def\csname PY@tok@s2\endcsname{\def\PY@tc##1{\textcolor[rgb]{0.73,0.13,0.13}{##1}}}
\expandafter\def\csname PY@tok@sh\endcsname{\def\PY@tc##1{\textcolor[rgb]{0.73,0.13,0.13}{##1}}}
\expandafter\def\csname PY@tok@s1\endcsname{\def\PY@tc##1{\textcolor[rgb]{0.73,0.13,0.13}{##1}}}
\expandafter\def\csname PY@tok@mb\endcsname{\def\PY@tc##1{\textcolor[rgb]{0.40,0.40,0.40}{##1}}}
\expandafter\def\csname PY@tok@mf\endcsname{\def\PY@tc##1{\textcolor[rgb]{0.40,0.40,0.40}{##1}}}
\expandafter\def\csname PY@tok@mh\endcsname{\def\PY@tc##1{\textcolor[rgb]{0.40,0.40,0.40}{##1}}}
\expandafter\def\csname PY@tok@mi\endcsname{\def\PY@tc##1{\textcolor[rgb]{0.40,0.40,0.40}{##1}}}
\expandafter\def\csname PY@tok@il\endcsname{\def\PY@tc##1{\textcolor[rgb]{0.40,0.40,0.40}{##1}}}
\expandafter\def\csname PY@tok@mo\endcsname{\def\PY@tc##1{\textcolor[rgb]{0.40,0.40,0.40}{##1}}}
\expandafter\def\csname PY@tok@ch\endcsname{\let\PY@it=\textit\def\PY@tc##1{\textcolor[rgb]{0.25,0.50,0.50}{##1}}}
\expandafter\def\csname PY@tok@cm\endcsname{\let\PY@it=\textit\def\PY@tc##1{\textcolor[rgb]{0.25,0.50,0.50}{##1}}}
\expandafter\def\csname PY@tok@cpf\endcsname{\let\PY@it=\textit\def\PY@tc##1{\textcolor[rgb]{0.25,0.50,0.50}{##1}}}
\expandafter\def\csname PY@tok@c1\endcsname{\let\PY@it=\textit\def\PY@tc##1{\textcolor[rgb]{0.25,0.50,0.50}{##1}}}
\expandafter\def\csname PY@tok@cs\endcsname{\let\PY@it=\textit\def\PY@tc##1{\textcolor[rgb]{0.25,0.50,0.50}{##1}}}

\def\PYZbs{\char`\\}
\def\PYZus{\char`\_}
\def\PYZob{\char`\{}
\def\PYZcb{\char`\}}
\def\PYZca{\char`\^}
\def\PYZam{\char`\&}
\def\PYZlt{\char`\<}
\def\PYZgt{\char`\>}
\def\PYZsh{\char`\#}
\def\PYZpc{\char`\%}
\def\PYZdl{\char`\$}
\def\PYZhy{\char`\-}
\def\PYZsq{\char`\'}
\def\PYZdq{\char`\"}
\def\PYZti{\char`\~}
% for compatibility with earlier versions
\def\PYZat{@}
\def\PYZlb{[}
\def\PYZrb{]}
\makeatother


    % For linebreaks inside Verbatim environment from package fancyvrb. 
    \makeatletter
        \newbox\Wrappedcontinuationbox 
        \newbox\Wrappedvisiblespacebox 
        \newcommand*\Wrappedvisiblespace {\textcolor{red}{\textvisiblespace}} 
        \newcommand*\Wrappedcontinuationsymbol {\textcolor{red}{\llap{\tiny$\m@th\hookrightarrow$}}} 
        \newcommand*\Wrappedcontinuationindent {3ex } 
        \newcommand*\Wrappedafterbreak {\kern\Wrappedcontinuationindent\copy\Wrappedcontinuationbox} 
        % Take advantage of the already applied Pygments mark-up to insert 
        % potential linebreaks for TeX processing. 
        %        {, <, #, %, $, ' and ": go to next line. 
        %        _, }, ^, &, >, - and ~: stay at end of broken line. 
        % Use of \textquotesingle for straight quote. 
        \newcommand*\Wrappedbreaksatspecials {% 
            \def\PYGZus{\discretionary{\char`\_}{\Wrappedafterbreak}{\char`\_}}% 
            \def\PYGZob{\discretionary{}{\Wrappedafterbreak\char`\{}{\char`\{}}% 
            \def\PYGZcb{\discretionary{\char`\}}{\Wrappedafterbreak}{\char`\}}}% 
            \def\PYGZca{\discretionary{\char`\^}{\Wrappedafterbreak}{\char`\^}}% 
            \def\PYGZam{\discretionary{\char`\&}{\Wrappedafterbreak}{\char`\&}}% 
            \def\PYGZlt{\discretionary{}{\Wrappedafterbreak\char`\<}{\char`\<}}% 
            \def\PYGZgt{\discretionary{\char`\>}{\Wrappedafterbreak}{\char`\>}}% 
            \def\PYGZsh{\discretionary{}{\Wrappedafterbreak\char`\#}{\char`\#}}% 
            \def\PYGZpc{\discretionary{}{\Wrappedafterbreak\char`\%}{\char`\%}}% 
            \def\PYGZdl{\discretionary{}{\Wrappedafterbreak\char`\$}{\char`\$}}% 
            \def\PYGZhy{\discretionary{\char`\-}{\Wrappedafterbreak}{\char`\-}}% 
            \def\PYGZsq{\discretionary{}{\Wrappedafterbreak\textquotesingle}{\textquotesingle}}% 
            \def\PYGZdq{\discretionary{}{\Wrappedafterbreak\char`\"}{\char`\"}}% 
            \def\PYGZti{\discretionary{\char`\~}{\Wrappedafterbreak}{\char`\~}}% 
        } 
        % Some characters . , ; ? ! / are not pygmentized. 
        % This macro makes them "active" and they will insert potential linebreaks 
        \newcommand*\Wrappedbreaksatpunct {% 
            \lccode`\~`\.\lowercase{\def~}{\discretionary{\hbox{\char`\.}}{\Wrappedafterbreak}{\hbox{\char`\.}}}% 
            \lccode`\~`\,\lowercase{\def~}{\discretionary{\hbox{\char`\,}}{\Wrappedafterbreak}{\hbox{\char`\,}}}% 
            \lccode`\~`\;\lowercase{\def~}{\discretionary{\hbox{\char`\;}}{\Wrappedafterbreak}{\hbox{\char`\;}}}% 
            \lccode`\~`\:\lowercase{\def~}{\discretionary{\hbox{\char`\:}}{\Wrappedafterbreak}{\hbox{\char`\:}}}% 
            \lccode`\~`\?\lowercase{\def~}{\discretionary{\hbox{\char`\?}}{\Wrappedafterbreak}{\hbox{\char`\?}}}% 
            \lccode`\~`\!\lowercase{\def~}{\discretionary{\hbox{\char`\!}}{\Wrappedafterbreak}{\hbox{\char`\!}}}% 
            \lccode`\~`\/\lowercase{\def~}{\discretionary{\hbox{\char`\/}}{\Wrappedafterbreak}{\hbox{\char`\/}}}% 
            \catcode`\.\active
            \catcode`\,\active 
            \catcode`\;\active
            \catcode`\:\active
            \catcode`\?\active
            \catcode`\!\active
            \catcode`\/\active 
            \lccode`\~`\~ 	
        }
    \makeatother

    \let\OriginalVerbatim=\Verbatim
    \makeatletter
    \renewcommand{\Verbatim}[1][1]{%
        %\parskip\z@skip
        \sbox\Wrappedcontinuationbox {\Wrappedcontinuationsymbol}%
        \sbox\Wrappedvisiblespacebox {\FV@SetupFont\Wrappedvisiblespace}%
        \def\FancyVerbFormatLine ##1{\hsize\linewidth
            \vtop{\raggedright\hyphenpenalty\z@\exhyphenpenalty\z@
                \doublehyphendemerits\z@\finalhyphendemerits\z@
                \strut ##1\strut}%
        }%
        % If the linebreak is at a space, the latter will be displayed as visible
        % space at end of first line, and a continuation symbol starts next line.
        % Stretch/shrink are however usually zero for typewriter font.
        \def\FV@Space {%
            \nobreak\hskip\z@ plus\fontdimen3\font minus\fontdimen4\font
            \discretionary{\copy\Wrappedvisiblespacebox}{\Wrappedafterbreak}
            {\kern\fontdimen2\font}%
        }%
        
        % Allow breaks at special characters using \PYG... macros.
        \Wrappedbreaksatspecials
        % Breaks at punctuation characters . , ; ? ! and / need catcode=\active 	
        \OriginalVerbatim[#1,fontsize=\small,codes*=\Wrappedbreaksatpunct]%
    }
    \makeatother

    % Exact colors from NB
    \definecolor{incolor}{HTML}{303F9F}
    \definecolor{outcolor}{HTML}{D84315}
    \definecolor{cellborder}{HTML}{CFCFCF}
    \definecolor{cellbackground}{HTML}{F7F7F7}
    
    % prompt
    \makeatletter
    \newcommand{\boxspacing}{\kern\kvtcb@left@rule\kern\kvtcb@boxsep}
    \makeatother
    \newcommand{\prompt}[4]{
        {\ttfamily\llap{{\color{#2}[#3]:\hspace{3pt}#4}}\vspace{-\baselineskip}}
    }
    

    
    % Prevent overflowing lines due to hard-to-break entities
    \sloppy 
    % Setup hyperref package
    \hypersetup{
      breaklinks=true,  % so long urls are correctly broken across lines
      colorlinks=true,
      urlcolor=urlcolor,
      linkcolor=linkcolor,
      citecolor=citecolor,
      }
    % Slightly bigger margins than the latex defaults
    
    \geometry{verbose,tmargin=1in,bmargin=1in,lmargin=1in,rmargin=1in}
    
    

\begin{document}
    
    \maketitle
    \tableofcontents

    

    
    \hypertarget{advanced-numpy}{%
\section{Advanced NumPy}\label{advanced-numpy}}

We already encountered NumPy arrays and their basic usage throughout
this course. In this unit, we will take a more in-depth look at NumPy.

\hypertarget{why-numpy-arrays}{%
\subsection{Why NumPy arrays?}\label{why-numpy-arrays}}

Why don't we just stick with built-in types such as Python lists to
store and process data? It turns out that while the built-in objects are
quite flexible, this flexibility comes at the cost of decreased
performance:

\begin{itemize}
\item
  \texttt{list} objects can store arbitrary data types, and the data
  type of any item can change:

\begin{verbatim}
items = ['foo']
items[0] = 1.0      # item was a string, now it's a float!
\end{verbatim}
\item
  There is no guarantee where in memory the data will be stored. In
  fact, two consecutive items could be very ``far'' from each other in
  memory, which imposes a performance penalty.
\item
  Even primitive data types such as \texttt{int} and \texttt{float} are
  not ``raw'' data, but full-fledged objects. That, again, is bad for
  performance.
\end{itemize}

On the other hand, the approach taken by NumPy is to store and process
data in a way very similar to low-level languages such as C and Fortran.
This means that

\begin{itemize}
\item
  arrays contain a \emph{homogenous} data type. \emph{All} elements are
  either 64-bit integers (\texttt{np.int64}), 64-bit floating-point
  numbers (\texttt{np.float64}), or some other of the many data types
  supported by NumPy.

  It is technically possible to get around this by specifying an array's
  data type (\texttt{dtype}) to be \texttt{object}, which is the most
  generic Python data type. However, we would never want to do this for
  numerical computations.
\item
  NumPy arrays are usually \emph{contiguous} in memory. This means that
  adjacent array elements are actually guaranteed to be stored next to
  each other, which allows for much more efficient computations.
\item
  NumPy arrays support numerous operations used in scientific computing.
  For example, with a NumPy array we can write

\begin{verbatim}
x = np.array([1, 2, 3])
y = x + 1       # We would expect this to work
\end{verbatim}

  With lists, however, we cannot:

\begin{verbatim}
x = [1, 2, 3]
y = x + 1       # Does not work!
\end{verbatim}

  Lists don't implement an addition operator that accepts integer
  arguments, so this code triggers an error.
\end{itemize}

You can see the performance uplift provided by NumPy arrays in this
simple example:

    \begin{tcolorbox}[breakable, size=fbox, boxrule=1pt, pad at break*=1mm,colback=cellbackground, colframe=cellborder]
\prompt{In}{incolor}{1}{\boxspacing}
\begin{Verbatim}[commandchars=\\\{\}]
\PY{c+c1}{\PYZsh{} Create list 0, 1, 2, ..., 999}
\PY{n}{lst} \PY{o}{=} \PY{n+nb}{list}\PY{p}{(}\PY{n+nb}{range}\PY{p}{(}\PY{l+m+mi}{1000}\PY{p}{)}\PY{p}{)}

\PY{c+c1}{\PYZsh{} Compute squares, time how long it takes}
\PY{o}{\PYZpc{}}\PY{k}{timeit} [i**2 for i in lst]
\end{Verbatim}
\end{tcolorbox}

    \begin{Verbatim}[commandchars=\\\{\}]
151 µs ± 2.68 µs per loop (mean ± std. dev. of 7 runs, 10000 loops each)
    \end{Verbatim}

    \begin{tcolorbox}[breakable, size=fbox, boxrule=1pt, pad at break*=1mm,colback=cellbackground, colframe=cellborder]
\prompt{In}{incolor}{2}{\boxspacing}
\begin{Verbatim}[commandchars=\\\{\}]
\PY{c+c1}{\PYZsh{} Repeat using NumPy arrays}
\PY{k+kn}{import} \PY{n+nn}{numpy} \PY{k}{as} \PY{n+nn}{np}
\PY{n}{arr} \PY{o}{=} \PY{n}{np}\PY{o}{.}\PY{n}{arange}\PY{p}{(}\PY{l+m+mi}{1000}\PY{p}{)}

\PY{o}{\PYZpc{}}\PY{k}{timeit} arr**2
\end{Verbatim}
\end{tcolorbox}

    \begin{Verbatim}[commandchars=\\\{\}]
849 ns ± 9.52 ns per loop (mean ± std. dev. of 7 runs, 1000000 loops each)
    \end{Verbatim}

    \begin{itemize}
\tightlist
\item
  On my machine, squaring 1000 elements of a \texttt{list} takes
  approximately 200 times longer than the computation using NumPy
  arrays!
\item
  Also, as mentioned above, NumPy supports squaring an array directly,
  while we have to manually loop through the \texttt{list} and square
  each element individually.
\end{itemize}

\emph{Note:} \texttt{\%timeit} is a so-called magic command that only
works in notebooks, but not in regular Python files.
{[}\href{https://ipython.readthedocs.io/en/stable/interactive/magics.html}{See
documentation}{]}


\hypertarget{creating-arrays}{%
\subsection{Creating arrays}\label{creating-arrays}}

We have already encountered some of the most frequently used array
creation routines:

\begin{itemize}
\tightlist
\item
  \texttt{np.array()} creates an array from a given argument, which can
  be

  \begin{itemize}
  \tightlist
  \item
    a scalar
  \item
    a collection such as a list or tuple
  \item
    some other iterable object, \eg something created by
    \texttt{range()}
  \end{itemize}
\item
  \texttt{np.empty()} allocates memory for a given array shape, but does
  not overwrite it with initial values.
\item
  \texttt{np.zeros()} creates an array of a given shape and initializes
  it to zeros.
\item
  \texttt{np.ones()} creates an array of a given shape and initializes
  it to ones.
\item
  \texttt{np.arange(start,stop,step)} creates an array with evenly
  spaced elements over the range \([start,stop)\).

  \begin{itemize}
  \tightlist
  \item
    \texttt{start} and \texttt{step} can be omitted and then default to
    \texttt{start=0} and \texttt{step=1}.
  \item
    Note that the number \texttt{stop} is never included in the
    resulting array!
  \end{itemize}
\item
  \texttt{np.linspace(start,stop,num)} returns a vector of \texttt{num}
  elements which are evenly spaced over the interval \([start,stop]\).
\item
  \texttt{np.identity(n)} returns the identity matrix of a size
  \(n \times n\).
\item
  \texttt{np.eye()} is a more flexible variant of \texttt{identity()}
  that can, for example, also create non-squared matrices.
\end{itemize}

There are many more array creation functions for more exotic use-cases,
see the NumPy
\href{https://numpy.org/doc/stable/reference/routines.array-creation.html}{documentation}
for details.

\emph{Examples:}

    \begin{tcolorbox}[breakable, size=fbox, boxrule=1pt, pad at break*=1mm,colback=cellbackground, colframe=cellborder]
\prompt{In}{incolor}{3}{\boxspacing}
\begin{Verbatim}[commandchars=\\\{\}]
\PY{k+kn}{import} \PY{n+nn}{numpy} \PY{k}{as} \PY{n+nn}{np}

\PY{c+c1}{\PYZsh{} Create array from list}
\PY{n}{lst} \PY{o}{=} \PY{p}{[}\PY{l+m+mi}{1}\PY{p}{,} \PY{l+m+mi}{2}\PY{p}{,} \PY{l+m+mi}{3}\PY{p}{]}
\PY{n}{np}\PY{o}{.}\PY{n}{array}\PY{p}{(}\PY{n}{lst}\PY{p}{)}
\end{Verbatim}
\end{tcolorbox}

            \begin{tcolorbox}[breakable, size=fbox, boxrule=.5pt, pad at break*=1mm, opacityfill=0]
\prompt{Out}{outcolor}{3}{\boxspacing}
\begin{Verbatim}[commandchars=\\\{\}]
array([1, 2, 3])
\end{Verbatim}
\end{tcolorbox}
        
    \begin{tcolorbox}[breakable, size=fbox, boxrule=1pt, pad at break*=1mm,colback=cellbackground, colframe=cellborder]
\prompt{In}{incolor}{4}{\boxspacing}
\begin{Verbatim}[commandchars=\\\{\}]
\PY{c+c1}{\PYZsh{} Create array from tuple}
\PY{n}{tpl} \PY{o}{=} \PY{l+m+mf}{1.0}\PY{p}{,} \PY{l+m+mf}{2.0}\PY{p}{,} \PY{l+m+mf}{3.0}
\PY{n}{np}\PY{o}{.}\PY{n}{array}\PY{p}{(}\PY{n}{tpl}\PY{p}{)}
\end{Verbatim}
\end{tcolorbox}

            \begin{tcolorbox}[breakable, size=fbox, boxrule=.5pt, pad at break*=1mm, opacityfill=0]
\prompt{Out}{outcolor}{4}{\boxspacing}
\begin{Verbatim}[commandchars=\\\{\}]
array([1., 2., 3.])
\end{Verbatim}
\end{tcolorbox}
        
    \begin{tcolorbox}[breakable, size=fbox, boxrule=1pt, pad at break*=1mm,colback=cellbackground, colframe=cellborder]
\prompt{In}{incolor}{5}{\boxspacing}
\begin{Verbatim}[commandchars=\\\{\}]
\PY{c+c1}{\PYZsh{} arange: end point is not included!}
\PY{n}{np}\PY{o}{.}\PY{n}{arange}\PY{p}{(}\PY{l+m+mi}{5}\PY{p}{)}
\end{Verbatim}
\end{tcolorbox}

            \begin{tcolorbox}[breakable, size=fbox, boxrule=.5pt, pad at break*=1mm, opacityfill=0]
\prompt{Out}{outcolor}{5}{\boxspacing}
\begin{Verbatim}[commandchars=\\\{\}]
array([0, 1, 2, 3, 4])
\end{Verbatim}
\end{tcolorbox}
        
    \begin{tcolorbox}[breakable, size=fbox, boxrule=1pt, pad at break*=1mm,colback=cellbackground, colframe=cellborder]
\prompt{In}{incolor}{6}{\boxspacing}
\begin{Verbatim}[commandchars=\\\{\}]
\PY{c+c1}{\PYZsh{} arange: increments can be negative too!}
\PY{n}{np}\PY{o}{.}\PY{n}{arange}\PY{p}{(}\PY{l+m+mi}{5}\PY{p}{,} \PY{l+m+mi}{1}\PY{p}{,} \PY{o}{\PYZhy{}}\PY{l+m+mi}{1}\PY{p}{)}
\end{Verbatim}
\end{tcolorbox}

            \begin{tcolorbox}[breakable, size=fbox, boxrule=.5pt, pad at break*=1mm, opacityfill=0]
\prompt{Out}{outcolor}{6}{\boxspacing}
\begin{Verbatim}[commandchars=\\\{\}]
array([5, 4, 3, 2])
\end{Verbatim}
\end{tcolorbox}
        
    \begin{tcolorbox}[breakable, size=fbox, boxrule=1pt, pad at break*=1mm,colback=cellbackground, colframe=cellborder]
\prompt{In}{incolor}{7}{\boxspacing}
\begin{Verbatim}[commandchars=\\\{\}]
\PY{c+c1}{\PYZsh{} arange also works on floats}
\PY{n}{np}\PY{o}{.}\PY{n}{arange}\PY{p}{(}\PY{l+m+mf}{1.0}\PY{p}{,} \PY{l+m+mf}{3.0}\PY{p}{,} \PY{l+m+mf}{0.5678}\PY{p}{)}
\end{Verbatim}
\end{tcolorbox}

            \begin{tcolorbox}[breakable, size=fbox, boxrule=.5pt, pad at break*=1mm, opacityfill=0]
\prompt{Out}{outcolor}{7}{\boxspacing}
\begin{Verbatim}[commandchars=\\\{\}]
array([1.    , 1.5678, 2.1356, 2.7034])
\end{Verbatim}
\end{tcolorbox}
        
    \begin{tcolorbox}[breakable, size=fbox, boxrule=1pt, pad at break*=1mm,colback=cellbackground, colframe=cellborder]
\prompt{In}{incolor}{8}{\boxspacing}
\begin{Verbatim}[commandchars=\\\{\}]
\PY{c+c1}{\PYZsh{} linspace DOES include the end point}
\PY{n}{np}\PY{o}{.}\PY{n}{linspace}\PY{p}{(}\PY{l+m+mf}{0.0}\PY{p}{,} \PY{l+m+mf}{1.0}\PY{p}{,} \PY{l+m+mi}{11}\PY{p}{)}
\end{Verbatim}
\end{tcolorbox}

            \begin{tcolorbox}[breakable, size=fbox, boxrule=.5pt, pad at break*=1mm, opacityfill=0]
\prompt{Out}{outcolor}{8}{\boxspacing}
\begin{Verbatim}[commandchars=\\\{\}]
array([0. , 0.1, 0.2, 0.3, 0.4, 0.5, 0.6, 0.7, 0.8, 0.9, 1. ])
\end{Verbatim}
\end{tcolorbox}
        

\hypertarget{array-shape}{%
\subsection{Array shape}\label{array-shape}}

Many of the array creation routines take the desired shape of the array
as their first argument. Array shapes are usually specified as tuples:

\begin{itemize}
\item
  A vector with 5 elements has shape \texttt{(5,\ )}.

  Note the comma \texttt{,}: we need to specify a tuple with a single
  element using this comma, since \texttt{(5)} is just the integer 5,
  not a tuple.

  It is worth pointing out that this is not the same as a 2-dimensional
  array with shape \texttt{(1,\ 5)} or \texttt{(5,\ 1)}, even though
  they have the same number of elements.
\item
  A \(2\times2\) matrix has shape \texttt{(2,\ 2)}.
\item
  A higher-dimensional array has shape \texttt{(k,\ l,\ m,\ n,\ ...)}.
\item
  A \emph{scalar} NumPy array has shape \texttt{()}, an empty tuple.

  While ``scalar array'' sounds like an oxymoron, it does exist.
\end{itemize}

We can query the shape of an array using the \texttt{shape} attribute,
and the number of dimensions is stored in the \texttt{ndim} attribute.

\emph{Examples:}

    \begin{tcolorbox}[breakable, size=fbox, boxrule=1pt, pad at break*=1mm,colback=cellbackground, colframe=cellborder]
\prompt{In}{incolor}{9}{\boxspacing}
\begin{Verbatim}[commandchars=\\\{\}]
\PY{k+kn}{import} \PY{n+nn}{numpy} \PY{k}{as} \PY{n+nn}{np}

\PY{c+c1}{\PYZsh{} Scalar array}
\PY{n}{x} \PY{o}{=} \PY{n}{np}\PY{o}{.}\PY{n}{array}\PY{p}{(}\PY{l+m+mf}{0.0}\PY{p}{)}
\PY{n+nb}{print}\PY{p}{(}\PY{l+s+s1}{\PYZsq{}}\PY{l+s+s1}{Scalar array with shape=}\PY{l+s+si}{\PYZob{}\PYZcb{}}\PY{l+s+s1}{ and ndim=}\PY{l+s+si}{\PYZob{}\PYZcb{}}\PY{l+s+s1}{\PYZsq{}}\PY{o}{.}\PY{n}{format}\PY{p}{(}\PY{n}{x}\PY{o}{.}\PY{n}{shape}\PY{p}{,} \PY{n}{x}\PY{o}{.}\PY{n}{ndim}\PY{p}{)}\PY{p}{)}
\end{Verbatim}
\end{tcolorbox}

    \begin{Verbatim}[commandchars=\\\{\}]
Scalar array with shape=() and ndim=0
    \end{Verbatim}

    Note that a scalar NumPy array is not the same as a Python scalar. The
built-in type \texttt{float} has neither a \texttt{shape}, nor an
\texttt{ndim}, nor any other of the NumPy array attributes.

    \begin{tcolorbox}[breakable, size=fbox, boxrule=1pt, pad at break*=1mm,colback=cellbackground, colframe=cellborder]
\prompt{In}{incolor}{10}{\boxspacing}
\begin{Verbatim}[commandchars=\\\{\}]
\PY{n}{scalar} \PY{o}{=} \PY{l+m+mf}{1.0}
\PY{n}{scalar}\PY{o}{.}\PY{n}{shape}
\end{Verbatim}
\end{tcolorbox}

    \begin{Verbatim}[commandchars=\\\{\}, frame=single, framerule=2mm, rulecolor=\color{outerrorbackground}]
\textcolor{ansi-red}{AttributeError}\textcolor{ansi-red}{:} 'float' object has no attribute 'shape'

    \end{Verbatim}

    \begin{tcolorbox}[breakable, size=fbox, boxrule=1pt, pad at break*=1mm,colback=cellbackground, colframe=cellborder]
\prompt{In}{incolor}{11}{\boxspacing}
\begin{Verbatim}[commandchars=\\\{\}]
\PY{c+c1}{\PYZsh{} 1\PYZhy{}dimensional array (vector), values not initialised}
\PY{n}{x} \PY{o}{=} \PY{n}{np}\PY{o}{.}\PY{n}{empty}\PY{p}{(}\PY{p}{(}\PY{l+m+mi}{5}\PY{p}{,}\PY{p}{)}\PY{p}{)}
\PY{n}{x}       \PY{c+c1}{\PYZsh{} could contain arbitrary garbage}
\end{Verbatim}
\end{tcolorbox}

            \begin{tcolorbox}[breakable, size=fbox, boxrule=.5pt, pad at break*=1mm, opacityfill=0]
\prompt{Out}{outcolor}{11}{\boxspacing}
\begin{Verbatim}[commandchars=\\\{\}]
array([0.0e+000, 4.9e-324, 9.9e-324, 1.5e-323, 2.0e-323])
\end{Verbatim}
\end{tcolorbox}
        
    An array created with \texttt{empty()} will contain arbitrary garbage
since the memory block assigned to the array is not initialised. The
result will most likely differ on each invocation and across computers.

Most function accept an integer value instead of a \texttt{tuple} when
creating 1-dimensional arrays, which is interpreted as the number of
elements:

    \begin{tcolorbox}[breakable, size=fbox, boxrule=1pt, pad at break*=1mm,colback=cellbackground, colframe=cellborder]
\prompt{In}{incolor}{12}{\boxspacing}
\begin{Verbatim}[commandchars=\\\{\}]
\PY{c+c1}{\PYZsh{} 1\PYZhy{}dimensional array}
\PY{n}{x} \PY{o}{=} \PY{n}{np}\PY{o}{.}\PY{n}{empty}\PY{p}{(}\PY{l+m+mi}{5}\PY{p}{)}         \PY{c+c1}{\PYZsh{} equivalent to np.empty((5,))}
\end{Verbatim}
\end{tcolorbox}

    Higher-dimensional arrays are creating by passing in tuples with more
than one element:

    \begin{tcolorbox}[breakable, size=fbox, boxrule=1pt, pad at break*=1mm,colback=cellbackground, colframe=cellborder]
\prompt{In}{incolor}{13}{\boxspacing}
\begin{Verbatim}[commandchars=\\\{\}]
\PY{n}{np}\PY{o}{.}\PY{n}{ones}\PY{p}{(}\PY{p}{(}\PY{l+m+mi}{1}\PY{p}{,} \PY{l+m+mi}{2}\PY{p}{,} \PY{l+m+mi}{3}\PY{p}{)}\PY{p}{)}      \PY{c+c1}{\PYZsh{} 3d\PYZhy{}array}
\end{Verbatim}
\end{tcolorbox}

            \begin{tcolorbox}[breakable, size=fbox, boxrule=.5pt, pad at break*=1mm, opacityfill=0]
\prompt{Out}{outcolor}{13}{\boxspacing}
\begin{Verbatim}[commandchars=\\\{\}]
array([[[1., 1., 1.],
        [1., 1., 1.]]])
\end{Verbatim}
\end{tcolorbox}
        
    Recall from unit 2 that we can use the \texttt{reshape()} method to
convert arrays to a different shape:

\begin{itemize}
\tightlist
\item
  The resulting number of elements must remain unchanged!
\item
  \emph{One} dimension can be specified using \texttt{-1}, which will
  prompt NumPy to compute the implied dimension size itself.
\end{itemize}

    \begin{tcolorbox}[breakable, size=fbox, boxrule=1pt, pad at break*=1mm,colback=cellbackground, colframe=cellborder]
\prompt{In}{incolor}{14}{\boxspacing}
\begin{Verbatim}[commandchars=\\\{\}]
\PY{n}{x} \PY{o}{=} \PY{n}{np}\PY{o}{.}\PY{n}{zeros}\PY{p}{(}\PY{p}{(}\PY{l+m+mi}{2}\PY{p}{,} \PY{l+m+mi}{1}\PY{p}{,} \PY{l+m+mi}{3}\PY{p}{)}\PY{p}{)}
\PY{n}{x} \PY{o}{=} \PY{n}{x}\PY{o}{.}\PY{n}{reshape}\PY{p}{(}\PY{p}{(}\PY{l+m+mi}{3}\PY{p}{,} \PY{o}{\PYZhy{}}\PY{l+m+mi}{1}\PY{p}{)}\PY{p}{)}      \PY{c+c1}{\PYZsh{} Infer number of columns}
\PY{n}{x}
\end{Verbatim}
\end{tcolorbox}

            \begin{tcolorbox}[breakable, size=fbox, boxrule=.5pt, pad at break*=1mm, opacityfill=0]
\prompt{Out}{outcolor}{14}{\boxspacing}
\begin{Verbatim}[commandchars=\\\{\}]
array([[0., 0.],
       [0., 0.],
       [0., 0.]])
\end{Verbatim}
\end{tcolorbox}
        
    We can reshape any array to a 1-dimensional vector using any of the
following expressions:

    \begin{tcolorbox}[breakable, size=fbox, boxrule=1pt, pad at break*=1mm,colback=cellbackground, colframe=cellborder]
\prompt{In}{incolor}{15}{\boxspacing}
\begin{Verbatim}[commandchars=\\\{\}]
\PY{n}{x}\PY{o}{.}\PY{n}{reshape}\PY{p}{(}\PY{p}{(}\PY{o}{\PYZhy{}}\PY{l+m+mi}{1}\PY{p}{,} \PY{p}{)}\PY{p}{)}          \PY{c+c1}{\PYZsh{} pass shape as tuple}
\PY{n}{x}\PY{o}{.}\PY{n}{reshape}\PY{p}{(}\PY{o}{\PYZhy{}}\PY{l+m+mi}{1}\PY{p}{)}              \PY{c+c1}{\PYZsh{} pass shape as integer}
\PY{n}{x}\PY{o}{.}\PY{n}{flatten}\PY{p}{(}\PY{p}{)}
\end{Verbatim}
\end{tcolorbox}

            \begin{tcolorbox}[breakable, size=fbox, boxrule=.5pt, pad at break*=1mm, opacityfill=0]
\prompt{Out}{outcolor}{15}{\boxspacing}
\begin{Verbatim}[commandchars=\\\{\}]
array([0., 0., 0., 0., 0., 0.])
\end{Verbatim}
\end{tcolorbox}
        
    This even works on scalar (0-dimensional) arrays:

    \begin{tcolorbox}[breakable, size=fbox, boxrule=1pt, pad at break*=1mm,colback=cellbackground, colframe=cellborder]
\prompt{In}{incolor}{16}{\boxspacing}
\begin{Verbatim}[commandchars=\\\{\}]
\PY{n}{np}\PY{o}{.}\PY{n}{array}\PY{p}{(}\PY{l+m+mf}{0.0}\PY{p}{)}\PY{o}{.}\PY{n}{flatten}\PY{p}{(}\PY{p}{)}
\end{Verbatim}
\end{tcolorbox}

            \begin{tcolorbox}[breakable, size=fbox, boxrule=.5pt, pad at break*=1mm, opacityfill=0]
\prompt{Out}{outcolor}{16}{\boxspacing}
\begin{Verbatim}[commandchars=\\\{\}]
array([0.])
\end{Verbatim}
\end{tcolorbox}
        

\hypertarget{advanced-indexing}{%
\subsection{Advanced indexing}\label{advanced-indexing}}

We previously discussed single element indexing and slicing, which works
the same way for both Python \texttt{list} and \texttt{tuple} objects as
well as NumPy arrays.

NumPy additionally implements more sophisticated indexing mechanisms
which we cover now.

\begin{itemize}
\tightlist
\item
  You might also want to consult the NumPy indexing
  \href{https://numpy.org/doc/stable/user/basics.indexing.html}{tutorial}
  and the detailed indexing
  \href{https://numpy.org/doc/stable/reference/arrays.indexing.html}{reference}.
\end{itemize}

\hypertarget{boolean-or-mask-indexing}{%
\subsubsection{Boolean or ``mask''
indexing}\label{boolean-or-mask-indexing}}

We can pass logical arrays as indices:

\begin{itemize}
\tightlist
\item
  Logical (or boolean) arrays consist of elements that can only take on
  values \texttt{True} and \texttt{False}
\item
  We usually don't create logical arrays manually, but apply an
  operation that results in \texttt{True}/\texttt{False} values, such as
  a comparison.
\item
  The boolean index array usually has the \emph{same} shape as the
  indexed array.
\end{itemize}

\emph{Examples:}

    \begin{tcolorbox}[breakable, size=fbox, boxrule=1pt, pad at break*=1mm,colback=cellbackground, colframe=cellborder]
\prompt{In}{incolor}{17}{\boxspacing}
\begin{Verbatim}[commandchars=\\\{\}]
\PY{k+kn}{import} \PY{n+nn}{numpy} \PY{k}{as} \PY{n+nn}{np}

\PY{n}{vec} \PY{o}{=} \PY{n}{np}\PY{o}{.}\PY{n}{arange}\PY{p}{(}\PY{l+m+mi}{5}\PY{p}{)}
\PY{n}{mask} \PY{o}{=} \PY{p}{(}\PY{n}{vec} \PY{o}{\PYZgt{}} \PY{l+m+mi}{1}\PY{p}{)}        \PY{c+c1}{\PYZsh{} apply comparison to create boolean array}
\PY{n}{mask}
\end{Verbatim}
\end{tcolorbox}

            \begin{tcolorbox}[breakable, size=fbox, boxrule=.5pt, pad at break*=1mm, opacityfill=0]
\prompt{Out}{outcolor}{17}{\boxspacing}
\begin{Verbatim}[commandchars=\\\{\}]
array([False, False,  True,  True,  True])
\end{Verbatim}
\end{tcolorbox}
        
    \begin{tcolorbox}[breakable, size=fbox, boxrule=1pt, pad at break*=1mm,colback=cellbackground, colframe=cellborder]
\prompt{In}{incolor}{18}{\boxspacing}
\begin{Verbatim}[commandchars=\\\{\}]
\PY{n}{vec}\PY{p}{[}\PY{n}{mask}\PY{p}{]}               \PY{c+c1}{\PYZsh{} use mask to retrieve only elements greater than 1}
\end{Verbatim}
\end{tcolorbox}

            \begin{tcolorbox}[breakable, size=fbox, boxrule=.5pt, pad at break*=1mm, opacityfill=0]
\prompt{Out}{outcolor}{18}{\boxspacing}
\begin{Verbatim}[commandchars=\\\{\}]
array([2, 3, 4])
\end{Verbatim}
\end{tcolorbox}
        
    We can even apply boolean indexing to multi-dimensional arrays. The
result will be flatted to a 1-dimensional array, though.

    \begin{tcolorbox}[breakable, size=fbox, boxrule=1pt, pad at break*=1mm,colback=cellbackground, colframe=cellborder]
\prompt{In}{incolor}{19}{\boxspacing}
\begin{Verbatim}[commandchars=\\\{\}]
\PY{n}{mat} \PY{o}{=} \PY{n}{np}\PY{o}{.}\PY{n}{arange}\PY{p}{(}\PY{l+m+mi}{6}\PY{p}{)}\PY{o}{.}\PY{n}{reshape}\PY{p}{(}\PY{p}{(}\PY{l+m+mi}{2}\PY{p}{,}\PY{l+m+mi}{3}\PY{p}{)}\PY{p}{)}
\PY{n}{mat}
\end{Verbatim}
\end{tcolorbox}

            \begin{tcolorbox}[breakable, size=fbox, boxrule=.5pt, pad at break*=1mm, opacityfill=0]
\prompt{Out}{outcolor}{19}{\boxspacing}
\begin{Verbatim}[commandchars=\\\{\}]
array([[0, 1, 2],
       [3, 4, 5]])
\end{Verbatim}
\end{tcolorbox}
        
    \begin{tcolorbox}[breakable, size=fbox, boxrule=1pt, pad at break*=1mm,colback=cellbackground, colframe=cellborder]
\prompt{In}{incolor}{20}{\boxspacing}
\begin{Verbatim}[commandchars=\\\{\}]
\PY{n}{mask} \PY{o}{=} \PY{p}{(}\PY{n}{mat} \PY{o}{\PYZgt{}} \PY{l+m+mi}{1}\PY{p}{)}        \PY{c+c1}{\PYZsh{} create boolean array}
\PY{n}{mask}
\end{Verbatim}
\end{tcolorbox}

            \begin{tcolorbox}[breakable, size=fbox, boxrule=.5pt, pad at break*=1mm, opacityfill=0]
\prompt{Out}{outcolor}{20}{\boxspacing}
\begin{Verbatim}[commandchars=\\\{\}]
array([[False, False,  True],
       [ True,  True,  True]])
\end{Verbatim}
\end{tcolorbox}
        
    \begin{tcolorbox}[breakable, size=fbox, boxrule=1pt, pad at break*=1mm,colback=cellbackground, colframe=cellborder]
\prompt{In}{incolor}{21}{\boxspacing}
\begin{Verbatim}[commandchars=\\\{\}]
\PY{n}{mat}\PY{p}{[}\PY{n}{mask}\PY{p}{]}            \PY{c+c1}{\PYZsh{} collapses result to 1\PYZhy{}d array}
\end{Verbatim}
\end{tcolorbox}

            \begin{tcolorbox}[breakable, size=fbox, boxrule=.5pt, pad at break*=1mm, opacityfill=0]
\prompt{Out}{outcolor}{21}{\boxspacing}
\begin{Verbatim}[commandchars=\\\{\}]
array([2, 3, 4, 5])
\end{Verbatim}
\end{tcolorbox}
        
    Note that logical indexing does \emph{not} work with \texttt{tuple} and
\texttt{list}

    \begin{tcolorbox}[breakable, size=fbox, boxrule=1pt, pad at break*=1mm,colback=cellbackground, colframe=cellborder]
\prompt{In}{incolor}{22}{\boxspacing}
\begin{Verbatim}[commandchars=\\\{\}]
\PY{n}{tpl} \PY{o}{=} \PY{p}{(}\PY{l+m+mi}{1}\PY{p}{,} \PY{l+m+mi}{2}\PY{p}{,} \PY{l+m+mi}{3}\PY{p}{)}
\PY{n}{mask} \PY{o}{=} \PY{p}{(}\PY{k+kc}{True}\PY{p}{,} \PY{k+kc}{False}\PY{p}{,} \PY{k+kc}{True}\PY{p}{)}
\PY{n}{tpl}\PY{p}{[}\PY{n}{mask}\PY{p}{]}               \PY{c+c1}{\PYZsh{} error}
\end{Verbatim}
\end{tcolorbox}

    \begin{Verbatim}[commandchars=\\\{\}, frame=single, framerule=2mm, rulecolor=\color{outerrorbackground}]
\textcolor{ansi-red}{TypeError}\textcolor{ansi-red}{:} tuple indices must be integers or slices, not tuple

    \end{Verbatim}

    \hypertarget{integer-index-arrays}{%
\subsubsection{Integer index arrays}\label{integer-index-arrays}}

We can also use index arrays of \emph{integer} type to select specific
elements on each axis. These are straightforward to use for
1-dimensional arrays, but can get fairly complex with multiple
dimensions.

    \begin{tcolorbox}[breakable, size=fbox, boxrule=1pt, pad at break*=1mm,colback=cellbackground, colframe=cellborder]
\prompt{In}{incolor}{23}{\boxspacing}
\begin{Verbatim}[commandchars=\\\{\}]
\PY{k+kn}{import} \PY{n+nn}{numpy} \PY{k}{as} \PY{n+nn}{np}

\PY{n}{data} \PY{o}{=} \PY{n}{np}\PY{o}{.}\PY{n}{arange}\PY{p}{(}\PY{l+m+mi}{10}\PY{p}{)}
\PY{n}{index} \PY{o}{=} \PY{p}{[}\PY{l+m+mi}{1}\PY{p}{,} \PY{l+m+mi}{2}\PY{p}{,} \PY{l+m+mi}{9}\PY{p}{]}       \PY{c+c1}{\PYZsh{} select second, third and 10th element}
\PY{n}{data}\PY{p}{[}\PY{n}{index}\PY{p}{]}
\end{Verbatim}
\end{tcolorbox}

            \begin{tcolorbox}[breakable, size=fbox, boxrule=.5pt, pad at break*=1mm, opacityfill=0]
\prompt{Out}{outcolor}{23}{\boxspacing}
\begin{Verbatim}[commandchars=\\\{\}]
array([1, 2, 9])
\end{Verbatim}
\end{tcolorbox}
        
    As you see, the index array does not have to be a NumPy array, but can
also be a list (not a tuple, though!).

In general, if we are using an index array to select elements along an
axis of length \(n\), then

\begin{itemize}
\tightlist
\item
  the index must only contain integers between \(0\) and \(n-1\), or
  negative integers from \(-n\) to \(-1\) (which, as usual, count from
  the end of the axis).
\item
  the index can be of arbitrary length. We can therefore select the same
  element multiple times.
\end{itemize}

    \begin{tcolorbox}[breakable, size=fbox, boxrule=1pt, pad at break*=1mm,colback=cellbackground, colframe=cellborder]
\prompt{In}{incolor}{24}{\boxspacing}
\begin{Verbatim}[commandchars=\\\{\}]
\PY{n}{data} \PY{o}{=} \PY{n}{np}\PY{o}{.}\PY{n}{arange}\PY{p}{(}\PY{l+m+mi}{5}\PY{p}{,} \PY{l+m+mi}{10}\PY{p}{)}     \PY{c+c1}{\PYZsh{} array with 5 elements, [5,...,9]}
\PY{n}{data}
\end{Verbatim}
\end{tcolorbox}

            \begin{tcolorbox}[breakable, size=fbox, boxrule=.5pt, pad at break*=1mm, opacityfill=0]
\prompt{Out}{outcolor}{24}{\boxspacing}
\begin{Verbatim}[commandchars=\\\{\}]
array([5, 6, 7, 8, 9])
\end{Verbatim}
\end{tcolorbox}
        
    \begin{tcolorbox}[breakable, size=fbox, boxrule=1pt, pad at break*=1mm,colback=cellbackground, colframe=cellborder]
\prompt{In}{incolor}{25}{\boxspacing}
\begin{Verbatim}[commandchars=\\\{\}]
\PY{n}{index} \PY{o}{=} \PY{p}{[}\PY{l+m+mi}{0}\PY{p}{,} \PY{l+m+mi}{1}\PY{p}{,} \PY{l+m+mi}{1}\PY{p}{,} \PY{l+m+mi}{2}\PY{p}{,} \PY{l+m+mi}{2}\PY{p}{,} \PY{l+m+mi}{3}\PY{p}{,} \PY{l+m+mi}{3}\PY{p}{,} \PY{l+m+mi}{4}\PY{p}{,} \PY{l+m+mi}{4}\PY{p}{]}         \PY{c+c1}{\PYZsh{} select elements multiple times}
\PY{n}{data}\PY{p}{[}\PY{n}{index}\PY{p}{]}
\end{Verbatim}
\end{tcolorbox}

            \begin{tcolorbox}[breakable, size=fbox, boxrule=.5pt, pad at break*=1mm, opacityfill=0]
\prompt{Out}{outcolor}{25}{\boxspacing}
\begin{Verbatim}[commandchars=\\\{\}]
array([5, 6, 6, 7, 7, 8, 8, 9, 9])
\end{Verbatim}
\end{tcolorbox}
        
    The same restrictions apply when indexing multi-dimensional arrays.
Moreover,

\begin{itemize}
\tightlist
\item
  if more than one axis is indexed using index arrays, the index arrays
  have to be of equal length.
\item
  we can combine integer array indexing on one axis with other types of
  indices on the remaining axes.
\end{itemize}

    \begin{tcolorbox}[breakable, size=fbox, boxrule=1pt, pad at break*=1mm,colback=cellbackground, colframe=cellborder]
\prompt{In}{incolor}{26}{\boxspacing}
\begin{Verbatim}[commandchars=\\\{\}]
\PY{n}{data} \PY{o}{=} \PY{n}{np}\PY{o}{.}\PY{n}{arange}\PY{p}{(}\PY{l+m+mi}{12}\PY{p}{)}\PY{o}{.}\PY{n}{reshape}\PY{p}{(}\PY{p}{(}\PY{l+m+mi}{3}\PY{p}{,} \PY{l+m+mi}{4}\PY{p}{)}\PY{p}{)}
\PY{n}{data}
\end{Verbatim}
\end{tcolorbox}

            \begin{tcolorbox}[breakable, size=fbox, boxrule=.5pt, pad at break*=1mm, opacityfill=0]
\prompt{Out}{outcolor}{26}{\boxspacing}
\begin{Verbatim}[commandchars=\\\{\}]
array([[ 0,  1,  2,  3],
       [ 4,  5,  6,  7],
       [ 8,  9, 10, 11]])
\end{Verbatim}
\end{tcolorbox}
        
    \begin{tcolorbox}[breakable, size=fbox, boxrule=1pt, pad at break*=1mm,colback=cellbackground, colframe=cellborder]
\prompt{In}{incolor}{27}{\boxspacing}
\begin{Verbatim}[commandchars=\\\{\}]
\PY{n}{index1} \PY{o}{=} \PY{p}{[}\PY{l+m+mi}{0}\PY{p}{,} \PY{l+m+mi}{2}\PY{p}{]}     \PY{c+c1}{\PYZsh{} row indices}
\PY{n}{index2} \PY{o}{=} \PY{p}{[}\PY{l+m+mi}{1}\PY{p}{,} \PY{l+m+mi}{3}\PY{p}{]}     \PY{c+c1}{\PYZsh{} column indices}
\PY{n}{data}\PY{p}{[}\PY{n}{index1}\PY{p}{,} \PY{n}{index2}\PY{p}{]}
\end{Verbatim}
\end{tcolorbox}

            \begin{tcolorbox}[breakable, size=fbox, boxrule=.5pt, pad at break*=1mm, opacityfill=0]
\prompt{Out}{outcolor}{27}{\boxspacing}
\begin{Verbatim}[commandchars=\\\{\}]
array([ 1, 11])
\end{Verbatim}
\end{tcolorbox}
        
    The code above selects two elements, the first at position
\texttt{(0,1)}, the second at position \texttt{(2,3)}.

We can combine index arrays on one axis with another indexing method on
a different axis:

    \begin{tcolorbox}[breakable, size=fbox, boxrule=1pt, pad at break*=1mm,colback=cellbackground, colframe=cellborder]
\prompt{In}{incolor}{28}{\boxspacing}
\begin{Verbatim}[commandchars=\\\{\}]
\PY{n}{data}\PY{p}{[}\PY{n}{index1}\PY{p}{,} \PY{l+m+mi}{2}\PY{p}{]}     \PY{c+c1}{\PYZsh{} return elements in 3rd column from rows given}
                    \PY{c+c1}{\PYZsh{} in index1}
\end{Verbatim}
\end{tcolorbox}

            \begin{tcolorbox}[breakable, size=fbox, boxrule=.5pt, pad at break*=1mm, opacityfill=0]
\prompt{Out}{outcolor}{28}{\boxspacing}
\begin{Verbatim}[commandchars=\\\{\}]
array([ 2, 10])
\end{Verbatim}
\end{tcolorbox}
        
    Using different indexing methods, in particular index arrays, on
higher-dimensional data can quickly become a mess, and you should be
extra careful to see if the results make sense.


\hypertarget{numerical-operations}{%
\subsection{Numerical operations}\label{numerical-operations}}

\hypertarget{element-wise-operations}{%
\subsubsection{Element-wise operations}\label{element-wise-operations}}

Element-wise operations are performed on each element individually and
leave the resulting array's shape unchanged.

There are three types of such operations:

\begin{enumerate}
\def\labelenumi{\arabic{enumi}.}
\tightlist
\item
  One operand is an array and one is a scalar.
\item
  Both operands are arrays, either of identical shape, or broadcastable
  to an identical shape (we discuss broadcasting below)
\item
  A function is applied to each array element.
\end{enumerate}

\emph{Case 1:} Array-scalar operations. These intuitively behave as you
would expect:

    \begin{tcolorbox}[breakable, size=fbox, boxrule=1pt, pad at break*=1mm,colback=cellbackground, colframe=cellborder]
\prompt{In}{incolor}{29}{\boxspacing}
\begin{Verbatim}[commandchars=\\\{\}]
\PY{k+kn}{import} \PY{n+nn}{numpy} \PY{k}{as} \PY{n+nn}{np}

\PY{n}{x} \PY{o}{=} \PY{n}{np}\PY{o}{.}\PY{n}{arange}\PY{p}{(}\PY{l+m+mi}{10}\PY{p}{)}
\PY{n}{scalar} \PY{o}{=} \PY{l+m+mi}{1}

\PY{c+c1}{\PYZsh{} The resulting array y has the same shape as x:}
\PY{n}{y} \PY{o}{=} \PY{n}{x} \PY{o}{+} \PY{n}{scalar}      \PY{c+c1}{\PYZsh{} addition}
\PY{n}{y} \PY{o}{=} \PY{n}{x} \PY{o}{\PYZhy{}} \PY{n}{scalar}      \PY{c+c1}{\PYZsh{} subtraction}
\PY{n}{y} \PY{o}{=} \PY{n}{x} \PY{o}{*} \PY{n}{scalar}      \PY{c+c1}{\PYZsh{} multiplication}
\PY{n}{y} \PY{o}{=} \PY{n}{x} \PY{o}{/} \PY{n}{scalar}      \PY{c+c1}{\PYZsh{} division}
\PY{n}{y} \PY{o}{=} \PY{n}{x} \PY{o}{/}\PY{o}{/} \PY{n}{scalar}     \PY{c+c1}{\PYZsh{} division with integer truncation}
\PY{n}{y} \PY{o}{=} \PY{n}{x} \PY{o}{\PYZpc{}} \PY{n}{scalar}      \PY{c+c1}{\PYZsh{} modulo operator}
\PY{n}{y} \PY{o}{=} \PY{n}{x} \PY{o}{*}\PY{o}{*} \PY{n}{scalar}     \PY{c+c1}{\PYZsh{} power function}
\PY{n}{y} \PY{o}{=} \PY{n}{x} \PY{o}{==} \PY{n}{scalar}     \PY{c+c1}{\PYZsh{} comparison: also \PYZgt{}, \PYZgt{}=, \PYZlt{}=, \PYZlt{}}
\end{Verbatim}
\end{tcolorbox}

    Note that unlike in Matlab, the ``standard'' operators work
element-wise, so \texttt{x\ *\ y} is \emph{not} matrix multiplication!

\emph{Case 2:} Both operands are arrays of equal shape:

    \begin{tcolorbox}[breakable, size=fbox, boxrule=1pt, pad at break*=1mm,colback=cellbackground, colframe=cellborder]
\prompt{In}{incolor}{30}{\boxspacing}
\begin{Verbatim}[commandchars=\\\{\}]
\PY{n}{x} \PY{o}{=} \PY{n}{np}\PY{o}{.}\PY{n}{arange}\PY{p}{(}\PY{l+m+mi}{10}\PY{p}{)}
\PY{n}{y} \PY{o}{=} \PY{n}{np}\PY{o}{.}\PY{n}{arange}\PY{p}{(}\PY{l+m+mi}{10}\PY{p}{,} \PY{l+m+mi}{20}\PY{p}{)}       \PY{c+c1}{\PYZsh{} has same shape as x}

\PY{c+c1}{\PYZsh{} Resulting array z has the same shape as x and y:}
\PY{n}{z} \PY{o}{=} \PY{n}{x} \PY{o}{+} \PY{n}{y}           \PY{c+c1}{\PYZsh{} addition}
\PY{n}{z} \PY{o}{=} \PY{n}{x} \PY{o}{\PYZhy{}} \PY{n}{y}           \PY{c+c1}{\PYZsh{} subtraction}
\PY{n}{z} \PY{o}{=} \PY{n}{x} \PY{o}{*} \PY{n}{y}           \PY{c+c1}{\PYZsh{} multiplication}
\PY{n}{z} \PY{o}{=} \PY{n}{x} \PY{o}{/} \PY{n}{y}           \PY{c+c1}{\PYZsh{} division}
\PY{n}{z} \PY{o}{=} \PY{n}{x} \PY{o}{/}\PY{o}{/} \PY{n}{y}          \PY{c+c1}{\PYZsh{} division with integer truncation}
\PY{n}{z} \PY{o}{=} \PY{n}{x} \PY{o}{\PYZpc{}} \PY{n}{y}           \PY{c+c1}{\PYZsh{} modulo operator}
\PY{n}{z} \PY{o}{=} \PY{n}{x} \PY{o}{*}\PY{o}{*} \PY{n}{y}          \PY{c+c1}{\PYZsh{} power function}
\PY{n}{z} \PY{o}{=} \PY{n}{x} \PY{o}{==} \PY{n}{y}          \PY{c+c1}{\PYZsh{} comparison: also \PYZgt{}, \PYZgt{}=, \PYZlt{}=, \PYZlt{}}
\end{Verbatim}
\end{tcolorbox}

    \emph{Case 3:} Applying element-wise functions. This case covers
numerous functions defined in NumPy, such as

\begin{itemize}
\tightlist
\item
  \texttt{np.sqrt}: square root
\item
  \texttt{np.exp}, \texttt{np.log}, \texttt{np.log10}: exponential and
  logarithmic functions
\item
  \texttt{np.sin}, \texttt{np.cos}, etc.: trigonometric functions
\end{itemize}

    You can find a complete list of mathematical functions in the NumPy
\href{https://numpy.org/doc/stable/reference/routines.math.html}{documentation}
(not all functions listed there operate element-wise, though!).

    \begin{tcolorbox}[breakable, size=fbox, boxrule=1pt, pad at break*=1mm,colback=cellbackground, colframe=cellborder]
\prompt{In}{incolor}{31}{\boxspacing}
\begin{Verbatim}[commandchars=\\\{\}]
\PY{c+c1}{\PYZsh{} element\PYZhy{}wise functions}
\PY{n}{x} \PY{o}{=} \PY{n}{np}\PY{o}{.}\PY{n}{arange}\PY{p}{(}\PY{l+m+mi}{1}\PY{p}{,} \PY{l+m+mi}{11}\PY{p}{)}
\PY{n}{y} \PY{o}{=} \PY{n}{np}\PY{o}{.}\PY{n}{exp}\PY{p}{(}\PY{n}{x}\PY{p}{)}       \PY{c+c1}{\PYZsh{} apply exponential function}
\PY{n}{y} \PY{o}{=} \PY{n}{np}\PY{o}{.}\PY{n}{log}\PY{p}{(}\PY{n}{x}\PY{p}{)}       \PY{c+c1}{\PYZsh{} apply natural logarithm}
\end{Verbatim}
\end{tcolorbox}

    \hypertarget{matrix-operations}{%
\subsubsection{Matrix operations}\label{matrix-operations}}

\textbf{Transpose}

You can transpose a matrix using the \texttt{T} attribute:

    \begin{tcolorbox}[breakable, size=fbox, boxrule=1pt, pad at break*=1mm,colback=cellbackground, colframe=cellborder]
\prompt{In}{incolor}{32}{\boxspacing}
\begin{Verbatim}[commandchars=\\\{\}]
\PY{n}{mat} \PY{o}{=} \PY{n}{np}\PY{o}{.}\PY{n}{arange}\PY{p}{(}\PY{l+m+mi}{6}\PY{p}{)}\PY{o}{.}\PY{n}{reshape}\PY{p}{(}\PY{p}{(}\PY{l+m+mi}{2}\PY{p}{,}\PY{l+m+mi}{3}\PY{p}{)}\PY{p}{)}
\PY{n}{mat}
\end{Verbatim}
\end{tcolorbox}

            \begin{tcolorbox}[breakable, size=fbox, boxrule=.5pt, pad at break*=1mm, opacityfill=0]
\prompt{Out}{outcolor}{32}{\boxspacing}
\begin{Verbatim}[commandchars=\\\{\}]
array([[0, 1, 2],
       [3, 4, 5]])
\end{Verbatim}
\end{tcolorbox}
        
    \begin{tcolorbox}[breakable, size=fbox, boxrule=1pt, pad at break*=1mm,colback=cellbackground, colframe=cellborder]
\prompt{In}{incolor}{33}{\boxspacing}
\begin{Verbatim}[commandchars=\\\{\}]
\PY{n}{mat}\PY{o}{.}\PY{n}{T}
\end{Verbatim}
\end{tcolorbox}

            \begin{tcolorbox}[breakable, size=fbox, boxrule=.5pt, pad at break*=1mm, opacityfill=0]
\prompt{Out}{outcolor}{33}{\boxspacing}
\begin{Verbatim}[commandchars=\\\{\}]
array([[0, 3],
       [1, 4],
       [2, 5]])
\end{Verbatim}
\end{tcolorbox}
        
    For higher-dimensional arrays, the \texttt{np.transpose()} function can
be used to permute the axes of an array. For two-dimensional arrays,
\texttt{np.transpose(mat)} and \texttt{mat.T} are equivalent.

\textbf{Array multiplication}

Matrix multiplication is performed using the \texttt{np.dot()} function
(``dot product''). The operands need not be matrices but can be vectors
as well, or even high-dimensional arrays (the result is then not
entirely obvious and one should check the
\href{https://numpy.org/doc/stable/reference/generated/numpy.dot.html}{documentation}).

Every newer version of Python and NumPy additionally interprets
\texttt{@} as the matrix multiplication operator.

    \begin{tcolorbox}[breakable, size=fbox, boxrule=1pt, pad at break*=1mm,colback=cellbackground, colframe=cellborder]
\prompt{In}{incolor}{34}{\boxspacing}
\begin{Verbatim}[commandchars=\\\{\}]
\PY{k+kn}{import} \PY{n+nn}{numpy} \PY{k}{as} \PY{n+nn}{np}

\PY{n}{mat} \PY{o}{=} \PY{n}{np}\PY{o}{.}\PY{n}{arange}\PY{p}{(}\PY{l+m+mi}{9}\PY{p}{)}\PY{o}{.}\PY{n}{reshape}\PY{p}{(}\PY{p}{(}\PY{l+m+mi}{3}\PY{p}{,} \PY{l+m+mi}{3}\PY{p}{)}\PY{p}{)}      \PY{c+c1}{\PYZsh{} 3x3 matrix}
\PY{n}{vec} \PY{o}{=} \PY{n}{np}\PY{o}{.}\PY{n}{arange}\PY{p}{(}\PY{l+m+mi}{3}\PY{p}{)}                      \PY{c+c1}{\PYZsh{} vector of length 3}

\PY{c+c1}{\PYZsh{} matrix\PYZhy{}matrix multiplication}
\PY{n}{np}\PY{o}{.}\PY{n}{dot}\PY{p}{(}\PY{n}{mat}\PY{p}{,} \PY{n}{mat}\PY{p}{)}    \PY{c+c1}{\PYZsh{} or: mat @ mat}
\end{Verbatim}
\end{tcolorbox}

            \begin{tcolorbox}[breakable, size=fbox, boxrule=.5pt, pad at break*=1mm, opacityfill=0]
\prompt{Out}{outcolor}{34}{\boxspacing}
\begin{Verbatim}[commandchars=\\\{\}]
array([[ 15,  18,  21],
       [ 42,  54,  66],
       [ 69,  90, 111]])
\end{Verbatim}
\end{tcolorbox}
        
    \begin{tcolorbox}[breakable, size=fbox, boxrule=1pt, pad at break*=1mm,colback=cellbackground, colframe=cellborder]
\prompt{In}{incolor}{35}{\boxspacing}
\begin{Verbatim}[commandchars=\\\{\}]
\PY{c+c1}{\PYZsh{} vector dot product (returns a scalar)}
\PY{n}{np}\PY{o}{.}\PY{n}{dot}\PY{p}{(}\PY{n}{vec}\PY{p}{,} \PY{n}{vec}\PY{p}{)}    \PY{c+c1}{\PYZsh{} or: vec @ vec}
\end{Verbatim}
\end{tcolorbox}

            \begin{tcolorbox}[breakable, size=fbox, boxrule=.5pt, pad at break*=1mm, opacityfill=0]
\prompt{Out}{outcolor}{35}{\boxspacing}
\begin{Verbatim}[commandchars=\\\{\}]
5
\end{Verbatim}
\end{tcolorbox}
        
    \begin{tcolorbox}[breakable, size=fbox, boxrule=1pt, pad at break*=1mm,colback=cellbackground, colframe=cellborder]
\prompt{In}{incolor}{36}{\boxspacing}
\begin{Verbatim}[commandchars=\\\{\}]
\PY{c+c1}{\PYZsh{} matrix\PYZhy{}vector product (returns vector)}
\PY{n}{np}\PY{o}{.}\PY{n}{dot}\PY{p}{(}\PY{n}{mat}\PY{p}{,} \PY{n}{vec}\PY{p}{)}    \PY{c+c1}{\PYZsh{} or: mat @ vec}
\end{Verbatim}
\end{tcolorbox}

            \begin{tcolorbox}[breakable, size=fbox, boxrule=.5pt, pad at break*=1mm, opacityfill=0]
\prompt{Out}{outcolor}{36}{\boxspacing}
\begin{Verbatim}[commandchars=\\\{\}]
array([ 5, 14, 23])
\end{Verbatim}
\end{tcolorbox}
        
    We must of course make sure that matrices and vector have conformable
dimensions!

    \begin{tcolorbox}[breakable, size=fbox, boxrule=1pt, pad at break*=1mm,colback=cellbackground, colframe=cellborder]
\prompt{In}{incolor}{37}{\boxspacing}
\begin{Verbatim}[commandchars=\\\{\}]
\PY{n}{mat} \PY{o}{=} \PY{n}{np}\PY{o}{.}\PY{n}{arange}\PY{p}{(}\PY{l+m+mi}{6}\PY{p}{)}\PY{o}{.}\PY{n}{reshape}\PY{p}{(}\PY{p}{(}\PY{l+m+mi}{2}\PY{p}{,} \PY{l+m+mi}{3}\PY{p}{)}\PY{p}{)}
\PY{n}{mat}
\end{Verbatim}
\end{tcolorbox}

            \begin{tcolorbox}[breakable, size=fbox, boxrule=.5pt, pad at break*=1mm, opacityfill=0]
\prompt{Out}{outcolor}{37}{\boxspacing}
\begin{Verbatim}[commandchars=\\\{\}]
array([[0, 1, 2],
       [3, 4, 5]])
\end{Verbatim}
\end{tcolorbox}
        
    \begin{tcolorbox}[breakable, size=fbox, boxrule=1pt, pad at break*=1mm,colback=cellbackground, colframe=cellborder]
\prompt{In}{incolor}{38}{\boxspacing}
\begin{Verbatim}[commandchars=\\\{\}]
\PY{n}{np}\PY{o}{.}\PY{n}{dot}\PY{p}{(}\PY{n}{mat}\PY{p}{,} \PY{n}{mat}\PY{p}{)}        \PY{c+c1}{\PYZsh{} raises error, cannot multiply 2x3 matrix with }
                        \PY{c+c1}{\PYZsh{} 2x3 matrix}
\end{Verbatim}
\end{tcolorbox}

    \begin{Verbatim}[commandchars=\\\{\}, frame=single, framerule=2mm, rulecolor=\color{outerrorbackground}]
\textcolor{ansi-red}{ValueError}\textcolor{ansi-red}{:} shapes (2,3) and (2,3) not aligned: 3 (dim 1) != 2 (dim 0)

    \end{Verbatim}

    \begin{tcolorbox}[breakable, size=fbox, boxrule=1pt, pad at break*=1mm,colback=cellbackground, colframe=cellborder]
\prompt{In}{incolor}{39}{\boxspacing}
\begin{Verbatim}[commandchars=\\\{\}]
\PY{n}{np}\PY{o}{.}\PY{n}{dot}\PY{p}{(}\PY{n}{mat}\PY{p}{,} \PY{n}{mat}\PY{o}{.}\PY{n}{T}\PY{p}{)}      \PY{c+c1}{\PYZsh{} transpose second operand, works}
\end{Verbatim}
\end{tcolorbox}

            \begin{tcolorbox}[breakable, size=fbox, boxrule=.5pt, pad at break*=1mm, opacityfill=0]
\prompt{Out}{outcolor}{39}{\boxspacing}
\begin{Verbatim}[commandchars=\\\{\}]
array([[ 5, 14],
       [14, 50]])
\end{Verbatim}
\end{tcolorbox}
        
    \hypertarget{reductions}{%
\subsubsection{Reductions}\label{reductions}}

Reductions are operations that reduce the dimensionality of the data.
For example, computing the mean of an array reduces a collection of data
points to a single scalar, its mean.

Basic reduction operations include:

\begin{itemize}
\tightlist
\item
  \texttt{np.sum()}: sum of array elements
\item
  \texttt{np.prod()}: product of array elements
\item
  \texttt{np.amin()}, \texttt{np.amax()}: minimum and maximum element
\item
  \texttt{np.argmin()}, \texttt{np.argmax()}: location of minimum and
  maximum element
\item
  \texttt{np.mean()}, \texttt{np.average()}: mean of array elements
\item
  \texttt{np.median()}: median of array elements
\item
  \texttt{np.std()}, \texttt{np.var()}: standard deviation and variance
  of array elements
\item
  \texttt{np.percentile()}: percentiles of array elements
\end{itemize}

Most if not all reductions accept an \texttt{axis} argument which
restricts the operation to a specific axis.

\begin{itemize}
\tightlist
\item
  If an axis is specified, the resulting array will have one dimension
  less than the input.
\item
  If no axis is specified, the operation is applied to the whole
  (flattened) array.
\end{itemize}

\emph{Examples:}

    \begin{tcolorbox}[breakable, size=fbox, boxrule=1pt, pad at break*=1mm,colback=cellbackground, colframe=cellborder]
\prompt{In}{incolor}{40}{\boxspacing}
\begin{Verbatim}[commandchars=\\\{\}]
\PY{k+kn}{import} \PY{n+nn}{numpy} \PY{k}{as} \PY{n+nn}{np}

\PY{c+c1}{\PYZsh{} 1\PYZhy{}dimensional input data}
\PY{n}{data} \PY{o}{=} \PY{n}{np}\PY{o}{.}\PY{n}{linspace}\PY{p}{(}\PY{l+m+mf}{0.0}\PY{p}{,} \PY{l+m+mf}{1.0}\PY{p}{,} \PY{l+m+mi}{11}\PY{p}{)}

\PY{c+c1}{\PYZsh{} Compute mean and std. of input data}
\PY{n}{m} \PY{o}{=} \PY{n}{np}\PY{o}{.}\PY{n}{mean}\PY{p}{(}\PY{n}{data}\PY{p}{)}
\PY{n}{s} \PY{o}{=} \PY{n}{np}\PY{o}{.}\PY{n}{std}\PY{p}{(}\PY{n}{data}\PY{p}{)}
\PY{n+nb}{print}\PY{p}{(}\PY{l+s+s1}{\PYZsq{}}\PY{l+s+s1}{Mean: }\PY{l+s+si}{\PYZob{}:.2f\PYZcb{}}\PY{l+s+s1}{, std. dev.: }\PY{l+s+si}{\PYZob{}:.2f\PYZcb{}}\PY{l+s+s1}{\PYZsq{}}\PY{o}{.}\PY{n}{format}\PY{p}{(}\PY{n}{m}\PY{p}{,} \PY{n}{s}\PY{p}{)}\PY{p}{)}
\end{Verbatim}
\end{tcolorbox}

    \begin{Verbatim}[commandchars=\\\{\}]
Mean: 0.50, std. dev.: 0.32
    \end{Verbatim}

    \begin{tcolorbox}[breakable, size=fbox, boxrule=1pt, pad at break*=1mm,colback=cellbackground, colframe=cellborder]
\prompt{In}{incolor}{41}{\boxspacing}
\begin{Verbatim}[commandchars=\\\{\}]
\PY{c+c1}{\PYZsh{} 2\PYZhy{}dimensional input data}
\PY{n}{data} \PY{o}{=} \PY{n}{np}\PY{o}{.}\PY{n}{linspace}\PY{p}{(}\PY{l+m+mf}{0.0}\PY{p}{,} \PY{l+m+mf}{1.0}\PY{p}{,} \PY{l+m+mi}{21}\PY{p}{)}\PY{o}{.}\PY{n}{reshape}\PY{p}{(}\PY{p}{(}\PY{l+m+mi}{3}\PY{p}{,} \PY{l+m+mi}{7}\PY{p}{)}\PY{p}{)}
\PY{n}{data}
\end{Verbatim}
\end{tcolorbox}

            \begin{tcolorbox}[breakable, size=fbox, boxrule=.5pt, pad at break*=1mm, opacityfill=0]
\prompt{Out}{outcolor}{41}{\boxspacing}
\begin{Verbatim}[commandchars=\\\{\}]
array([[0.  , 0.05, 0.1 , 0.15, 0.2 , 0.25, 0.3 ],
       [0.35, 0.4 , 0.45, 0.5 , 0.55, 0.6 , 0.65],
       [0.7 , 0.75, 0.8 , 0.85, 0.9 , 0.95, 1.  ]])
\end{Verbatim}
\end{tcolorbox}
        
    \begin{tcolorbox}[breakable, size=fbox, boxrule=1pt, pad at break*=1mm,colback=cellbackground, colframe=cellborder]
\prompt{In}{incolor}{42}{\boxspacing}
\begin{Verbatim}[commandchars=\\\{\}]
\PY{c+c1}{\PYZsh{} Compute mean of each row, ie along the column axis}
\PY{n}{m} \PY{o}{=} \PY{n}{np}\PY{o}{.}\PY{n}{mean}\PY{p}{(}\PY{n}{data}\PY{p}{,} \PY{n}{axis}\PY{o}{=}\PY{l+m+mi}{1}\PY{p}{)}
\PY{n}{m}           \PY{c+c1}{\PYZsh{} Result is a vector of 3 elements, one for each row}
\end{Verbatim}
\end{tcolorbox}

            \begin{tcolorbox}[breakable, size=fbox, boxrule=.5pt, pad at break*=1mm, opacityfill=0]
\prompt{Out}{outcolor}{42}{\boxspacing}
\begin{Verbatim}[commandchars=\\\{\}]
array([0.15, 0.5 , 0.85])
\end{Verbatim}
\end{tcolorbox}
        
    \hypertarget{broadcasting}{%
\subsubsection{Broadcasting}\label{broadcasting}}

Element-wise operations in most programming languages require input
arrays to have identical shapes. NumPy relaxes this constraint and
allows us to use arrays with different shapes that can be ``broadcast''
to identical shapes.

\textbf{Simple example}

What do we mean by ``broadcasting''? We introduce the concept using a
specific example, and will discuss the technical details below.

\begin{itemize}
\tightlist
\item
  Imagine we want to add a \(2 \times 3\) matrix to a length-2 vector.
\item
  This operation does not make sense, unless we interpret the (column)
  vector as a \(2 \times 1\) matrix, and replicate it 3 times to obtain
  a \(2 \times 3\) matrix. This is exactly what NumPy does.
\end{itemize}

\emph{Examples:}

    \begin{tcolorbox}[breakable, size=fbox, boxrule=1pt, pad at break*=1mm,colback=cellbackground, colframe=cellborder]
\prompt{In}{incolor}{43}{\boxspacing}
\begin{Verbatim}[commandchars=\\\{\}]
\PY{k+kn}{import} \PY{n+nn}{numpy} \PY{k}{as} \PY{n+nn}{np}

\PY{c+c1}{\PYZsh{} Create 3x2 matrix}
\PY{n}{mat} \PY{o}{=} \PY{n}{np}\PY{o}{.}\PY{n}{arange}\PY{p}{(}\PY{l+m+mi}{6}\PY{p}{)}\PY{o}{.}\PY{n}{reshape}\PY{p}{(}\PY{p}{(}\PY{l+m+mi}{2}\PY{p}{,} \PY{l+m+mi}{3}\PY{p}{)}\PY{p}{)}
\PY{n}{mat}
\end{Verbatim}
\end{tcolorbox}

            \begin{tcolorbox}[breakable, size=fbox, boxrule=.5pt, pad at break*=1mm, opacityfill=0]
\prompt{Out}{outcolor}{43}{\boxspacing}
\begin{Verbatim}[commandchars=\\\{\}]
array([[0, 1, 2],
       [3, 4, 5]])
\end{Verbatim}
\end{tcolorbox}
        
    \begin{tcolorbox}[breakable, size=fbox, boxrule=1pt, pad at break*=1mm,colback=cellbackground, colframe=cellborder]
\prompt{In}{incolor}{44}{\boxspacing}
\begin{Verbatim}[commandchars=\\\{\}]
\PY{c+c1}{\PYZsh{} Create 2\PYZhy{}element vector}
\PY{n}{vec} \PY{o}{=} \PY{n}{np}\PY{o}{.}\PY{n}{arange}\PY{p}{(}\PY{l+m+mi}{2}\PY{p}{)}
\PY{n}{vec}
\end{Verbatim}
\end{tcolorbox}

            \begin{tcolorbox}[breakable, size=fbox, boxrule=.5pt, pad at break*=1mm, opacityfill=0]
\prompt{Out}{outcolor}{44}{\boxspacing}
\begin{Verbatim}[commandchars=\\\{\}]
array([0, 1])
\end{Verbatim}
\end{tcolorbox}
        
    \begin{tcolorbox}[breakable, size=fbox, boxrule=1pt, pad at break*=1mm,colback=cellbackground, colframe=cellborder]
\prompt{In}{incolor}{45}{\boxspacing}
\begin{Verbatim}[commandchars=\\\{\}]
\PY{c+c1}{\PYZsh{} Trying to add matrix to vector fails}
\PY{n}{mat} \PY{o}{+} \PY{n}{vec}
\end{Verbatim}
\end{tcolorbox}

    \begin{Verbatim}[commandchars=\\\{\}, frame=single, framerule=2mm, rulecolor=\color{outerrorbackground}]
\textcolor{ansi-red}{ValueError}\textcolor{ansi-red}{:} operands could not be broadcast together with shapes (2,3) (2,) 

    \end{Verbatim}

    \begin{tcolorbox}[breakable, size=fbox, boxrule=1pt, pad at break*=1mm,colback=cellbackground, colframe=cellborder]
\prompt{In}{incolor}{46}{\boxspacing}
\begin{Verbatim}[commandchars=\\\{\}]
\PY{c+c1}{\PYZsh{} However, we can explicitly reshape the vector to a 2x1 column vector}
\PY{n}{colvec} \PY{o}{=} \PY{n}{vec}\PY{o}{.}\PY{n}{reshape}\PY{p}{(}\PY{p}{(}\PY{o}{\PYZhy{}}\PY{l+m+mi}{1}\PY{p}{,} \PY{l+m+mi}{1}\PY{p}{)}\PY{p}{)}
\PY{n}{colvec}
\end{Verbatim}
\end{tcolorbox}

            \begin{tcolorbox}[breakable, size=fbox, boxrule=.5pt, pad at break*=1mm, opacityfill=0]
\prompt{Out}{outcolor}{46}{\boxspacing}
\begin{Verbatim}[commandchars=\\\{\}]
array([[0],
       [1]])
\end{Verbatim}
\end{tcolorbox}
        
    \begin{tcolorbox}[breakable, size=fbox, boxrule=1pt, pad at break*=1mm,colback=cellbackground, colframe=cellborder]
\prompt{In}{incolor}{47}{\boxspacing}
\begin{Verbatim}[commandchars=\\\{\}]
\PY{c+c1}{\PYZsh{} Now, broadcasting replicates column vector to match matrix columns}
\PY{n}{mat} \PY{o}{+} \PY{n}{colvec}
\end{Verbatim}
\end{tcolorbox}

            \begin{tcolorbox}[breakable, size=fbox, boxrule=.5pt, pad at break*=1mm, opacityfill=0]
\prompt{Out}{outcolor}{47}{\boxspacing}
\begin{Verbatim}[commandchars=\\\{\}]
array([[0, 1, 2],
       [4, 5, 6]])
\end{Verbatim}
\end{tcolorbox}
        
    We do not need to \texttt{reshape()} data, but can instead use a feature
of NumPy that allows us to increase the number of dimensions on the
spot:

    \begin{tcolorbox}[breakable, size=fbox, boxrule=1pt, pad at break*=1mm,colback=cellbackground, colframe=cellborder]
\prompt{In}{incolor}{48}{\boxspacing}
\begin{Verbatim}[commandchars=\\\{\}]
\PY{c+c1}{\PYZsh{} use vec[:, None] to append an additional dimension to vec}
\PY{n}{mat} \PY{o}{+} \PY{n}{vec}\PY{p}{[}\PY{p}{:}\PY{p}{,} \PY{k+kc}{None}\PY{p}{]}
\end{Verbatim}
\end{tcolorbox}

            \begin{tcolorbox}[breakable, size=fbox, boxrule=.5pt, pad at break*=1mm, opacityfill=0]
\prompt{Out}{outcolor}{48}{\boxspacing}
\begin{Verbatim}[commandchars=\\\{\}]
array([[0, 1, 2],
       [4, 5, 6]])
\end{Verbatim}
\end{tcolorbox}
        
    For more examples, see the official NumPy
\href{https://numpy.org/doc/stable/user/theory.broadcasting.html}{tutorial}
on broadcasting.

\textbf{Technical details}

We are now ready to look at the technical details underlying
broadcasting. The NumPy
\href{https://numpy.org/doc/stable/reference/ufuncs.html\#broadcasting}{documentation}
is quite helpful, so we will just summarise the points made there.

Broadcasting is applied in four steps:

\begin{enumerate}
\def\labelenumi{\arabic{enumi}.}
\item
  Determine the largest dimension (\texttt{ndim} attribute) among all
  arrays involved in an operation. Any array of smaller dimension will
  have 1's \emph{prepended} to its shape until its dimension corresponds
  to the largest one.

  \emph{Example:} given array \texttt{a} with shape \texttt{(m,n)} and
  array \texttt{b} with shape \texttt{(n,)}, the maximum dimension is 2,
  and \texttt{b} will be implicitly reshaped to \texttt{(1,n)}.
\item
  The size of the output array is determined as the maximum size of all
  arrays along each dimension.

  \emph{Example:} Continuing with our example from above, the maximum
  size along dimension 1 is \texttt{m}, and the maximum size along
  dimension 2 is \texttt{n}, so the output array has shape
  \texttt{(m,n)}.
\item
  An input array can be used in the computation if for every dimension
  its size either matches the output size or is equal to 1. If this is
  not the case, broadcasting cannot be applied and the operation fails.

  \emph{Example:} In the above example, the shape of \texttt{a} matches
  the output shape exactly. The implied shape of \texttt{b} is
  \texttt{(1,n)}, so it matches exactly along the second dimension, and
  is 1 along the first, and thus can be used.
\item
  For any input array with size 1 along some dimension, the (unique)
  element in this dimension will be used for all calculations along that
  dimension.

  \emph{Example:} Any element \texttt{a{[}i,j{]}} will be matched with
  the element \texttt{b{[}0,j{]}} to the calculate the output value at
  \texttt{(i,j)}.
\end{enumerate}

Because additional dimensions are added \emph{at the beginning} to
create the desired output shape, broadcasting will not work
automatically if we want to multiply arrays of shape
\texttt{a.shape\ =\ (m,n)} and \texttt{b.shape\ =\ (m,)}.

\begin{itemize}
\tightlist
\item
  Following the above steps, \texttt{b} will implicitly be reshaped to
  \texttt{(1,m)} and the operation will fail at step 3:
\end{itemize}

    \begin{tcolorbox}[breakable, size=fbox, boxrule=1pt, pad at break*=1mm,colback=cellbackground, colframe=cellborder]
\prompt{In}{incolor}{49}{\boxspacing}
\begin{Verbatim}[commandchars=\\\{\}]
\PY{n}{m} \PY{o}{=} \PY{l+m+mi}{3}
\PY{n}{n} \PY{o}{=} \PY{l+m+mi}{2}
\PY{n}{a} \PY{o}{=} \PY{n}{np}\PY{o}{.}\PY{n}{arange}\PY{p}{(}\PY{n}{m}\PY{o}{*}\PY{n}{n}\PY{p}{)}\PY{o}{.}\PY{n}{reshape}\PY{p}{(}\PY{p}{(}\PY{n}{m}\PY{p}{,} \PY{n}{n}\PY{p}{)}\PY{p}{)}      \PY{c+c1}{\PYZsh{} matrix of shape (m,n)}
\PY{n}{b} \PY{o}{=} \PY{n}{np}\PY{o}{.}\PY{n}{arange}\PY{p}{(}\PY{l+m+mi}{1}\PY{p}{,} \PY{n}{m}\PY{o}{+}\PY{l+m+mi}{1}\PY{p}{)}                   \PY{c+c1}{\PYZsh{} vector of shape (m,)}

\PY{n}{a} \PY{o}{*} \PY{n}{b}           \PY{c+c1}{\PYZsh{} will not work!}
\end{Verbatim}
\end{tcolorbox}

    \begin{Verbatim}[commandchars=\\\{\}, frame=single, framerule=2mm, rulecolor=\color{outerrorbackground}]
\textcolor{ansi-red}{ValueError}\textcolor{ansi-red}{:} operands could not be broadcast together with shapes (3,2) (3,) 

    \end{Verbatim}

    We therefore have to explicitly \emph{append} a degenerate axis to
\texttt{b} such that both arrays have the same dimension:

    \begin{tcolorbox}[breakable, size=fbox, boxrule=1pt, pad at break*=1mm,colback=cellbackground, colframe=cellborder]
\prompt{In}{incolor}{50}{\boxspacing}
\begin{Verbatim}[commandchars=\\\{\}]
\PY{n}{b} \PY{o}{=} \PY{n}{b}\PY{p}{[}\PY{p}{:}\PY{p}{,}\PY{k+kc}{None}\PY{p}{]}
\PY{n}{a} \PY{o}{*} \PY{n}{b}
\end{Verbatim}
\end{tcolorbox}

            \begin{tcolorbox}[breakable, size=fbox, boxrule=.5pt, pad at break*=1mm, opacityfill=0]
\prompt{Out}{outcolor}{50}{\boxspacing}
\begin{Verbatim}[commandchars=\\\{\}]
array([[ 0,  1],
       [ 4,  6],
       [12, 15]])
\end{Verbatim}
\end{tcolorbox}
        
    Because \texttt{a} has shape \texttt{(m,n)} and \texttt{b} now has shape
\texttt{(m,1)}, \texttt{b{[}:,0{]}} will be replicated across all
columns of \texttt{a} to perform the operation.

It is worthwhile to take some time to master broadcasting as it's
essential to using NumPy efficiently. You might think that one can
simply replicate array operands along some dimension to get the same
effect, which is what we do in languages that do not support
broadcasting.

\begin{itemize}
\item
  This included Matlab until release R2016b, where implicit expansion
  for some arithmetic and logical operations was introduced.

  Prior to that, users had to manually expand input arrays using
  \texttt{repmat()}, or use the rather inelegant \texttt{bsxfun()}
  function.

  Note that even today, NumPy broadcasting goes beyond Matlab's
  capabilities.
\end{itemize}

To illustrate the difference between broadcasting and manual replication
of data, we perform the element-wise multiplication of a 3-dimensional
array with a (1-dimensional) vector:

    \begin{tcolorbox}[breakable, size=fbox, boxrule=1pt, pad at break*=1mm,colback=cellbackground, colframe=cellborder]
\prompt{In}{incolor}{51}{\boxspacing}
\begin{Verbatim}[commandchars=\\\{\}]
\PY{c+c1}{\PYZsh{} Dimensions of 3d array}
\PY{n}{k} \PY{o}{=} \PY{l+m+mi}{10}
\PY{n}{m} \PY{o}{=} \PY{l+m+mi}{11}
\PY{n}{n} \PY{o}{=} \PY{l+m+mi}{12}

\PY{n}{a} \PY{o}{=} \PY{n}{np}\PY{o}{.}\PY{n}{arange}\PY{p}{(}\PY{n}{k}\PY{o}{*}\PY{n}{m}\PY{o}{*}\PY{n}{n}\PY{p}{)}\PY{o}{.}\PY{n}{reshape}\PY{p}{(}\PY{p}{(}\PY{n}{k}\PY{p}{,} \PY{n}{m}\PY{p}{,} \PY{n}{n}\PY{p}{)}\PY{p}{)}     \PY{c+c1}{\PYZsh{} create 3d array}
\PY{n}{b} \PY{o}{=} \PY{n}{np}\PY{o}{.}\PY{n}{arange}\PY{p}{(}\PY{n}{n}\PY{p}{)}                            \PY{c+c1}{\PYZsh{} create 1d vector}
\end{Verbatim}
\end{tcolorbox}

    We can manually expand the vector to have the same shape as the array
\texttt{a} using \texttt{np.tile()} which creates: \texttt{k\ *\ m}
copies of the \texttt{n} elements in \texttt{b}:

    \begin{tcolorbox}[breakable, size=fbox, boxrule=1pt, pad at break*=1mm,colback=cellbackground, colframe=cellborder]
\prompt{In}{incolor}{52}{\boxspacing}
\begin{Verbatim}[commandchars=\\\{\}]
\PY{n}{b\PYZus{}exp} \PY{o}{=} \PY{n}{np}\PY{o}{.}\PY{n}{tile}\PY{p}{(}\PY{n}{b}\PY{p}{,} \PY{n}{reps}\PY{o}{=}\PY{p}{(}\PY{n}{k}\PY{p}{,} \PY{n}{m}\PY{p}{,} \PY{l+m+mi}{1}\PY{p}{)}\PY{p}{)}
\PY{n}{b\PYZus{}exp}\PY{o}{.}\PY{n}{shape}
\end{Verbatim}
\end{tcolorbox}

            \begin{tcolorbox}[breakable, size=fbox, boxrule=.5pt, pad at break*=1mm, opacityfill=0]
\prompt{Out}{outcolor}{52}{\boxspacing}
\begin{Verbatim}[commandchars=\\\{\}]
(10, 11, 12)
\end{Verbatim}
\end{tcolorbox}
        
    The following code compares the execution time of computing
\texttt{a\ *\ b} using broadcasting to the case where we first
explicitly expand \texttt{b}:

    \begin{tcolorbox}[breakable, size=fbox, boxrule=1pt, pad at break*=1mm,colback=cellbackground, colframe=cellborder]
\prompt{In}{incolor}{53}{\boxspacing}
\begin{Verbatim}[commandchars=\\\{\}]
\PY{c+c1}{\PYZsh{} Multiplication with broadcasting}
\PY{o}{\PYZpc{}}\PY{k}{timeit} a * b
\end{Verbatim}
\end{tcolorbox}

    \begin{Verbatim}[commandchars=\\\{\}]
1.81 µs ± 13.9 ns per loop (mean ± std. dev. of 7 runs, 1000000 loops each)
    \end{Verbatim}

    \begin{tcolorbox}[breakable, size=fbox, boxrule=1pt, pad at break*=1mm,colback=cellbackground, colframe=cellborder]
\prompt{In}{incolor}{54}{\boxspacing}
\begin{Verbatim}[commandchars=\\\{\}]
\PY{c+c1}{\PYZsh{} Multiplication with explicitly expanded operands}
\PY{o}{\PYZpc{}}\PY{k}{timeit} a * np.tile(b, reps=(k, m, 1))
\end{Verbatim}
\end{tcolorbox}

    \begin{Verbatim}[commandchars=\\\{\}]
6.52 µs ± 107 ns per loop (mean ± std. dev. of 7 runs, 100000 loops each)
    \end{Verbatim}

    As you see, not only is the second operation more complex and difficult
to read, it also takes approximately three times as long to execute! The
reason is that \texttt{np.tile()} replicates the data in memory, which
is expensive. No copying is done when using broadcasting.


\hypertarget{vectorisation}{%
\subsection{Vectorisation}\label{vectorisation}}

Vectorisation is the concept of applying operations to whole arrays of
data instead of every singular element (note that the term also has
other meanings in computer science). In Python, as well as languages
such as Matlab and R, we use this programming technique to increase
performance for two reasons:

\begin{enumerate}
\def\labelenumi{\arabic{enumi}.}
\tightlist
\item
  Looping over elements is slow.
\item
  Calling a function on every single element is also slow.
\end{enumerate}

These performance penalties are less pronounced for compiled languages
such as C or Fortran, so we try to move the looping to code written in
one of these languages. In particular, since NumPy's core parts are
implemented in C, we always want to do looping ``within'' NumPy.

For example, consider element-wise addition of two arrays, \texttt{a}
and \texttt{b}:

    \begin{tcolorbox}[breakable, size=fbox, boxrule=1pt, pad at break*=1mm,colback=cellbackground, colframe=cellborder]
\prompt{In}{incolor}{55}{\boxspacing}
\begin{Verbatim}[commandchars=\\\{\}]
\PY{k+kn}{import} \PY{n+nn}{numpy} \PY{k}{as} \PY{n+nn}{np}

\PY{c+c1}{\PYZsh{} array size}
\PY{n}{N} \PY{o}{=} \PY{l+m+mi}{100}

\PY{c+c1}{\PYZsh{} input arrays}
\PY{n}{a} \PY{o}{=} \PY{n}{np}\PY{o}{.}\PY{n}{linspace}\PY{p}{(}\PY{l+m+mf}{0.0}\PY{p}{,} \PY{l+m+mf}{1.0}\PY{p}{,} \PY{n}{N}\PY{p}{)}
\PY{n}{b} \PY{o}{=} \PY{n}{np}\PY{o}{.}\PY{n}{linspace}\PY{p}{(}\PY{l+m+mf}{1.0}\PY{p}{,} \PY{l+m+mf}{2.0}\PY{p}{,} \PY{n}{N}\PY{p}{)}
\end{Verbatim}
\end{tcolorbox}

    Benchmarking a non-vectorised loop in pure Python against NumPy's
vectorised implementation reveals some striking differences:

    \begin{tcolorbox}[breakable, size=fbox, boxrule=1pt, pad at break*=1mm,colback=cellbackground, colframe=cellborder]
\prompt{In}{incolor}{56}{\boxspacing}
\begin{Verbatim}[commandchars=\\\{\}]
\PY{o}{\PYZpc{}\PYZpc{}time}it
\PY{c+c1}{\PYZsh{} Compute c = a + b using Python loops}
\PY{n}{c} \PY{o}{=} \PY{n}{np}\PY{o}{.}\PY{n}{empty}\PY{p}{(}\PY{n}{N}\PY{p}{)}         \PY{c+c1}{\PYZsh{} allocate output array}
\PY{k}{for} \PY{n}{i} \PY{o+ow}{in} \PY{n+nb}{range}\PY{p}{(}\PY{n}{N}\PY{p}{)}\PY{p}{:}
    \PY{n}{c}\PY{p}{[}\PY{n}{i}\PY{p}{]} \PY{o}{=} \PY{n}{a}\PY{p}{[}\PY{n}{i}\PY{p}{]} \PY{o}{+} \PY{n}{b}\PY{p}{[}\PY{n}{i}\PY{p}{]}
\end{Verbatim}
\end{tcolorbox}

    \begin{Verbatim}[commandchars=\\\{\}]
29 µs ± 95.2 ns per loop (mean ± std. dev. of 7 runs, 10000 loops each)
    \end{Verbatim}

    \begin{tcolorbox}[breakable, size=fbox, boxrule=1pt, pad at break*=1mm,colback=cellbackground, colframe=cellborder]
\prompt{In}{incolor}{57}{\boxspacing}
\begin{Verbatim}[commandchars=\\\{\}]
\PY{c+c1}{\PYZsh{} Compute c = a + b using vectorised addition}
\PY{o}{\PYZpc{}}\PY{k}{timeit} c = a + b
\end{Verbatim}
\end{tcolorbox}

    \begin{Verbatim}[commandchars=\\\{\}]
363 ns ± 3.01 ns per loop (mean ± std. dev. of 7 runs, 1000000 loops each)
    \end{Verbatim}

    We see that the vectorised variant is about 100 times faster! What is
going on?

\begin{itemize}
\tightlist
\item
  NumPy implements a vectorised operator \texttt{+} which accepts arrays
  as operands.
\item
  NumPy performs looping over individual elements in C which is compiled
  to high-performance machine code.
\end{itemize}

Of course this example is somewhat artificial: you would not have
implemented a loop in Python in the first place, you probably just
assumed that \texttt{+} works with array operands.

However, we can easily construct more complex computations which may not
be implemented in NumPy but which can use vectorised building blocks to
speed up computations.

As an example, consider the following function which computes the sum of
finite of elements of an array, ignoring infinity and NaNs.

    \begin{tcolorbox}[breakable, size=fbox, boxrule=1pt, pad at break*=1mm,colback=cellbackground, colframe=cellborder]
\prompt{In}{incolor}{58}{\boxspacing}
\begin{Verbatim}[commandchars=\\\{\}]
\PY{c+c1}{\PYZsh{} Compute sum of finite elements in x}
\PY{k}{def} \PY{n+nf}{finite\PYZus{}sum}\PY{p}{(}\PY{n}{x}\PY{p}{)}\PY{p}{:}
    \PY{c+c1}{\PYZsh{} initialise sum}
    \PY{n}{s} \PY{o}{=} \PY{l+m+mf}{0.0}
    \PY{c+c1}{\PYZsh{} loop over array elements}
    \PY{k}{for} \PY{n}{xi} \PY{o+ow}{in} \PY{n}{x}\PY{p}{:}
        \PY{c+c1}{\PYZsh{} Check whether a value is finite}
        \PY{k}{if} \PY{n}{np}\PY{o}{.}\PY{n}{isfinite}\PY{p}{(}\PY{n}{xi}\PY{p}{)}\PY{p}{:}
            \PY{c+c1}{\PYZsh{} Add to running sum}
            \PY{n}{s} \PY{o}{+}\PY{o}{=} \PY{n}{xi}
    \PY{k}{return} \PY{n}{s}
\end{Verbatim}
\end{tcolorbox}

    Next, we create a sample array with a few NaNs and infinite values and
to test \texttt{finite\_sum()}:

    \begin{tcolorbox}[breakable, size=fbox, boxrule=1pt, pad at break*=1mm,colback=cellbackground, colframe=cellborder]
\prompt{In}{incolor}{59}{\boxspacing}
\begin{Verbatim}[commandchars=\\\{\}]
\PY{n}{arr} \PY{o}{=} \PY{n}{np}\PY{o}{.}\PY{n}{linspace}\PY{p}{(}\PY{l+m+mf}{0.0}\PY{p}{,} \PY{l+m+mf}{1.0}\PY{p}{,} \PY{l+m+mi}{100}\PY{p}{)}
\PY{n}{arr}\PY{p}{[}\PY{p}{:}\PY{p}{:}\PY{l+m+mi}{3}\PY{p}{]} \PY{o}{=} \PY{n}{np}\PY{o}{.}\PY{n}{inf}       \PY{c+c1}{\PYZsh{} assign infinity to every 3rd element}
\PY{n}{arr}\PY{p}{[}\PY{p}{:}\PY{p}{:}\PY{l+m+mi}{5}\PY{p}{]} \PY{o}{=} \PY{n}{np}\PY{o}{.}\PY{n}{nan}       \PY{c+c1}{\PYZsh{} assign NaN to every 5th element}
\end{Verbatim}
\end{tcolorbox}

    Comparing our \texttt{finite\_sum()} to a vectorised version again shows
the performance advantage of vectorised code:

    \begin{tcolorbox}[breakable, size=fbox, boxrule=1pt, pad at break*=1mm,colback=cellbackground, colframe=cellborder]
\prompt{In}{incolor}{60}{\boxspacing}
\begin{Verbatim}[commandchars=\\\{\}]
\PY{o}{\PYZpc{}}\PY{k}{timeit} finite\PYZus{}sum(arr)
\end{Verbatim}
\end{tcolorbox}

    \begin{Verbatim}[commandchars=\\\{\}]
112 µs ± 606 ns per loop (mean ± std. dev. of 7 runs, 10000 loops each)
    \end{Verbatim}

    \begin{tcolorbox}[breakable, size=fbox, boxrule=1pt, pad at break*=1mm,colback=cellbackground, colframe=cellborder]
\prompt{In}{incolor}{61}{\boxspacing}
\begin{Verbatim}[commandchars=\\\{\}]
\PY{o}{\PYZpc{}}\PY{k}{timeit} np.sum(arr[np.isfinite(arr)])
\end{Verbatim}
\end{tcolorbox}

    \begin{Verbatim}[commandchars=\\\{\}]
3.99 µs ± 20.3 ns per loop (mean ± std. dev. of 7 runs, 100000 loops each)
    \end{Verbatim}

    The vectorised version is approximately 25 times faster, despite the
fact that

\begin{itemize}
\item
  NumPy does not implement this specific function; and
\item
  the vectorised implementation creates two temporary arrays:

  \begin{itemize}
  \tightlist
  \item
    one when calling \texttt{np.isfinite()}; and
  \item
    one when indexing \texttt{arr} with the boolean array returned by
    \texttt{np.isfinite()}. As we discuss below, indexing with boolean
    arrays always creates a copy!
  \end{itemize}
\end{itemize}

This illustrates that creating vectorised code by combining several
vectorised functions also yields considerable speed-ups.

As an aside, note that NumPy actually implements \texttt{np.nansum()}
which drops NaNs, but it does not discard infinite values.


\hypertarget{copies-and-views-advanced}{%
\subsection{Copies and views
(advanced)}\label{copies-and-views-advanced}}

Recall that assignment in Python does \emph{not} create a copy (unlike
in C, Fortran or Matlab):

    \begin{tcolorbox}[breakable, size=fbox, boxrule=1pt, pad at break*=1mm,colback=cellbackground, colframe=cellborder]
\prompt{In}{incolor}{62}{\boxspacing}
\begin{Verbatim}[commandchars=\\\{\}]
\PY{n}{a} \PY{o}{=} \PY{p}{[}\PY{l+m+mi}{0}\PY{p}{,} \PY{l+m+mi}{0}\PY{p}{,} \PY{l+m+mi}{0}\PY{p}{]}
\PY{n}{b} \PY{o}{=} \PY{n}{a}           \PY{c+c1}{\PYZsh{} b references the same object as a}
\PY{n}{b}\PY{p}{[}\PY{l+m+mi}{1}\PY{p}{]} \PY{o}{=} \PY{l+m+mi}{1}        \PY{c+c1}{\PYZsh{} modify second element of b (and a!)}
\PY{n}{a} \PY{o}{==} \PY{n}{b}          \PY{c+c1}{\PYZsh{} a and b are still the same}
\end{Verbatim}
\end{tcolorbox}

            \begin{tcolorbox}[breakable, size=fbox, boxrule=.5pt, pad at break*=1mm, opacityfill=0]
\prompt{Out}{outcolor}{62}{\boxspacing}
\begin{Verbatim}[commandchars=\\\{\}]
True
\end{Verbatim}
\end{tcolorbox}
        
    NumPy adds another layer to this sort of data sharing: whenever you
perform an assignment or indexing operation, NumPy tries hard \emph{not}
to copy the underlying data but instead creates a so-called view which
points to the same block of memory. It does so for performance reasons
(copying is expensive).

We can illustrate this using array slicing:

    \begin{tcolorbox}[breakable, size=fbox, boxrule=1pt, pad at break*=1mm,colback=cellbackground, colframe=cellborder]
\prompt{In}{incolor}{63}{\boxspacing}
\begin{Verbatim}[commandchars=\\\{\}]
\PY{k+kn}{import} \PY{n+nn}{numpy} \PY{k}{as} \PY{n+nn}{np}

\PY{n}{x} \PY{o}{=} \PY{n}{np}\PY{o}{.}\PY{n}{arange}\PY{p}{(}\PY{l+m+mi}{10}\PY{p}{)}
\PY{n}{x}
\end{Verbatim}
\end{tcolorbox}

            \begin{tcolorbox}[breakable, size=fbox, boxrule=.5pt, pad at break*=1mm, opacityfill=0]
\prompt{Out}{outcolor}{63}{\boxspacing}
\begin{Verbatim}[commandchars=\\\{\}]
array([0, 1, 2, 3, 4, 5, 6, 7, 8, 9])
\end{Verbatim}
\end{tcolorbox}
        
    \begin{tcolorbox}[breakable, size=fbox, boxrule=1pt, pad at break*=1mm,colback=cellbackground, colframe=cellborder]
\prompt{In}{incolor}{64}{\boxspacing}
\begin{Verbatim}[commandchars=\\\{\}]
\PY{n}{y} \PY{o}{=} \PY{n}{x}\PY{p}{[}\PY{l+m+mi}{3}\PY{p}{:}\PY{l+m+mi}{8}\PY{p}{]}          \PY{c+c1}{\PYZsh{} Create array that points to elements 4\PYZhy{}8 of x}
\PY{n}{y}
\end{Verbatim}
\end{tcolorbox}

            \begin{tcolorbox}[breakable, size=fbox, boxrule=.5pt, pad at break*=1mm, opacityfill=0]
\prompt{Out}{outcolor}{64}{\boxspacing}
\begin{Verbatim}[commandchars=\\\{\}]
array([3, 4, 5, 6, 7])
\end{Verbatim}
\end{tcolorbox}
        
    The arrays \texttt{x} and \texttt{y} are two different Python objects,
which we can verify using the built-in \texttt{id()} function:

    \begin{tcolorbox}[breakable, size=fbox, boxrule=1pt, pad at break*=1mm,colback=cellbackground, colframe=cellborder]
\prompt{In}{incolor}{65}{\boxspacing}
\begin{Verbatim}[commandchars=\\\{\}]
\PY{n+nb}{print}\PY{p}{(}\PY{n+nb}{id}\PY{p}{(}\PY{n}{x}\PY{p}{)}\PY{p}{)}
\PY{n+nb}{print}\PY{p}{(}\PY{n+nb}{id}\PY{p}{(}\PY{n}{y}\PY{p}{)}\PY{p}{)}
\end{Verbatim}
\end{tcolorbox}

    \begin{Verbatim}[commandchars=\\\{\}]
140699715380928
140699715337312
    \end{Verbatim}

    \begin{tcolorbox}[breakable, size=fbox, boxrule=1pt, pad at break*=1mm,colback=cellbackground, colframe=cellborder]
\prompt{In}{incolor}{66}{\boxspacing}
\begin{Verbatim}[commandchars=\\\{\}]
\PY{n+nb}{id}\PY{p}{(}\PY{n}{x}\PY{p}{)} \PY{o}{==} \PY{n+nb}{id}\PY{p}{(}\PY{n}{y}\PY{p}{)}
\end{Verbatim}
\end{tcolorbox}

            \begin{tcolorbox}[breakable, size=fbox, boxrule=.5pt, pad at break*=1mm, opacityfill=0]
\prompt{Out}{outcolor}{66}{\boxspacing}
\begin{Verbatim}[commandchars=\\\{\}]
False
\end{Verbatim}
\end{tcolorbox}
        
    And yet, the NumPy implementation makes sure that they reference the
same block of memory!

We can see this easily by modifying \texttt{y}:

    \begin{tcolorbox}[breakable, size=fbox, boxrule=1pt, pad at break*=1mm,colback=cellbackground, colframe=cellborder]
\prompt{In}{incolor}{67}{\boxspacing}
\begin{Verbatim}[commandchars=\\\{\}]
\PY{n}{y}\PY{p}{[}\PY{p}{:}\PY{p}{]} \PY{o}{=} \PY{l+m+mi}{0}        \PY{c+c1}{\PYZsh{} overwrite all elements of y with zeros}
\PY{n}{y}
\end{Verbatim}
\end{tcolorbox}

            \begin{tcolorbox}[breakable, size=fbox, boxrule=.5pt, pad at break*=1mm, opacityfill=0]
\prompt{Out}{outcolor}{67}{\boxspacing}
\begin{Verbatim}[commandchars=\\\{\}]
array([0, 0, 0, 0, 0])
\end{Verbatim}
\end{tcolorbox}
        
    \begin{tcolorbox}[breakable, size=fbox, boxrule=1pt, pad at break*=1mm,colback=cellbackground, colframe=cellborder]
\prompt{In}{incolor}{68}{\boxspacing}
\begin{Verbatim}[commandchars=\\\{\}]
\PY{n}{x}               \PY{c+c1}{\PYZsh{} elements of x that are also referenced by y}
                \PY{c+c1}{\PYZsh{} are now also zero!}
\end{Verbatim}
\end{tcolorbox}

            \begin{tcolorbox}[breakable, size=fbox, boxrule=.5pt, pad at break*=1mm, opacityfill=0]
\prompt{Out}{outcolor}{68}{\boxspacing}
\begin{Verbatim}[commandchars=\\\{\}]
array([0, 1, 2, 0, 0, 0, 0, 0, 8, 9])
\end{Verbatim}
\end{tcolorbox}
        
    This behaviour is even triggered when \texttt{y} references non-adjacent
elements in \texttt{x}. For example, we can let \texttt{y} be a view
onto every \emph{second} element in \texttt{x}:

    \begin{tcolorbox}[breakable, size=fbox, boxrule=1pt, pad at break*=1mm,colback=cellbackground, colframe=cellborder]
\prompt{In}{incolor}{69}{\boxspacing}
\begin{Verbatim}[commandchars=\\\{\}]
\PY{n}{x} \PY{o}{=} \PY{n}{np}\PY{o}{.}\PY{n}{arange}\PY{p}{(}\PY{l+m+mi}{10}\PY{p}{)}
\PY{n}{y} \PY{o}{=} \PY{n}{x}\PY{p}{[}\PY{p}{:}\PY{p}{:}\PY{l+m+mi}{2}\PY{p}{]}      \PY{c+c1}{\PYZsh{} y now points to every second element of x}
\PY{n}{y}\PY{p}{[}\PY{p}{:}\PY{p}{]} \PY{o}{=} \PY{l+m+mi}{0}        \PY{c+c1}{\PYZsh{} overwrite all elements of y with zeros}
\PY{n}{x}               \PY{c+c1}{\PYZsh{} every second element in x is now zero!}
\end{Verbatim}
\end{tcolorbox}

            \begin{tcolorbox}[breakable, size=fbox, boxrule=.5pt, pad at break*=1mm, opacityfill=0]
\prompt{Out}{outcolor}{69}{\boxspacing}
\begin{Verbatim}[commandchars=\\\{\}]
array([0, 1, 0, 3, 0, 5, 0, 7, 0, 9])
\end{Verbatim}
\end{tcolorbox}
        
    As a rule of thumb, NumPy will create a view as opposed to copying data
if

\begin{itemize}
\tightlist
\item
  An array is created from another array via slicing (\ie indexing
  using the \texttt{start:stop:step} triplet)
\end{itemize}

Conversely, a \emph{copy} is created whenever

\begin{itemize}
\tightlist
\item
  An array is created from another array via boolean (mask) indexing.
\item
  An array is created from another array via integer array indexing.
\end{itemize}

Moreover, you can always force NumPy to create a copy by calling
\texttt{np.copy()}!

\emph{Examples:}

    \begin{tcolorbox}[breakable, size=fbox, boxrule=1pt, pad at break*=1mm,colback=cellbackground, colframe=cellborder]
\prompt{In}{incolor}{70}{\boxspacing}
\begin{Verbatim}[commandchars=\\\{\}]
\PY{c+c1}{\PYZsh{} Copies are created with boolean indexing}
\PY{n}{x} \PY{o}{=} \PY{n}{np}\PY{o}{.}\PY{n}{arange}\PY{p}{(}\PY{l+m+mi}{10}\PY{p}{)}
\PY{n}{mask} \PY{o}{=} \PY{p}{(}\PY{n}{x} \PY{o}{\PYZgt{}} \PY{l+m+mi}{4}\PY{p}{)}      \PY{c+c1}{\PYZsh{} boolean mask}
\PY{n}{y} \PY{o}{=} \PY{n}{x}\PY{p}{[}\PY{n}{mask}\PY{p}{]}         \PY{c+c1}{\PYZsh{} create y using boolean indexing}
\PY{n}{y}\PY{p}{[}\PY{p}{:}\PY{p}{]} \PY{o}{=} \PY{l+m+mi}{0}
\PY{n}{x}                   \PY{c+c1}{\PYZsh{} x is unmodified}
\end{Verbatim}
\end{tcolorbox}

            \begin{tcolorbox}[breakable, size=fbox, boxrule=.5pt, pad at break*=1mm, opacityfill=0]
\prompt{Out}{outcolor}{70}{\boxspacing}
\begin{Verbatim}[commandchars=\\\{\}]
array([0, 1, 2, 3, 4, 5, 6, 7, 8, 9])
\end{Verbatim}
\end{tcolorbox}
        
    \begin{tcolorbox}[breakable, size=fbox, boxrule=1pt, pad at break*=1mm,colback=cellbackground, colframe=cellborder]
\prompt{In}{incolor}{71}{\boxspacing}
\begin{Verbatim}[commandchars=\\\{\}]
\PY{c+c1}{\PYZsh{} Copies are created with integer array indexing}
\PY{n}{x} \PY{o}{=} \PY{n}{np}\PY{o}{.}\PY{n}{arange}\PY{p}{(}\PY{l+m+mi}{10}\PY{p}{)}
\PY{n}{index} \PY{o}{=} \PY{p}{[}\PY{l+m+mi}{3}\PY{p}{,} \PY{l+m+mi}{4}\PY{p}{,} \PY{l+m+mi}{5}\PY{p}{]}   \PY{c+c1}{\PYZsh{} List of indices to include in y}
\PY{n}{y} \PY{o}{=} \PY{n}{x}\PY{p}{[}\PY{n}{index}\PY{p}{]}
\PY{n}{y}\PY{p}{[}\PY{p}{:}\PY{p}{]} \PY{o}{=} \PY{l+m+mi}{0}
\PY{n}{x}                   \PY{c+c1}{\PYZsh{} x is unmodified}
\end{Verbatim}
\end{tcolorbox}

            \begin{tcolorbox}[breakable, size=fbox, boxrule=.5pt, pad at break*=1mm, opacityfill=0]
\prompt{Out}{outcolor}{71}{\boxspacing}
\begin{Verbatim}[commandchars=\\\{\}]
array([0, 1, 2, 3, 4, 5, 6, 7, 8, 9])
\end{Verbatim}
\end{tcolorbox}
        
    \begin{tcolorbox}[breakable, size=fbox, boxrule=1pt, pad at break*=1mm,colback=cellbackground, colframe=cellborder]
\prompt{In}{incolor}{72}{\boxspacing}
\begin{Verbatim}[commandchars=\\\{\}]
\PY{c+c1}{\PYZsh{} Forced copy with slicing}
\PY{n}{x} \PY{o}{=} \PY{n}{np}\PY{o}{.}\PY{n}{arange}\PY{p}{(}\PY{l+m+mi}{10}\PY{p}{)}
\PY{n}{y} \PY{o}{=} \PY{n}{np}\PY{o}{.}\PY{n}{copy}\PY{p}{(}\PY{n}{x}\PY{p}{[}\PY{l+m+mi}{3}\PY{p}{:}\PY{l+m+mi}{8}\PY{p}{]}\PY{p}{)} \PY{c+c1}{\PYZsh{} force copy with np.copy()}
\PY{n}{y}\PY{p}{[}\PY{p}{:}\PY{p}{]} \PY{o}{=} \PY{l+m+mi}{0}
\PY{n}{x}                   \PY{c+c1}{\PYZsh{} x is unmodified}
\end{Verbatim}
\end{tcolorbox}

            \begin{tcolorbox}[breakable, size=fbox, boxrule=.5pt, pad at break*=1mm, opacityfill=0]
\prompt{Out}{outcolor}{72}{\boxspacing}
\begin{Verbatim}[commandchars=\\\{\}]
array([0, 1, 2, 3, 4, 5, 6, 7, 8, 9])
\end{Verbatim}
\end{tcolorbox}
        
    As an alternative to \texttt{np.copy()}, we can directly call the
\texttt{copy()} method of an array:

    \begin{tcolorbox}[breakable, size=fbox, boxrule=1pt, pad at break*=1mm,colback=cellbackground, colframe=cellborder]
\prompt{In}{incolor}{73}{\boxspacing}
\begin{Verbatim}[commandchars=\\\{\}]
\PY{n}{y} \PY{o}{=} \PY{n}{x}\PY{p}{[}\PY{l+m+mi}{3}\PY{p}{:}\PY{l+m+mi}{8}\PY{p}{]}\PY{o}{.}\PY{n}{copy}\PY{p}{(}\PY{p}{)}
\end{Verbatim}
\end{tcolorbox}


\hypertarget{numpy-data-types-advanced}{%
\subsection{NumPy data types
(advanced)}\label{numpy-data-types-advanced}}

\hypertarget{default-data-types}{%
\subsubsection{Default data types}\label{default-data-types}}

We have already touched upon the numerical data types used by NumPy. If
we do not explicitly request a data type using the \texttt{dtype}
keyword argument, NumPy by default behaves as follows:

\begin{enumerate}
\def\labelenumi{\arabic{enumi}.}
\item
  The default data type for most array creation routines which create
  arrays of a given shape or size, such as \texttt{np.empty()},
  \texttt{np.ones()} and \texttt{np.zeros()}, is a 64-bit floating-point
  number (\texttt{np.float64}).
\item
  Array creation routines that accept numerical input data will use the
  data type of this input data to determine the array data type.

  Examples of such functions are \texttt{np.arange()} and
  \texttt{np.array()}.
\item
  Arrays that are implicitly created as a result of an operation
  (addition, etc.) are assigned the most suitable type to represent the
  result.

  For example, when adding a floating-point and an integer array, the
  result will be a floating-point array.
\end{enumerate}

\emph{Examples:}

\emph{Case 1:} default data type is \texttt{np.float64}:

    \begin{tcolorbox}[breakable, size=fbox, boxrule=1pt, pad at break*=1mm,colback=cellbackground, colframe=cellborder]
\prompt{In}{incolor}{74}{\boxspacing}
\begin{Verbatim}[commandchars=\\\{\}]
\PY{k+kn}{import} \PY{n+nn}{numpy} \PY{k}{as} \PY{n+nn}{np}

\PY{n}{x} \PY{o}{=} \PY{n}{np}\PY{o}{.}\PY{n}{ones}\PY{p}{(}\PY{l+m+mi}{1}\PY{p}{)}      \PY{c+c1}{\PYZsh{} length\PYZhy{}1 vector of ones}
\PY{n}{x}\PY{o}{.}\PY{n}{dtype}             \PY{c+c1}{\PYZsh{} default type: float64}
\end{Verbatim}
\end{tcolorbox}

            \begin{tcolorbox}[breakable, size=fbox, boxrule=.5pt, pad at break*=1mm, opacityfill=0]
\prompt{Out}{outcolor}{74}{\boxspacing}
\begin{Verbatim}[commandchars=\\\{\}]
dtype('float64')
\end{Verbatim}
\end{tcolorbox}
        
    \emph{Case 2:} data type depends on input data:

    \begin{tcolorbox}[breakable, size=fbox, boxrule=1pt, pad at break*=1mm,colback=cellbackground, colframe=cellborder]
\prompt{In}{incolor}{75}{\boxspacing}
\begin{Verbatim}[commandchars=\\\{\}]
\PY{c+c1}{\PYZsh{} Argument is an integer}
\PY{n}{x} \PY{o}{=} \PY{n}{np}\PY{o}{.}\PY{n}{arange}\PY{p}{(}\PY{l+m+mi}{5}\PY{p}{)}
\PY{n}{x}\PY{o}{.}\PY{n}{dtype}             \PY{c+c1}{\PYZsh{} data type is np.int64 (on most platforms)}
\end{Verbatim}
\end{tcolorbox}

            \begin{tcolorbox}[breakable, size=fbox, boxrule=.5pt, pad at break*=1mm, opacityfill=0]
\prompt{Out}{outcolor}{75}{\boxspacing}
\begin{Verbatim}[commandchars=\\\{\}]
dtype('int64')
\end{Verbatim}
\end{tcolorbox}
        
    \begin{tcolorbox}[breakable, size=fbox, boxrule=1pt, pad at break*=1mm,colback=cellbackground, colframe=cellborder]
\prompt{In}{incolor}{76}{\boxspacing}
\begin{Verbatim}[commandchars=\\\{\}]
\PY{c+c1}{\PYZsh{} Argument is a float}
\PY{n}{x} \PY{o}{=} \PY{n}{np}\PY{o}{.}\PY{n}{arange}\PY{p}{(}\PY{l+m+mf}{5.0}\PY{p}{)}
\PY{n}{x}\PY{o}{.}\PY{n}{dtype}             \PY{c+c1}{\PYZsh{} data type is np.float64}
\end{Verbatim}
\end{tcolorbox}

            \begin{tcolorbox}[breakable, size=fbox, boxrule=.5pt, pad at break*=1mm, opacityfill=0]
\prompt{Out}{outcolor}{76}{\boxspacing}
\begin{Verbatim}[commandchars=\\\{\}]
dtype('float64')
\end{Verbatim}
\end{tcolorbox}
        
    \emph{Case 3:} data type determined to accommodate result

    \begin{tcolorbox}[breakable, size=fbox, boxrule=1pt, pad at break*=1mm,colback=cellbackground, colframe=cellborder]
\prompt{In}{incolor}{77}{\boxspacing}
\begin{Verbatim}[commandchars=\\\{\}]
\PY{c+c1}{\PYZsh{} Add two integer arrays}
\PY{n}{arr1} \PY{o}{=} \PY{n}{np}\PY{o}{.}\PY{n}{arange}\PY{p}{(}\PY{l+m+mi}{3}\PY{p}{)}
\PY{n}{arr2} \PY{o}{=} \PY{n}{np}\PY{o}{.}\PY{n}{arange}\PY{p}{(}\PY{l+m+mi}{3}\PY{p}{,} \PY{l+m+mi}{0}\PY{p}{,} \PY{o}{\PYZhy{}}\PY{l+m+mi}{1}\PY{p}{)}      \PY{c+c1}{\PYZsh{} creates [3, 2, 1]}
\PY{n}{result} \PY{o}{=} \PY{n}{arr1} \PY{o}{+} \PY{n}{arr2}
\PY{n+nb}{print}\PY{p}{(}\PY{n}{result}\PY{p}{)}
\PY{n}{result}\PY{o}{.}\PY{n}{dtype}                    \PY{c+c1}{\PYZsh{} data type is np.int64}
\end{Verbatim}
\end{tcolorbox}

    \begin{Verbatim}[commandchars=\\\{\}]
[3 3 3]
    \end{Verbatim}

            \begin{tcolorbox}[breakable, size=fbox, boxrule=.5pt, pad at break*=1mm, opacityfill=0]
\prompt{Out}{outcolor}{77}{\boxspacing}
\begin{Verbatim}[commandchars=\\\{\}]
dtype('int64')
\end{Verbatim}
\end{tcolorbox}
        
    \begin{tcolorbox}[breakable, size=fbox, boxrule=1pt, pad at break*=1mm,colback=cellbackground, colframe=cellborder]
\prompt{In}{incolor}{78}{\boxspacing}
\begin{Verbatim}[commandchars=\\\{\}]
\PY{c+c1}{\PYZsh{} Add integer to floating\PYZhy{}point array}
\PY{n}{arr1} \PY{o}{=} \PY{n}{np}\PY{o}{.}\PY{n}{arange}\PY{p}{(}\PY{l+m+mi}{3}\PY{p}{)}
\PY{n}{arr2} \PY{o}{=} \PY{n}{np}\PY{o}{.}\PY{n}{arange}\PY{p}{(}\PY{l+m+mf}{3.0}\PY{p}{,} \PY{l+m+mf}{0.0}\PY{p}{,} \PY{o}{\PYZhy{}}\PY{l+m+mf}{1.0}\PY{p}{)}    \PY{c+c1}{\PYZsh{} creates [3.0, 2.0, 1.0]}
\PY{n}{result} \PY{o}{=} \PY{n}{arr1} \PY{o}{+} \PY{n}{arr2}
\PY{n+nb}{print}\PY{p}{(}\PY{n}{result}\PY{p}{)}
\PY{n}{result}\PY{o}{.}\PY{n}{dtype}                        \PY{c+c1}{\PYZsh{} data type is np.float64}
\end{Verbatim}
\end{tcolorbox}

    \begin{Verbatim}[commandchars=\\\{\}]
[3. 3. 3.]
    \end{Verbatim}

            \begin{tcolorbox}[breakable, size=fbox, boxrule=.5pt, pad at break*=1mm, opacityfill=0]
\prompt{Out}{outcolor}{78}{\boxspacing}
\begin{Verbatim}[commandchars=\\\{\}]
dtype('float64')
\end{Verbatim}
\end{tcolorbox}
        
    Even though the resulting array is \texttt{{[}3.0,\ 3.0,\ 3.0{]}} and
can thus be represented as integers without loss of data, NumPy only
takes into account that one of the operands is floating-point, and thus
the result has to be of floating-point type!

    \hypertarget{explicit-data-types}{%
\subsubsection{Explicit data types}\label{explicit-data-types}}

We can almost always explicitly request an array to be of a particular
data type by passing the \texttt{dtype} keyword argument. The most
common types are:

\begin{itemize}
\item
  \texttt{np.float64}: a 64-bit floating-point number, also called
  \emph{double precision} in other languages.

  This is the most commonly used floating-point data type. It can
  represent numbers with up to 16 decimal digits, and covers a range of
  approximately \(\pm 10^{308}\).
\item
  \texttt{np.int64}: a 64-bit integer which can represent integer values
  on the interval of (approximately) \(\pm10^{19}\).

  Unlike floating-point, the integer representation is \emph{exact}, but
  covers a much smaller range (and, obviously, no fractional numbers)
\item
  \texttt{np.float32}, \texttt{np.float16}: single-precision and
  half-precision floating-point numbers. These occupy only 32 and 16
  bits of memory, respectively.

  They thus trade off storage requirements for a loss of precision and
  range.
\item
  \texttt{np.int32}, \texttt{np.int16}, \texttt{np.int8} represent
  integers using 32, 16 and 8 bits, respectively.

  They require less memory, but can represent only a smaller range of
  integers. For example, \texttt{np.int8} can only store integer values
  from -128 to 127.
\item
  NumPy also supports complex numerical types to represent imaginary
  numbers. We will not be using those in this tutorial.
\end{itemize}

Would we ever want to use anything other than the default data types,
which in most cases are either \texttt{np.float64} and
\texttt{np.int64}? These, after all, support the largest range and
highest precision. This is true in general, but there are special cases
where other data types need to be used:

\begin{enumerate}
\def\labelenumi{\arabic{enumi}.}
\item
  \emph{Storage requirements:} if you work with large amounts of data,
  for example arrays with many dimensions, you can run out of memory or
  storage space (when saving results to files).

  In this case, you can store data as \texttt{np.float32} instead of
  \texttt{np.float64}, which halves the storage requirement.

  Similarly, if you know that your integer data only takes on values
  between -128 and 127, you can store them as \texttt{np.int8} which
  consumes only 1/8 of the space compared to \texttt{np.int64}!
\item
  \emph{Performance:} Some tasks simply don't require high precision or
  range. For example, some machine learning tasks can be performed using
  only 8-bit integers, and companies like Google have developed
  dedicated processors to considerably speed up workloads using 8-bit
  integers.

  Even if you are not using any specialised CPUs or GPUs, data has to be
  transferred from memory to the processor and this is a major
  performance bottleneck. The less data needs to be transferred, the
  better!

  In general, this is nothing you need to worry about at this point, but
  might become relevant once you start writing complex high-performance
  code.
\end{enumerate}

\emph{Examples:}

    \begin{tcolorbox}[breakable, size=fbox, boxrule=1pt, pad at break*=1mm,colback=cellbackground, colframe=cellborder]
\prompt{In}{incolor}{79}{\boxspacing}
\begin{Verbatim}[commandchars=\\\{\}]
\PY{k+kn}{import} \PY{n+nn}{numpy} \PY{k}{as} \PY{n+nn}{np}

\PY{c+c1}{\PYZsh{} Explicitly specify data type}
\PY{n}{x} \PY{o}{=} \PY{n}{np}\PY{o}{.}\PY{n}{ones}\PY{p}{(}\PY{l+m+mi}{1}\PY{p}{,} \PY{n}{dtype}\PY{o}{=}\PY{n}{np}\PY{o}{.}\PY{n}{float16}\PY{p}{)}
\PY{n}{x}       \PY{c+c1}{\PYZsh{} prints np.float16}
\end{Verbatim}
\end{tcolorbox}

            \begin{tcolorbox}[breakable, size=fbox, boxrule=.5pt, pad at break*=1mm, opacityfill=0]
\prompt{Out}{outcolor}{79}{\boxspacing}
\begin{Verbatim}[commandchars=\\\{\}]
array([1.], dtype=float16)
\end{Verbatim}
\end{tcolorbox}
        
    We can use \texttt{dtype} to override the data type inferred from input
data:

    \begin{tcolorbox}[breakable, size=fbox, boxrule=1pt, pad at break*=1mm,colback=cellbackground, colframe=cellborder]
\prompt{In}{incolor}{80}{\boxspacing}
\begin{Verbatim}[commandchars=\\\{\}]
\PY{n}{lst} \PY{o}{=} \PY{p}{[}\PY{l+m+mi}{1}\PY{p}{,} \PY{l+m+mi}{2}\PY{p}{,} \PY{l+m+mi}{3}\PY{p}{]}
\PY{n}{x} \PY{o}{=} \PY{n}{np}\PY{o}{.}\PY{n}{array}\PY{p}{(}\PY{n}{lst}\PY{p}{)}       \PY{c+c1}{\PYZsh{} given list of integers, creates integer array}
\PY{n}{x}\PY{o}{.}\PY{n}{dtype}                 \PY{c+c1}{\PYZsh{} prints np.int64}
\end{Verbatim}
\end{tcolorbox}

            \begin{tcolorbox}[breakable, size=fbox, boxrule=.5pt, pad at break*=1mm, opacityfill=0]
\prompt{Out}{outcolor}{80}{\boxspacing}
\begin{Verbatim}[commandchars=\\\{\}]
dtype('int64')
\end{Verbatim}
\end{tcolorbox}
        
    \begin{tcolorbox}[breakable, size=fbox, boxrule=1pt, pad at break*=1mm,colback=cellbackground, colframe=cellborder]
\prompt{In}{incolor}{81}{\boxspacing}
\begin{Verbatim}[commandchars=\\\{\}]
\PY{c+c1}{\PYZsh{} override inferred data type:}
\PY{c+c1}{\PYZsh{} created floating\PYZhy{}point array even if integers were given}
\PY{n}{x} \PY{o}{=} \PY{n}{np}\PY{o}{.}\PY{n}{array}\PY{p}{(}\PY{n}{lst}\PY{p}{,} \PY{n}{dtype}\PY{o}{=}\PY{n}{np}\PY{o}{.}\PY{n}{float64}\PY{p}{)}
\PY{n}{x}\PY{o}{.}\PY{n}{dtype}                 \PY{c+c1}{\PYZsh{} prints np.float64}
\end{Verbatim}
\end{tcolorbox}

            \begin{tcolorbox}[breakable, size=fbox, boxrule=.5pt, pad at break*=1mm, opacityfill=0]
\prompt{Out}{outcolor}{81}{\boxspacing}
\begin{Verbatim}[commandchars=\\\{\}]
dtype('float64')
\end{Verbatim}
\end{tcolorbox}
        
    \begin{tcolorbox}[breakable, size=fbox, boxrule=1pt, pad at break*=1mm,colback=cellbackground, colframe=cellborder]
\prompt{In}{incolor}{82}{\boxspacing}
\begin{Verbatim}[commandchars=\\\{\}]
\PY{c+c1}{\PYZsh{} override inferred data type:}
\PY{c+c1}{\PYZsh{} created integer array even if floats were given,}
\PY{c+c1}{\PYZsh{} thus truncating input data!}
\PY{n}{lst} \PY{o}{=} \PY{p}{[}\PY{l+m+mf}{1.234}\PY{p}{,} \PY{l+m+mf}{4.567}\PY{p}{,} \PY{l+m+mf}{6.789}\PY{p}{]}
\PY{n}{x} \PY{o}{=} \PY{n}{np}\PY{o}{.}\PY{n}{array}\PY{p}{(}\PY{n}{lst}\PY{p}{,} \PY{n}{dtype}\PY{o}{=}\PY{n}{np}\PY{o}{.}\PY{n}{int}\PY{p}{)}
\PY{n+nb}{print}\PY{p}{(}\PY{n}{x}\PY{p}{)}                \PY{c+c1}{\PYZsh{} prints [1, 4, 6]}
\PY{n}{x}\PY{o}{.}\PY{n}{dtype}                 \PY{c+c1}{\PYZsh{} prints np.int64}
\end{Verbatim}
\end{tcolorbox}

    \begin{Verbatim}[commandchars=\\\{\}]
[1 4 6]
    \end{Verbatim}

            \begin{tcolorbox}[breakable, size=fbox, boxrule=.5pt, pad at break*=1mm, opacityfill=0]
\prompt{Out}{outcolor}{82}{\boxspacing}
\begin{Verbatim}[commandchars=\\\{\}]
dtype('int64')
\end{Verbatim}
\end{tcolorbox}
        

\hypertarget{array-storage-order-advanced}{%
\subsection{Array storage order
(advanced)}\label{array-storage-order-advanced}}

Computer memory is linear, so a multi-dimensional array is mapped to a
one-dimensional block in memory. This can be done in two ways:

\begin{enumerate}
\def\labelenumi{\arabic{enumi}.}
\tightlist
\item
  NumPy uses the so-called \emph{row-major order} (also called \emph{C
  order}, because its the same as in C programming language)
\item
  This is exactly the opposite of Matlab, which uses \emph{column-major
  order} (also called \emph{F order}, because its the same as in the
  Fortran programming language)
\end{enumerate}

    \begin{tcolorbox}[breakable, size=fbox, boxrule=1pt, pad at break*=1mm,colback=cellbackground, colframe=cellborder]
\prompt{In}{incolor}{83}{\boxspacing}
\begin{Verbatim}[commandchars=\\\{\}]
\PY{k+kn}{import} \PY{n+nn}{numpy} \PY{k}{as} \PY{n+nn}{np}

\PY{n}{mat} \PY{o}{=} \PY{n}{np}\PY{o}{.}\PY{n}{arange}\PY{p}{(}\PY{l+m+mi}{6}\PY{p}{)}\PY{o}{.}\PY{n}{reshape}\PY{p}{(}\PY{p}{(}\PY{l+m+mi}{2}\PY{p}{,}\PY{l+m+mi}{3}\PY{p}{)}\PY{p}{)}
\PY{n}{mat}
\end{Verbatim}
\end{tcolorbox}

            \begin{tcolorbox}[breakable, size=fbox, boxrule=.5pt, pad at break*=1mm, opacityfill=0]
\prompt{Out}{outcolor}{83}{\boxspacing}
\begin{Verbatim}[commandchars=\\\{\}]
array([[0, 1, 2],
       [3, 4, 5]])
\end{Verbatim}
\end{tcolorbox}
        
    \begin{tcolorbox}[breakable, size=fbox, boxrule=1pt, pad at break*=1mm,colback=cellbackground, colframe=cellborder]
\prompt{In}{incolor}{84}{\boxspacing}
\begin{Verbatim}[commandchars=\\\{\}]
\PY{c+c1}{\PYZsh{} The matrix mat is stored in memory like this}
\PY{n}{mat}\PY{o}{.}\PY{n}{reshape}\PY{p}{(}\PY{o}{\PYZhy{}}\PY{l+m+mi}{1}\PY{p}{,} \PY{n}{order}\PY{o}{=}\PY{l+s+s1}{\PYZsq{}}\PY{l+s+s1}{C}\PY{l+s+s1}{\PYZsq{}}\PY{p}{)}
\end{Verbatim}
\end{tcolorbox}

            \begin{tcolorbox}[breakable, size=fbox, boxrule=.5pt, pad at break*=1mm, opacityfill=0]
\prompt{Out}{outcolor}{84}{\boxspacing}
\begin{Verbatim}[commandchars=\\\{\}]
array([0, 1, 2, 3, 4, 5])
\end{Verbatim}
\end{tcolorbox}
        
    \begin{tcolorbox}[breakable, size=fbox, boxrule=1pt, pad at break*=1mm,colback=cellbackground, colframe=cellborder]
\prompt{In}{incolor}{85}{\boxspacing}
\begin{Verbatim}[commandchars=\\\{\}]
\PY{c+c1}{\PYZsh{} ... and NOT like this}
\PY{n}{mat}\PY{o}{.}\PY{n}{reshape}\PY{p}{(}\PY{o}{\PYZhy{}}\PY{l+m+mi}{1}\PY{p}{,} \PY{n}{order}\PY{o}{=}\PY{l+s+s1}{\PYZsq{}}\PY{l+s+s1}{F}\PY{l+s+s1}{\PYZsq{}}\PY{p}{)}      \PY{c+c1}{\PYZsh{} use order=\PYZsq{}F\PYZsq{} to convert to column\PYZhy{}major storage order}
\end{Verbatim}
\end{tcolorbox}

            \begin{tcolorbox}[breakable, size=fbox, boxrule=.5pt, pad at break*=1mm, opacityfill=0]
\prompt{Out}{outcolor}{85}{\boxspacing}
\begin{Verbatim}[commandchars=\\\{\}]
array([0, 3, 1, 4, 2, 5])
\end{Verbatim}
\end{tcolorbox}
        
    While this is not particularly important initially, as an advanced user
you should remember that you never want to perform on non-contiguous
blocks of memory. This can have devastating effects on performance!

    \begin{tcolorbox}[breakable, size=fbox, boxrule=1pt, pad at break*=1mm,colback=cellbackground, colframe=cellborder]
\prompt{In}{incolor}{86}{\boxspacing}
\begin{Verbatim}[commandchars=\\\{\}]
\PY{c+c1}{\PYZsh{} Avoid operations on non\PYZhy{}contiguous array sections such as}
\PY{n}{mat}\PY{p}{[}\PY{p}{:}\PY{p}{,} \PY{l+m+mi}{1}\PY{p}{]}

\PY{c+c1}{\PYZsh{} Contiguous array sections are fine}
\PY{n}{mat}\PY{p}{[}\PY{l+m+mi}{1}\PY{p}{]}
\end{Verbatim}
\end{tcolorbox}

            \begin{tcolorbox}[breakable, size=fbox, boxrule=.5pt, pad at break*=1mm, opacityfill=0]
\prompt{Out}{outcolor}{86}{\boxspacing}
\begin{Verbatim}[commandchars=\\\{\}]
array([3, 4, 5])
\end{Verbatim}
\end{tcolorbox}
        

\hypertarget{exercises}{%
\subsection{Exercises}\label{exercises}}

    \hypertarget{exercise-1-broadcasting}{%
\subsubsection{Exercise 1: Broadcasting}\label{exercise-1-broadcasting}}

Let \texttt{m\ =\ 2}, \texttt{n\ =\ 3} and \texttt{k\ =\ 4}. Create an
array \texttt{a} with shape \texttt{(m,n)} like this:

\begin{verbatim}
a = np.arange(m*n).reshape((m,n))
\end{verbatim}

Perform the following tasks:

\begin{enumerate}
\def\labelenumi{\arabic{enumi}.}
\item
  Define the vector \texttt{b\ =\ np.arange(n)\ *\ 10} and use
  broadcasting to compute \texttt{c\ =\ a\ *\ b} such that \texttt{c}
  has shape \texttt{(m,n)}.
\item
  Define the vector \texttt{b\ =\ np.arange(m)\ *\ 10} and use
  broadcasting to compute \texttt{c\ =\ a\ *\ b} such that \texttt{c}
  has shape \texttt{(m,n)}.
\item
  Define the 3-dimensional array \texttt{b},

\begin{verbatim}
b = np.arange(m*n*k).reshape((m,k,n)) * 10
\end{verbatim}

  and use broadcasting to compute \texttt{c\ =\ a\ *\ b} such that
  \texttt{c} has shape \texttt{(m,k,n)}.
\end{enumerate}

    \hypertarget{exercise-2-boolean-indexing}{%
\subsubsection{Exercise 2: Boolean
indexing}\label{exercise-2-boolean-indexing}}

Let \texttt{m\ =\ 8} and \texttt{n\ =\ 9}. Create an array \texttt{a}
with shape \texttt{(m,n)} as follows:

\begin{verbatim}
a = np.arange(m*n).reshape((m,n)) % 7
\end{verbatim}

The \texttt{\%} is the modulo operator which returns the remainder of a
division of one number by another (in this case the division by 7). The
resulting array \texttt{a} will therefore contain integers between 0 and
6.

\begin{enumerate}
\def\labelenumi{\arabic{enumi}.}
\item
  Create a boolean array called \texttt{mask} which has the same shape
  as \texttt{a} and is \texttt{True} whenever an element in \texttt{a}
  is between 1 and 4 (inclusive).

  \emph{Hint:} The character \texttt{\&} works as a logical \texttt{and}
  operator for NumPy arrays. Alternatively, you can use the function
  \texttt{np.logical\_and()}.
\item
  Compute the number of elements in \texttt{a} that satisfy this
  criterion.
\item
  Compute the average of these elements.
\end{enumerate}

    \hypertarget{exercise-3-diagonal-matrices}{%
\subsubsection{Exercise 3: Diagonal
matrices}\label{exercise-3-diagonal-matrices}}

In this exercise, we'll create diagonal matrices using integer array
indexing.

\begin{enumerate}
\def\labelenumi{\arabic{enumi}.}
\item
  Create a square matrix of zeros with shape \texttt{(n,n)} for
  \texttt{n\ =\ 5} and \texttt{dtype\ =\ int} as its data type:
  \texttt{a\ =\ np.zeros((n,n),\ dtype=int)} Use integer array indexing
  to modify its diagonal to construct the following matrix, where
  omitted elements are zero: \[
  \begin{bmatrix}
  1 &   &   &   & \\
    & 2 &   &   & \\
    &   & 3 &   & \\
    &   &   & 4 & \\
    &   &   &   & 5     
  \end{bmatrix}
  \]
\item
  Repeat the exercise, but now use a non-squared matrix with shape
  \texttt{(4,5)}, and insert \texttt{1,2,...} as the values of the first
  diagonal above the main diagonal: \[
  \begin{bmatrix}
   0 & 1 &   &   &   \\
   &   & 2 &   &   \\
   &   &   & 3 &   \\
   &   &   &   & 4 \\
  \end{bmatrix}
  \]
\item
  Repeat the exercise, but now use a matrix with shape \texttt{(6,5)}.
  Adapt your code so that it can handle matrix shapes \texttt{(m,n)} for
  cases \texttt{n\ \textgreater{}\ m}, \texttt{n\ =\ m} and
  \texttt{n\ \textless{}\ m}: \[
  \begin{bmatrix}
   0 & 1 &   &   &   \\
   &   & 2 &   &   \\
   &   &   & 3 &   \\
   &   &   &   & 4 \\
   &&&& \\
   &&&&
  \end{bmatrix}
  \]
\end{enumerate}

    \hypertarget{exercise-4-triangular-matrices}{%
\subsubsection{Exercise 4: Triangular
matrices}\label{exercise-4-triangular-matrices}}

Create a matrix of zeros with shape \texttt{(m,n)}, with
\texttt{m\ =\ 4}, \texttt{n\ =\ 5}, and \texttt{dtype\ =\ int}:

\begin{verbatim}
a = np.zeros((m,n), dtype=int)
\end{verbatim}

Transform it to an upper-triangular matrix so that it looks like this:
\[
\begin{bmatrix}
1 & 2 & 3 & 4 & 5 \\
 & 6 & 7 & 8 & 9 \\
  & & 10 & 11 & 12 \\
  & & & 13 & 14
\end{bmatrix}
\] where the omitted elements are zeros. Do this without using loops.

\emph{Hint:} For any upper-diagonal element at position \((i,j)\) it
holds that \(j \geq i\). Create arrays of row and column indices and use
these to build a mask with shape \texttt{(m,n)} which selects all
upper-triangular elements. Exploit the fact that logical array
operations support broadcasting!

\emph{Challenge:} Modify your code to construct the following
upper-triangular matrix instead: \[
\begin{bmatrix}
1 & 2 & 4 & 7 & 11 \\
  & 3 & 5 & 8 & 12 \\
  &   & 6 & 9 & 13 \\
  &   &   & 10 & 14 \\
\end{bmatrix}
\]

\emph{Hint:} One solution is to build a lower-triangular matrix and
transpose it!

    \hypertarget{exercise-5-row-averages-ignoring-nans}{%
\subsubsection{Exercise 5: Row averages (ignoring
NaNs)}\label{exercise-5-row-averages-ignoring-nans}}

Construct a matrix with shape \texttt{(m,n)} where \texttt{m\ =\ 5} and
\texttt{n\ =\ 8} as follows:

\begin{verbatim}
a = np.arange(m*n, dtype=np.float).reshape((m, n))
mask = (a % 5) == 0
a[mask] = np.nan
\end{verbatim}

Lines two and three set all elements of \texttt{a} which are divisible
by 5 without remainder to \texttt{np.nan}, the floating-point value
signalling that something is ``not a number'' (NaN).

Write code to perform the following tasks:

\begin{enumerate}
\def\labelenumi{\arabic{enumi}.}
\item
  Define a function \texttt{rowmeans(x)} which takes a matrix \texttt{x}
  as an argument and returns a vector of row averages of elements in
  \texttt{x} which are not NaN. The return vector should therefore have
  the same length as the number of rows in \texttt{x}.

  \emph{Hint:} Use \texttt{np.isnan()} to check whether something is
  NaN.
\item
  Compare your results to the output of \texttt{np.nanmean()} with
  argument \texttt{axis\ =\ 1}.
\item
  Use the \texttt{\%timeit} magic to benchmark the run time of your
  routine against the (vectorised) \texttt{np.nanmean()}.
\end{enumerate}

    \hypertarget{exercise-6-locating-maxima-advanced}{%
\subsubsection{Exercise 6: Locating maxima
(advanced)}\label{exercise-6-locating-maxima-advanced}}

Consider the following quadratic polynomial in \(x\) which is
parametrised by the positive real numbers \(a\), \(b\) and \(c\): \[
p(x;a,b,c) = -a (x-b)^2 + c
\]

Imagine that we have a set of \texttt{m\ =\ 10} such functions, each
with different values for \(a\), \(b\) and \(c\). These parameters take
on the following values:

\begin{verbatim}
m = 10

a = np.linspace(0.4, 2.0, m)
b = np.linspace(-1.0, 0.0, m)
c = np.linspace(0.0, 3.0, m)
\end{verbatim}

The parameters for the first polynomial are thus \texttt{a{[}0{]}},
\texttt{b{[}0{]}} and \texttt{c{[}0{]}}, and similarly for the remaining
polynomials.

Perform the following tasks:

\begin{enumerate}
\def\labelenumi{\arabic{enumi}.}
\tightlist
\item
  Evaluate each polynomial on an equidistant grid of \texttt{n\ =\ 50}
  points on the interval \texttt{{[}-2.0,\ 2.0{]}}. Store the results as
  an \(m \times n\) matrix called \texttt{pvalues}.
\item
  Create a (single) graph which plots all \texttt{m} polynomials.
\item
  Write a function \texttt{find\_max(pvalues)} which accepts this matrix
  as an argument and returns an integer array of length \texttt{m}. Each
  element \texttt{i} should contain the location of the maximum for the
  \texttt{i}-th row of \texttt{pvalues}.
\item
  Add the maxima you found to the graph you created: each maximum should
  be marked as a black dot at the correct \((x,y)\) coordinates.
\item
  Use vectorised code to perform the same task as the function
  \texttt{find\_max()}:

  \begin{enumerate}
  \def\labelenumii{\arabic{enumii}.}
  \tightlist
  \item
    You actually don't have to do anything here since NumPy implements
    the function \texttt{np.argmax()} which does what you need (just
    correctly specify the \texttt{axis} argument.)
  \item
    Use the \texttt{\%timeit} magic to compare the runtime of your
    \texttt{find\_max()} to NumPy's \texttt{np.argmax()}.
  \end{enumerate}
\end{enumerate}


\hypertarget{solutions}{%
\subsection{Solutions}\label{solutions}}

    \hypertarget{solution-for-exercise-1}{%
\subsubsection{Solution for exercise 1}\label{solution-for-exercise-1}}

To solve this exercise, we need to make sure that the broadcasting rules
can be applied by inserting an additional axis in either \texttt{a} or
\texttt{b} as needed.

    \begin{tcolorbox}[breakable, size=fbox, boxrule=1pt, pad at break*=1mm,colback=cellbackground, colframe=cellborder]
\prompt{In}{incolor}{87}{\boxspacing}
\begin{Verbatim}[commandchars=\\\{\}]
\PY{c+c1}{\PYZsh{} Define problem dimensions}
\PY{n}{m} \PY{o}{=} \PY{l+m+mi}{2}
\PY{n}{n} \PY{o}{=} \PY{l+m+mi}{3}
\PY{n}{k} \PY{o}{=} \PY{l+m+mi}{4}

\PY{c+c1}{\PYZsh{} Create array a}
\PY{n}{a} \PY{o}{=} \PY{n}{np}\PY{o}{.}\PY{n}{arange}\PY{p}{(}\PY{l+m+mi}{6}\PY{p}{)}\PY{o}{.}\PY{n}{reshape}\PY{p}{(}\PY{p}{(}\PY{n}{m}\PY{p}{,} \PY{n}{n}\PY{p}{)}\PY{p}{)}
\PY{n}{a}
\end{Verbatim}
\end{tcolorbox}

            \begin{tcolorbox}[breakable, size=fbox, boxrule=.5pt, pad at break*=1mm, opacityfill=0]
\prompt{Out}{outcolor}{87}{\boxspacing}
\begin{Verbatim}[commandchars=\\\{\}]
array([[0, 1, 2],
       [3, 4, 5]])
\end{Verbatim}
\end{tcolorbox}
        
    \begin{tcolorbox}[breakable, size=fbox, boxrule=1pt, pad at break*=1mm,colback=cellbackground, colframe=cellborder]
\prompt{In}{incolor}{88}{\boxspacing}
\begin{Verbatim}[commandchars=\\\{\}]
\PY{c+c1}{\PYZsh{} Task 1}
\PY{n}{b} \PY{o}{=} \PY{n}{np}\PY{o}{.}\PY{n}{arange}\PY{p}{(}\PY{n}{n}\PY{p}{)} \PY{o}{*} \PY{l+m+mi}{10}
\PY{c+c1}{\PYZsh{} Broadcasing works as is, axis will be prepended to b}
\PY{n}{c} \PY{o}{=} \PY{n}{a} \PY{o}{*} \PY{n}{b}

\PY{c+c1}{\PYZsh{} If we really want, we can prepend new axis manually:}
\PY{n}{c} \PY{o}{=} \PY{n}{a} \PY{o}{*} \PY{n}{b}\PY{p}{[}\PY{k+kc}{None}\PY{p}{]}
\PY{n}{c}
\end{Verbatim}
\end{tcolorbox}

            \begin{tcolorbox}[breakable, size=fbox, boxrule=.5pt, pad at break*=1mm, opacityfill=0]
\prompt{Out}{outcolor}{88}{\boxspacing}
\begin{Verbatim}[commandchars=\\\{\}]
array([[  0,  10,  40],
       [  0,  40, 100]])
\end{Verbatim}
\end{tcolorbox}
        
    \begin{tcolorbox}[breakable, size=fbox, boxrule=1pt, pad at break*=1mm,colback=cellbackground, colframe=cellborder]
\prompt{In}{incolor}{89}{\boxspacing}
\begin{Verbatim}[commandchars=\\\{\}]
\PY{c+c1}{\PYZsh{} Task 2}
\PY{n}{b} \PY{o}{=} \PY{n}{np}\PY{o}{.}\PY{n}{arange}\PY{p}{(}\PY{n}{m}\PY{p}{)} \PY{o}{*} \PY{l+m+mi}{10}
\PY{c+c1}{\PYZsh{} Manually append new axis to b to make broadcasing work}
\PY{n}{c} \PY{o}{=} \PY{n}{a} \PY{o}{*} \PY{n}{b}\PY{p}{[}\PY{p}{:}\PY{p}{,}\PY{k+kc}{None}\PY{p}{]}
\PY{n}{c}
\end{Verbatim}
\end{tcolorbox}

            \begin{tcolorbox}[breakable, size=fbox, boxrule=.5pt, pad at break*=1mm, opacityfill=0]
\prompt{Out}{outcolor}{89}{\boxspacing}
\begin{Verbatim}[commandchars=\\\{\}]
array([[ 0,  0,  0],
       [30, 40, 50]])
\end{Verbatim}
\end{tcolorbox}
        
    \begin{tcolorbox}[breakable, size=fbox, boxrule=1pt, pad at break*=1mm,colback=cellbackground, colframe=cellborder]
\prompt{In}{incolor}{90}{\boxspacing}
\begin{Verbatim}[commandchars=\\\{\}]
\PY{c+c1}{\PYZsh{} Task 3}
\PY{n}{b} \PY{o}{=} \PY{n}{np}\PY{o}{.}\PY{n}{arange}\PY{p}{(}\PY{n}{n}\PY{o}{*}\PY{n}{m}\PY{o}{*}\PY{n}{k}\PY{p}{)}\PY{o}{.}\PY{n}{reshape}\PY{p}{(}\PY{p}{(}\PY{n}{m}\PY{p}{,} \PY{n}{k}\PY{p}{,} \PY{n}{n}\PY{p}{)}\PY{p}{)} \PY{o}{*} \PY{l+m+mi}{10}
\PY{n}{b}
\end{Verbatim}
\end{tcolorbox}

            \begin{tcolorbox}[breakable, size=fbox, boxrule=.5pt, pad at break*=1mm, opacityfill=0]
\prompt{Out}{outcolor}{90}{\boxspacing}
\begin{Verbatim}[commandchars=\\\{\}]
array([[[  0,  10,  20],
        [ 30,  40,  50],
        [ 60,  70,  80],
        [ 90, 100, 110]],

       [[120, 130, 140],
        [150, 160, 170],
        [180, 190, 200],
        [210, 220, 230]]])
\end{Verbatim}
\end{tcolorbox}
        
    \begin{tcolorbox}[breakable, size=fbox, boxrule=1pt, pad at break*=1mm,colback=cellbackground, colframe=cellborder]
\prompt{In}{incolor}{91}{\boxspacing}
\begin{Verbatim}[commandchars=\\\{\}]
\PY{c+c1}{\PYZsh{} Need to insert new axis inbetween existing axes in a}
\PY{n}{c} \PY{o}{=} \PY{n}{a}\PY{p}{[}\PY{p}{:}\PY{p}{,}\PY{k+kc}{None}\PY{p}{]} \PY{o}{*} \PY{n}{b}
\PY{n}{c}
\end{Verbatim}
\end{tcolorbox}

            \begin{tcolorbox}[breakable, size=fbox, boxrule=.5pt, pad at break*=1mm, opacityfill=0]
\prompt{Out}{outcolor}{91}{\boxspacing}
\begin{Verbatim}[commandchars=\\\{\}]
array([[[   0,   10,   40],
        [   0,   40,  100],
        [   0,   70,  160],
        [   0,  100,  220]],

       [[ 360,  520,  700],
        [ 450,  640,  850],
        [ 540,  760, 1000],
        [ 630,  880, 1150]]])
\end{Verbatim}
\end{tcolorbox}
        
    \hypertarget{solution-for-exercise-2}{%
\subsubsection{Solution for exercise 2}\label{solution-for-exercise-2}}

    \begin{tcolorbox}[breakable, size=fbox, boxrule=1pt, pad at break*=1mm,colback=cellbackground, colframe=cellborder]
\prompt{In}{incolor}{92}{\boxspacing}
\begin{Verbatim}[commandchars=\\\{\}]
\PY{k+kn}{import} \PY{n+nn}{numpy} \PY{k}{as} \PY{n+nn}{np}

\PY{c+c1}{\PYZsh{} Array dimensions}
\PY{n}{m} \PY{o}{=} \PY{l+m+mi}{8}
\PY{n}{n} \PY{o}{=} \PY{l+m+mi}{9}

\PY{c+c1}{\PYZsh{} Create array a}
\PY{n}{a} \PY{o}{=} \PY{n}{np}\PY{o}{.}\PY{n}{arange}\PY{p}{(}\PY{n}{m}\PY{o}{*}\PY{n}{n}\PY{p}{)}\PY{o}{.}\PY{n}{reshape}\PY{p}{(}\PY{p}{(}\PY{n}{m}\PY{p}{,}\PY{n}{n}\PY{p}{)}\PY{p}{)} \PY{o}{\PYZpc{}} \PY{l+m+mi}{7}

\PY{c+c1}{\PYZsh{} Create mask that selects all elements }
\PY{c+c1}{\PYZsh{} between 1 and 4}
\PY{n}{mask} \PY{o}{=} \PY{p}{(}\PY{n}{a} \PY{o}{\PYZgt{}} \PY{l+m+mi}{0}\PY{p}{)} \PY{o}{\PYZam{}} \PY{p}{(}\PY{n}{a} \PY{o}{\PYZlt{}} \PY{l+m+mi}{5}\PY{p}{)}    \PY{c+c1}{\PYZsh{} same as np.logical\PYZus{}and(a \PYZgt{} 0, a \PYZlt{} 5)}
\PY{n}{mask}
\end{Verbatim}
\end{tcolorbox}

            \begin{tcolorbox}[breakable, size=fbox, boxrule=.5pt, pad at break*=1mm, opacityfill=0]
\prompt{Out}{outcolor}{92}{\boxspacing}
\begin{Verbatim}[commandchars=\\\{\}]
array([[False,  True,  True,  True,  True, False, False, False,  True],
       [ True,  True,  True, False, False, False,  True,  True,  True],
       [ True, False, False, False,  True,  True,  True,  True, False],
       [False, False,  True,  True,  True,  True, False, False, False],
       [ True,  True,  True,  True, False, False, False,  True,  True],
       [ True,  True, False, False, False,  True,  True,  True,  True],
       [False, False, False,  True,  True,  True,  True, False, False],
       [False,  True,  True,  True,  True, False, False, False,  True]])
\end{Verbatim}
\end{tcolorbox}
        
    We use \texttt{\&} to obtain the set of elements for which the
conditions \texttt{a\ \textgreater{}\ 0} and \texttt{a\ \textless{}\ 5}
are \texttt{True} at the same time.

Boolean arrays only contain values \texttt{False} and \texttt{True}.
However, arithmetic operations automatically interpret \texttt{False} as
0 and \texttt{True} as 1, so we can simply sum over the array to obtain
the number of elements that are \texttt{True}:

    \begin{tcolorbox}[breakable, size=fbox, boxrule=1pt, pad at break*=1mm,colback=cellbackground, colframe=cellborder]
\prompt{In}{incolor}{93}{\boxspacing}
\begin{Verbatim}[commandchars=\\\{\}]
\PY{c+c1}{\PYZsh{} Count number of elements}
\PY{n}{nmask} \PY{o}{=} \PY{n}{np}\PY{o}{.}\PY{n}{sum}\PY{p}{(}\PY{n}{mask}\PY{p}{)}
\PY{n}{nmask}
\end{Verbatim}
\end{tcolorbox}

            \begin{tcolorbox}[breakable, size=fbox, boxrule=.5pt, pad at break*=1mm, opacityfill=0]
\prompt{Out}{outcolor}{93}{\boxspacing}
\begin{Verbatim}[commandchars=\\\{\}]
41
\end{Verbatim}
\end{tcolorbox}
        
    Finally, we compute the average as follows:

    \begin{tcolorbox}[breakable, size=fbox, boxrule=1pt, pad at break*=1mm,colback=cellbackground, colframe=cellborder]
\prompt{In}{incolor}{94}{\boxspacing}
\begin{Verbatim}[commandchars=\\\{\}]
\PY{c+c1}{\PYZsh{} Compute average of elements selected by mask}
\PY{c+c1}{\PYZsh{} We do this by summing over selected elements and}
\PY{c+c1}{\PYZsh{} dividing by the number of such elements.}
\PY{n}{mean} \PY{o}{=} \PY{n}{np}\PY{o}{.}\PY{n}{sum}\PY{p}{(}\PY{n}{a}\PY{p}{[}\PY{n}{mask}\PY{p}{]}\PY{p}{)} \PY{o}{/} \PY{n}{nmask}
\PY{n}{mean}
\end{Verbatim}
\end{tcolorbox}

            \begin{tcolorbox}[breakable, size=fbox, boxrule=.5pt, pad at break*=1mm, opacityfill=0]
\prompt{Out}{outcolor}{94}{\boxspacing}
\begin{Verbatim}[commandchars=\\\{\}]
2.4634146341463414
\end{Verbatim}
\end{tcolorbox}
        
    \hypertarget{solution-for-exercise-3}{%
\subsubsection{Solution for exercise 3}\label{solution-for-exercise-3}}

    \begin{tcolorbox}[breakable, size=fbox, boxrule=1pt, pad at break*=1mm,colback=cellbackground, colframe=cellborder]
\prompt{In}{incolor}{95}{\boxspacing}
\begin{Verbatim}[commandchars=\\\{\}]
\PY{k+kn}{import} \PY{n+nn}{numpy} \PY{k}{as} \PY{n+nn}{np} 

\PY{c+c1}{\PYZsh{} Square matrix with 1,2,...n diagonal}

\PY{n}{n} \PY{o}{=} \PY{l+m+mi}{5}
\PY{c+c1}{\PYZsh{} Create matrix of zeros}
\PY{n}{mat} \PY{o}{=} \PY{n}{np}\PY{o}{.}\PY{n}{zeros}\PY{p}{(}\PY{p}{(}\PY{n}{n}\PY{p}{,}\PY{n}{n}\PY{p}{)}\PY{p}{,} \PY{n}{dtype}\PY{o}{=}\PY{n+nb}{int}\PY{p}{)}

\PY{c+c1}{\PYZsh{} Create row and column indices for the diagonal elements}
\PY{n}{irow} \PY{o}{=} \PY{n}{np}\PY{o}{.}\PY{n}{arange}\PY{p}{(}\PY{n}{n}\PY{p}{)}
\PY{n}{icol} \PY{o}{=} \PY{n}{irow}             \PY{c+c1}{\PYZsh{} column indices are the same since }
                        \PY{c+c1}{\PYZsh{} this is the diagonal of a square matrix}

\PY{c+c1}{\PYZsh{} Vector to insert as diagonal}
\PY{n}{diag} \PY{o}{=} \PY{n}{np}\PY{o}{.}\PY{n}{arange}\PY{p}{(}\PY{l+m+mi}{1}\PY{p}{,} \PY{n}{n}\PY{o}{+}\PY{l+m+mi}{1}\PY{p}{)}

\PY{c+c1}{\PYZsh{} Set new diagonal}
\PY{n}{mat}\PY{p}{[}\PY{n}{irow}\PY{p}{,} \PY{n}{icol}\PY{p}{]} \PY{o}{=} \PY{n}{diag}

\PY{n}{mat}         \PY{c+c1}{\PYZsh{} print result}
\end{Verbatim}
\end{tcolorbox}

            \begin{tcolorbox}[breakable, size=fbox, boxrule=.5pt, pad at break*=1mm, opacityfill=0]
\prompt{Out}{outcolor}{95}{\boxspacing}
\begin{Verbatim}[commandchars=\\\{\}]
array([[1, 0, 0, 0, 0],
       [0, 2, 0, 0, 0],
       [0, 0, 3, 0, 0],
       [0, 0, 0, 4, 0],
       [0, 0, 0, 0, 5]])
\end{Verbatim}
\end{tcolorbox}
        
    You can achieve the same result with NumPy's \texttt{np.diag()}
function:

    \begin{tcolorbox}[breakable, size=fbox, boxrule=1pt, pad at break*=1mm,colback=cellbackground, colframe=cellborder]
\prompt{In}{incolor}{96}{\boxspacing}
\begin{Verbatim}[commandchars=\\\{\}]
\PY{n}{np}\PY{o}{.}\PY{n}{diag}\PY{p}{(}\PY{n}{np}\PY{o}{.}\PY{n}{arange}\PY{p}{(}\PY{n}{n}\PY{p}{)} \PY{o}{+} \PY{l+m+mi}{1}\PY{p}{)}
\end{Verbatim}
\end{tcolorbox}

            \begin{tcolorbox}[breakable, size=fbox, boxrule=.5pt, pad at break*=1mm, opacityfill=0]
\prompt{Out}{outcolor}{96}{\boxspacing}
\begin{Verbatim}[commandchars=\\\{\}]
array([[1, 0, 0, 0, 0],
       [0, 2, 0, 0, 0],
       [0, 0, 3, 0, 0],
       [0, 0, 0, 4, 0],
       [0, 0, 0, 0, 5]])
\end{Verbatim}
\end{tcolorbox}
        
    For task 2, we need to specify row and column index arrays that are no
longer identical:

    \begin{tcolorbox}[breakable, size=fbox, boxrule=1pt, pad at break*=1mm,colback=cellbackground, colframe=cellborder]
\prompt{In}{incolor}{97}{\boxspacing}
\begin{Verbatim}[commandchars=\\\{\}]
\PY{c+c1}{\PYZsh{} Task 2: set upper diagonal elements of non\PYZhy{}square matrix}
\PY{k+kn}{import} \PY{n+nn}{numpy} \PY{k}{as} \PY{n+nn}{np}

\PY{n}{m} \PY{o}{=} \PY{l+m+mi}{4}
\PY{n}{n} \PY{o}{=} \PY{l+m+mi}{5}

\PY{c+c1}{\PYZsh{} Create matrix of zeros}
\PY{n}{mat} \PY{o}{=} \PY{n}{np}\PY{o}{.}\PY{n}{zeros}\PY{p}{(}\PY{p}{(}\PY{n}{m}\PY{p}{,} \PY{n}{n}\PY{p}{)}\PY{p}{,} \PY{n}{dtype}\PY{o}{=}\PY{n+nb}{int}\PY{p}{)}

\PY{c+c1}{\PYZsh{} Row and column indices}
\PY{n}{irow} \PY{o}{=} \PY{n}{np}\PY{o}{.}\PY{n}{arange}\PY{p}{(}\PY{n}{m}\PY{p}{)}
\PY{n}{icol} \PY{o}{=} \PY{n}{np}\PY{o}{.}\PY{n}{arange}\PY{p}{(}\PY{n}{n}\PY{o}{\PYZhy{}}\PY{l+m+mi}{1}\PY{p}{)} \PY{o}{+} \PY{l+m+mi}{1}

\PY{n}{diag} \PY{o}{=} \PY{n}{np}\PY{o}{.}\PY{n}{arange}\PY{p}{(}\PY{n}{m}\PY{p}{)} \PY{o}{+} \PY{l+m+mi}{1}

\PY{c+c1}{\PYZsh{} Set new diagonal}
\PY{n}{mat}\PY{p}{[}\PY{n}{irow}\PY{p}{,} \PY{n}{icol}\PY{p}{]} \PY{o}{=} \PY{n}{diag}

\PY{n}{mat}                 \PY{c+c1}{\PYZsh{} print result}
\end{Verbatim}
\end{tcolorbox}

            \begin{tcolorbox}[breakable, size=fbox, boxrule=.5pt, pad at break*=1mm, opacityfill=0]
\prompt{Out}{outcolor}{97}{\boxspacing}
\begin{Verbatim}[commandchars=\\\{\}]
array([[0, 1, 0, 0, 0],
       [0, 0, 2, 0, 0],
       [0, 0, 0, 3, 0],
       [0, 0, 0, 0, 4]])
\end{Verbatim}
\end{tcolorbox}
        
    We now create a more generic version of the code above that can handle
the cases \texttt{n\ \textgreater{}\ m}, \texttt{n\ =\ m} and
\texttt{n\ \textless{}\ m}. The above code will fail for
\texttt{m\ \textgreater{}\ n}.

    \begin{tcolorbox}[breakable, size=fbox, boxrule=1pt, pad at break*=1mm,colback=cellbackground, colframe=cellborder]
\prompt{In}{incolor}{98}{\boxspacing}
\begin{Verbatim}[commandchars=\\\{\}]
\PY{k+kn}{import} \PY{n+nn}{numpy} \PY{k}{as} \PY{n+nn}{np}

\PY{n}{m} \PY{o}{=} \PY{l+m+mi}{6}
\PY{n}{n} \PY{o}{=} \PY{l+m+mi}{5}

\PY{c+c1}{\PYZsh{} Create matrix of zeros}
\PY{n}{mat} \PY{o}{=} \PY{n}{np}\PY{o}{.}\PY{n}{zeros}\PY{p}{(}\PY{p}{(}\PY{n}{m}\PY{p}{,} \PY{n}{n}\PY{p}{)}\PY{p}{,} \PY{n}{dtype}\PY{o}{=}\PY{n+nb}{int}\PY{p}{)}

\PY{c+c1}{\PYZsh{} Row indices: this is the min. of the number of rows,}
\PY{c+c1}{\PYZsh{}       and the number of cols\PYZhy{}1, since the first col}
\PY{c+c1}{\PYZsh{}       is newer included.}
\PY{n}{irow} \PY{o}{=} \PY{n}{np}\PY{o}{.}\PY{n}{arange}\PY{p}{(}\PY{n+nb}{min}\PY{p}{(}\PY{n}{m}\PY{p}{,}\PY{n}{n}\PY{o}{\PYZhy{}}\PY{l+m+mi}{1}\PY{p}{)}\PY{p}{)}

\PY{c+c1}{\PYZsh{} Col indices: array needs to have same length as}
\PY{c+c1}{\PYZsh{} row indices. First col is omitted so we }
\PY{c+c1}{\PYZsh{} need to shift all indices by 1.}
\PY{n}{icol} \PY{o}{=} \PY{n}{np}\PY{o}{.}\PY{n}{arange}\PY{p}{(}\PY{n+nb}{len}\PY{p}{(}\PY{n}{irow}\PY{p}{)}\PY{p}{)} \PY{o}{+} \PY{l+m+mi}{1}

\PY{c+c1}{\PYZsh{} Number of elements to insert needs to be identical}
\PY{c+c1}{\PYZsh{} to length of array index.}
\PY{n}{diag} \PY{o}{=} \PY{n}{np}\PY{o}{.}\PY{n}{arange}\PY{p}{(}\PY{n+nb}{len}\PY{p}{(}\PY{n}{irow}\PY{p}{)}\PY{p}{)} \PY{o}{+} \PY{l+m+mi}{1}

\PY{c+c1}{\PYZsh{} Set new diagonal}
\PY{n}{mat}\PY{p}{[}\PY{n}{irow}\PY{p}{,} \PY{n}{icol}\PY{p}{]} \PY{o}{=} \PY{n}{diag}

\PY{n}{mat}                 \PY{c+c1}{\PYZsh{} print result}
\end{Verbatim}
\end{tcolorbox}

            \begin{tcolorbox}[breakable, size=fbox, boxrule=.5pt, pad at break*=1mm, opacityfill=0]
\prompt{Out}{outcolor}{98}{\boxspacing}
\begin{Verbatim}[commandchars=\\\{\}]
array([[0, 1, 0, 0, 0],
       [0, 0, 2, 0, 0],
       [0, 0, 0, 3, 0],
       [0, 0, 0, 0, 4],
       [0, 0, 0, 0, 0],
       [0, 0, 0, 0, 0]])
\end{Verbatim}
\end{tcolorbox}
        
    For the special case of \texttt{m\ =\ n}, NumPy's \texttt{np.diag()} can
also insert values at diagonals other than the main diagonal. For
example, for \texttt{m\ =\ n\ =\ 5}, we can create a similar matrix as
follows:

    \begin{tcolorbox}[breakable, size=fbox, boxrule=1pt, pad at break*=1mm,colback=cellbackground, colframe=cellborder]
\prompt{In}{incolor}{99}{\boxspacing}
\begin{Verbatim}[commandchars=\\\{\}]
\PY{n}{n} \PY{o}{=} \PY{l+m+mi}{5}
\PY{c+c1}{\PYZsh{} the k argument tells diag() which diagonal to manipulate.}
\PY{c+c1}{\PYZsh{} k = 1 corresponds to first diagonal above}
\PY{c+c1}{\PYZsh{} the main diagonal.}
\PY{n}{np}\PY{o}{.}\PY{n}{diag}\PY{p}{(}\PY{n}{np}\PY{o}{.}\PY{n}{arange}\PY{p}{(}\PY{n}{n}\PY{o}{\PYZhy{}}\PY{l+m+mi}{1}\PY{p}{)}\PY{o}{+}\PY{l+m+mi}{1}\PY{p}{,} \PY{n}{k}\PY{o}{=}\PY{l+m+mi}{1}\PY{p}{)}
\end{Verbatim}
\end{tcolorbox}

            \begin{tcolorbox}[breakable, size=fbox, boxrule=.5pt, pad at break*=1mm, opacityfill=0]
\prompt{Out}{outcolor}{99}{\boxspacing}
\begin{Verbatim}[commandchars=\\\{\}]
array([[0, 1, 0, 0, 0],
       [0, 0, 2, 0, 0],
       [0, 0, 0, 3, 0],
       [0, 0, 0, 0, 4],
       [0, 0, 0, 0, 0]])
\end{Verbatim}
\end{tcolorbox}
        
    \hypertarget{solution-for-exercise-4}{%
\subsubsection{Solution for exercise 4}\label{solution-for-exercise-4}}

    \begin{tcolorbox}[breakable, size=fbox, boxrule=1pt, pad at break*=1mm,colback=cellbackground, colframe=cellborder]
\prompt{In}{incolor}{100}{\boxspacing}
\begin{Verbatim}[commandchars=\\\{\}]
\PY{n}{m} \PY{o}{=} \PY{l+m+mi}{4}
\PY{n}{n} \PY{o}{=} \PY{l+m+mi}{5}

\PY{c+c1}{\PYZsh{} Create arrays of valid row and column indices.}
\PY{n}{irow} \PY{o}{=} \PY{n}{np}\PY{o}{.}\PY{n}{arange}\PY{p}{(}\PY{n}{m}\PY{p}{)}
\PY{n}{icol} \PY{o}{=} \PY{n}{np}\PY{o}{.}\PY{n}{arange}\PY{p}{(}\PY{n}{n}\PY{p}{)}

\PY{c+c1}{\PYZsh{} Use broadcasting to idetify all upper\PYZhy{}triangular elements}
\PY{n}{mask} \PY{o}{=} \PY{n}{icol}\PY{p}{[}\PY{k+kc}{None}\PY{p}{]} \PY{o}{\PYZgt{}}\PY{o}{=} \PY{n}{irow}\PY{p}{[}\PY{p}{:}\PY{p}{,}\PY{k+kc}{None}\PY{p}{]}
\PY{n}{mask}
\end{Verbatim}
\end{tcolorbox}

            \begin{tcolorbox}[breakable, size=fbox, boxrule=.5pt, pad at break*=1mm, opacityfill=0]
\prompt{Out}{outcolor}{100}{\boxspacing}
\begin{Verbatim}[commandchars=\\\{\}]
array([[ True,  True,  True,  True,  True],
       [False,  True,  True,  True,  True],
       [False, False,  True,  True,  True],
       [False, False, False,  True,  True]])
\end{Verbatim}
\end{tcolorbox}
        
    \begin{tcolorbox}[breakable, size=fbox, boxrule=1pt, pad at break*=1mm,colback=cellbackground, colframe=cellborder]
\prompt{In}{incolor}{101}{\boxspacing}
\begin{Verbatim}[commandchars=\\\{\}]
\PY{c+c1}{\PYZsh{} Create matrix of zeros}
\PY{n}{mat} \PY{o}{=} \PY{n}{np}\PY{o}{.}\PY{n}{zeros}\PY{p}{(}\PY{p}{(}\PY{n}{m}\PY{p}{,} \PY{n}{n}\PY{p}{)}\PY{p}{,} \PY{n}{dtype}\PY{o}{=}\PY{n+nb}{int}\PY{p}{)}

\PY{c+c1}{\PYZsh{} count number of True values in mask}
\PY{n}{ntrue} \PY{o}{=} \PY{n}{np}\PY{o}{.}\PY{n}{sum}\PY{p}{(}\PY{n}{mask}\PY{p}{)}

\PY{c+c1}{\PYZsh{} Values to insert into upper\PYZhy{}triangular part:}
\PY{c+c1}{\PYZsh{} 1, 2, ... }
\PY{n}{values} \PY{o}{=} \PY{n}{np}\PY{o}{.}\PY{n}{arange}\PY{p}{(}\PY{n}{ntrue}\PY{p}{)} \PY{o}{+} \PY{l+m+mi}{1}

\PY{n}{mat}\PY{p}{[}\PY{n}{mask}\PY{p}{]} \PY{o}{=} \PY{n}{values}
\PY{n}{mat}
\end{Verbatim}
\end{tcolorbox}

            \begin{tcolorbox}[breakable, size=fbox, boxrule=.5pt, pad at break*=1mm, opacityfill=0]
\prompt{Out}{outcolor}{101}{\boxspacing}
\begin{Verbatim}[commandchars=\\\{\}]
array([[ 1,  2,  3,  4,  5],
       [ 0,  6,  7,  8,  9],
       [ 0,  0, 10, 11, 12],
       [ 0,  0,  0, 13, 14]])
\end{Verbatim}
\end{tcolorbox}
        
    To create the second matrix, we need to make sure that the sequence of
integers is arranged column-wise instead of by row, as in the code
above.

This is most likely not possible to achieve with masked indexing and
C-ordered arrays. If we want to use masked indexing, we instead create
the transposed matrix using the same approach as above, and then
transpose it to get the final matrix.

    \begin{tcolorbox}[breakable, size=fbox, boxrule=1pt, pad at break*=1mm,colback=cellbackground, colframe=cellborder]
\prompt{In}{incolor}{102}{\boxspacing}
\begin{Verbatim}[commandchars=\\\{\}]
\PY{c+c1}{\PYZsh{} dimensions of TRANSPOSED matrix}
\PY{n}{m} \PY{o}{=} \PY{l+m+mi}{5}
\PY{n}{n} \PY{o}{=} \PY{l+m+mi}{4}

\PY{c+c1}{\PYZsh{} row and column indices of TRANSPOSED matrix}
\PY{n}{irow} \PY{o}{=} \PY{n}{np}\PY{o}{.}\PY{n}{arange}\PY{p}{(}\PY{n}{m}\PY{p}{)}
\PY{n}{icol} \PY{o}{=} \PY{n}{np}\PY{o}{.}\PY{n}{arange}\PY{p}{(}\PY{n}{n}\PY{p}{)}

\PY{c+c1}{\PYZsh{} mask to select lower\PYZhy{}triangular elements}
\PY{c+c1}{\PYZsh{} of transposed matrix}
\PY{n}{mask} \PY{o}{=} \PY{n}{irow}\PY{p}{[}\PY{p}{:}\PY{p}{,} \PY{k+kc}{None}\PY{p}{]} \PY{o}{\PYZgt{}}\PY{o}{=} \PY{n}{icol}\PY{p}{[}\PY{k+kc}{None}\PY{p}{]}
\PY{n}{mask}
\end{Verbatim}
\end{tcolorbox}

            \begin{tcolorbox}[breakable, size=fbox, boxrule=.5pt, pad at break*=1mm, opacityfill=0]
\prompt{Out}{outcolor}{102}{\boxspacing}
\begin{Verbatim}[commandchars=\\\{\}]
array([[ True, False, False, False],
       [ True,  True, False, False],
       [ True,  True,  True, False],
       [ True,  True,  True,  True],
       [ True,  True,  True,  True]])
\end{Verbatim}
\end{tcolorbox}
        
    \begin{tcolorbox}[breakable, size=fbox, boxrule=1pt, pad at break*=1mm,colback=cellbackground, colframe=cellborder]
\prompt{In}{incolor}{103}{\boxspacing}
\begin{Verbatim}[commandchars=\\\{\}]
\PY{c+c1}{\PYZsh{} Create matrix of zeros}
\PY{n}{mat} \PY{o}{=} \PY{n}{np}\PY{o}{.}\PY{n}{zeros}\PY{p}{(}\PY{p}{(}\PY{n}{m}\PY{p}{,} \PY{n}{n}\PY{p}{)}\PY{p}{,} \PY{n}{dtype}\PY{o}{=}\PY{n+nb}{int}\PY{p}{)}

\PY{c+c1}{\PYZsh{} Number of True elements in mask}
\PY{n}{ntrue} \PY{o}{=} \PY{n}{np}\PY{o}{.}\PY{n}{sum}\PY{p}{(}\PY{n}{mask}\PY{p}{)}

\PY{c+c1}{\PYZsh{} Values to insert into lower\PYZhy{}triangular part:}
\PY{c+c1}{\PYZsh{} 1, 2, ... }
\PY{n}{values} \PY{o}{=} \PY{n}{np}\PY{o}{.}\PY{n}{arange}\PY{p}{(}\PY{n}{ntrue}\PY{p}{)} \PY{o}{+} \PY{l+m+mi}{1}
\PY{n}{mat}\PY{p}{[}\PY{n}{mask}\PY{p}{]} \PY{o}{=} \PY{n}{values}

\PY{c+c1}{\PYZsh{} Transpose to get final matrix}
\PY{n}{mat} \PY{o}{=} \PY{n}{mat}\PY{o}{.}\PY{n}{T}
\PY{n}{mat}
\end{Verbatim}
\end{tcolorbox}

            \begin{tcolorbox}[breakable, size=fbox, boxrule=.5pt, pad at break*=1mm, opacityfill=0]
\prompt{Out}{outcolor}{103}{\boxspacing}
\begin{Verbatim}[commandchars=\\\{\}]
array([[ 1,  2,  4,  7, 11],
       [ 0,  3,  5,  8, 12],
       [ 0,  0,  6,  9, 13],
       [ 0,  0,  0, 10, 14]])
\end{Verbatim}
\end{tcolorbox}
        
    Of course, you can also solve this using the brute-force way with loops,
but that does not help us practice using NumPy:

    \begin{tcolorbox}[breakable, size=fbox, boxrule=1pt, pad at break*=1mm,colback=cellbackground, colframe=cellborder]
\prompt{In}{incolor}{104}{\boxspacing}
\begin{Verbatim}[commandchars=\\\{\}]
\PY{n}{m} \PY{o}{=} \PY{l+m+mi}{4}
\PY{n}{n} \PY{o}{=} \PY{l+m+mi}{5}

\PY{n}{mat} \PY{o}{=} \PY{n}{np}\PY{o}{.}\PY{n}{zeros}\PY{p}{(}\PY{p}{(}\PY{n}{m}\PY{p}{,} \PY{n}{n}\PY{p}{)}\PY{p}{,} \PY{n}{dtype}\PY{o}{=}\PY{n+nb}{int}\PY{p}{)}

\PY{c+c1}{\PYZsh{} keep track of current value to be inserted}
\PY{c+c1}{\PYZsh{} into matrix}
\PY{n}{value} \PY{o}{=} \PY{l+m+mi}{1}

\PY{c+c1}{\PYZsh{} loop over columns}
\PY{k}{for} \PY{n}{j} \PY{o+ow}{in} \PY{n+nb}{range}\PY{p}{(}\PY{n}{n}\PY{p}{)}\PY{p}{:}
    \PY{c+c1}{\PYZsh{} loop over rows}
    \PY{k}{for} \PY{n}{i} \PY{o+ow}{in} \PY{n+nb}{range}\PY{p}{(}\PY{n}{m}\PY{p}{)}\PY{p}{:}
        \PY{k}{if} \PY{n}{j} \PY{o}{\PYZgt{}}\PY{o}{=} \PY{n}{i}\PY{p}{:}
            \PY{n}{mat}\PY{p}{[}\PY{n}{i}\PY{p}{,}\PY{n}{j}\PY{p}{]} \PY{o}{=} \PY{n}{value}
            \PY{c+c1}{\PYZsh{} increment value for next applicable}
            \PY{c+c1}{\PYZsh{} element}
            \PY{n}{value} \PY{o}{+}\PY{o}{=} \PY{l+m+mi}{1}

\PY{n}{mat}
\end{Verbatim}
\end{tcolorbox}

            \begin{tcolorbox}[breakable, size=fbox, boxrule=.5pt, pad at break*=1mm, opacityfill=0]
\prompt{Out}{outcolor}{104}{\boxspacing}
\begin{Verbatim}[commandchars=\\\{\}]
array([[ 1,  2,  4,  7, 11],
       [ 0,  3,  5,  8, 12],
       [ 0,  0,  6,  9, 13],
       [ 0,  0,  0, 10, 14]])
\end{Verbatim}
\end{tcolorbox}
        
    \hypertarget{solution-for-exercise-5}{%
\subsubsection{Solution for exercise 5}\label{solution-for-exercise-5}}

One implementation of \texttt{rowmeans()} could look as follows:

    \begin{tcolorbox}[breakable, size=fbox, boxrule=1pt, pad at break*=1mm,colback=cellbackground, colframe=cellborder]
\prompt{In}{incolor}{105}{\boxspacing}
\begin{Verbatim}[commandchars=\\\{\}]
\PY{k+kn}{import} \PY{n+nn}{numpy} \PY{k}{as} \PY{n+nn}{np}

\PY{k}{def} \PY{n+nf}{rowmeans}\PY{p}{(}\PY{n}{x}\PY{p}{)}\PY{p}{:}
    \PY{c+c1}{\PYZsh{} Number of rows and columns in x}
    \PY{n}{m}\PY{p}{,} \PY{n}{n} \PY{o}{=} \PY{n}{x}\PY{o}{.}\PY{n}{shape}

    \PY{n}{means} \PY{o}{=} \PY{n}{np}\PY{o}{.}\PY{n}{zeros}\PY{p}{(}\PY{n}{m}\PY{p}{)}

    \PY{c+c1}{\PYZsh{} Loop over rows}
    \PY{k}{for} \PY{n}{i} \PY{o+ow}{in} \PY{n+nb}{range}\PY{p}{(}\PY{n}{m}\PY{p}{)}\PY{p}{:}
        \PY{c+c1}{\PYZsh{} number of non\PYZhy{}NaN elements in current row}
        \PY{n}{count} \PY{o}{=} \PY{l+m+mi}{0}
        \PY{c+c1}{\PYZsh{} sum of non\PYZhy{}Nan elements in current row}
        \PY{n}{s} \PY{o}{=} \PY{l+m+mf}{0.0}
        \PY{k}{for} \PY{n}{j} \PY{o+ow}{in} \PY{n+nb}{range}\PY{p}{(}\PY{n}{n}\PY{p}{)}\PY{p}{:}
            \PY{n}{value} \PY{o}{=} \PY{n}{x}\PY{p}{[}\PY{n}{i}\PY{p}{,}\PY{n}{j}\PY{p}{]}
            \PY{k}{if} \PY{o+ow}{not} \PY{n}{np}\PY{o}{.}\PY{n}{isnan}\PY{p}{(}\PY{n}{value}\PY{p}{)}\PY{p}{:}
                \PY{n}{s} \PY{o}{+}\PY{o}{=} \PY{n}{value}
                \PY{c+c1}{\PYZsh{} increments number of non\PYZhy{}NaN elements}
                \PY{n}{count} \PY{o}{+}\PY{o}{=} \PY{l+m+mi}{1}

        \PY{c+c1}{\PYZsh{} compute mean, store in output vector}
        \PY{n}{means}\PY{p}{[}\PY{n}{i}\PY{p}{]} \PY{o}{=} \PY{n}{s} \PY{o}{/} \PY{n}{count} 

    \PY{k}{return} \PY{n}{means}
\end{Verbatim}
\end{tcolorbox}

    We create the array given in the exercise to test our function:

    \begin{tcolorbox}[breakable, size=fbox, boxrule=1pt, pad at break*=1mm,colback=cellbackground, colframe=cellborder]
\prompt{In}{incolor}{106}{\boxspacing}
\begin{Verbatim}[commandchars=\\\{\}]
\PY{n}{m} \PY{o}{=} \PY{l+m+mi}{5}
\PY{n}{n} \PY{o}{=} \PY{l+m+mi}{8}
\PY{n}{a} \PY{o}{=} \PY{n}{np}\PY{o}{.}\PY{n}{arange}\PY{p}{(}\PY{n}{m}\PY{o}{*}\PY{n}{n}\PY{p}{,} \PY{n}{dtype}\PY{o}{=}\PY{n}{np}\PY{o}{.}\PY{n}{float}\PY{p}{)}\PY{o}{.}\PY{n}{reshape}\PY{p}{(}\PY{p}{(}\PY{n}{m}\PY{p}{,} \PY{n}{n}\PY{p}{)}\PY{p}{)}
\PY{n}{mask} \PY{o}{=} \PY{p}{(}\PY{n}{a} \PY{o}{\PYZpc{}} \PY{l+m+mi}{5}\PY{p}{)} \PY{o}{==} \PY{l+m+mi}{0}
\PY{n}{a}\PY{p}{[}\PY{n}{mask}\PY{p}{]} \PY{o}{=} \PY{n}{np}\PY{o}{.}\PY{n}{nan}
\PY{n}{a}
\end{Verbatim}
\end{tcolorbox}

            \begin{tcolorbox}[breakable, size=fbox, boxrule=.5pt, pad at break*=1mm, opacityfill=0]
\prompt{Out}{outcolor}{106}{\boxspacing}
\begin{Verbatim}[commandchars=\\\{\}]
array([[nan,  1.,  2.,  3.,  4., nan,  6.,  7.],
       [ 8.,  9., nan, 11., 12., 13., 14., nan],
       [16., 17., 18., 19., nan, 21., 22., 23.],
       [24., nan, 26., 27., 28., 29., nan, 31.],
       [32., 33., 34., nan, 36., 37., 38., 39.]])
\end{Verbatim}
\end{tcolorbox}
        
    As you see, NumPy indicates elements that are NaN using the string
`nan'.

    \begin{tcolorbox}[breakable, size=fbox, boxrule=1pt, pad at break*=1mm,colback=cellbackground, colframe=cellborder]
\prompt{In}{incolor}{107}{\boxspacing}
\begin{Verbatim}[commandchars=\\\{\}]
\PY{c+c1}{\PYZsh{} call rowmeans() using test data}
\PY{n}{means} \PY{o}{=} \PY{n}{rowmeans}\PY{p}{(}\PY{n}{a}\PY{p}{)}
\PY{n}{means} 
\end{Verbatim}
\end{tcolorbox}

            \begin{tcolorbox}[breakable, size=fbox, boxrule=.5pt, pad at break*=1mm, opacityfill=0]
\prompt{Out}{outcolor}{107}{\boxspacing}
\begin{Verbatim}[commandchars=\\\{\}]
array([ 3.83333333, 11.16666667, 19.42857143, 27.5       , 35.57142857])
\end{Verbatim}
\end{tcolorbox}
        
    We verify our results using the NumPy routine \texttt{np.nanmean()}
which implements the same functionality. Since we are computing row
averages (and thus compute averages \emph{across} columns), we need to
pass in the argument \texttt{axis\ =\ 1}: axes are numbered starting at
0, so \texttt{axis\ =\ 1} refers to the second axis, \ie the columns.

    \begin{tcolorbox}[breakable, size=fbox, boxrule=1pt, pad at break*=1mm,colback=cellbackground, colframe=cellborder]
\prompt{In}{incolor}{108}{\boxspacing}
\begin{Verbatim}[commandchars=\\\{\}]
\PY{n}{means2} \PY{o}{=} \PY{n}{np}\PY{o}{.}\PY{n}{nanmean}\PY{p}{(}\PY{n}{a}\PY{p}{,} \PY{n}{axis}\PY{o}{=}\PY{l+m+mi}{1}\PY{p}{)}
\PY{n}{means2} 
\end{Verbatim}
\end{tcolorbox}

            \begin{tcolorbox}[breakable, size=fbox, boxrule=.5pt, pad at break*=1mm, opacityfill=0]
\prompt{Out}{outcolor}{108}{\boxspacing}
\begin{Verbatim}[commandchars=\\\{\}]
array([ 3.83333333, 11.16666667, 19.42857143, 27.5       , 35.57142857])
\end{Verbatim}
\end{tcolorbox}
        
    To test whether the results are the same, we check whether their
absolute difference is below some tolerance level, in this case
\(10^{-8}\). The routine \texttt{np.all()} evaluates to \texttt{True} if
this is the case for \emph{all} elements.

    \begin{tcolorbox}[breakable, size=fbox, boxrule=1pt, pad at break*=1mm,colback=cellbackground, colframe=cellborder]
\prompt{In}{incolor}{109}{\boxspacing}
\begin{Verbatim}[commandchars=\\\{\}]
\PY{n}{np}\PY{o}{.}\PY{n}{all}\PY{p}{(}\PY{n}{np}\PY{o}{.}\PY{n}{abs}\PY{p}{(}\PY{n}{means} \PY{o}{\PYZhy{}} \PY{n}{means2}\PY{p}{)} \PY{o}{\PYZlt{}} \PY{l+m+mf}{1.0e\PYZhy{}8}\PY{p}{)}
\end{Verbatim}
\end{tcolorbox}

            \begin{tcolorbox}[breakable, size=fbox, boxrule=.5pt, pad at break*=1mm, opacityfill=0]
\prompt{Out}{outcolor}{109}{\boxspacing}
\begin{Verbatim}[commandchars=\\\{\}]
True
\end{Verbatim}
\end{tcolorbox}
        
    Note that we rarely want to compare floating-point numbers resulting
from computations for exact equality using \texttt{==}. Floating-point
has limited precision, and different implementations can potentially
produce different rounding errors. It is therefore unlikely that two
floating-point results will be \emph{exactly} identical.

Finally, we use the \%timeit magic to benchmark our implementation
against NumPy's.

    \begin{tcolorbox}[breakable, size=fbox, boxrule=1pt, pad at break*=1mm,colback=cellbackground, colframe=cellborder]
\prompt{In}{incolor}{110}{\boxspacing}
\begin{Verbatim}[commandchars=\\\{\}]
\PY{o}{\PYZpc{}}\PY{k}{timeit} rowmeans(a)
\end{Verbatim}
\end{tcolorbox}

    \begin{Verbatim}[commandchars=\\\{\}]
51.5 µs ± 544 ns per loop (mean ± std. dev. of 7 runs, 10000 loops each)
    \end{Verbatim}

    \begin{tcolorbox}[breakable, size=fbox, boxrule=1pt, pad at break*=1mm,colback=cellbackground, colframe=cellborder]
\prompt{In}{incolor}{111}{\boxspacing}
\begin{Verbatim}[commandchars=\\\{\}]
\PY{o}{\PYZpc{}}\PY{k}{timeit} np.nanmean(a, axis=1)
\end{Verbatim}
\end{tcolorbox}

    \begin{Verbatim}[commandchars=\\\{\}]
26.9 µs ± 492 ns per loop (mean ± std. dev. of 7 runs, 10000 loops each)
    \end{Verbatim}

    You may be surprised that \texttt{np.nanmean()} is only twice as fast as
our implementation, but this is purely due to the small array size, as
then there are only a few iterations performed in our Python loop.

Try increasing the array dimensions to \texttt{(500,800)} instead of
\texttt{(5,8)} and you will see that then NumPy is about 500 times
faster!

    \hypertarget{solution-for-exercise-6}{%
\subsubsection{Solution for exercise 6}\label{solution-for-exercise-6}}

We first create the matrix \texttt{pvalues} which contains the all
polynomials (for all \texttt{m\ =\ 10}) parametrisations evaluated on a
common grid of x-values. Each row of this matrix represents a different
polynomial parametrisation.

    \begin{tcolorbox}[breakable, size=fbox, boxrule=1pt, pad at break*=1mm,colback=cellbackground, colframe=cellborder]
\prompt{In}{incolor}{112}{\boxspacing}
\begin{Verbatim}[commandchars=\\\{\}]
\PY{k+kn}{import} \PY{n+nn}{numpy} \PY{k}{as} \PY{n+nn}{np}
\PY{k+kn}{import} \PY{n+nn}{matplotlib}\PY{n+nn}{.}\PY{n+nn}{pyplot} \PY{k}{as} \PY{n+nn}{plt}

\PY{n}{m} \PY{o}{=} \PY{l+m+mi}{10}
\PY{n}{n} \PY{o}{=} \PY{l+m+mi}{50}

\PY{c+c1}{\PYZsh{} Parameters for each polynomial}
\PY{n}{a} \PY{o}{=} \PY{n}{np}\PY{o}{.}\PY{n}{linspace}\PY{p}{(}\PY{l+m+mf}{0.4}\PY{p}{,} \PY{l+m+mf}{2.0}\PY{p}{,} \PY{n}{m}\PY{p}{)}
\PY{n}{b} \PY{o}{=} \PY{n}{np}\PY{o}{.}\PY{n}{linspace}\PY{p}{(}\PY{o}{\PYZhy{}}\PY{l+m+mf}{1.0}\PY{p}{,} \PY{l+m+mf}{0.0}\PY{p}{,} \PY{n}{m}\PY{p}{)}
\PY{n}{c} \PY{o}{=} \PY{n}{np}\PY{o}{.}\PY{n}{linspace}\PY{p}{(}\PY{l+m+mf}{0.0}\PY{p}{,} \PY{l+m+mf}{3.0}\PY{p}{,} \PY{n}{m}\PY{p}{)}

\PY{c+c1}{\PYZsh{} grid of x values on which to evaluate polynomials}
\PY{n}{xgrid} \PY{o}{=} \PY{n}{np}\PY{o}{.}\PY{n}{linspace}\PY{p}{(}\PY{o}{\PYZhy{}}\PY{l+m+mf}{2.0}\PY{p}{,} \PY{l+m+mf}{2.0}\PY{p}{,} \PY{n}{n}\PY{p}{)}

\PY{c+c1}{\PYZsh{} polynomials on x, for each parameter tuple (a,b,c)}
\PY{n}{pvalues} \PY{o}{=} \PY{o}{\PYZhy{}} \PY{n}{a}\PY{p}{[}\PY{p}{:}\PY{p}{,}\PY{k+kc}{None}\PY{p}{]} \PY{o}{*} \PY{p}{(}\PY{n}{xgrid}\PY{p}{[}\PY{k+kc}{None}\PY{p}{]} \PY{o}{\PYZhy{}} \PY{n}{b}\PY{p}{[}\PY{p}{:}\PY{p}{,}\PY{k+kc}{None}\PY{p}{]}\PY{p}{)}\PY{o}{*}\PY{o}{*}\PY{l+m+mf}{2.0} \PY{o}{+} \PY{n}{c}\PY{p}{[}\PY{p}{:}\PY{p}{,}\PY{k+kc}{None}\PY{p}{]}
\end{Verbatim}
\end{tcolorbox}

    Next, we create a function to plot all polynomials. We will be reusing
this code, so it is convenient to encapsulate it in a function instead
of copy-pasting it again and again!

    \begin{tcolorbox}[breakable, size=fbox, boxrule=1pt, pad at break*=1mm,colback=cellbackground, colframe=cellborder]
\prompt{In}{incolor}{113}{\boxspacing}
\begin{Verbatim}[commandchars=\\\{\}]
\PY{c+c1}{\PYZsh{} function to plot quadratic polynomials}
\PY{k}{def} \PY{n+nf}{plot\PYZus{}quad}\PY{p}{(}\PY{n}{xvalues}\PY{p}{,} \PY{n}{pvalues}\PY{p}{)}\PY{p}{:}
    \PY{c+c1}{\PYZsh{} Number of different polynomials in pvalues}
    \PY{n}{m} \PY{o}{=} \PY{n}{pvalues}\PY{o}{.}\PY{n}{shape}\PY{p}{[}\PY{l+m+mi}{0}\PY{p}{]}

    \PY{c+c1}{\PYZsh{} Use different transparency (alpha) level for each polynomial}
    \PY{n}{alphas} \PY{o}{=} \PY{n}{np}\PY{o}{.}\PY{n}{linspace}\PY{p}{(}\PY{l+m+mf}{0.2}\PY{p}{,} \PY{l+m+mf}{0.8}\PY{p}{,} \PY{n}{m}\PY{p}{)}
    
    \PY{c+c1}{\PYZsh{} Plot each row against the common x\PYZhy{}values}
    \PY{k}{for} \PY{n}{i} \PY{o+ow}{in} \PY{n+nb}{range}\PY{p}{(}\PY{n}{m}\PY{p}{)}\PY{p}{:}
        \PY{n}{plt}\PY{o}{.}\PY{n}{plot}\PY{p}{(}\PY{n}{xvalues}\PY{p}{,} \PY{n}{pvalues}\PY{p}{[}\PY{n}{i}\PY{p}{]}\PY{p}{,} \PY{n}{color}\PY{o}{=}\PY{l+s+s1}{\PYZsq{}}\PY{l+s+s1}{darkblue}\PY{l+s+s1}{\PYZsq{}}\PY{p}{,} \PY{n}{alpha}\PY{o}{=}\PY{n}{alphas}\PY{p}{[}\PY{n}{i}\PY{p}{]}\PY{p}{)}
    
    \PY{c+c1}{\PYZsh{} Label axes, figure}
    \PY{n}{plt}\PY{o}{.}\PY{n}{xlabel}\PY{p}{(}\PY{l+s+sa}{r}\PY{l+s+s1}{\PYZsq{}}\PY{l+s+s1}{\PYZdl{}x\PYZdl{}}\PY{l+s+s1}{\PYZsq{}}\PY{p}{)}
    \PY{n}{plt}\PY{o}{.}\PY{n}{ylabel}\PY{p}{(}\PY{l+s+sa}{r}\PY{l+s+s1}{\PYZsq{}}\PY{l+s+s1}{\PYZdl{}p(x;a,b,c)\PYZdl{}}\PY{l+s+s1}{\PYZsq{}}\PY{p}{)}
    \PY{n}{plt}\PY{o}{.}\PY{n}{title}\PY{p}{(}\PY{l+s+s1}{\PYZsq{}}\PY{l+s+s1}{Quadratic functions}\PY{l+s+s1}{\PYZsq{}}\PY{p}{)}
\end{Verbatim}
\end{tcolorbox}

    \begin{tcolorbox}[breakable, size=fbox, boxrule=1pt, pad at break*=1mm,colback=cellbackground, colframe=cellborder]
\prompt{In}{incolor}{114}{\boxspacing}
\begin{Verbatim}[commandchars=\\\{\}]
\PY{c+c1}{\PYZsh{} Call plotting routine, passing x\PYZhy{}values and y\PYZhy{}values}
\PY{c+c1}{\PYZsh{} as arguments}
\PY{n}{plot\PYZus{}quad}\PY{p}{(}\PY{n}{xgrid}\PY{p}{,} \PY{n}{pvalues}\PY{p}{)}
\end{Verbatim}
\end{tcolorbox}

    \begin{center}
    \adjustimage{max size={0.9\linewidth}{0.9\paperheight}}{unit06_files/unit06_197_0.pdf}
    \end{center}
    { \hspace*{\fill} \\}
    
    \hypertarget{locating-maxima-using-loops}{%
\subsubsection{Locating maxima using
loops}\label{locating-maxima-using-loops}}

Below is one possible way to implement a function that returns a vector
of indices, each index storing the location of the maximum element in
the corresponding row.

    \begin{tcolorbox}[breakable, size=fbox, boxrule=1pt, pad at break*=1mm,colback=cellbackground, colframe=cellborder]
\prompt{In}{incolor}{115}{\boxspacing}
\begin{Verbatim}[commandchars=\\\{\}]
\PY{c+c1}{\PYZsh{} Function to locate the maximum value in each row}
\PY{k}{def} \PY{n+nf}{find\PYZus{}max}\PY{p}{(}\PY{n}{pvalues}\PY{p}{)}\PY{p}{:}
    \PY{c+c1}{\PYZsh{} unpack rows and columns from shape attribute}
    \PY{n}{nrow}\PY{p}{,} \PY{n}{ncol} \PY{o}{=} \PY{n}{pvalues}\PY{o}{.}\PY{n}{shape}
    \PY{c+c1}{\PYZsh{} Create array to store location of maximum for each row.}
    \PY{c+c1}{\PYZsh{} Location is an index, so choose integer array type!}
    \PY{n}{imax} \PY{o}{=} \PY{n}{np}\PY{o}{.}\PY{n}{zeros}\PY{p}{(}\PY{n}{nrow}\PY{p}{,} \PY{n}{dtype}\PY{o}{=}\PY{n}{np}\PY{o}{.}\PY{n}{int}\PY{p}{)}

    \PY{c+c1}{\PYZsh{} iterate over all row}
    \PY{k}{for} \PY{n}{i} \PY{o+ow}{in} \PY{n+nb}{range}\PY{p}{(}\PY{n}{nrow}\PY{p}{)}\PY{p}{:}
        \PY{c+c1}{\PYZsh{} initial guess for location of row maximum}
        \PY{n}{jmax} \PY{o}{=} \PY{l+m+mi}{0}
        \PY{c+c1}{\PYZsh{} iterate over all columns, locate index of maximum}
        \PY{k}{for} \PY{n}{j} \PY{o+ow}{in} \PY{n+nb}{range}\PY{p}{(}\PY{l+m+mi}{1}\PY{p}{,} \PY{n}{ncol}\PY{p}{)}\PY{p}{:}
            \PY{k}{if} \PY{n}{pvalues}\PY{p}{[}\PY{n}{i}\PY{p}{,}\PY{n}{j}\PY{p}{]} \PY{o}{\PYZgt{}} \PY{n}{pvalues}\PY{p}{[}\PY{n}{i}\PY{p}{,}\PY{n}{jmax}\PY{p}{]}\PY{p}{:}
                \PY{c+c1}{\PYZsh{} value at (i,j) is larger than}
                \PY{c+c1}{\PYZsh{} value at current max: }
                \PY{c+c1}{\PYZsh{} update jmax}
                \PY{n}{jmax} \PY{o}{=} \PY{n}{j}
        \PY{c+c1}{\PYZsh{} store index of maximum for current row}
        \PY{n}{imax}\PY{p}{[}\PY{n}{i}\PY{p}{]} \PY{o}{=} \PY{n}{jmax}
    
    \PY{k}{return} \PY{n}{imax}
\end{Verbatim}
\end{tcolorbox}

    \begin{tcolorbox}[breakable, size=fbox, boxrule=1pt, pad at break*=1mm,colback=cellbackground, colframe=cellborder]
\prompt{In}{incolor}{116}{\boxspacing}
\begin{Verbatim}[commandchars=\\\{\}]
\PY{c+c1}{\PYZsh{} use find\PYZus{}max() to locale indices of each row maximum}
\PY{n}{ipmax} \PY{o}{=} \PY{n}{find\PYZus{}max}\PY{p}{(}\PY{n}{pvalues}\PY{p}{)}
\PY{n}{ipmax}
\end{Verbatim}
\end{tcolorbox}

            \begin{tcolorbox}[breakable, size=fbox, boxrule=.5pt, pad at break*=1mm, opacityfill=0]
\prompt{Out}{outcolor}{116}{\boxspacing}
\begin{Verbatim}[commandchars=\\\{\}]
array([12, 14, 15, 16, 18, 19, 20, 22, 23, 24])
\end{Verbatim}
\end{tcolorbox}
        
    We plot the polynomials using the function we defined above. We then add
the maxima to the \emph{same} plot.

    \begin{tcolorbox}[breakable, size=fbox, boxrule=1pt, pad at break*=1mm,colback=cellbackground, colframe=cellborder]
\prompt{In}{incolor}{117}{\boxspacing}
\begin{Verbatim}[commandchars=\\\{\}]
\PY{c+c1}{\PYZsh{} Recreate original plot from above}
\PY{n}{plot\PYZus{}quad}\PY{p}{(}\PY{n}{xgrid}\PY{p}{,} \PY{n}{pvalues}\PY{p}{)}

\PY{c+c1}{\PYZsh{} Plot maxima on top of previous graph}
\PY{n}{ix} \PY{o}{=} \PY{n}{np}\PY{o}{.}\PY{n}{arange}\PY{p}{(}\PY{n}{m}\PY{p}{)}
\PY{n}{plt}\PY{o}{.}\PY{n}{scatter}\PY{p}{(}\PY{n}{xgrid}\PY{p}{[}\PY{n}{ipmax}\PY{p}{]}\PY{p}{,} \PY{n}{pvalues}\PY{p}{[}\PY{n}{ix}\PY{p}{,}\PY{n}{ipmax}\PY{p}{]}\PY{p}{,} \PY{n}{c}\PY{o}{=}\PY{l+s+s1}{\PYZsq{}}\PY{l+s+s1}{black}\PY{l+s+s1}{\PYZsq{}}\PY{p}{,} \PY{n}{label}\PY{o}{=}\PY{l+s+s1}{\PYZsq{}}\PY{l+s+s1}{Maximum}\PY{l+s+s1}{\PYZsq{}}\PY{p}{)}
\PY{n}{plt}\PY{o}{.}\PY{n}{legend}\PY{p}{(}\PY{p}{)}
\end{Verbatim}
\end{tcolorbox}

            \begin{tcolorbox}[breakable, size=fbox, boxrule=.5pt, pad at break*=1mm, opacityfill=0]
\prompt{Out}{outcolor}{117}{\boxspacing}
\begin{Verbatim}[commandchars=\\\{\}]
<matplotlib.legend.Legend at 0x7ff71f72bca0>
\end{Verbatim}
\end{tcolorbox}
        
    \begin{center}
    \adjustimage{max size={0.9\linewidth}{0.9\paperheight}}{unit06_files/unit06_202_1.pdf}
    \end{center}
    { \hspace*{\fill} \\}
    
    \hypertarget{vectorised-version}{%
\subsubsection{Vectorised version}\label{vectorised-version}}

The vectorised version simply uses NumPy's \texttt{np.argmax()}
function. We need to pass the argument \texttt{axis\ =\ 1} as the
maximum should be computed across all columns for any given row.

    \begin{tcolorbox}[breakable, size=fbox, boxrule=1pt, pad at break*=1mm,colback=cellbackground, colframe=cellborder]
\prompt{In}{incolor}{118}{\boxspacing}
\begin{Verbatim}[commandchars=\\\{\}]
\PY{c+c1}{\PYZsh{} Find row maxima using np.argmax()}

\PY{n}{ipmax2} \PY{o}{=} \PY{n}{np}\PY{o}{.}\PY{n}{argmax}\PY{p}{(}\PY{n}{pvalues}\PY{p}{,} \PY{n}{axis}\PY{o}{=}\PY{l+m+mi}{1}\PY{p}{)}
\PY{n}{ipmax2}
\end{Verbatim}
\end{tcolorbox}

            \begin{tcolorbox}[breakable, size=fbox, boxrule=.5pt, pad at break*=1mm, opacityfill=0]
\prompt{Out}{outcolor}{118}{\boxspacing}
\begin{Verbatim}[commandchars=\\\{\}]
array([12, 14, 15, 16, 18, 19, 20, 22, 23, 24])
\end{Verbatim}
\end{tcolorbox}
        
    We verify that the results of our and NumPy's implementation are
identical. Since the values here are integers, we can directly compare
them using \texttt{==}.

    \begin{tcolorbox}[breakable, size=fbox, boxrule=1pt, pad at break*=1mm,colback=cellbackground, colframe=cellborder]
\prompt{In}{incolor}{119}{\boxspacing}
\begin{Verbatim}[commandchars=\\\{\}]
\PY{c+c1}{\PYZsh{} Check that these are the same as what we computed above}
\PY{n}{np}\PY{o}{.}\PY{n}{all}\PY{p}{(}\PY{n}{ipmax} \PY{o}{==} \PY{n}{ipmax2}\PY{p}{)}
\end{Verbatim}
\end{tcolorbox}

            \begin{tcolorbox}[breakable, size=fbox, boxrule=.5pt, pad at break*=1mm, opacityfill=0]
\prompt{Out}{outcolor}{119}{\boxspacing}
\begin{Verbatim}[commandchars=\\\{\}]
True
\end{Verbatim}
\end{tcolorbox}
        
    Finally, we benchmark both implementations using \texttt{\%timeit}. The
NumPy version is approximately 100 times faster!

    \begin{tcolorbox}[breakable, size=fbox, boxrule=1pt, pad at break*=1mm,colback=cellbackground, colframe=cellborder]
\prompt{In}{incolor}{120}{\boxspacing}
\begin{Verbatim}[commandchars=\\\{\}]
\PY{c+c1}{\PYZsh{} Time our manual implementation}
\PY{o}{\PYZpc{}}\PY{k}{timeit} find\PYZus{}max(pvalues)
\end{Verbatim}
\end{tcolorbox}

    \begin{Verbatim}[commandchars=\\\{\}]
126 µs ± 1.1 µs per loop (mean ± std. dev. of 7 runs, 10000 loops each)
    \end{Verbatim}

    \begin{tcolorbox}[breakable, size=fbox, boxrule=1pt, pad at break*=1mm,colback=cellbackground, colframe=cellborder]
\prompt{In}{incolor}{121}{\boxspacing}
\begin{Verbatim}[commandchars=\\\{\}]
\PY{c+c1}{\PYZsh{} Time NumPy\PYZsq{}s implementation}
\PY{o}{\PYZpc{}}\PY{k}{timeit} np.argmax(pvalues, axis=1)
\end{Verbatim}
\end{tcolorbox}

    \begin{Verbatim}[commandchars=\\\{\}]
1.51 µs ± 3.11 ns per loop (mean ± std. dev. of 7 runs, 1000000 loops each)
    \end{Verbatim}


    % Add a bibliography block to the postdoc
    
    
    
\end{document}
