\documentclass[10pt]{scrartcl}

    % included at the top of the generated TeX file

\usepackage{tgcursor}

\usepackage[utf8]{inputenc}

\KOMAoptions{parskip=half*}
\KOMAoptions{paper=a4,twoside=false}
\KOMAoptions{numbers=noendperiod}

\addtokomafont{disposition}{\rmfamily}
\setcounter{tocdepth}{\subsectiontocdepth}

    \usepackage[breakable]{tcolorbox}
    \usepackage{parskip} % Stop auto-indenting (to mimic markdown behaviour)
    
    \usepackage{iftex}
    \ifPDFTeX
    	\usepackage[T1]{fontenc}
    	\usepackage{mathpazo}
    \else
    	\usepackage{fontspec}
    \fi

    % Basic figure setup, for now with no caption control since it's done
    % automatically by Pandoc (which extracts ![](path) syntax from Markdown).
    \usepackage{graphicx}
    % Maintain compatibility with old templates. Remove in nbconvert 6.0
    \let\Oldincludegraphics\includegraphics
    % Ensure that by default, figures have no caption (until we provide a
    % proper Figure object with a Caption API and a way to capture that
    % in the conversion process - todo).
    \usepackage{caption}
    \DeclareCaptionFormat{nocaption}{}
    \captionsetup{format=nocaption,aboveskip=0pt,belowskip=0pt}

    \usepackage{float}
    \floatplacement{figure}{H} % forces figures to be placed at the correct location
    \usepackage{xcolor} % Allow colors to be defined
    \usepackage{enumerate} % Needed for markdown enumerations to work
    \usepackage{geometry} % Used to adjust the document margins
    \usepackage{amsmath} % Equations
    \usepackage{amssymb} % Equations
    \usepackage{textcomp} % defines textquotesingle
    % Hack from http://tex.stackexchange.com/a/47451/13684:
    \AtBeginDocument{%
        \def\PYZsq{\textquotesingle}% Upright quotes in Pygmentized code
    }
    \usepackage{upquote} % Upright quotes for verbatim code
    \usepackage{eurosym} % defines \euro
    \usepackage[mathletters]{ucs} % Extended unicode (utf-8) support
    \usepackage{fancyvrb} % verbatim replacement that allows latex
    \usepackage{grffile} % extends the file name processing of package graphics 
                         % to support a larger range
    \makeatletter % fix for old versions of grffile with XeLaTeX
    \@ifpackagelater{grffile}{2019/11/01}
    {
      % Do nothing on new versions
    }
    {
      \def\Gread@@xetex#1{%
        \IfFileExists{"\Gin@base".bb}%
        {\Gread@eps{\Gin@base.bb}}%
        {\Gread@@xetex@aux#1}%
      }
    }
    \makeatother
    \usepackage[Export]{adjustbox} % Used to constrain images to a maximum size
    \adjustboxset{max size={0.9\linewidth}{0.9\paperheight}}

    % The hyperref package gives us a pdf with properly built
    % internal navigation ('pdf bookmarks' for the table of contents,
    % internal cross-reference links, web links for URLs, etc.)
    \usepackage{hyperref}
    % The default LaTeX title has an obnoxious amount of whitespace. By default,
    % titling removes some of it. It also provides customization options.
    \usepackage{titling}
    \usepackage{longtable} % longtable support required by pandoc >1.10
    \usepackage{booktabs}  % table support for pandoc > 1.12.2
    \usepackage[inline]{enumitem} % IRkernel/repr support (it uses the enumerate* environment)
    \usepackage[normalem]{ulem} % ulem is needed to support strikethroughs (\sout)
                                % normalem makes italics be italics, not underlines
    \usepackage{mathrsfs}
    

    
    % Colors for the hyperref package
    \definecolor{urlcolor}{rgb}{0,.145,.698}
    \definecolor{linkcolor}{rgb}{.71,0.21,0.01}
    \definecolor{citecolor}{rgb}{.12,.54,.11}

    % ANSI colors
    \definecolor{ansi-black}{HTML}{3E424D}
    \definecolor{ansi-black-intense}{HTML}{282C36}
    \definecolor{ansi-red}{HTML}{E75C58}
    \definecolor{ansi-red-intense}{HTML}{B22B31}
    \definecolor{ansi-green}{HTML}{00A250}
    \definecolor{ansi-green-intense}{HTML}{007427}
    \definecolor{ansi-yellow}{HTML}{DDB62B}
    \definecolor{ansi-yellow-intense}{HTML}{B27D12}
    \definecolor{ansi-blue}{HTML}{208FFB}
    \definecolor{ansi-blue-intense}{HTML}{0065CA}
    \definecolor{ansi-magenta}{HTML}{D160C4}
    \definecolor{ansi-magenta-intense}{HTML}{A03196}
    \definecolor{ansi-cyan}{HTML}{60C6C8}
    \definecolor{ansi-cyan-intense}{HTML}{258F8F}
    \definecolor{ansi-white}{HTML}{C5C1B4}
    \definecolor{ansi-white-intense}{HTML}{A1A6B2}
    \definecolor{ansi-default-inverse-fg}{HTML}{FFFFFF}
    \definecolor{ansi-default-inverse-bg}{HTML}{000000}

    % common color for the border for error outputs.
    \definecolor{outerrorbackground}{HTML}{FFDFDF}

    % commands and environments needed by pandoc snippets
    % extracted from the output of `pandoc -s`
    \providecommand{\tightlist}{%
      \setlength{\itemsep}{0pt}\setlength{\parskip}{0pt}}
    \DefineVerbatimEnvironment{Highlighting}{Verbatim}{commandchars=\\\{\}}
    % Add ',fontsize=\small' for more characters per line
    \newenvironment{Shaded}{}{}
    \newcommand{\KeywordTok}[1]{\textcolor[rgb]{0.00,0.44,0.13}{\textbf{{#1}}}}
    \newcommand{\DataTypeTok}[1]{\textcolor[rgb]{0.56,0.13,0.00}{{#1}}}
    \newcommand{\DecValTok}[1]{\textcolor[rgb]{0.25,0.63,0.44}{{#1}}}
    \newcommand{\BaseNTok}[1]{\textcolor[rgb]{0.25,0.63,0.44}{{#1}}}
    \newcommand{\FloatTok}[1]{\textcolor[rgb]{0.25,0.63,0.44}{{#1}}}
    \newcommand{\CharTok}[1]{\textcolor[rgb]{0.25,0.44,0.63}{{#1}}}
    \newcommand{\StringTok}[1]{\textcolor[rgb]{0.25,0.44,0.63}{{#1}}}
    \newcommand{\CommentTok}[1]{\textcolor[rgb]{0.38,0.63,0.69}{\textit{{#1}}}}
    \newcommand{\OtherTok}[1]{\textcolor[rgb]{0.00,0.44,0.13}{{#1}}}
    \newcommand{\AlertTok}[1]{\textcolor[rgb]{1.00,0.00,0.00}{\textbf{{#1}}}}
    \newcommand{\FunctionTok}[1]{\textcolor[rgb]{0.02,0.16,0.49}{{#1}}}
    \newcommand{\RegionMarkerTok}[1]{{#1}}
    \newcommand{\ErrorTok}[1]{\textcolor[rgb]{1.00,0.00,0.00}{\textbf{{#1}}}}
    \newcommand{\NormalTok}[1]{{#1}}
    
    % Additional commands for more recent versions of Pandoc
    \newcommand{\ConstantTok}[1]{\textcolor[rgb]{0.53,0.00,0.00}{{#1}}}
    \newcommand{\SpecialCharTok}[1]{\textcolor[rgb]{0.25,0.44,0.63}{{#1}}}
    \newcommand{\VerbatimStringTok}[1]{\textcolor[rgb]{0.25,0.44,0.63}{{#1}}}
    \newcommand{\SpecialStringTok}[1]{\textcolor[rgb]{0.73,0.40,0.53}{{#1}}}
    \newcommand{\ImportTok}[1]{{#1}}
    \newcommand{\DocumentationTok}[1]{\textcolor[rgb]{0.73,0.13,0.13}{\textit{{#1}}}}
    \newcommand{\AnnotationTok}[1]{\textcolor[rgb]{0.38,0.63,0.69}{\textbf{\textit{{#1}}}}}
    \newcommand{\CommentVarTok}[1]{\textcolor[rgb]{0.38,0.63,0.69}{\textbf{\textit{{#1}}}}}
    \newcommand{\VariableTok}[1]{\textcolor[rgb]{0.10,0.09,0.49}{{#1}}}
    \newcommand{\ControlFlowTok}[1]{\textcolor[rgb]{0.00,0.44,0.13}{\textbf{{#1}}}}
    \newcommand{\OperatorTok}[1]{\textcolor[rgb]{0.40,0.40,0.40}{{#1}}}
    \newcommand{\BuiltInTok}[1]{{#1}}
    \newcommand{\ExtensionTok}[1]{{#1}}
    \newcommand{\PreprocessorTok}[1]{\textcolor[rgb]{0.74,0.48,0.00}{{#1}}}
    \newcommand{\AttributeTok}[1]{\textcolor[rgb]{0.49,0.56,0.16}{{#1}}}
    \newcommand{\InformationTok}[1]{\textcolor[rgb]{0.38,0.63,0.69}{\textbf{\textit{{#1}}}}}
    \newcommand{\WarningTok}[1]{\textcolor[rgb]{0.38,0.63,0.69}{\textbf{\textit{{#1}}}}}
    
    
    % Define a nice break command that doesn't care if a line doesn't already
    % exist.
    \def\br{\hspace*{\fill} \\* }
    % Math Jax compatibility definitions
    \def\gt{>}
    \def\lt{<}
    \let\Oldtex\TeX
    \let\Oldlatex\LaTeX
    \renewcommand{\TeX}{\textrm{\Oldtex}}
    \renewcommand{\LaTeX}{\textrm{\Oldlatex}}
    % Document parameters
    % Document title
    \newcommand*{\mytitle}{Unit 8: Handling data with pandas}

    % Included at the bottom of the preamble

\usepackage{microtype}


\title{\mytitle}
\author{Richard Foltyn}
\hypersetup{pdfauthor={Richard Foltyn}, pdftitle={\mytitle}}

% Remove horizontal rules in a very hackish way
\renewcommand{\rule}[2]{}

\RequirePackage{xspace}


\newcommand*{\eg}{e.g.\@\xspace}
\newcommand*{\Eg}{E.g.\@\xspace}
\newcommand*{\etc}{etc.\@\xspace}
\newcommand*{\ie}{i.e.\@\xspace}
\newcommand*{\vs}{vs.\@\xspace}
\newcommand*{\viz}{viz.\@\xspace}
\newcommand*{\US}{U.S.\@\xspace}

    
    
    
    
    
% Pygments definitions
\makeatletter
\def\PY@reset{\let\PY@it=\relax \let\PY@bf=\relax%
    \let\PY@ul=\relax \let\PY@tc=\relax%
    \let\PY@bc=\relax \let\PY@ff=\relax}
\def\PY@tok#1{\csname PY@tok@#1\endcsname}
\def\PY@toks#1+{\ifx\relax#1\empty\else%
    \PY@tok{#1}\expandafter\PY@toks\fi}
\def\PY@do#1{\PY@bc{\PY@tc{\PY@ul{%
    \PY@it{\PY@bf{\PY@ff{#1}}}}}}}
\def\PY#1#2{\PY@reset\PY@toks#1+\relax+\PY@do{#2}}

\expandafter\def\csname PY@tok@w\endcsname{\def\PY@tc##1{\textcolor[rgb]{0.73,0.73,0.73}{##1}}}
\expandafter\def\csname PY@tok@c\endcsname{\let\PY@it=\textit\def\PY@tc##1{\textcolor[rgb]{0.25,0.50,0.50}{##1}}}
\expandafter\def\csname PY@tok@cp\endcsname{\def\PY@tc##1{\textcolor[rgb]{0.74,0.48,0.00}{##1}}}
\expandafter\def\csname PY@tok@k\endcsname{\let\PY@bf=\textbf\def\PY@tc##1{\textcolor[rgb]{0.00,0.50,0.00}{##1}}}
\expandafter\def\csname PY@tok@kp\endcsname{\def\PY@tc##1{\textcolor[rgb]{0.00,0.50,0.00}{##1}}}
\expandafter\def\csname PY@tok@kt\endcsname{\def\PY@tc##1{\textcolor[rgb]{0.69,0.00,0.25}{##1}}}
\expandafter\def\csname PY@tok@o\endcsname{\def\PY@tc##1{\textcolor[rgb]{0.40,0.40,0.40}{##1}}}
\expandafter\def\csname PY@tok@ow\endcsname{\let\PY@bf=\textbf\def\PY@tc##1{\textcolor[rgb]{0.67,0.13,1.00}{##1}}}
\expandafter\def\csname PY@tok@nb\endcsname{\def\PY@tc##1{\textcolor[rgb]{0.00,0.50,0.00}{##1}}}
\expandafter\def\csname PY@tok@nf\endcsname{\def\PY@tc##1{\textcolor[rgb]{0.00,0.00,1.00}{##1}}}
\expandafter\def\csname PY@tok@nc\endcsname{\let\PY@bf=\textbf\def\PY@tc##1{\textcolor[rgb]{0.00,0.00,1.00}{##1}}}
\expandafter\def\csname PY@tok@nn\endcsname{\let\PY@bf=\textbf\def\PY@tc##1{\textcolor[rgb]{0.00,0.00,1.00}{##1}}}
\expandafter\def\csname PY@tok@ne\endcsname{\let\PY@bf=\textbf\def\PY@tc##1{\textcolor[rgb]{0.82,0.25,0.23}{##1}}}
\expandafter\def\csname PY@tok@nv\endcsname{\def\PY@tc##1{\textcolor[rgb]{0.10,0.09,0.49}{##1}}}
\expandafter\def\csname PY@tok@no\endcsname{\def\PY@tc##1{\textcolor[rgb]{0.53,0.00,0.00}{##1}}}
\expandafter\def\csname PY@tok@nl\endcsname{\def\PY@tc##1{\textcolor[rgb]{0.63,0.63,0.00}{##1}}}
\expandafter\def\csname PY@tok@ni\endcsname{\let\PY@bf=\textbf\def\PY@tc##1{\textcolor[rgb]{0.60,0.60,0.60}{##1}}}
\expandafter\def\csname PY@tok@na\endcsname{\def\PY@tc##1{\textcolor[rgb]{0.49,0.56,0.16}{##1}}}
\expandafter\def\csname PY@tok@nt\endcsname{\let\PY@bf=\textbf\def\PY@tc##1{\textcolor[rgb]{0.00,0.50,0.00}{##1}}}
\expandafter\def\csname PY@tok@nd\endcsname{\def\PY@tc##1{\textcolor[rgb]{0.67,0.13,1.00}{##1}}}
\expandafter\def\csname PY@tok@s\endcsname{\def\PY@tc##1{\textcolor[rgb]{0.73,0.13,0.13}{##1}}}
\expandafter\def\csname PY@tok@sd\endcsname{\let\PY@it=\textit\def\PY@tc##1{\textcolor[rgb]{0.73,0.13,0.13}{##1}}}
\expandafter\def\csname PY@tok@si\endcsname{\let\PY@bf=\textbf\def\PY@tc##1{\textcolor[rgb]{0.73,0.40,0.53}{##1}}}
\expandafter\def\csname PY@tok@se\endcsname{\let\PY@bf=\textbf\def\PY@tc##1{\textcolor[rgb]{0.73,0.40,0.13}{##1}}}
\expandafter\def\csname PY@tok@sr\endcsname{\def\PY@tc##1{\textcolor[rgb]{0.73,0.40,0.53}{##1}}}
\expandafter\def\csname PY@tok@ss\endcsname{\def\PY@tc##1{\textcolor[rgb]{0.10,0.09,0.49}{##1}}}
\expandafter\def\csname PY@tok@sx\endcsname{\def\PY@tc##1{\textcolor[rgb]{0.00,0.50,0.00}{##1}}}
\expandafter\def\csname PY@tok@m\endcsname{\def\PY@tc##1{\textcolor[rgb]{0.40,0.40,0.40}{##1}}}
\expandafter\def\csname PY@tok@gh\endcsname{\let\PY@bf=\textbf\def\PY@tc##1{\textcolor[rgb]{0.00,0.00,0.50}{##1}}}
\expandafter\def\csname PY@tok@gu\endcsname{\let\PY@bf=\textbf\def\PY@tc##1{\textcolor[rgb]{0.50,0.00,0.50}{##1}}}
\expandafter\def\csname PY@tok@gd\endcsname{\def\PY@tc##1{\textcolor[rgb]{0.63,0.00,0.00}{##1}}}
\expandafter\def\csname PY@tok@gi\endcsname{\def\PY@tc##1{\textcolor[rgb]{0.00,0.63,0.00}{##1}}}
\expandafter\def\csname PY@tok@gr\endcsname{\def\PY@tc##1{\textcolor[rgb]{1.00,0.00,0.00}{##1}}}
\expandafter\def\csname PY@tok@ge\endcsname{\let\PY@it=\textit}
\expandafter\def\csname PY@tok@gs\endcsname{\let\PY@bf=\textbf}
\expandafter\def\csname PY@tok@gp\endcsname{\let\PY@bf=\textbf\def\PY@tc##1{\textcolor[rgb]{0.00,0.00,0.50}{##1}}}
\expandafter\def\csname PY@tok@go\endcsname{\def\PY@tc##1{\textcolor[rgb]{0.53,0.53,0.53}{##1}}}
\expandafter\def\csname PY@tok@gt\endcsname{\def\PY@tc##1{\textcolor[rgb]{0.00,0.27,0.87}{##1}}}
\expandafter\def\csname PY@tok@err\endcsname{\def\PY@bc##1{\setlength{\fboxsep}{0pt}\fcolorbox[rgb]{1.00,0.00,0.00}{1,1,1}{\strut ##1}}}
\expandafter\def\csname PY@tok@kc\endcsname{\let\PY@bf=\textbf\def\PY@tc##1{\textcolor[rgb]{0.00,0.50,0.00}{##1}}}
\expandafter\def\csname PY@tok@kd\endcsname{\let\PY@bf=\textbf\def\PY@tc##1{\textcolor[rgb]{0.00,0.50,0.00}{##1}}}
\expandafter\def\csname PY@tok@kn\endcsname{\let\PY@bf=\textbf\def\PY@tc##1{\textcolor[rgb]{0.00,0.50,0.00}{##1}}}
\expandafter\def\csname PY@tok@kr\endcsname{\let\PY@bf=\textbf\def\PY@tc##1{\textcolor[rgb]{0.00,0.50,0.00}{##1}}}
\expandafter\def\csname PY@tok@bp\endcsname{\def\PY@tc##1{\textcolor[rgb]{0.00,0.50,0.00}{##1}}}
\expandafter\def\csname PY@tok@fm\endcsname{\def\PY@tc##1{\textcolor[rgb]{0.00,0.00,1.00}{##1}}}
\expandafter\def\csname PY@tok@vc\endcsname{\def\PY@tc##1{\textcolor[rgb]{0.10,0.09,0.49}{##1}}}
\expandafter\def\csname PY@tok@vg\endcsname{\def\PY@tc##1{\textcolor[rgb]{0.10,0.09,0.49}{##1}}}
\expandafter\def\csname PY@tok@vi\endcsname{\def\PY@tc##1{\textcolor[rgb]{0.10,0.09,0.49}{##1}}}
\expandafter\def\csname PY@tok@vm\endcsname{\def\PY@tc##1{\textcolor[rgb]{0.10,0.09,0.49}{##1}}}
\expandafter\def\csname PY@tok@sa\endcsname{\def\PY@tc##1{\textcolor[rgb]{0.73,0.13,0.13}{##1}}}
\expandafter\def\csname PY@tok@sb\endcsname{\def\PY@tc##1{\textcolor[rgb]{0.73,0.13,0.13}{##1}}}
\expandafter\def\csname PY@tok@sc\endcsname{\def\PY@tc##1{\textcolor[rgb]{0.73,0.13,0.13}{##1}}}
\expandafter\def\csname PY@tok@dl\endcsname{\def\PY@tc##1{\textcolor[rgb]{0.73,0.13,0.13}{##1}}}
\expandafter\def\csname PY@tok@s2\endcsname{\def\PY@tc##1{\textcolor[rgb]{0.73,0.13,0.13}{##1}}}
\expandafter\def\csname PY@tok@sh\endcsname{\def\PY@tc##1{\textcolor[rgb]{0.73,0.13,0.13}{##1}}}
\expandafter\def\csname PY@tok@s1\endcsname{\def\PY@tc##1{\textcolor[rgb]{0.73,0.13,0.13}{##1}}}
\expandafter\def\csname PY@tok@mb\endcsname{\def\PY@tc##1{\textcolor[rgb]{0.40,0.40,0.40}{##1}}}
\expandafter\def\csname PY@tok@mf\endcsname{\def\PY@tc##1{\textcolor[rgb]{0.40,0.40,0.40}{##1}}}
\expandafter\def\csname PY@tok@mh\endcsname{\def\PY@tc##1{\textcolor[rgb]{0.40,0.40,0.40}{##1}}}
\expandafter\def\csname PY@tok@mi\endcsname{\def\PY@tc##1{\textcolor[rgb]{0.40,0.40,0.40}{##1}}}
\expandafter\def\csname PY@tok@il\endcsname{\def\PY@tc##1{\textcolor[rgb]{0.40,0.40,0.40}{##1}}}
\expandafter\def\csname PY@tok@mo\endcsname{\def\PY@tc##1{\textcolor[rgb]{0.40,0.40,0.40}{##1}}}
\expandafter\def\csname PY@tok@ch\endcsname{\let\PY@it=\textit\def\PY@tc##1{\textcolor[rgb]{0.25,0.50,0.50}{##1}}}
\expandafter\def\csname PY@tok@cm\endcsname{\let\PY@it=\textit\def\PY@tc##1{\textcolor[rgb]{0.25,0.50,0.50}{##1}}}
\expandafter\def\csname PY@tok@cpf\endcsname{\let\PY@it=\textit\def\PY@tc##1{\textcolor[rgb]{0.25,0.50,0.50}{##1}}}
\expandafter\def\csname PY@tok@c1\endcsname{\let\PY@it=\textit\def\PY@tc##1{\textcolor[rgb]{0.25,0.50,0.50}{##1}}}
\expandafter\def\csname PY@tok@cs\endcsname{\let\PY@it=\textit\def\PY@tc##1{\textcolor[rgb]{0.25,0.50,0.50}{##1}}}

\def\PYZbs{\char`\\}
\def\PYZus{\char`\_}
\def\PYZob{\char`\{}
\def\PYZcb{\char`\}}
\def\PYZca{\char`\^}
\def\PYZam{\char`\&}
\def\PYZlt{\char`\<}
\def\PYZgt{\char`\>}
\def\PYZsh{\char`\#}
\def\PYZpc{\char`\%}
\def\PYZdl{\char`\$}
\def\PYZhy{\char`\-}
\def\PYZsq{\char`\'}
\def\PYZdq{\char`\"}
\def\PYZti{\char`\~}
% for compatibility with earlier versions
\def\PYZat{@}
\def\PYZlb{[}
\def\PYZrb{]}
\makeatother


    % For linebreaks inside Verbatim environment from package fancyvrb. 
    \makeatletter
        \newbox\Wrappedcontinuationbox 
        \newbox\Wrappedvisiblespacebox 
        \newcommand*\Wrappedvisiblespace {\textcolor{red}{\textvisiblespace}} 
        \newcommand*\Wrappedcontinuationsymbol {\textcolor{red}{\llap{\tiny$\m@th\hookrightarrow$}}} 
        \newcommand*\Wrappedcontinuationindent {3ex } 
        \newcommand*\Wrappedafterbreak {\kern\Wrappedcontinuationindent\copy\Wrappedcontinuationbox} 
        % Take advantage of the already applied Pygments mark-up to insert 
        % potential linebreaks for TeX processing. 
        %        {, <, #, %, $, ' and ": go to next line. 
        %        _, }, ^, &, >, - and ~: stay at end of broken line. 
        % Use of \textquotesingle for straight quote. 
        \newcommand*\Wrappedbreaksatspecials {% 
            \def\PYGZus{\discretionary{\char`\_}{\Wrappedafterbreak}{\char`\_}}% 
            \def\PYGZob{\discretionary{}{\Wrappedafterbreak\char`\{}{\char`\{}}% 
            \def\PYGZcb{\discretionary{\char`\}}{\Wrappedafterbreak}{\char`\}}}% 
            \def\PYGZca{\discretionary{\char`\^}{\Wrappedafterbreak}{\char`\^}}% 
            \def\PYGZam{\discretionary{\char`\&}{\Wrappedafterbreak}{\char`\&}}% 
            \def\PYGZlt{\discretionary{}{\Wrappedafterbreak\char`\<}{\char`\<}}% 
            \def\PYGZgt{\discretionary{\char`\>}{\Wrappedafterbreak}{\char`\>}}% 
            \def\PYGZsh{\discretionary{}{\Wrappedafterbreak\char`\#}{\char`\#}}% 
            \def\PYGZpc{\discretionary{}{\Wrappedafterbreak\char`\%}{\char`\%}}% 
            \def\PYGZdl{\discretionary{}{\Wrappedafterbreak\char`\$}{\char`\$}}% 
            \def\PYGZhy{\discretionary{\char`\-}{\Wrappedafterbreak}{\char`\-}}% 
            \def\PYGZsq{\discretionary{}{\Wrappedafterbreak\textquotesingle}{\textquotesingle}}% 
            \def\PYGZdq{\discretionary{}{\Wrappedafterbreak\char`\"}{\char`\"}}% 
            \def\PYGZti{\discretionary{\char`\~}{\Wrappedafterbreak}{\char`\~}}% 
        } 
        % Some characters . , ; ? ! / are not pygmentized. 
        % This macro makes them "active" and they will insert potential linebreaks 
        \newcommand*\Wrappedbreaksatpunct {% 
            \lccode`\~`\.\lowercase{\def~}{\discretionary{\hbox{\char`\.}}{\Wrappedafterbreak}{\hbox{\char`\.}}}% 
            \lccode`\~`\,\lowercase{\def~}{\discretionary{\hbox{\char`\,}}{\Wrappedafterbreak}{\hbox{\char`\,}}}% 
            \lccode`\~`\;\lowercase{\def~}{\discretionary{\hbox{\char`\;}}{\Wrappedafterbreak}{\hbox{\char`\;}}}% 
            \lccode`\~`\:\lowercase{\def~}{\discretionary{\hbox{\char`\:}}{\Wrappedafterbreak}{\hbox{\char`\:}}}% 
            \lccode`\~`\?\lowercase{\def~}{\discretionary{\hbox{\char`\?}}{\Wrappedafterbreak}{\hbox{\char`\?}}}% 
            \lccode`\~`\!\lowercase{\def~}{\discretionary{\hbox{\char`\!}}{\Wrappedafterbreak}{\hbox{\char`\!}}}% 
            \lccode`\~`\/\lowercase{\def~}{\discretionary{\hbox{\char`\/}}{\Wrappedafterbreak}{\hbox{\char`\/}}}% 
            \catcode`\.\active
            \catcode`\,\active 
            \catcode`\;\active
            \catcode`\:\active
            \catcode`\?\active
            \catcode`\!\active
            \catcode`\/\active 
            \lccode`\~`\~ 	
        }
    \makeatother

    \let\OriginalVerbatim=\Verbatim
    \makeatletter
    \renewcommand{\Verbatim}[1][1]{%
        %\parskip\z@skip
        \sbox\Wrappedcontinuationbox {\Wrappedcontinuationsymbol}%
        \sbox\Wrappedvisiblespacebox {\FV@SetupFont\Wrappedvisiblespace}%
        \def\FancyVerbFormatLine ##1{\hsize\linewidth
            \vtop{\raggedright\hyphenpenalty\z@\exhyphenpenalty\z@
                \doublehyphendemerits\z@\finalhyphendemerits\z@
                \strut ##1\strut}%
        }%
        % If the linebreak is at a space, the latter will be displayed as visible
        % space at end of first line, and a continuation symbol starts next line.
        % Stretch/shrink are however usually zero for typewriter font.
        \def\FV@Space {%
            \nobreak\hskip\z@ plus\fontdimen3\font minus\fontdimen4\font
            \discretionary{\copy\Wrappedvisiblespacebox}{\Wrappedafterbreak}
            {\kern\fontdimen2\font}%
        }%
        
        % Allow breaks at special characters using \PYG... macros.
        \Wrappedbreaksatspecials
        % Breaks at punctuation characters . , ; ? ! and / need catcode=\active 	
        \OriginalVerbatim[#1,fontsize=\small,codes*=\Wrappedbreaksatpunct]%
    }
    \makeatother

    % Exact colors from NB
    \definecolor{incolor}{HTML}{303F9F}
    \definecolor{outcolor}{HTML}{D84315}
    \definecolor{cellborder}{HTML}{CFCFCF}
    \definecolor{cellbackground}{HTML}{F7F7F7}
    
    % prompt
    \makeatletter
    \newcommand{\boxspacing}{\kern\kvtcb@left@rule\kern\kvtcb@boxsep}
    \makeatother
    \newcommand{\prompt}[4]{
        {\ttfamily\llap{{\color{#2}[#3]:\hspace{3pt}#4}}\vspace{-\baselineskip}}
    }
    

    
    % Prevent overflowing lines due to hard-to-break entities
    \sloppy 
    % Setup hyperref package
    \hypersetup{
      breaklinks=true,  % so long urls are correctly broken across lines
      colorlinks=true,
      urlcolor=urlcolor,
      linkcolor=linkcolor,
      citecolor=citecolor,
      }
    % Slightly bigger margins than the latex defaults
    
    \geometry{verbose,tmargin=1in,bmargin=1in,lmargin=1in,rmargin=1in}
    
    

\begin{document}
    
    \maketitle
    \tableofcontents

    

    
    \hypertarget{handling-data-with-pandas}{%
\section{Handling data with pandas}\label{handling-data-with-pandas}}

\hypertarget{motivation}{%
\subsection{Motivation}\label{motivation}}

So far, we have encountered NumPy arrays as the only way to store
numerical data (we mostly ignored the built-in containers provided
directly in Python). However, while NumPy arrays are great for storing
homogenous data without any particular structure, they are somewhat
limited when we want to use them for high-level data analysis.

For example, we usually want to process data sets with

\begin{enumerate}
\def\labelenumi{\arabic{enumi}.}
\tightlist
\item
  several variables;
\item
  multiple observations, which need not be identical across variables
  (some values may be missing);
\item
  non-homogenous data types: for examples, names need to be stored as
  strings, birthdays as dates and income as a floating-point number.
\end{enumerate}

While NumPy can in principle handle such situations, it puts all the
burden on the user. Most users would prefer to not have to deal with
such low-level details.

Imagine we want to store names, birth dates and annual income for two
people:

\begin{longtable}[]{@{}lll@{}}
\toprule
Name & Date of birth & Income\tabularnewline
\midrule
\endhead
Alice & 1985-01-01 & 30,000\tabularnewline
Bob & 1997-05-12 & -\tabularnewline
\bottomrule
\end{longtable}

No income was reported for Bob, so it's missing. With NumPy, we could do
this as follows:

    \begin{tcolorbox}[breakable, size=fbox, boxrule=1pt, pad at break*=1mm,colback=cellbackground, colframe=cellborder]
\prompt{In}{incolor}{1}{\boxspacing}
\begin{Verbatim}[commandchars=\\\{\}]
\PY{k+kn}{import} \PY{n+nn}{numpy} \PY{k}{as} \PY{n+nn}{np}
\PY{k+kn}{from} \PY{n+nn}{datetime} \PY{k+kn}{import} \PY{n}{date}

\PY{n}{date1} \PY{o}{=} \PY{n}{date}\PY{p}{(}\PY{l+m+mi}{1985}\PY{p}{,} \PY{l+m+mi}{1}\PY{p}{,} \PY{l+m+mi}{1}\PY{p}{)}            \PY{c+c1}{\PYZsh{} birth date for Alice}
\PY{n}{date2} \PY{o}{=} \PY{n}{date}\PY{p}{(}\PY{l+m+mi}{1997}\PY{p}{,} \PY{l+m+mi}{5}\PY{p}{,} \PY{l+m+mi}{12}\PY{p}{)}           \PY{c+c1}{\PYZsh{} birth date for Bob}

\PY{n}{data} \PY{o}{=} \PY{n}{np}\PY{o}{.}\PY{n}{array}\PY{p}{(}\PY{p}{[}\PY{p}{[}\PY{l+s+s1}{\PYZsq{}}\PY{l+s+s1}{Alice}\PY{l+s+s1}{\PYZsq{}}\PY{p}{,} \PY{n}{date1}\PY{p}{,} \PY{l+m+mf}{30000.0}\PY{p}{]}\PY{p}{,}
                 \PY{p}{[}\PY{l+s+s1}{\PYZsq{}}\PY{l+s+s1}{Bob}\PY{l+s+s1}{\PYZsq{}}\PY{p}{,} \PY{n}{date2}\PY{p}{,} \PY{k+kc}{None}\PY{p}{]}\PY{p}{]}\PY{p}{)}

\PY{n}{data}
\end{Verbatim}
\end{tcolorbox}

            \begin{tcolorbox}[breakable, size=fbox, boxrule=.5pt, pad at break*=1mm, opacityfill=0]
\prompt{Out}{outcolor}{1}{\boxspacing}
\begin{Verbatim}[commandchars=\\\{\}]
array([['Alice', datetime.date(1985, 1, 1), 30000.0],
       ['Bob', datetime.date(1997, 5, 12), None]], dtype=object)
\end{Verbatim}
\end{tcolorbox}
        
    \begin{tcolorbox}[breakable, size=fbox, boxrule=1pt, pad at break*=1mm,colback=cellbackground, colframe=cellborder]
\prompt{In}{incolor}{2}{\boxspacing}
\begin{Verbatim}[commandchars=\\\{\}]
\PY{n}{data}\PY{o}{.}\PY{n}{dtype}          \PY{c+c1}{\PYZsh{} print array data type}
\end{Verbatim}
\end{tcolorbox}

            \begin{tcolorbox}[breakable, size=fbox, boxrule=.5pt, pad at break*=1mm, opacityfill=0]
\prompt{Out}{outcolor}{2}{\boxspacing}
\begin{Verbatim}[commandchars=\\\{\}]
dtype('O')
\end{Verbatim}
\end{tcolorbox}
        
    While we can create such arrays, they are almost useless for data
analysis, in particular since everything is stored as a generic
\texttt{object}.

\begin{itemize}
\item
  To be fair, NumPy offers an alternative array type called
  \href{https://numpy.org/doc/stable/user/basics.rec.html}{``record'' or
  ``structured'' array} which can handle fields of different data types.

  However, the pandas library offers much more beyond that, so there is
  little reason to use structured arrays.
\end{itemize}

Pandas was created to offer more versatile data structures that are
straightforward to use for storing, manipulating and analysing
heterogeneous data:

\begin{enumerate}
\def\labelenumi{\arabic{enumi}.}
\item
  Data is clearly organised in \emph{variables} and \emph{observations},
  similar to econometrics programs such as Stata.
\item
  Each variable is permitted to have a different data type.
\item
  We can use \emph{labels} to select observations, instead of having to
  use a linear numerical index as with NumPy.

  We could, for example, index a data set using National Insurance
  Numbers.
\item
  Pandas offers many convenient data aggregation and reduction routines
  that can be applied to subsets of data.

  For example, we can easily group observations by city and compute
  average incomes.
\item
  Pandas also offers many convenient data import / export functions that
  go beyond what's in NumPy.
\end{enumerate}

Should we be using pandas at all times, then? No!

\begin{itemize}
\tightlist
\item
  For low-level tasks where performance is essential, use NumPy.
\item
  For homogenous data without any particular data structure, use NumPy.
\item
  On the other hand, if data is heterogeneous, needs to be imported from
  an external data source and cleaned or transformed before performing
  computations, use pandas.
\end{itemize}

There are numerous tutorials on pandas on the internet, so we will keep
this unit short and illustrate only the main concepts. Useful references
to additional material include:

\begin{itemize}
\tightlist
\item
  The official
  \href{https://pandas.pydata.org/docs/user_guide/index.html}{user
  guide}.
\item
  The official
  \href{https://pandas.pydata.org/Pandas_Cheat_Sheet.pdf}{pandas cheat
  sheet} which nicely illustrates the most frequently used operations.
\item
  The official
  \href{https://pandas.pydata.org/docs/reference/index.html}{API
  reference} with details on every pandas object and function.
\item
  There are numerous tutorials (including videos) available on the
  internet. See
  \href{https://pandas.pydata.org/docs/getting_started/tutorials.html}{here}
  for a list.
\end{itemize}


\hypertarget{creating-pandas-data-structures}{%
\subsection{Creating pandas data
structures}\label{creating-pandas-data-structures}}

Pandas has two main data structures:

\begin{enumerate}
\def\labelenumi{\arabic{enumi}.}
\tightlist
\item
  \texttt{Series} represents observations of a single variable.
\item
  \texttt{DataFrame} is a container for several variables. You can think
  of each individual column of a \texttt{DataFrame} as a
  \texttt{Series}, and each row represents one observation.
\end{enumerate}

The easiest way to create a \texttt{Series} or \texttt{DataFrame} is to
create them from pre-existing data.

To access pandas data structures and routines, we need to import them
first. The near-universal convention is to make pandas available using
the name \texttt{pd}:

    \begin{tcolorbox}[breakable, size=fbox, boxrule=1pt, pad at break*=1mm,colback=cellbackground, colframe=cellborder]
\prompt{In}{incolor}{3}{\boxspacing}
\begin{Verbatim}[commandchars=\\\{\}]
\PY{k+kn}{import} \PY{n+nn}{pandas} \PY{k}{as} \PY{n+nn}{pd}
\end{Verbatim}
\end{tcolorbox}

    \emph{Examples:}

We can create a \texttt{DataFrame} from a NumPy array:

    \begin{tcolorbox}[breakable, size=fbox, boxrule=1pt, pad at break*=1mm,colback=cellbackground, colframe=cellborder]
\prompt{In}{incolor}{4}{\boxspacing}
\begin{Verbatim}[commandchars=\\\{\}]
\PY{k+kn}{import} \PY{n+nn}{numpy} \PY{k}{as} \PY{n+nn}{np}
\PY{k+kn}{import} \PY{n+nn}{pandas} \PY{k}{as} \PY{n+nn}{pd}             \PY{c+c1}{\PYZsh{} universal convention: import using pd}
\PY{k+kn}{from} \PY{n+nn}{numpy}\PY{n+nn}{.}\PY{n+nn}{random} \PY{k+kn}{import} \PY{n}{default\PYZus{}rng}

\PY{c+c1}{\PYZsh{} Draw normally distributed data}
\PY{n}{rng} \PY{o}{=} \PY{n}{default\PYZus{}rng}\PY{p}{(}\PY{l+m+mi}{123}\PY{p}{)}
\PY{n}{data} \PY{o}{=} \PY{n}{rng}\PY{o}{.}\PY{n}{normal}\PY{p}{(}\PY{n}{size}\PY{o}{=}\PY{p}{(}\PY{l+m+mi}{10}\PY{p}{,}\PY{l+m+mi}{3}\PY{p}{)}\PY{p}{)}

\PY{c+c1}{\PYZsh{} Define variable (or column) names}
\PY{n}{varnames} \PY{o}{=} \PY{p}{[}\PY{l+s+s1}{\PYZsq{}}\PY{l+s+s1}{A}\PY{l+s+s1}{\PYZsq{}}\PY{p}{,} \PY{l+s+s1}{\PYZsq{}}\PY{l+s+s1}{B}\PY{l+s+s1}{\PYZsq{}}\PY{p}{,} \PY{l+s+s1}{\PYZsq{}}\PY{l+s+s1}{C}\PY{l+s+s1}{\PYZsq{}}\PY{p}{]}

\PY{c+c1}{\PYZsh{} Create pandas DataFrame}
\PY{n}{pd}\PY{o}{.}\PY{n}{DataFrame}\PY{p}{(}\PY{n}{data}\PY{p}{,} \PY{n}{columns}\PY{o}{=}\PY{n}{varnames}\PY{p}{)}
\end{Verbatim}
\end{tcolorbox}

            \begin{tcolorbox}[breakable, size=fbox, boxrule=.5pt, pad at break*=1mm, opacityfill=0]
\prompt{Out}{outcolor}{4}{\boxspacing}
\begin{Verbatim}[commandchars=\\\{\}]
          A         B         C
0 -0.989121 -0.367787  1.287925
1  0.193974  0.920231  0.577104
2 -0.636464  0.541952 -0.316595
3 -0.322389  0.097167 -1.525930
4  1.192166 -0.671090  1.000269
5  0.136321  1.532033 -0.659969
6 -0.311795  0.337769 -2.207471
7  0.827921  1.541630  1.126807
8  0.754770 -0.145978  1.281902
9  1.074031  0.392621  0.005114
\end{Verbatim}
\end{tcolorbox}
        
    This code creates a \texttt{DataFrame} of three variables called
\texttt{A}, \texttt{B} and \texttt{C} with 10 observations each.

Alternatively, we can create a \texttt{DataFrame} from non-homogenous
data as follows:

    \begin{tcolorbox}[breakable, size=fbox, boxrule=1pt, pad at break*=1mm,colback=cellbackground, colframe=cellborder]
\prompt{In}{incolor}{5}{\boxspacing}
\begin{Verbatim}[commandchars=\\\{\}]
\PY{c+c1}{\PYZsh{} Names (strings)}
\PY{n}{names} \PY{o}{=} \PY{p}{[}\PY{l+s+s1}{\PYZsq{}}\PY{l+s+s1}{Alice}\PY{l+s+s1}{\PYZsq{}}\PY{p}{,} \PY{l+s+s1}{\PYZsq{}}\PY{l+s+s1}{Bob}\PY{l+s+s1}{\PYZsq{}}\PY{p}{]}

\PY{c+c1}{\PYZsh{} Birth dates (datetime objects)}
\PY{n}{bdates} \PY{o}{=} \PY{n}{pd}\PY{o}{.}\PY{n}{to\PYZus{}datetime}\PY{p}{(}\PY{p}{[}\PY{l+s+s1}{\PYZsq{}}\PY{l+s+s1}{1985\PYZhy{}01\PYZhy{}01}\PY{l+s+s1}{\PYZsq{}}\PY{p}{,} \PY{l+s+s1}{\PYZsq{}}\PY{l+s+s1}{1997\PYZhy{}05\PYZhy{}12}\PY{l+s+s1}{\PYZsq{}}\PY{p}{]}\PY{p}{)}

\PY{c+c1}{\PYZsh{} Incomes (floats)}
\PY{n}{incomes} \PY{o}{=} \PY{n}{np}\PY{o}{.}\PY{n}{array}\PY{p}{(}\PY{p}{[}\PY{l+m+mi}{35000}\PY{p}{,} \PY{n}{np}\PY{o}{.}\PY{n}{nan}\PY{p}{]}\PY{p}{)}         \PY{c+c1}{\PYZsh{} code missing income as NaN}

\PY{c+c1}{\PYZsh{} create DataFrame from dictionary}
\PY{n}{pd}\PY{o}{.}\PY{n}{DataFrame}\PY{p}{(}\PY{p}{\PYZob{}}\PY{l+s+s1}{\PYZsq{}}\PY{l+s+s1}{Name}\PY{l+s+s1}{\PYZsq{}}\PY{p}{:} \PY{n}{names}\PY{p}{,} \PY{l+s+s1}{\PYZsq{}}\PY{l+s+s1}{Birthdate}\PY{l+s+s1}{\PYZsq{}}\PY{p}{:} \PY{n}{bdates}\PY{p}{,} \PY{l+s+s1}{\PYZsq{}}\PY{l+s+s1}{Income}\PY{l+s+s1}{\PYZsq{}}\PY{p}{:} \PY{n}{incomes}\PY{p}{\PYZcb{}}\PY{p}{)}
\end{Verbatim}
\end{tcolorbox}

            \begin{tcolorbox}[breakable, size=fbox, boxrule=.5pt, pad at break*=1mm, opacityfill=0]
\prompt{Out}{outcolor}{5}{\boxspacing}
\begin{Verbatim}[commandchars=\\\{\}]
    Name  Birthdate   Income
0  Alice 1985-01-01  35000.0
1    Bob 1997-05-12      NaN
\end{Verbatim}
\end{tcolorbox}
        
    If data types differ across columns, as in the above example, it is
often convenient to create the \texttt{DataFrame} by passing a
dictionary as an argument. Each key represents a column name and each
corresponding value contains the data for that variable.


\hypertarget{viewing-data}{%
\subsection{Viewing data}\label{viewing-data}}

With large data sets, you hardly ever want to print the entire
\texttt{DataFrame}. Pandas by default limits the amount of data shown.
You can use the \texttt{head()} and \texttt{tail()} methods to
explicitly display a specific number of rows from the top or the end of
a \texttt{DataFrame}.

To illustrate, we use a data set of 23 UK universities that contains the
following variables:

\begin{itemize}
\tightlist
\item
  \texttt{Instititution}: Name of the institution
\item
  \texttt{Country}: Country/nation within the UK (England, Scotland,
  \ldots)
\item
  \texttt{Founded}: Year in which university (or a predecessor
  institution) was founded
\item
  \texttt{Students}: Total number of students
\item
  \texttt{Staff}: Number of academic staff
\item
  \texttt{Admin}: Number of administrative staff
\item
  \texttt{Budget}: Budget in million pounds
\item
  \texttt{Russell}: Binary indicator whether university is a member of
  the \href{https://en.wikipedia.org/wiki/Russell_Group}{Russell Group},
  an association of the UK's top research universities.
\end{itemize}

The data was compiled based on information from Wikipedia.

We read in the data stored in the file \texttt{universities.csv} (from
the \texttt{data/} folder) like this:

    \begin{tcolorbox}[breakable, size=fbox, boxrule=1pt, pad at break*=1mm,colback=cellbackground, colframe=cellborder]
\prompt{In}{incolor}{6}{\boxspacing}
\begin{Verbatim}[commandchars=\\\{\}]
\PY{k+kn}{import} \PY{n+nn}{pandas} \PY{k}{as} \PY{n+nn}{pd}

\PY{c+c1}{\PYZsh{} relative path to CSV file}
\PY{n}{file} \PY{o}{=} \PY{l+s+s1}{\PYZsq{}}\PY{l+s+s1}{../data/universities.csv}\PY{l+s+s1}{\PYZsq{}}

\PY{c+c1}{\PYZsh{} Load sample data set of UK universities}
\PY{n}{df} \PY{o}{=} \PY{n}{pd}\PY{o}{.}\PY{n}{read\PYZus{}csv}\PY{p}{(}\PY{n}{file}\PY{p}{,} \PY{n}{sep}\PY{o}{=}\PY{l+s+s1}{\PYZsq{}}\PY{l+s+s1}{;}\PY{l+s+s1}{\PYZsq{}}\PY{p}{)}
\end{Verbatim}
\end{tcolorbox}

    We can now display the first and last three rows:

    \begin{tcolorbox}[breakable, size=fbox, boxrule=1pt, pad at break*=1mm,colback=cellbackground, colframe=cellborder]
\prompt{In}{incolor}{7}{\boxspacing}
\begin{Verbatim}[commandchars=\\\{\}]
\PY{n}{df}\PY{o}{.}\PY{n}{head}\PY{p}{(}\PY{l+m+mi}{3}\PY{p}{)}      \PY{c+c1}{\PYZsh{} show first three rows}
\end{Verbatim}
\end{tcolorbox}

            \begin{tcolorbox}[breakable, size=fbox, boxrule=.5pt, pad at break*=1mm, opacityfill=0]
\prompt{Out}{outcolor}{7}{\boxspacing}
\begin{Verbatim}[commandchars=\\\{\}]
                Institution   Country  Founded  Students   Staff   Admin  \textbackslash{}
0     University of Glasgow  Scotland     1451     30805  2942.0  4003.0
1   University of Edinburgh  Scotland     1583     34275  4589.0  6107.0
2  University of St Andrews  Scotland     1413      8984  1137.0  1576.0

   Budget  Russell
0   626.5        1
1  1102.0        1
2   251.2        0
\end{Verbatim}
\end{tcolorbox}
        
    \begin{tcolorbox}[breakable, size=fbox, boxrule=1pt, pad at break*=1mm,colback=cellbackground, colframe=cellborder]
\prompt{In}{incolor}{8}{\boxspacing}
\begin{Verbatim}[commandchars=\\\{\}]
\PY{n}{df}\PY{o}{.}\PY{n}{tail}\PY{p}{(}\PY{l+m+mi}{3}\PY{p}{)}      \PY{c+c1}{\PYZsh{} show last three rows}
\end{Verbatim}
\end{tcolorbox}

            \begin{tcolorbox}[breakable, size=fbox, boxrule=.5pt, pad at break*=1mm, opacityfill=0]
\prompt{Out}{outcolor}{8}{\boxspacing}
\begin{Verbatim}[commandchars=\\\{\}]
                   Institution           Country  Founded  Students   Staff  \textbackslash{}
20      University of Stirling          Scotland     1967      9548     NaN
21  Queen’s University Belfast  Northern Ireland     1810     18438  2414.0
22          Swansea University             Wales     1920     20620     NaN

     Admin  Budget  Russell
20  1872.0   113.3        0
21  1489.0   369.2        1
22  3290.0     NaN        0
\end{Verbatim}
\end{tcolorbox}
        
    To quickly compute some descriptive statistics for the \emph{numerical}
variables in the \texttt{DataFrame}, we use \texttt{describe()}:

    \begin{tcolorbox}[breakable, size=fbox, boxrule=1pt, pad at break*=1mm,colback=cellbackground, colframe=cellborder]
\prompt{In}{incolor}{9}{\boxspacing}
\begin{Verbatim}[commandchars=\\\{\}]
\PY{n}{df}\PY{o}{.}\PY{n}{describe}\PY{p}{(}\PY{p}{)}
\end{Verbatim}
\end{tcolorbox}

            \begin{tcolorbox}[breakable, size=fbox, boxrule=.5pt, pad at break*=1mm, opacityfill=0]
\prompt{Out}{outcolor}{9}{\boxspacing}
\begin{Verbatim}[commandchars=\\\{\}]
           Founded      Students        Staff        Admin       Budget  \textbackslash{}
count    23.000000     23.000000    20.000000    19.000000    22.000000
mean   1745.652174  24106.782609  3664.250000  3556.736842   768.609091
std     256.992149   9093.000735  2025.638038  1550.434342   608.234948
min    1096.000000   8984.000000  1086.000000  1489.000000   113.300000
25\%    1589.000000  18776.500000  2294.250000  2193.500000   340.850000
50\%    1826.000000  23247.000000  3307.500000  3485.000000   643.750000
75\%    1941.500000  30801.500000  4439.750000  4347.500000  1023.500000
max    2004.000000  41180.000000  7913.000000  6199.000000  2450.000000

         Russell
count  23.000000
mean    0.739130
std     0.448978
min     0.000000
25\%     0.500000
50\%     1.000000
75\%     1.000000
max     1.000000
\end{Verbatim}
\end{tcolorbox}
        
    Note that this automatically ignores the columns \texttt{Institution}
and \texttt{Country} as they contain strings and computing the mean,
etc. of a string variable does not make sense.

To see low-level information about the data type used in each column, we
call \texttt{info()}:

    \begin{tcolorbox}[breakable, size=fbox, boxrule=1pt, pad at break*=1mm,colback=cellbackground, colframe=cellborder]
\prompt{In}{incolor}{10}{\boxspacing}
\begin{Verbatim}[commandchars=\\\{\}]
\PY{n}{df}\PY{o}{.}\PY{n}{info}\PY{p}{(}\PY{p}{)}
\end{Verbatim}
\end{tcolorbox}

    \begin{Verbatim}[commandchars=\\\{\}]
<class 'pandas.core.frame.DataFrame'>
RangeIndex: 23 entries, 0 to 22
Data columns (total 8 columns):
 \#   Column       Non-Null Count  Dtype
---  ------       --------------  -----
 0   Institution  23 non-null     object
 1   Country      23 non-null     object
 2   Founded      23 non-null     int64
 3   Students     23 non-null     int64
 4   Staff        20 non-null     float64
 5   Admin        19 non-null     float64
 6   Budget       22 non-null     float64
 7   Russell      23 non-null     int64
dtypes: float64(3), int64(3), object(2)
memory usage: 1.6+ KB
    \end{Verbatim}

    Pandas automatically discards missing information in computations. For
example, the number of academic staff is missing for several
universities, so the number of \emph{non-null} entries reported in the
table above is less than 23, the overall sample size.


\hypertarget{indexing}{%
\subsection{Indexing}\label{indexing}}

Pandas supports two types of indexing:

\begin{enumerate}
\def\labelenumi{\arabic{enumi}.}
\item
  Indexing by position. This is basically identical to the indexing of
  other Python and NumPy containers.
\item
  Indexing by label, \ie by the values assigned to the row or column
  index. These labels need not be integers in increasing order, as is
  the case for NumPy.

  We will see how to assign labels below.
\end{enumerate}

Pandas indexing is performed either by using brackets \texttt{{[}{]}},
or by using \texttt{.loc{[}{]}} for label indexing, or
\texttt{.iloc{[}{]}} for positional indexing.

Indexing via \texttt{{[}{]}} can be somewhat confusing:

\begin{itemize}
\tightlist
\item
  specifying \texttt{df{[}\textquotesingle{}name\textquotesingle{}{]}}
  returns the column \texttt{name} as a \texttt{Series} object.
\item
  On the other hand, specifying a range such as \texttt{df{[}5:10{]}}
  returns the \emph{rows} associated with the \emph{positions}
  5,\ldots,9.
\end{itemize}

\emph{Examples:}

    \begin{tcolorbox}[breakable, size=fbox, boxrule=1pt, pad at break*=1mm,colback=cellbackground, colframe=cellborder]
\prompt{In}{incolor}{11}{\boxspacing}
\begin{Verbatim}[commandchars=\\\{\}]
\PY{k+kn}{import} \PY{n+nn}{pandas} \PY{k}{as} \PY{n+nn}{pd}

\PY{c+c1}{\PYZsh{} Load sample data set of UK universities}
\PY{n}{df} \PY{o}{=} \PY{n}{pd}\PY{o}{.}\PY{n}{read\PYZus{}csv}\PY{p}{(}\PY{l+s+s1}{\PYZsq{}}\PY{l+s+s1}{../data/universities.csv}\PY{l+s+s1}{\PYZsq{}}\PY{p}{,} \PY{n}{sep}\PY{o}{=}\PY{l+s+s1}{\PYZsq{}}\PY{l+s+s1}{;}\PY{l+s+s1}{\PYZsq{}}\PY{p}{)}
\PY{n}{df}\PY{p}{[}\PY{l+s+s1}{\PYZsq{}}\PY{l+s+s1}{Institution}\PY{l+s+s1}{\PYZsq{}}\PY{p}{]}               \PY{c+c1}{\PYZsh{} select a single column}
\end{Verbatim}
\end{tcolorbox}

            \begin{tcolorbox}[breakable, size=fbox, boxrule=.5pt, pad at break*=1mm, opacityfill=0]
\prompt{Out}{outcolor}{11}{\boxspacing}
\begin{Verbatim}[commandchars=\\\{\}]
0               University of Glasgow
1             University of Edinburgh
2            University of St Andrews
3              University of Aberdeen
4           University of Strathclyde
5                                 LSE
6                                 UCL
7             University of Cambridge
8                University of Oxford
9               University of Warwick
10            Imperial College London
11              King’s College London
12           University of Manchester
13              University of Bristol
14           University of Birmingham
15    Queen Mary University of London
16                 University of York
17           University of Nottingham
18               University of Dundee
19                 Cardiff University
20             University of Stirling
21         Queen’s University Belfast
22                 Swansea University
Name: Institution, dtype: object
\end{Verbatim}
\end{tcolorbox}
        
    \begin{tcolorbox}[breakable, size=fbox, boxrule=1pt, pad at break*=1mm,colback=cellbackground, colframe=cellborder]
\prompt{In}{incolor}{12}{\boxspacing}
\begin{Verbatim}[commandchars=\\\{\}]
\PY{n}{df}\PY{p}{[}\PY{p}{[}\PY{l+s+s1}{\PYZsq{}}\PY{l+s+s1}{Institution}\PY{l+s+s1}{\PYZsq{}}\PY{p}{,} \PY{l+s+s1}{\PYZsq{}}\PY{l+s+s1}{Students}\PY{l+s+s1}{\PYZsq{}}\PY{p}{]}\PY{p}{]}     \PY{c+c1}{\PYZsh{} select multiple columns using a list}
\end{Verbatim}
\end{tcolorbox}

            \begin{tcolorbox}[breakable, size=fbox, boxrule=.5pt, pad at break*=1mm, opacityfill=0]
\prompt{Out}{outcolor}{12}{\boxspacing}
\begin{Verbatim}[commandchars=\\\{\}]
                        Institution  Students
0             University of Glasgow     30805
1           University of Edinburgh     34275
2          University of St Andrews      8984
3            University of Aberdeen     14775
4         University of Strathclyde     22640
5                               LSE     11850
6                               UCL     41180
7           University of Cambridge     23247
8              University of Oxford     24515
9             University of Warwick     27278
10          Imperial College London     19115
11            King’s College London     32895
12         University of Manchester     40250
13            University of Bristol     25955
14         University of Birmingham     35445
15  Queen Mary University of London     20560
16               University of York     19470
17         University of Nottingham     30798
18             University of Dundee     15915
19               Cardiff University     25898
20           University of Stirling      9548
21       Queen’s University Belfast     18438
22               Swansea University     20620
\end{Verbatim}
\end{tcolorbox}
        
    To return the rows at positions 1, 2 and 3 we use

    \begin{tcolorbox}[breakable, size=fbox, boxrule=1pt, pad at break*=1mm,colback=cellbackground, colframe=cellborder]
\prompt{In}{incolor}{13}{\boxspacing}
\begin{Verbatim}[commandchars=\\\{\}]
\PY{n}{df}\PY{p}{[}\PY{l+m+mi}{1}\PY{p}{:}\PY{l+m+mi}{4}\PY{p}{]}
\end{Verbatim}
\end{tcolorbox}

            \begin{tcolorbox}[breakable, size=fbox, boxrule=.5pt, pad at break*=1mm, opacityfill=0]
\prompt{Out}{outcolor}{13}{\boxspacing}
\begin{Verbatim}[commandchars=\\\{\}]
                Institution   Country  Founded  Students   Staff   Admin  \textbackslash{}
1   University of Edinburgh  Scotland     1583     34275  4589.0  6107.0
2  University of St Andrews  Scotland     1413      8984  1137.0  1576.0
3    University of Aberdeen  Scotland     1495     14775  1086.0  1489.0

   Budget  Russell
1  1102.0        1
2   251.2        0
3   219.5        0
\end{Verbatim}
\end{tcolorbox}
        
    Pandas follows the Python convention that indices are 0-based, and the
endpoint of a slice is not included.

    \hypertarget{manipulating-indices}{%
\subsubsection{Manipulating indices}\label{manipulating-indices}}

Pandas uses \emph{labels} to index and align data. These can be integer
values starting at 0 with increments of 1 for each additional element,
which is the default, but they need not be. The two main methods to
manipulate indices are:

\begin{itemize}
\tightlist
\item
  \texttt{set\_index(keys={[}\textquotesingle{}column1\textquotesingle{},\ ...{]})}:
  uses the values of \texttt{column1} and optionally additional columns
  as indices, discarding the current index.
\item
  \texttt{reset\_index()}: resets the index to its default value, a
  sequence of increasing integers starting at 0.
\end{itemize}

Both methods return a new \texttt{DataFrame} and leave the original
\texttt{DataFrame} unchanged. If we want to change the existing
\texttt{DataFrame}, we need to pass the argument \texttt{inplace=True}.

For example, we can replace the row index and use the Roman lower-case
characters \texttt{a}, \texttt{b}, \texttt{c}, \ldots{} as labels
instead of integers:

    \begin{tcolorbox}[breakable, size=fbox, boxrule=1pt, pad at break*=1mm,colback=cellbackground, colframe=cellborder]
\prompt{In}{incolor}{14}{\boxspacing}
\begin{Verbatim}[commandchars=\\\{\}]
\PY{k+kn}{import} \PY{n+nn}{pandas} \PY{k}{as} \PY{n+nn}{pd}
\PY{n}{df} \PY{o}{=} \PY{n}{pd}\PY{o}{.}\PY{n}{read\PYZus{}csv}\PY{p}{(}\PY{l+s+s1}{\PYZsq{}}\PY{l+s+s1}{../data/universities.csv}\PY{l+s+s1}{\PYZsq{}}\PY{p}{,} \PY{n}{sep}\PY{o}{=}\PY{l+s+s1}{\PYZsq{}}\PY{l+s+s1}{;}\PY{l+s+s1}{\PYZsq{}}\PY{p}{)}

\PY{c+c1}{\PYZsh{} Create list of lower\PYZhy{}case letters which has same}
\PY{c+c1}{\PYZsh{} length as the number of observations.}
\PY{n}{index} \PY{o}{=} \PY{p}{[}\PY{n+nb}{chr}\PY{p}{(}\PY{l+m+mi}{97}\PY{o}{+}\PY{n}{i}\PY{p}{)} \PY{k}{for} \PY{n}{i} \PY{o+ow}{in} \PY{n+nb}{range}\PY{p}{(}\PY{n+nb}{len}\PY{p}{(}\PY{n}{df}\PY{p}{)}\PY{p}{)}\PY{p}{]}     \PY{c+c1}{\PYZsh{} len(df) returns number of obs.}
\PY{n}{index}
\end{Verbatim}
\end{tcolorbox}

            \begin{tcolorbox}[breakable, size=fbox, boxrule=.5pt, pad at break*=1mm, opacityfill=0]
\prompt{Out}{outcolor}{14}{\boxspacing}
\begin{Verbatim}[commandchars=\\\{\}]
['a',
 'b',
 'c',
 'd',
 'e',
 'f',
 'g',
 'h',
 'i',
 'j',
 'k',
 'l',
 'm',
 'n',
 'o',
 'p',
 'q',
 'r',
 's',
 't',
 'u',
 'v',
 'w']
\end{Verbatim}
\end{tcolorbox}
        
    \begin{tcolorbox}[breakable, size=fbox, boxrule=1pt, pad at break*=1mm,colback=cellbackground, colframe=cellborder]
\prompt{In}{incolor}{15}{\boxspacing}
\begin{Verbatim}[commandchars=\\\{\}]
\PY{n}{df}\PY{p}{[}\PY{l+s+s1}{\PYZsq{}}\PY{l+s+s1}{index}\PY{l+s+s1}{\PYZsq{}}\PY{p}{]} \PY{o}{=} \PY{n}{index}                             \PY{c+c1}{\PYZsh{} create new column \PYZsq{}index\PYZsq{}}
\PY{n}{df}\PY{o}{.}\PY{n}{set\PYZus{}index}\PY{p}{(}\PY{n}{keys}\PY{o}{=}\PY{p}{[}\PY{l+s+s1}{\PYZsq{}}\PY{l+s+s1}{index}\PY{l+s+s1}{\PYZsq{}}\PY{p}{]}\PY{p}{,} \PY{n}{inplace}\PY{o}{=}\PY{k+kc}{True}\PY{p}{)}      \PY{c+c1}{\PYZsh{} set letters as index!}

\PY{c+c1}{\PYZsh{} print first 3 rows using labels}
\PY{n}{df}\PY{p}{[}\PY{l+s+s1}{\PYZsq{}}\PY{l+s+s1}{a}\PY{l+s+s1}{\PYZsq{}}\PY{p}{:}\PY{l+s+s1}{\PYZsq{}}\PY{l+s+s1}{c}\PY{l+s+s1}{\PYZsq{}}\PY{p}{]}             \PY{c+c1}{\PYZsh{} This is the same as df[:3]}
\end{Verbatim}
\end{tcolorbox}

            \begin{tcolorbox}[breakable, size=fbox, boxrule=.5pt, pad at break*=1mm, opacityfill=0]
\prompt{Out}{outcolor}{15}{\boxspacing}
\begin{Verbatim}[commandchars=\\\{\}]
                    Institution   Country  Founded  Students   Staff   Admin  \textbackslash{}
index
a         University of Glasgow  Scotland     1451     30805  2942.0  4003.0
b       University of Edinburgh  Scotland     1583     34275  4589.0  6107.0
c      University of St Andrews  Scotland     1413      8984  1137.0  1576.0

       Budget  Russell
index
a       626.5        1
b      1102.0        1
c       251.2        0
\end{Verbatim}
\end{tcolorbox}
        
    To add to the confusion, note that when specifying a range in terms of
labels, the last element \emph{is} included! Hence the row with index
\texttt{c} in the above example is shown.

We can reset the index to its default integer values using the
\texttt{reset\_index()} method:

    \begin{tcolorbox}[breakable, size=fbox, boxrule=1pt, pad at break*=1mm,colback=cellbackground, colframe=cellborder]
\prompt{In}{incolor}{16}{\boxspacing}
\begin{Verbatim}[commandchars=\\\{\}]
\PY{c+c1}{\PYZsh{} Reset index labels to default value (integers 0, 1, 2, ...)}
\PY{n}{df\PYZus{}new} \PY{o}{=} \PY{n}{df}\PY{o}{.}\PY{n}{reset\PYZus{}index}\PY{p}{(}\PY{n}{drop}\PY{o}{=}\PY{k+kc}{True}\PY{p}{)}
\PY{n}{df\PYZus{}new}\PY{o}{.}\PY{n}{head}\PY{p}{(}\PY{l+m+mi}{3}\PY{p}{)}          \PY{c+c1}{\PYZsh{} print first 3 rows of new DataFrame}
\end{Verbatim}
\end{tcolorbox}

            \begin{tcolorbox}[breakable, size=fbox, boxrule=.5pt, pad at break*=1mm, opacityfill=0]
\prompt{Out}{outcolor}{16}{\boxspacing}
\begin{Verbatim}[commandchars=\\\{\}]
                Institution   Country  Founded  Students   Staff   Admin  \textbackslash{}
0     University of Glasgow  Scotland     1451     30805  2942.0  4003.0
1   University of Edinburgh  Scotland     1583     34275  4589.0  6107.0
2  University of St Andrews  Scotland     1413      8984  1137.0  1576.0

   Budget  Russell
0   626.5        1
1  1102.0        1
2   251.2        0
\end{Verbatim}
\end{tcolorbox}
        
    The \texttt{drop=True} argument tells pandas to throw away the old index
values instead of storing them as a column of the resulting
\texttt{DataFrame}.

\hypertarget{selecting-elements}{%
\subsubsection{Selecting elements}\label{selecting-elements}}

To more clearly distinguish between selection by label and by position,
pandas provides the \texttt{.loc{[}{]}} and \texttt{.iloc{[}{]}} methods
of indexing. To make your intention obvious, you should therefore adhere
to the following rules:

\begin{enumerate}
\def\labelenumi{\arabic{enumi}.}
\tightlist
\item
  Use \texttt{df{[}\textquotesingle{}name\textquotesingle{}{]}} only to
  select \emph{columns} and nothing else.
\item
  Use \texttt{.loc{[}{]}} to select by label.
\item
  Use \texttt{.iloc{[}{]}} to select by position.
\end{enumerate}

\textbf{Selection by label}

To illustrate, using \texttt{.loc{[}{]}} unambiguously indexes by label:

    \begin{tcolorbox}[breakable, size=fbox, boxrule=1pt, pad at break*=1mm,colback=cellbackground, colframe=cellborder]
\prompt{In}{incolor}{17}{\boxspacing}
\begin{Verbatim}[commandchars=\\\{\}]
\PY{n}{df}\PY{o}{.}\PY{n}{loc}\PY{p}{[}\PY{l+s+s1}{\PYZsq{}}\PY{l+s+s1}{d}\PY{l+s+s1}{\PYZsq{}}\PY{p}{:}\PY{l+s+s1}{\PYZsq{}}\PY{l+s+s1}{f}\PY{l+s+s1}{\PYZsq{}}\PY{p}{,} \PY{p}{[}\PY{l+s+s1}{\PYZsq{}}\PY{l+s+s1}{Institution}\PY{l+s+s1}{\PYZsq{}}\PY{p}{,} \PY{l+s+s1}{\PYZsq{}}\PY{l+s+s1}{Students}\PY{l+s+s1}{\PYZsq{}}\PY{p}{]}\PY{p}{]}
\end{Verbatim}
\end{tcolorbox}

            \begin{tcolorbox}[breakable, size=fbox, boxrule=.5pt, pad at break*=1mm, opacityfill=0]
\prompt{Out}{outcolor}{17}{\boxspacing}
\begin{Verbatim}[commandchars=\\\{\}]
                     Institution  Students
index
d         University of Aberdeen     14775
e      University of Strathclyde     22640
f                            LSE     11850
\end{Verbatim}
\end{tcolorbox}
        
    With \texttt{.loc{[}{]}} we can even perform slicing on column names,
which is not possible with the simpler \texttt{df{[}{]}} syntax:

    \begin{tcolorbox}[breakable, size=fbox, boxrule=1pt, pad at break*=1mm,colback=cellbackground, colframe=cellborder]
\prompt{In}{incolor}{18}{\boxspacing}
\begin{Verbatim}[commandchars=\\\{\}]
\PY{n}{df}\PY{o}{.}\PY{n}{loc}\PY{p}{[}\PY{l+s+s1}{\PYZsq{}}\PY{l+s+s1}{d}\PY{l+s+s1}{\PYZsq{}}\PY{p}{:}\PY{l+s+s1}{\PYZsq{}}\PY{l+s+s1}{f}\PY{l+s+s1}{\PYZsq{}}\PY{p}{,} \PY{l+s+s1}{\PYZsq{}}\PY{l+s+s1}{Institution}\PY{l+s+s1}{\PYZsq{}}\PY{p}{:}\PY{l+s+s1}{\PYZsq{}}\PY{l+s+s1}{Founded}\PY{l+s+s1}{\PYZsq{}}\PY{p}{]}
\end{Verbatim}
\end{tcolorbox}

            \begin{tcolorbox}[breakable, size=fbox, boxrule=.5pt, pad at break*=1mm, opacityfill=0]
\prompt{Out}{outcolor}{18}{\boxspacing}
\begin{Verbatim}[commandchars=\\\{\}]
                     Institution   Country  Founded
index
d         University of Aberdeen  Scotland     1495
e      University of Strathclyde  Scotland     1964
f                            LSE   England     1895
\end{Verbatim}
\end{tcolorbox}
        
    This includes all the columns between \texttt{Institution} and
\texttt{Founded}, where the latter is included since we are slicing by
label.

Trying to pass in positional arguments will return an error for the
given \texttt{DataFrame} since the index labels are \texttt{a},
\texttt{b}, \texttt{c},\ldots{} and not 0, 1, 2\ldots{}

    \begin{tcolorbox}[breakable, size=fbox, boxrule=1pt, pad at break*=1mm,colback=cellbackground, colframe=cellborder]
\prompt{In}{incolor}{19}{\boxspacing}
\begin{Verbatim}[commandchars=\\\{\}]
\PY{n}{df}\PY{o}{.}\PY{n}{loc}\PY{p}{[}\PY{l+m+mi}{0}\PY{p}{:}\PY{l+m+mi}{4}\PY{p}{]}
\end{Verbatim}
\end{tcolorbox}

    \begin{Verbatim}[commandchars=\\\{\}, frame=single, framerule=2mm, rulecolor=\color{outerrorbackground}]
\textcolor{ansi-red}{TypeError}\textcolor{ansi-red}{:} cannot do slice indexing on Index with these indexers [0] of type int

    \end{Verbatim}

    However, we can reset the index to its default value. Then the index
labels are integers and coincide with their position, so that
\texttt{.loc{[}{]}} works:

    \begin{tcolorbox}[breakable, size=fbox, boxrule=1pt, pad at break*=1mm,colback=cellbackground, colframe=cellborder]
\prompt{In}{incolor}{20}{\boxspacing}
\begin{Verbatim}[commandchars=\\\{\}]
\PY{n}{df}\PY{o}{.}\PY{n}{reset\PYZus{}index}\PY{p}{(}\PY{n}{inplace}\PY{o}{=}\PY{k+kc}{True}\PY{p}{,} \PY{n}{drop}\PY{o}{=}\PY{k+kc}{True}\PY{p}{)}     \PY{c+c1}{\PYZsh{} reset index labels to integers,}
                                            \PY{c+c1}{\PYZsh{} drop original index}
\PY{n}{df}\PY{o}{.}\PY{n}{loc}\PY{p}{[}\PY{l+m+mi}{0}\PY{p}{:}\PY{l+m+mi}{4}\PY{p}{]}
\end{Verbatim}
\end{tcolorbox}

            \begin{tcolorbox}[breakable, size=fbox, boxrule=.5pt, pad at break*=1mm, opacityfill=0]
\prompt{Out}{outcolor}{20}{\boxspacing}
\begin{Verbatim}[commandchars=\\\{\}]
                 Institution   Country  Founded  Students   Staff   Admin  \textbackslash{}
0      University of Glasgow  Scotland     1451     30805  2942.0  4003.0
1    University of Edinburgh  Scotland     1583     34275  4589.0  6107.0
2   University of St Andrews  Scotland     1413      8984  1137.0  1576.0
3     University of Aberdeen  Scotland     1495     14775  1086.0  1489.0
4  University of Strathclyde  Scotland     1964     22640     NaN  3200.0

   Budget  Russell
0   626.5        1
1  1102.0        1
2   251.2        0
3   219.5        0
4   304.4        0
\end{Verbatim}
\end{tcolorbox}
        
    Again, the end point with label \texttt{4} is included because we are
selecting by label.

Somewhat surprisingly, we can also pass boolean arrays to
\texttt{.loc{[}{]}} even though these are clearly not labels:

    \begin{tcolorbox}[breakable, size=fbox, boxrule=1pt, pad at break*=1mm,colback=cellbackground, colframe=cellborder]
\prompt{In}{incolor}{21}{\boxspacing}
\begin{Verbatim}[commandchars=\\\{\}]
\PY{n}{df}\PY{o}{.}\PY{n}{loc}\PY{p}{[}\PY{n}{df}\PY{p}{[}\PY{l+s+s1}{\PYZsq{}}\PY{l+s+s1}{Country}\PY{l+s+s1}{\PYZsq{}}\PY{p}{]} \PY{o}{==} \PY{l+s+s1}{\PYZsq{}}\PY{l+s+s1}{Scotland}\PY{l+s+s1}{\PYZsq{}}\PY{p}{]}
\end{Verbatim}
\end{tcolorbox}

            \begin{tcolorbox}[breakable, size=fbox, boxrule=.5pt, pad at break*=1mm, opacityfill=0]
\prompt{Out}{outcolor}{21}{\boxspacing}
\begin{Verbatim}[commandchars=\\\{\}]
                  Institution   Country  Founded  Students   Staff   Admin  \textbackslash{}
0       University of Glasgow  Scotland     1451     30805  2942.0  4003.0
1     University of Edinburgh  Scotland     1583     34275  4589.0  6107.0
2    University of St Andrews  Scotland     1413      8984  1137.0  1576.0
3      University of Aberdeen  Scotland     1495     14775  1086.0  1489.0
4   University of Strathclyde  Scotland     1964     22640     NaN  3200.0
18       University of Dundee  Scotland     1967     15915  1410.0  1805.0
20     University of Stirling  Scotland     1967      9548     NaN  1872.0

    Budget  Russell
0    626.5        1
1   1102.0        1
2    251.2        0
3    219.5        0
4    304.4        0
18   256.4        0
20   113.3        0
\end{Verbatim}
\end{tcolorbox}
        
    Indexing via \texttt{.loc{[}{]}} supports a few more types of arguments,
see the
\href{https://pandas.pydata.org/docs/user_guide/indexing.html\#selection-by-label}{official
documentation} for details.

    \textbf{Selection by position}

Conversely, if we want to select items exclusively by their position and
ignore their labels, we use \texttt{.iloc{[}{]}}:

    \begin{tcolorbox}[breakable, size=fbox, boxrule=1pt, pad at break*=1mm,colback=cellbackground, colframe=cellborder]
\prompt{In}{incolor}{22}{\boxspacing}
\begin{Verbatim}[commandchars=\\\{\}]
\PY{n}{df}\PY{o}{.}\PY{n}{iloc}\PY{p}{[}\PY{l+m+mi}{0}\PY{p}{:}\PY{l+m+mi}{4}\PY{p}{,} \PY{l+m+mi}{0}\PY{p}{:}\PY{l+m+mi}{2}\PY{p}{]}           \PY{c+c1}{\PYZsh{} select first 4 rows, first 2 columns}
\end{Verbatim}
\end{tcolorbox}

            \begin{tcolorbox}[breakable, size=fbox, boxrule=.5pt, pad at break*=1mm, opacityfill=0]
\prompt{Out}{outcolor}{22}{\boxspacing}
\begin{Verbatim}[commandchars=\\\{\}]
                Institution   Country
0     University of Glasgow  Scotland
1   University of Edinburgh  Scotland
2  University of St Andrews  Scotland
3    University of Aberdeen  Scotland
\end{Verbatim}
\end{tcolorbox}
        
    Again, \texttt{.iloc{[}{]}} supports a multitude of other arguments,
including boolean arrays. See the
\href{https://pandas.pydata.org/docs/user_guide/indexing.html\#selection-by-position}{official
documentation} for details.


\hypertarget{aggregation-and-reduction}{%
\subsection{Aggregation and reduction}\label{aggregation-and-reduction}}

\hypertarget{working-with-entire-dataframes}{%
\subsubsection{Working with entire
DataFrames}\label{working-with-entire-dataframes}}

The simplest way to perform data reduction is to invoke the desired
routine on the entire \texttt{DataFrame}:

    \begin{tcolorbox}[breakable, size=fbox, boxrule=1pt, pad at break*=1mm,colback=cellbackground, colframe=cellborder]
\prompt{In}{incolor}{23}{\boxspacing}
\begin{Verbatim}[commandchars=\\\{\}]
\PY{k+kn}{import} \PY{n+nn}{pandas} \PY{k}{as} \PY{n+nn}{pd}

\PY{n}{df} \PY{o}{=} \PY{n}{pd}\PY{o}{.}\PY{n}{read\PYZus{}csv}\PY{p}{(}\PY{l+s+s1}{\PYZsq{}}\PY{l+s+s1}{../data/universities.csv}\PY{l+s+s1}{\PYZsq{}}\PY{p}{,} \PY{n}{sep}\PY{o}{=}\PY{l+s+s1}{\PYZsq{}}\PY{l+s+s1}{;}\PY{l+s+s1}{\PYZsq{}}\PY{p}{)}
\PY{n}{df}\PY{o}{.}\PY{n}{mean}\PY{p}{(}\PY{p}{)}
\end{Verbatim}
\end{tcolorbox}

            \begin{tcolorbox}[breakable, size=fbox, boxrule=.5pt, pad at break*=1mm, opacityfill=0]
\prompt{Out}{outcolor}{23}{\boxspacing}
\begin{Verbatim}[commandchars=\\\{\}]
Founded      1745.652174
Students    24106.782609
Staff        3664.250000
Admin        3556.736842
Budget        768.609091
Russell         0.739130
dtype: float64
\end{Verbatim}
\end{tcolorbox}
        
    Methods such as \texttt{mean()} are by default applied column-wise to
each numerical column.

One big advantage over NumPy is that missing values (represented by
\texttt{np.nan}) are automatically ignored:

    \begin{tcolorbox}[breakable, size=fbox, boxrule=1pt, pad at break*=1mm,colback=cellbackground, colframe=cellborder]
\prompt{In}{incolor}{24}{\boxspacing}
\begin{Verbatim}[commandchars=\\\{\}]
\PY{c+c1}{\PYZsh{} mean() automatically drops 3 missing observations}
\PY{n}{df}\PY{p}{[}\PY{l+s+s1}{\PYZsq{}}\PY{l+s+s1}{Staff}\PY{l+s+s1}{\PYZsq{}}\PY{p}{]}\PY{o}{.}\PY{n}{mean}\PY{p}{(}\PY{p}{)}
\end{Verbatim}
\end{tcolorbox}

            \begin{tcolorbox}[breakable, size=fbox, boxrule=.5pt, pad at break*=1mm, opacityfill=0]
\prompt{Out}{outcolor}{24}{\boxspacing}
\begin{Verbatim}[commandchars=\\\{\}]
3664.25
\end{Verbatim}
\end{tcolorbox}
        
    \hypertarget{splitting-and-grouping}{%
\subsubsection{Splitting and grouping}\label{splitting-and-grouping}}

Applying aggregation functions to the entire \texttt{DataFrame} is
similar to what we can do with NumPy. The added flexibility of pandas
becomes obvious once we want to apply these functions to subsets of
data, \ie groups, which we can define based on values or index labels.

For example, we can easily group our universities by country:

    \begin{tcolorbox}[breakable, size=fbox, boxrule=1pt, pad at break*=1mm,colback=cellbackground, colframe=cellborder]
\prompt{In}{incolor}{25}{\boxspacing}
\begin{Verbatim}[commandchars=\\\{\}]
\PY{k+kn}{import} \PY{n+nn}{pandas} \PY{k}{as} \PY{n+nn}{pd}

\PY{n}{df} \PY{o}{=} \PY{n}{pd}\PY{o}{.}\PY{n}{read\PYZus{}csv}\PY{p}{(}\PY{l+s+s1}{\PYZsq{}}\PY{l+s+s1}{../data/universities.csv}\PY{l+s+s1}{\PYZsq{}}\PY{p}{,} \PY{n}{sep}\PY{o}{=}\PY{l+s+s1}{\PYZsq{}}\PY{l+s+s1}{;}\PY{l+s+s1}{\PYZsq{}}\PY{p}{)}

\PY{n}{groups} \PY{o}{=} \PY{n}{df}\PY{o}{.}\PY{n}{groupby}\PY{p}{(}\PY{p}{[}\PY{l+s+s1}{\PYZsq{}}\PY{l+s+s1}{Country}\PY{l+s+s1}{\PYZsq{}}\PY{p}{]}\PY{p}{)}
\end{Verbatim}
\end{tcolorbox}

    Here \texttt{groups} is a special pandas objects which can subsequently
be used to process group-specific data. To compute the group-wise
averages, we can simply run

    \begin{tcolorbox}[breakable, size=fbox, boxrule=1pt, pad at break*=1mm,colback=cellbackground, colframe=cellborder]
\prompt{In}{incolor}{26}{\boxspacing}
\begin{Verbatim}[commandchars=\\\{\}]
\PY{n}{groups}\PY{o}{.}\PY{n}{mean}\PY{p}{(}\PY{p}{)}
\end{Verbatim}
\end{tcolorbox}

            \begin{tcolorbox}[breakable, size=fbox, boxrule=.5pt, pad at break*=1mm, opacityfill=0]
\prompt{Out}{outcolor}{26}{\boxspacing}
\begin{Verbatim}[commandchars=\\\{\}]
                      Founded      Students        Staff        Admin  \textbackslash{}
Country
England           1745.923077  27119.846154  4336.692308  4112.000000
Northern Ireland  1810.000000  18438.000000  2414.000000  1489.000000
Scotland          1691.428571  19563.142857  2232.800000  2864.571429
Wales             1901.500000  23259.000000  3330.000000  4514.500000

                       Budget   Russell
Country
England           1001.700000  1.000000
Northern Ireland   369.200000  1.000000
Scotland           410.471429  0.285714
Wales              644.800000  0.500000
\end{Verbatim}
\end{tcolorbox}
        
    Groups support column indexing: if we want to only compute the total
number of students for each country in our sample, we can do this as
follows:

    \begin{tcolorbox}[breakable, size=fbox, boxrule=1pt, pad at break*=1mm,colback=cellbackground, colframe=cellborder]
\prompt{In}{incolor}{27}{\boxspacing}
\begin{Verbatim}[commandchars=\\\{\}]
\PY{n}{groups}\PY{p}{[}\PY{l+s+s1}{\PYZsq{}}\PY{l+s+s1}{Students}\PY{l+s+s1}{\PYZsq{}}\PY{p}{]}\PY{o}{.}\PY{n}{sum}\PY{p}{(}\PY{p}{)}
\end{Verbatim}
\end{tcolorbox}

            \begin{tcolorbox}[breakable, size=fbox, boxrule=.5pt, pad at break*=1mm, opacityfill=0]
\prompt{Out}{outcolor}{27}{\boxspacing}
\begin{Verbatim}[commandchars=\\\{\}]
Country
England             352558
Northern Ireland     18438
Scotland            136942
Wales                46518
Name: Students, dtype: int64
\end{Verbatim}
\end{tcolorbox}
        
    There are numerous routines to aggregate grouped data, for example:

\begin{itemize}
\tightlist
\item
  \texttt{mean()}, \texttt{sum()}: averages and sums over numerical
  items within groups.
\item
  \texttt{std()}, \texttt{var()}: within-group std. dev. and variances
\item
  \texttt{size()}: group sizes
\item
  \texttt{first()}, \texttt{last()}: first and last elements in each
  group
\item
  \texttt{min()}, \texttt{max()}: minimum and maximum elements within a
  group
\end{itemize}

    \emph{Examples:}

    \begin{tcolorbox}[breakable, size=fbox, boxrule=1pt, pad at break*=1mm,colback=cellbackground, colframe=cellborder]
\prompt{In}{incolor}{28}{\boxspacing}
\begin{Verbatim}[commandchars=\\\{\}]
\PY{n}{groups}\PY{o}{.}\PY{n}{size}\PY{p}{(}\PY{p}{)}       \PY{c+c1}{\PYZsh{} return number of elements in each group}
\end{Verbatim}
\end{tcolorbox}

            \begin{tcolorbox}[breakable, size=fbox, boxrule=.5pt, pad at break*=1mm, opacityfill=0]
\prompt{Out}{outcolor}{28}{\boxspacing}
\begin{Verbatim}[commandchars=\\\{\}]
Country
England             13
Northern Ireland     1
Scotland             7
Wales                2
dtype: int64
\end{Verbatim}
\end{tcolorbox}
        
    \begin{tcolorbox}[breakable, size=fbox, boxrule=1pt, pad at break*=1mm,colback=cellbackground, colframe=cellborder]
\prompt{In}{incolor}{29}{\boxspacing}
\begin{Verbatim}[commandchars=\\\{\}]
\PY{n}{groups}\PY{o}{.}\PY{n}{first}\PY{p}{(}\PY{p}{)}      \PY{c+c1}{\PYZsh{} return first element in each group}
\end{Verbatim}
\end{tcolorbox}

            \begin{tcolorbox}[breakable, size=fbox, boxrule=.5pt, pad at break*=1mm, opacityfill=0]
\prompt{Out}{outcolor}{29}{\boxspacing}
\begin{Verbatim}[commandchars=\\\{\}]
                                 Institution  Founded  Students   Staff  \textbackslash{}
Country
England                                  LSE     1895     11850  1725.0
Northern Ireland  Queen’s University Belfast     1810     18438  2414.0
Scotland               University of Glasgow     1451     30805  2942.0
Wales                     Cardiff University     1883     25898  3330.0

                   Admin  Budget  Russell
Country
England           2515.0   415.1        1
Northern Ireland  1489.0   369.2        1
Scotland          4003.0   626.5        1
Wales             5739.0   644.8        1
\end{Verbatim}
\end{tcolorbox}
        
    We can create custom aggregation routines by calling \texttt{agg()} or
\texttt{aggregate()} on the grouped object. To illustrate, we count the
number of universities in each country that have more than 20,000
students:

    \begin{tcolorbox}[breakable, size=fbox, boxrule=1pt, pad at break*=1mm,colback=cellbackground, colframe=cellborder]
\prompt{In}{incolor}{30}{\boxspacing}
\begin{Verbatim}[commandchars=\\\{\}]
\PY{n}{groups}\PY{p}{[}\PY{l+s+s1}{\PYZsq{}}\PY{l+s+s1}{Students}\PY{l+s+s1}{\PYZsq{}}\PY{p}{]}\PY{o}{.}\PY{n}{agg}\PY{p}{(}\PY{k}{lambda} \PY{n}{x}\PY{p}{:} \PY{n}{np}\PY{o}{.}\PY{n}{sum}\PY{p}{(}\PY{n}{x} \PY{o}{\PYZgt{}}\PY{o}{=} \PY{l+m+mi}{20000}\PY{p}{)}\PY{p}{)}
\end{Verbatim}
\end{tcolorbox}

            \begin{tcolorbox}[breakable, size=fbox, boxrule=.5pt, pad at break*=1mm, opacityfill=0]
\prompt{Out}{outcolor}{30}{\boxspacing}
\begin{Verbatim}[commandchars=\\\{\}]
Country
England             10
Northern Ireland     0
Scotland             3
Wales                2
Name: Students, dtype: int64
\end{Verbatim}
\end{tcolorbox}
        
    Note that we called \texttt{agg()} only on the column \texttt{Students},
otherwise the function would be applied to every column separately,
which is not what we want.

The most flexible aggregation method is \texttt{apply()} which calls a
given function, passing the entire group-specific subset of data
(including all columns) as an argument, and glues together the results.

For example, if we want to compute the average budget per student (in
pounds), we can do this as follows:

    \begin{tcolorbox}[breakable, size=fbox, boxrule=1pt, pad at break*=1mm,colback=cellbackground, colframe=cellborder]
\prompt{In}{incolor}{31}{\boxspacing}
\begin{Verbatim}[commandchars=\\\{\}]
\PY{c+c1}{\PYZsh{} Budget is in millions of pounds, rescale by 1.0e6}
\PY{n}{groups}\PY{o}{.}\PY{n}{apply}\PY{p}{(}\PY{k}{lambda} \PY{n}{x}\PY{p}{:} \PY{n}{x}\PY{p}{[}\PY{l+s+s1}{\PYZsq{}}\PY{l+s+s1}{Budget}\PY{l+s+s1}{\PYZsq{}}\PY{p}{]}\PY{o}{.}\PY{n}{sum}\PY{p}{(}\PY{p}{)} \PY{o}{/} \PY{n}{x}\PY{p}{[}\PY{l+s+s1}{\PYZsq{}}\PY{l+s+s1}{Students}\PY{l+s+s1}{\PYZsq{}}\PY{p}{]}\PY{o}{.}\PY{n}{sum}\PY{p}{(}\PY{p}{)} \PY{o}{*} \PY{l+m+mf}{1.0e6}\PY{p}{)}
\end{Verbatim}
\end{tcolorbox}

            \begin{tcolorbox}[breakable, size=fbox, boxrule=.5pt, pad at break*=1mm, opacityfill=0]
\prompt{Out}{outcolor}{31}{\boxspacing}
\begin{Verbatim}[commandchars=\\\{\}]
Country
England             36936.050239
Northern Ireland    20023.863760
Scotland            20981.875539
Wales               13861.301002
dtype: float64
\end{Verbatim}
\end{tcolorbox}
        
    We couldn't have done this with \texttt{agg()}, since \texttt{agg()}
never gets to see the entire chunk of data but only one column at a
time.

This section provided only a first look at pandas's
``split-apply-combine'' functionality implemented via \texttt{groupby}.
See the
\href{https://pandas.pydata.org/pandas-docs/stable/user_guide/groupby.html}{official
documentation} for more details.


\hypertarget{visualisation}{%
\subsection{Visualisation}\label{visualisation}}

We covered plotting with Matplotlib in earlier units. Pandas itself
implements some convenience wrappers around Matplotlib plotting routines
which allow us to quickly inspect data stored in \texttt{DataFrames}.
Alternatively, we can extract the numerical data and pass it to
Matplotlib's routines manually.

For example, to plot student numbers as a bar chart, we can directly use
pandas:

    \begin{tcolorbox}[breakable, size=fbox, boxrule=1pt, pad at break*=1mm,colback=cellbackground, colframe=cellborder]
\prompt{In}{incolor}{32}{\boxspacing}
\begin{Verbatim}[commandchars=\\\{\}]
\PY{k+kn}{import} \PY{n+nn}{pandas} \PY{k}{as} \PY{n+nn}{pd}

\PY{n}{df} \PY{o}{=} \PY{n}{pd}\PY{o}{.}\PY{n}{read\PYZus{}csv}\PY{p}{(}\PY{l+s+s1}{\PYZsq{}}\PY{l+s+s1}{../data/universities.csv}\PY{l+s+s1}{\PYZsq{}}\PY{p}{,} \PY{n}{sep}\PY{o}{=}\PY{l+s+s1}{\PYZsq{}}\PY{l+s+s1}{;}\PY{l+s+s1}{\PYZsq{}}\PY{p}{)}

\PY{c+c1}{\PYZsh{} set institution as label so they automatically show up in plot}
\PY{n}{df2} \PY{o}{=} \PY{n}{df}\PY{o}{.}\PY{n}{set\PYZus{}index}\PY{p}{(}\PY{n}{keys}\PY{o}{=}\PY{p}{[}\PY{l+s+s1}{\PYZsq{}}\PY{l+s+s1}{Institution}\PY{l+s+s1}{\PYZsq{}}\PY{p}{]}\PY{p}{)}
\PY{n}{df2}\PY{p}{[}\PY{l+s+s1}{\PYZsq{}}\PY{l+s+s1}{Students}\PY{l+s+s1}{\PYZsq{}}\PY{p}{]}\PY{o}{.}\PY{n}{plot}\PY{p}{(}\PY{n}{kind}\PY{o}{=}\PY{l+s+s1}{\PYZsq{}}\PY{l+s+s1}{bar}\PY{l+s+s1}{\PYZsq{}}\PY{p}{)}        \PY{c+c1}{\PYZsh{} same as df2[\PYZsq{}Students\PYZsq{}].plot.bar()}
\end{Verbatim}
\end{tcolorbox}

            \begin{tcolorbox}[breakable, size=fbox, boxrule=.5pt, pad at break*=1mm, opacityfill=0]
\prompt{Out}{outcolor}{32}{\boxspacing}
\begin{Verbatim}[commandchars=\\\{\}]
<AxesSubplot:xlabel='Institution'>
\end{Verbatim}
\end{tcolorbox}
        
    \begin{center}
    \adjustimage{max size={0.9\linewidth}{0.9\paperheight}}{unit8_files/unit8_66_1.pdf}
    \end{center}
    { \hspace*{\fill} \\}
    
    Alternatively, we can construct the graph using Matplotlib ourselves:

    \begin{tcolorbox}[breakable, size=fbox, boxrule=1pt, pad at break*=1mm,colback=cellbackground, colframe=cellborder]
\prompt{In}{incolor}{33}{\boxspacing}
\begin{Verbatim}[commandchars=\\\{\}]
\PY{k+kn}{import} \PY{n+nn}{matplotlib}\PY{n+nn}{.}\PY{n+nn}{pyplot} \PY{k}{as} \PY{n+nn}{plt}

\PY{n}{labels} \PY{o}{=} \PY{n}{df}\PY{p}{[}\PY{l+s+s1}{\PYZsq{}}\PY{l+s+s1}{Institution}\PY{l+s+s1}{\PYZsq{}}\PY{p}{]}\PY{o}{.}\PY{n}{to\PYZus{}list}\PY{p}{(}\PY{p}{)}        \PY{c+c1}{\PYZsh{} labels as list}
\PY{n}{values} \PY{o}{=} \PY{n}{df}\PY{p}{[}\PY{l+s+s1}{\PYZsq{}}\PY{l+s+s1}{Students}\PY{l+s+s1}{\PYZsq{}}\PY{p}{]}\PY{o}{.}\PY{n}{to\PYZus{}numpy}\PY{p}{(}\PY{p}{)}          \PY{c+c1}{\PYZsh{} data as NumPy array}
\PY{n}{plt}\PY{o}{.}\PY{n}{bar}\PY{p}{(}\PY{n}{labels}\PY{p}{,} \PY{n}{values}\PY{p}{)}
\PY{n}{plt}\PY{o}{.}\PY{n}{tick\PYZus{}params}\PY{p}{(}\PY{n}{axis}\PY{o}{=}\PY{l+s+s1}{\PYZsq{}}\PY{l+s+s1}{x}\PY{l+s+s1}{\PYZsq{}}\PY{p}{,} \PY{n}{labelrotation}\PY{o}{=}\PY{l+m+mi}{90}\PY{p}{)}
\end{Verbatim}
\end{tcolorbox}

    \begin{center}
    \adjustimage{max size={0.9\linewidth}{0.9\paperheight}}{unit8_files/unit8_68_0.pdf}
    \end{center}
    { \hspace*{\fill} \\}
    
    Sometimes Matplotlib's routines directly work with pandas's data
structures, sometimes they don't. In cases where they don't, we can
convert a \texttt{DataFrame} or \texttt{Series} object to a NumPy array
using the \texttt{to\_numpy()} method, and convert a \texttt{Series} to
a Python list using \texttt{to\_list()}, as illustrated in the example
above.

To plot timeseries-like data, we can use the \texttt{plot()} method,
which optionally accepts arguments to specify which columns should be
used for the \(x\)-axis and which for the \(y\)-axis:

    \begin{tcolorbox}[breakable, size=fbox, boxrule=1pt, pad at break*=1mm,colback=cellbackground, colframe=cellborder]
\prompt{In}{incolor}{34}{\boxspacing}
\begin{Verbatim}[commandchars=\\\{\}]
\PY{k+kn}{import} \PY{n+nn}{numpy} \PY{k}{as} \PY{n+nn}{np}
\PY{k+kn}{import} \PY{n+nn}{pandas} \PY{k}{as} \PY{n+nn}{pd}

\PY{c+c1}{\PYZsh{} Instantiate RNG}
\PY{n}{rng} \PY{o}{=} \PY{n}{np}\PY{o}{.}\PY{n}{random}\PY{o}{.}\PY{n}{default\PYZus{}rng}\PY{p}{(}\PY{l+m+mi}{123}\PY{p}{)}

\PY{c+c1}{\PYZsh{} Create pandas DataFrame}
\PY{n}{nobs} \PY{o}{=} \PY{l+m+mi}{50}
\PY{n}{df} \PY{o}{=} \PY{n}{pd}\PY{o}{.}\PY{n}{DataFrame}\PY{p}{(}\PY{p}{\PYZob{}}\PY{l+s+s1}{\PYZsq{}}\PY{l+s+s1}{A}\PY{l+s+s1}{\PYZsq{}}\PY{p}{:} \PY{n}{np}\PY{o}{.}\PY{n}{arange}\PY{p}{(}\PY{n}{nobs}\PY{p}{)}\PY{p}{,} \PY{l+s+s1}{\PYZsq{}}\PY{l+s+s1}{B}\PY{l+s+s1}{\PYZsq{}}\PY{p}{:} \PY{n}{rng}\PY{o}{.}\PY{n}{normal}\PY{p}{(}\PY{n}{size}\PY{o}{=}\PY{n}{nobs}\PY{p}{)}\PY{p}{\PYZcb{}}\PY{p}{)}

\PY{n}{df}\PY{o}{.}\PY{n}{plot}\PY{p}{(}\PY{n}{x}\PY{o}{=}\PY{l+s+s1}{\PYZsq{}}\PY{l+s+s1}{A}\PY{l+s+s1}{\PYZsq{}}\PY{p}{,} \PY{n}{y}\PY{o}{=}\PY{l+s+s1}{\PYZsq{}}\PY{l+s+s1}{B}\PY{l+s+s1}{\PYZsq{}}\PY{p}{)}       \PY{c+c1}{\PYZsh{} plot A on x\PYZhy{}axis, B on y\PYZhy{}axis}
\end{Verbatim}
\end{tcolorbox}

            \begin{tcolorbox}[breakable, size=fbox, boxrule=.5pt, pad at break*=1mm, opacityfill=0]
\prompt{Out}{outcolor}{34}{\boxspacing}
\begin{Verbatim}[commandchars=\\\{\}]
<AxesSubplot:xlabel='A'>
\end{Verbatim}
\end{tcolorbox}
        
    \begin{center}
    \adjustimage{max size={0.9\linewidth}{0.9\paperheight}}{unit8_files/unit8_70_1.pdf}
    \end{center}
    { \hspace*{\fill} \\}
    
    To quickly generate some descriptive statistics, we can use the built-in
box plot:

    \begin{tcolorbox}[breakable, size=fbox, boxrule=1pt, pad at break*=1mm,colback=cellbackground, colframe=cellborder]
\prompt{In}{incolor}{35}{\boxspacing}
\begin{Verbatim}[commandchars=\\\{\}]
\PY{k+kn}{import} \PY{n+nn}{numpy} \PY{k}{as} \PY{n+nn}{np}
\PY{k+kn}{import} \PY{n+nn}{pandas} \PY{k}{as} \PY{n+nn}{pd}

\PY{c+c1}{\PYZsh{} Instantiate RNG}
\PY{n}{rng} \PY{o}{=} \PY{n}{np}\PY{o}{.}\PY{n}{random}\PY{o}{.}\PY{n}{default\PYZus{}rng}\PY{p}{(}\PY{l+m+mi}{123}\PY{p}{)}

\PY{c+c1}{\PYZsh{} Create pandas DataFrame}
\PY{n}{df} \PY{o}{=} \PY{n}{pd}\PY{o}{.}\PY{n}{DataFrame}\PY{p}{(}\PY{n}{rng}\PY{o}{.}\PY{n}{normal}\PY{p}{(}\PY{n}{size}\PY{o}{=}\PY{p}{(}\PY{l+m+mi}{100}\PY{p}{,} \PY{l+m+mi}{5}\PY{p}{)}\PY{p}{)}\PY{p}{,}
                  \PY{n}{columns}\PY{o}{=}\PY{p}{[}\PY{l+s+s1}{\PYZsq{}}\PY{l+s+s1}{A}\PY{l+s+s1}{\PYZsq{}}\PY{p}{,} \PY{l+s+s1}{\PYZsq{}}\PY{l+s+s1}{B}\PY{l+s+s1}{\PYZsq{}}\PY{p}{,} \PY{l+s+s1}{\PYZsq{}}\PY{l+s+s1}{C}\PY{l+s+s1}{\PYZsq{}}\PY{p}{,} \PY{l+s+s1}{\PYZsq{}}\PY{l+s+s1}{D}\PY{l+s+s1}{\PYZsq{}}\PY{p}{,} \PY{l+s+s1}{\PYZsq{}}\PY{l+s+s1}{E}\PY{l+s+s1}{\PYZsq{}}\PY{p}{]}\PY{p}{)}

\PY{n}{df}\PY{o}{.}\PY{n}{plot}\PY{o}{.}\PY{n}{box}\PY{p}{(}\PY{p}{)}           \PY{c+c1}{\PYZsh{} same as df.plot(kind=\PYZsq{}box\PYZsq{})}
\end{Verbatim}
\end{tcolorbox}

            \begin{tcolorbox}[breakable, size=fbox, boxrule=.5pt, pad at break*=1mm, opacityfill=0]
\prompt{Out}{outcolor}{35}{\boxspacing}
\begin{Verbatim}[commandchars=\\\{\}]
<AxesSubplot:>
\end{Verbatim}
\end{tcolorbox}
        
    \begin{center}
    \adjustimage{max size={0.9\linewidth}{0.9\paperheight}}{unit8_files/unit8_72_1.pdf}
    \end{center}
    { \hspace*{\fill} \\}
    
    Similarly, we can generate scatter plots, plotting one column against
another:

    \begin{tcolorbox}[breakable, size=fbox, boxrule=1pt, pad at break*=1mm,colback=cellbackground, colframe=cellborder]
\prompt{In}{incolor}{36}{\boxspacing}
\begin{Verbatim}[commandchars=\\\{\}]
\PY{c+c1}{\PYZsh{} Create pandas DataFrame}
\PY{n}{df} \PY{o}{=} \PY{n}{pd}\PY{o}{.}\PY{n}{DataFrame}\PY{p}{(}\PY{n}{rng}\PY{o}{.}\PY{n}{uniform}\PY{p}{(}\PY{n}{size}\PY{o}{=}\PY{p}{(}\PY{l+m+mi}{100}\PY{p}{,} \PY{l+m+mi}{5}\PY{p}{)}\PY{p}{)}\PY{p}{,}
                  \PY{n}{columns}\PY{o}{=}\PY{p}{[}\PY{l+s+s1}{\PYZsq{}}\PY{l+s+s1}{A}\PY{l+s+s1}{\PYZsq{}}\PY{p}{,} \PY{l+s+s1}{\PYZsq{}}\PY{l+s+s1}{B}\PY{l+s+s1}{\PYZsq{}}\PY{p}{,} \PY{l+s+s1}{\PYZsq{}}\PY{l+s+s1}{C}\PY{l+s+s1}{\PYZsq{}}\PY{p}{,} \PY{l+s+s1}{\PYZsq{}}\PY{l+s+s1}{D}\PY{l+s+s1}{\PYZsq{}}\PY{p}{,} \PY{l+s+s1}{\PYZsq{}}\PY{l+s+s1}{E}\PY{l+s+s1}{\PYZsq{}}\PY{p}{]}\PY{p}{)}

\PY{n}{df}\PY{o}{.}\PY{n}{plot}\PY{o}{.}\PY{n}{scatter}\PY{p}{(}\PY{n}{x}\PY{o}{=}\PY{l+s+s1}{\PYZsq{}}\PY{l+s+s1}{A}\PY{l+s+s1}{\PYZsq{}}\PY{p}{,} \PY{n}{y}\PY{o}{=}\PY{l+s+s1}{\PYZsq{}}\PY{l+s+s1}{E}\PY{l+s+s1}{\PYZsq{}}\PY{p}{,} \PY{n}{label}\PY{o}{=}\PY{l+s+s1}{\PYZsq{}}\PY{l+s+s1}{Pandas scatter plot}\PY{l+s+s1}{\PYZsq{}}\PY{p}{)}
\end{Verbatim}
\end{tcolorbox}

            \begin{tcolorbox}[breakable, size=fbox, boxrule=.5pt, pad at break*=1mm, opacityfill=0]
\prompt{Out}{outcolor}{36}{\boxspacing}
\begin{Verbatim}[commandchars=\\\{\}]
<AxesSubplot:xlabel='A', ylabel='E'>
\end{Verbatim}
\end{tcolorbox}
        
    \begin{center}
    \adjustimage{max size={0.9\linewidth}{0.9\paperheight}}{unit8_files/unit8_74_1.pdf}
    \end{center}
    { \hspace*{\fill} \\}
    
    \begin{tcolorbox}[breakable, size=fbox, boxrule=1pt, pad at break*=1mm,colback=cellbackground, colframe=cellborder]
\prompt{In}{incolor}{37}{\boxspacing}
\begin{Verbatim}[commandchars=\\\{\}]
\PY{c+c1}{\PYZsh{} We can even use a column to specify the dot size!}
\PY{n}{df}\PY{o}{.}\PY{n}{plot}\PY{o}{.}\PY{n}{scatter}\PY{p}{(}\PY{n}{x}\PY{o}{=}\PY{l+s+s1}{\PYZsq{}}\PY{l+s+s1}{A}\PY{l+s+s1}{\PYZsq{}}\PY{p}{,} \PY{n}{y}\PY{o}{=}\PY{l+s+s1}{\PYZsq{}}\PY{l+s+s1}{E}\PY{l+s+s1}{\PYZsq{}}\PY{p}{,} \PY{n}{s}\PY{o}{=}\PY{n}{df}\PY{p}{[}\PY{l+s+s1}{\PYZsq{}}\PY{l+s+s1}{B}\PY{l+s+s1}{\PYZsq{}}\PY{p}{]}\PY{o}{*}\PY{l+m+mf}{100.0}\PY{p}{)}
\end{Verbatim}
\end{tcolorbox}

            \begin{tcolorbox}[breakable, size=fbox, boxrule=.5pt, pad at break*=1mm, opacityfill=0]
\prompt{Out}{outcolor}{37}{\boxspacing}
\begin{Verbatim}[commandchars=\\\{\}]
<AxesSubplot:xlabel='A', ylabel='E'>
\end{Verbatim}
\end{tcolorbox}
        
    \begin{center}
    \adjustimage{max size={0.9\linewidth}{0.9\paperheight}}{unit8_files/unit8_75_1.pdf}
    \end{center}
    { \hspace*{\fill} \\}
    
    In general, the wrappers implemented in pandas are useful to get an idea
how the data looks like. For reusable code or more complex graphs, we'll
usually want to directly use Matplotlib and pass the data converted to
NumPy arrays.


\hypertarget{exercises}{%
\subsection{Exercises}\label{exercises}}

The following exercises use data files from the \texttt{data/} folder.

    \hypertarget{exercise-1-basic-data-manipulations}{%
\subsubsection{Exercise 1: Basic data
manipulations}\label{exercise-1-basic-data-manipulations}}

In this exercise, we will perform some basic data manipulation and plot
the results.

\begin{enumerate}
\def\labelenumi{\arabic{enumi}.}
\item
  Load the CSV file \texttt{FRED\_QTR.csv} (using
  \texttt{sep=\textquotesingle{},\textquotesingle{}}). Set the columns
  \texttt{Year} and \texttt{Quarter} as (joint) indices.

  \emph{Hint:} You can do this by specifying these column names in the
  \texttt{index\_col} argument of
  \href{https://pandas.pydata.org/pandas-docs/stable/reference/api/pandas.read_csv.html}{\texttt{read\_csv()}}.
  Alternatively, you can cell
  \href{https://pandas.pydata.org/pandas-docs/stable/reference/api/pandas.DataFrame.set_index.html}{\texttt{set\_index()}}
  once you have loaded the data.
\item
  This data comes at a quarterly frequency. Convert it to annual values
  by computing the average values for each year.

  \emph{Hint:} Group the data by \texttt{Year} using the
  \href{https://pandas.pydata.org/pandas-docs/stable/reference/api/pandas.DataFrame.groupby.html}{\texttt{groupby()}}
  function and compute the mean on the grouped data.
\item
  Compute two new variables from the annualised data and add them to the
  \texttt{DataFrame}:

  \begin{itemize}
  \tightlist
  \item
    \texttt{Inflation}, defined as the growth rate of \texttt{CPI}
    (consumer price index)
  \item
    \texttt{GDP\_growth}, defined as the growth rate of \texttt{GDP}
  \end{itemize}
\item
  Drop all rows with missing values (these show up as \texttt{NaN}).

  \emph{Hint:} There is no need to manually filter out \texttt{NaN}
  values, you can use the
  \href{https://pandas.pydata.org/pandas-docs/stable/reference/api/pandas.DataFrame.dropna.html}{\texttt{dropna()}}
  method instead.
\item
  Plot the columns \texttt{GDP\_growth}, \texttt{Inflation},
  \texttt{UNRATE} (unemployment rate) and \texttt{LFPART} (labour force
  participation) using the pandas plotting routines. Use the option
  \texttt{subplots=True} and \texttt{layout=(2,2)} to create a
  \(2\times2\) grid. See the documentation for
  \href{https://pandas.pydata.org/docs/reference/api/pandas.DataFrame.plot.html\#pandas.DataFrame.plot}{\texttt{plot()}}
  for details.
\end{enumerate}

    \hypertarget{exercise-2-decade-averages}{%
\subsubsection{Exercise 2: Decade
averages}\label{exercise-2-decade-averages}}

Load the FRED data from the CSV file \texttt{FRED\_QTR.csv} (using
\texttt{sep=\textquotesingle{},\textquotesingle{}}) and perform the
following tasks:

\begin{enumerate}
\def\labelenumi{\arabic{enumi}.}
\tightlist
\item
  Compute the quarterly GDP growth rate and inflation, similar to what
  you did in the previous exercise.
\item
  Add the column \texttt{Decade} which contains the decade for every
  observation. Use 1940 to code the 40s, 1950 for the 50s, etc.
\item
  We want to retain only observations for decades for which all 40
  quarters are present:

  \begin{enumerate}
  \def\labelenumii{\arabic{enumii}.}
  \tightlist
  \item
    Group the data by \texttt{Decade} and count the number of
    observations using
    \href{https://pandas.pydata.org/pandas-docs/stable/reference/api/pandas.DataFrame.count.html}{\texttt{count()}}.
  \item
    A decade should be kept in the data set only if \emph{all} variables
    have the full 40 observations.
  \item
    Drop all observations for which this is not the case.
  \end{enumerate}
\item
  With the remaining observations, compute the decade averages for
  quarterly GDP growth, inflation and the unemployment rate
  (\texttt{UNRATE}). Annualise the GDP growth and inflation figures by
  multiplying them by 4.
\item
  Create a bar chart that plots these three variables by decade.
\end{enumerate}

    \hypertarget{exercise-3-group-averages}{%
\subsubsection{Exercise 3: Group
averages}\label{exercise-3-group-averages}}

Load the universities data from the CSV file \texttt{universities.csv}
(using \texttt{sep=\textquotesingle{};\textquotesingle{}}) and perform
the following tasks:

\begin{enumerate}
\def\labelenumi{\arabic{enumi}.}
\item
  Group the data by Russell Group membership using the indicator
  variable \texttt{Russell}. For each group, compute the averages of the
  following ratios using
  \href{https://pandas.pydata.org/pandas-docs/stable/reference/api/pandas.core.groupby.GroupBy.apply.html}{\texttt{apply()}}:

  \begin{itemize}
  \tightlist
  \item
    The ratio of academic staff (\texttt{Staff}) to students
    (\texttt{Students})
  \item
    The ratio of administrative staff (\texttt{Admin}) to students.
  \item
    The budget (\texttt{Budget}) per student in pounds.
  \end{itemize}

  Additionally, compute the number of universities is each group.
\item
  Repeat the task using a different approach:

  \begin{enumerate}
  \def\labelenumii{\arabic{enumii}.}
  \tightlist
  \item
    Compute the above ratios and add them as new columns to the initial
    \texttt{DataFrame}.
  \item
    Group the data by Russell Group membership.
  \item
    Compute the mean of each ratio using
    \href{https://pandas.pydata.org/pandas-docs/stable/reference/api/pandas.core.groupby.GroupBy.mean.html}{\texttt{mean()}}.
  \item
    Compute the number of universities in each group using
    \texttt{count()}, and store the result in the column \texttt{Count}
    in the \texttt{DataFrame} you obtained in the previous step.
  \end{enumerate}
\item
  Create a bar chart, plotting the value for universities in and outside
  of the Russell Group for each of the four statistics computed above.
\end{enumerate}

    \hypertarget{exercise-4-grouping-by-multiple-dimensions}{%
\subsubsection{Exercise 4: Grouping by multiple
dimensions}\label{exercise-4-grouping-by-multiple-dimensions}}

Load the universities data from the CSV file \texttt{universities.csv}
(using \texttt{sep=\textquotesingle{};\textquotesingle{}}) and perform
the following tasks:

\begin{enumerate}
\def\labelenumi{\arabic{enumi}.}
\item
  Create an indicator \texttt{Pre1800} which is \texttt{True} for
  universities founded before the year 1800.
\item
  Group the data by \texttt{Country} and the value of \texttt{Pre1800}.

  \emph{Hint:} You need to pass a list of column names to
  \href{https://pandas.pydata.org/pandas-docs/stable/reference/api/pandas.DataFrame.groupby.html}{\texttt{groupby()}}.
\item
  Compute the number of universities for each combination of
  \texttt{(Country,\ Pre1800)}.
\item
  Create a bar chart showing the number of pre- and post-1800
  universities by country (\ie create four groups of bars, each group
  showing one bar for pre- and one for post-1800).
\item
  Create a bar chart showing the number of universities by country by
  pre- and post-1800 period (\ie create two groups of bars, each group
  showing four bars, one for each country.)
\end{enumerate}

    \hypertarget{exercise-5-okuns-law-advanced}{%
\subsubsection{Exercise 5: Okun's law
(advanced)}\label{exercise-5-okuns-law-advanced}}

In this exercise, we will estimate
\href{https://en.wikipedia.org/wiki/Okun\%27s_law}{Okun's law} on
quarterly data for each of the last eight decades.

Okun's law relates unemployment to the output gap. One version (see
Jones: Macroeconomics, 2019) is stated as follows: \[
u_t - \overline{u}_{t} = 
\alpha + \beta \left(\frac{Y_t - \overline{Y}_t}{\overline{Y}_t}\right)
\] where \(u_t\) is the unemployment rate, \(\overline{u}_{t}\) is the
natural rate of unemployment, \(Y_t\) is output (GDP) and
\(\overline{Y}_{t}\) is potential output. We will refer to
\(u_t-\overline{u}_{t}\) as ``cyclical unemployment'' and to the term in
parenthesis on the right-hand side as the ``output gap.'' Okun's law
says that the coefficient \(\beta\) is negative, \ie cyclical
unemployment is higher when the output gap is low (negative) because the
economy is in a recession.

Load the FRED data from the CSV file \texttt{FRED\_QTR.csv} (using
\texttt{sep=\textquotesingle{},\textquotesingle{}}) and perform the
following tasks:

\begin{enumerate}
\def\labelenumi{\arabic{enumi}.}
\item
  Compute the output gap and cyclical unemployment rate as defined above
  and add them as columns to the \texttt{DataFrame}.
\item
  Assign each observation to a decade as you did in previous exercises.
\item
  Write a function \texttt{regress\_okun()} which accepts a
  \texttt{DataFrame} containing a decade-spefic sub-sample as the only
  argument, and estimates the coefficients \(\alpha\) (the intercept)
  and \(\beta\) (the slope) of the above regression equation.

  This function should return a \texttt{DataFrame} of a single row and
  two columns which store the intercept and slope.

  \emph{Hint:} Use NumPy's
  \href{https://numpy.org/doc/stable/reference/generated/numpy.linalg.lstsq.html}{\texttt{lstsq()}}
  to perform the regression. To regress the dependent variable
  \texttt{y} on regressors \texttt{X}, you need to call
  \texttt{lstsq(X,\ y)}. To include the intercept, you will manually
  have to create \texttt{X} such that the first column contains only
  ones.
\item
  Group the data by decade and call the
  \href{https://pandas.pydata.org/pandas-docs/stable/reference/api/pandas.core.groupby.GroupBy.apply.html}{\texttt{apply()}}
  method, passing \texttt{regress\_okun} you wrote as the argument.
\item
  Plot your results: for each decade, create a scatter plot of the raw
  data and overlay it with the regression line you estimated.
\end{enumerate}


\hypertarget{solutions}{%
\subsection{Solutions}\label{solutions}}

These solutions illustrate \emph{one} possible way to solve the
exercises. Pandas is extremely flexible (maybe too flexible) and allows
us to perform these tasks in many different ways, so your implementation
might look very different.

    \hypertarget{solution-for-exercise-1}{%
\subsubsection{Solution for exercise 1}\label{solution-for-exercise-1}}

One possible implementation looks as follows:

    \begin{tcolorbox}[breakable, size=fbox, boxrule=1pt, pad at break*=1mm,colback=cellbackground, colframe=cellborder]
\prompt{In}{incolor}{38}{\boxspacing}
\begin{Verbatim}[commandchars=\\\{\}]
\PY{k+kn}{import} \PY{n+nn}{pandas} \PY{k}{as} \PY{n+nn}{pd}
\PY{k+kn}{import} \PY{n+nn}{numpy} \PY{k}{as} \PY{n+nn}{np}

\PY{n}{filepath} \PY{o}{=} \PY{l+s+s1}{\PYZsq{}}\PY{l+s+s1}{../data/FRED\PYZus{}QTR.csv}\PY{l+s+s1}{\PYZsq{}}

\PY{n}{df} \PY{o}{=} \PY{n}{pd}\PY{o}{.}\PY{n}{read\PYZus{}csv}\PY{p}{(}\PY{n}{filepath}\PY{p}{,} \PY{n}{sep}\PY{o}{=}\PY{l+s+s1}{\PYZsq{}}\PY{l+s+s1}{,}\PY{l+s+s1}{\PYZsq{}}\PY{p}{,} \PY{n}{index\PYZus{}col}\PY{o}{=}\PY{p}{[}\PY{l+s+s1}{\PYZsq{}}\PY{l+s+s1}{Year}\PY{l+s+s1}{\PYZsq{}}\PY{p}{,} \PY{l+s+s1}{\PYZsq{}}\PY{l+s+s1}{Quarter}\PY{l+s+s1}{\PYZsq{}}\PY{p}{]}\PY{p}{)}
\PY{c+c1}{\PYZsh{} Alternatively, set index columns later}
\PY{c+c1}{\PYZsh{} df = pd.read\PYZus{}csv(filepath, sep=\PYZsq{},\PYZsq{})}
\PY{c+c1}{\PYZsh{} df.set\PYZus{}index(keys=[\PYZsq{}Year\PYZsq{}, \PYZsq{}Quarter\PYZsq{}], inplace=True)}

\PY{c+c1}{\PYZsh{} Convert to annual frequency}
\PY{c+c1}{\PYZsh{} Group by year}
\PY{n}{grp} \PY{o}{=} \PY{n}{df}\PY{o}{.}\PY{n}{groupby}\PY{p}{(}\PY{p}{[}\PY{l+s+s1}{\PYZsq{}}\PY{l+s+s1}{Year}\PY{l+s+s1}{\PYZsq{}}\PY{p}{]}\PY{p}{)}
\PY{c+c1}{\PYZsh{} Compute annual data as mean of quarterly values}
\PY{n}{df\PYZus{}year} \PY{o}{=} \PY{n}{grp}\PY{o}{.}\PY{n}{mean}\PY{p}{(}\PY{p}{)}

\PY{c+c1}{\PYZsh{} Alternative ways to perform the same aggregation:}
\PY{c+c1}{\PYZsh{} df\PYZus{}year = grp.agg(\PYZsq{}mean\PYZsq{})}
\PY{c+c1}{\PYZsh{} df\PYZus{}year = grp.agg(np.mean)}

\PY{c+c1}{\PYZsh{} Compute CPI and GDP growth rates (in percent)}
\PY{n}{df\PYZus{}year}\PY{p}{[}\PY{l+s+s1}{\PYZsq{}}\PY{l+s+s1}{Inflation}\PY{l+s+s1}{\PYZsq{}}\PY{p}{]} \PY{o}{=} \PY{n}{df\PYZus{}year}\PY{p}{[}\PY{l+s+s1}{\PYZsq{}}\PY{l+s+s1}{CPI}\PY{l+s+s1}{\PYZsq{}}\PY{p}{]}\PY{o}{.}\PY{n}{diff}\PY{p}{(}\PY{p}{)} \PY{o}{/} \PY{n}{df\PYZus{}year}\PY{p}{[}\PY{l+s+s1}{\PYZsq{}}\PY{l+s+s1}{CPI}\PY{l+s+s1}{\PYZsq{}}\PY{p}{]}\PY{o}{.}\PY{n}{shift}\PY{p}{(}\PY{p}{)} \PY{o}{*} \PY{l+m+mf}{100.0}
\PY{n}{df\PYZus{}year}\PY{p}{[}\PY{l+s+s1}{\PYZsq{}}\PY{l+s+s1}{GDP\PYZus{}growth}\PY{l+s+s1}{\PYZsq{}}\PY{p}{]} \PY{o}{=} \PY{n}{df\PYZus{}year}\PY{p}{[}\PY{l+s+s1}{\PYZsq{}}\PY{l+s+s1}{GDP}\PY{l+s+s1}{\PYZsq{}}\PY{p}{]}\PY{o}{.}\PY{n}{diff}\PY{p}{(}\PY{p}{)} \PY{o}{/} \PY{n}{df\PYZus{}year}\PY{p}{[}\PY{l+s+s1}{\PYZsq{}}\PY{l+s+s1}{GDP}\PY{l+s+s1}{\PYZsq{}}\PY{p}{]}\PY{o}{.}\PY{n}{shift}\PY{p}{(}\PY{p}{)} \PY{o}{*} \PY{l+m+mf}{100.0}

\PY{c+c1}{\PYZsh{} Drop all rows that contain any NaNs}
\PY{n}{df\PYZus{}year} \PY{o}{=} \PY{n}{df\PYZus{}year}\PY{o}{.}\PY{n}{dropna}\PY{p}{(}\PY{n}{axis}\PY{o}{=}\PY{l+m+mi}{0}\PY{p}{)}

\PY{c+c1}{\PYZsh{} Columns to plot}
\PY{n}{varnames} \PY{o}{=} \PY{p}{[}\PY{l+s+s1}{\PYZsq{}}\PY{l+s+s1}{GDP\PYZus{}growth}\PY{l+s+s1}{\PYZsq{}}\PY{p}{,} \PY{l+s+s1}{\PYZsq{}}\PY{l+s+s1}{Inflation}\PY{l+s+s1}{\PYZsq{}}\PY{p}{,} \PY{l+s+s1}{\PYZsq{}}\PY{l+s+s1}{UNRATE}\PY{l+s+s1}{\PYZsq{}}\PY{p}{,} \PY{l+s+s1}{\PYZsq{}}\PY{l+s+s1}{LFPART}\PY{l+s+s1}{\PYZsq{}}\PY{p}{]}
\PY{n}{df\PYZus{}year}\PY{o}{.}\PY{n}{plot}\PY{o}{.}\PY{n}{line}\PY{p}{(}\PY{n}{y}\PY{o}{=}\PY{n}{varnames}\PY{p}{,} \PY{n}{subplots}\PY{o}{=}\PY{k+kc}{True}\PY{p}{,} \PY{n}{layout}\PY{o}{=}\PY{p}{(}\PY{l+m+mi}{2}\PY{p}{,} \PY{l+m+mi}{2}\PY{p}{)}\PY{p}{,}  
                  \PY{n}{sharex}\PY{o}{=}\PY{k+kc}{True}\PY{p}{,} \PY{n}{figsize}\PY{o}{=}\PY{p}{(}\PY{l+m+mi}{10}\PY{p}{,} \PY{l+m+mi}{10}\PY{p}{)}\PY{p}{)}
                
\PY{c+c1}{\PYZsh{} Alternatively, we can call plot() directly, which}
\PY{c+c1}{\PYZsh{} defaults to generating a line plot:}
\PY{c+c1}{\PYZsh{}}
\PY{c+c1}{\PYZsh{} df\PYZus{}year.plot(y=varnames, subplots=True, layout=(2, 2),  }
\PY{c+c1}{\PYZsh{}              sharex=True, figsize=(10, 10))}
\end{Verbatim}
\end{tcolorbox}

            \begin{tcolorbox}[breakable, size=fbox, boxrule=.5pt, pad at break*=1mm, opacityfill=0]
\prompt{Out}{outcolor}{38}{\boxspacing}
\begin{Verbatim}[commandchars=\\\{\}]
array([[<AxesSubplot:xlabel='Year'>, <AxesSubplot:xlabel='Year'>],
       [<AxesSubplot:xlabel='Year'>, <AxesSubplot:xlabel='Year'>]],
      dtype=object)
\end{Verbatim}
\end{tcolorbox}
        
    \begin{center}
    \adjustimage{max size={0.9\linewidth}{0.9\paperheight}}{unit8_files/unit8_85_1.pdf}
    \end{center}
    { \hspace*{\fill} \\}
    
    A few comments:

\begin{enumerate}
\def\labelenumi{\arabic{enumi}.}
\item
  We can set the index column when loading a CSV file by passing the
  column names as \texttt{index\_col}:

\begin{verbatim}
df = pd.read_csv(filepath, sep=',', index_col=['Year', 'Quarter'])
\end{verbatim}

  Alternatively, we can first load the CSV file and set the index later:

\begin{verbatim}
df = pd.read_csv(filepath, sep=',')
df.set_index(keys=['Year', 'Quarter'], inplace=True)
\end{verbatim}
\item
  There are several ways to compute the means of grouped data:

  \begin{enumerate}
  \def\labelenumii{\arabic{enumii}.}
  \item
    We can call \texttt{mean()} on the group object directly:

\begin{verbatim}
df_year = grp.mean()
\end{verbatim}
  \item
    Alternatively, we can call \texttt{agg()} and pass it the
    aggregation routine that should be applied:

\begin{verbatim}
df_year = grp.agg('mean')
df_year = grp.agg(np.mean)
\end{verbatim}

    Here we again have multiple options: pandas understands
    \texttt{\textquotesingle{}mean\textquotesingle{}} if passed as a
    string (which might not be the case for some other functions), or we
    pass an actual function such as \texttt{np.mean}.
  \end{enumerate}
\item
  The easiest way to compute differences between adjacent rows is to use
  the
  \href{https://pandas.pydata.org/pandas-docs/stable/reference/api/pandas.DataFrame.diff.html}{\texttt{diff()}}
  method, which returns \(x_t - x_{t-1}\). Pandas then automatically
  matches the correct values and sets the first observation to
  \texttt{NaN} as there is no preceding value to compute the difference.

  To compute a growth rate \((x_t - x_{t-1})/x_{t-1}\), we additionally
  need to lag a variable to get the correct period in the denominator.
  In pandas this is achieved using the
  \href{https://pandas.pydata.org/pandas-docs/stable/reference/api/pandas.DataFrame.shift.html}{\texttt{shift()}}
  method (which defaults to shifting by 1 period).
\end{enumerate}

    \hypertarget{solution-for-exercise-2}{%
\subsubsection{Solution for exercise 2}\label{solution-for-exercise-2}}

This time we do not specify \texttt{index\_cols} when reading in the CSV
data since we need \texttt{Year} as a regular variable, not as the
index.

We then compute the decade for each year, using the fact that
\texttt{//} performs division with integer truncation. As an example,
1951 // 10 is 195, and (1951 // 10) * 10 = 1950, which we use to
represent the 1950s.

    \begin{tcolorbox}[breakable, size=fbox, boxrule=1pt, pad at break*=1mm,colback=cellbackground, colframe=cellborder]
\prompt{In}{incolor}{39}{\boxspacing}
\begin{Verbatim}[commandchars=\\\{\}]
\PY{k+kn}{import} \PY{n+nn}{pandas} \PY{k}{as} \PY{n+nn}{pd} 

\PY{n}{filepath} \PY{o}{=} \PY{l+s+s1}{\PYZsq{}}\PY{l+s+s1}{../data/FRED\PYZus{}QTR.csv}\PY{l+s+s1}{\PYZsq{}}

\PY{n}{df} \PY{o}{=} \PY{n}{pd}\PY{o}{.}\PY{n}{read\PYZus{}csv}\PY{p}{(}\PY{n}{filepath}\PY{p}{,} \PY{n}{sep}\PY{o}{=}\PY{l+s+s1}{\PYZsq{}}\PY{l+s+s1}{,}\PY{l+s+s1}{\PYZsq{}}\PY{p}{)}

\PY{c+c1}{\PYZsh{} Compute GDP growth rates, inflation (in percent)}
\PY{n}{df}\PY{p}{[}\PY{l+s+s1}{\PYZsq{}}\PY{l+s+s1}{GDP\PYZus{}growth}\PY{l+s+s1}{\PYZsq{}}\PY{p}{]} \PY{o}{=} \PY{n}{df}\PY{p}{[}\PY{l+s+s1}{\PYZsq{}}\PY{l+s+s1}{GDP}\PY{l+s+s1}{\PYZsq{}}\PY{p}{]}\PY{o}{.}\PY{n}{diff}\PY{p}{(}\PY{p}{)} \PY{o}{/} \PY{n}{df}\PY{p}{[}\PY{l+s+s1}{\PYZsq{}}\PY{l+s+s1}{GDP}\PY{l+s+s1}{\PYZsq{}}\PY{p}{]}\PY{o}{.}\PY{n}{shift}\PY{p}{(}\PY{p}{)} \PY{o}{*} \PY{l+m+mf}{100.0}
\PY{n}{df}\PY{p}{[}\PY{l+s+s1}{\PYZsq{}}\PY{l+s+s1}{Inflation}\PY{l+s+s1}{\PYZsq{}}\PY{p}{]} \PY{o}{=} \PY{n}{df}\PY{p}{[}\PY{l+s+s1}{\PYZsq{}}\PY{l+s+s1}{CPI}\PY{l+s+s1}{\PYZsq{}}\PY{p}{]}\PY{o}{.}\PY{n}{diff}\PY{p}{(}\PY{p}{)} \PY{o}{/} \PY{n}{df}\PY{p}{[}\PY{l+s+s1}{\PYZsq{}}\PY{l+s+s1}{CPI}\PY{l+s+s1}{\PYZsq{}}\PY{p}{]}\PY{o}{.}\PY{n}{shift}\PY{p}{(}\PY{p}{)} \PY{o}{*} \PY{l+m+mf}{100.0}

\PY{c+c1}{\PYZsh{} Assign decade using // to truncate division to}
\PY{c+c1}{\PYZsh{} integer part. So we have 194x // 10 = 194 for any x.}
\PY{n}{df}\PY{p}{[}\PY{l+s+s1}{\PYZsq{}}\PY{l+s+s1}{Decade}\PY{l+s+s1}{\PYZsq{}}\PY{p}{]} \PY{o}{=} \PY{p}{(}\PY{n}{df}\PY{p}{[}\PY{l+s+s1}{\PYZsq{}}\PY{l+s+s1}{Year}\PY{l+s+s1}{\PYZsq{}}\PY{p}{]} \PY{o}{/}\PY{o}{/} \PY{l+m+mi}{10}\PY{p}{)} \PY{o}{*} \PY{l+m+mi}{10}

\PY{n}{grp} \PY{o}{=} \PY{n}{df}\PY{o}{.}\PY{n}{groupby}\PY{p}{(}\PY{p}{[}\PY{l+s+s1}{\PYZsq{}}\PY{l+s+s1}{Decade}\PY{l+s+s1}{\PYZsq{}}\PY{p}{]}\PY{p}{)}

\PY{c+c1}{\PYZsh{} Print number of obs. by decade}
\PY{n+nb}{print}\PY{p}{(}\PY{n}{grp}\PY{o}{.}\PY{n}{count}\PY{p}{(}\PY{p}{)}\PY{p}{)}

\PY{c+c1}{\PYZsh{} Create series that contains True for each}
\PY{c+c1}{\PYZsh{} decade if all variables have 40 observations.}
\PY{n}{use\PYZus{}decade} \PY{o}{=} \PY{p}{(}\PY{n}{grp}\PY{o}{.}\PY{n}{count}\PY{p}{(}\PY{p}{)} \PY{o}{==} \PY{l+m+mi}{40}\PY{p}{)}\PY{o}{.}\PY{n}{all}\PY{p}{(}\PY{n}{axis}\PY{o}{=}\PY{l+m+mi}{1}\PY{p}{)}
\PY{c+c1}{\PYZsh{} Convert series to DataFrame, assign column name \PYZsq{}Keep\PYZsq{}}
\PY{n}{df\PYZus{}decade} \PY{o}{=} \PY{n}{use\PYZus{}decade}\PY{o}{.}\PY{n}{to\PYZus{}frame}\PY{p}{(}\PY{l+s+s1}{\PYZsq{}}\PY{l+s+s1}{Keep}\PY{l+s+s1}{\PYZsq{}}\PY{p}{)}
\PY{c+c1}{\PYZsh{} merge into original DataFrame, matching rows on value}
\PY{c+c1}{\PYZsh{} of column \PYZsq{}Decade\PYZsq{}}
\PY{n}{df} \PY{o}{=} \PY{n}{df}\PY{o}{.}\PY{n}{merge}\PY{p}{(}\PY{n}{df\PYZus{}decade}\PY{p}{,} \PY{n}{on}\PY{o}{=}\PY{l+s+s1}{\PYZsq{}}\PY{l+s+s1}{Decade}\PY{l+s+s1}{\PYZsq{}}\PY{p}{)}
\PY{c+c1}{\PYZsh{} Restrict data only to rows which are part of complete decade}
\PY{n}{df} \PY{o}{=} \PY{n}{df}\PY{o}{.}\PY{n}{loc}\PY{p}{[}\PY{n}{df}\PY{p}{[}\PY{l+s+s1}{\PYZsq{}}\PY{l+s+s1}{Keep}\PY{l+s+s1}{\PYZsq{}}\PY{p}{]}\PY{p}{,} \PY{p}{:}\PY{p}{]}\PY{o}{.}\PY{n}{copy}\PY{p}{(}\PY{p}{)}
\PY{c+c1}{\PYZsh{} Drop \PYZsq{}Keep\PYZsq{} column}
\PY{k}{del} \PY{n}{df}\PY{p}{[}\PY{l+s+s1}{\PYZsq{}}\PY{l+s+s1}{Keep}\PY{l+s+s1}{\PYZsq{}}\PY{p}{]}

\PY{c+c1}{\PYZsh{} Compute average growth rates and unemployment rate by decade}
\PY{n}{grp} \PY{o}{=} \PY{n}{df}\PY{o}{.}\PY{n}{groupby}\PY{p}{(}\PY{p}{[}\PY{l+s+s1}{\PYZsq{}}\PY{l+s+s1}{Decade}\PY{l+s+s1}{\PYZsq{}}\PY{p}{]}\PY{p}{)}

\PY{n}{df\PYZus{}avg} \PY{o}{=} \PY{n}{grp}\PY{p}{[}\PY{p}{[}\PY{l+s+s1}{\PYZsq{}}\PY{l+s+s1}{GDP\PYZus{}growth}\PY{l+s+s1}{\PYZsq{}}\PY{p}{,} \PY{l+s+s1}{\PYZsq{}}\PY{l+s+s1}{Inflation}\PY{l+s+s1}{\PYZsq{}}\PY{p}{,} \PY{l+s+s1}{\PYZsq{}}\PY{l+s+s1}{UNRATE}\PY{l+s+s1}{\PYZsq{}}\PY{p}{]}\PY{p}{]}\PY{o}{.}\PY{n}{mean}\PY{p}{(}\PY{p}{)}
\PY{c+c1}{\PYZsh{} Convert to (approximate) annualised growth rates}
\PY{n}{df\PYZus{}avg}\PY{p}{[}\PY{l+s+s1}{\PYZsq{}}\PY{l+s+s1}{GDP\PYZus{}growth}\PY{l+s+s1}{\PYZsq{}}\PY{p}{]} \PY{o}{*}\PY{o}{=} \PY{l+m+mf}{4.0}
\PY{n}{df\PYZus{}avg}\PY{p}{[}\PY{l+s+s1}{\PYZsq{}}\PY{l+s+s1}{Inflation}\PY{l+s+s1}{\PYZsq{}}\PY{p}{]} \PY{o}{*}\PY{o}{=} \PY{l+m+mf}{4.0}
\end{Verbatim}
\end{tcolorbox}

    \begin{Verbatim}[commandchars=\\\{\}]
        Year  Quarter  GDP  CPI  UNRATE  LFPART  GDPPOT  NROU  GDP\_growth  \textbackslash{}
Decade
1940       8        8    8    8       8       8       4     4           7
1950      40       40   40   40      40      40      40    40          40
1960      40       40   40   40      40      40      40    40          40
1970      40       40   40   40      40      40      40    40          40
1980      40       40   40   40      40      40      40    40          40
1990      40       40   40   40      40      40      40    40          40
2000      40       40   40   40      40      40      40    40          40
2010      40       40   40   40      40      40      40    40          40

        Inflation
Decade
1940            7
1950           40
1960           40
1970           40
1980           40
1990           40
2000           40
2010           40
    \end{Verbatim}

    The tricky part is to keep only observations for ``complete'' decades
that have 40 quarters of data. We see that this is not the case for the
1940s:

\begin{enumerate}
\def\labelenumi{\arabic{enumi}.}
\tightlist
\item
  We group by \texttt{Decade} and use \texttt{count()} to determine the
  number of non-missing observations for each variable.
\item
  \texttt{count()\ ==\ 40} evaluates to \texttt{True} for some variable
  if it has 40 observations.
\item
  We then use
  \href{https://pandas.pydata.org/pandas-docs/stable/reference/api/pandas.DataFrame.all.html}{\texttt{all()}}
  to aggregate across all variables, \ie we require 40 observations for
  every variable to keep the decade.
\item
  Finally, we merge the indicator whether a decade should be kept in the
  data set using
  \href{https://pandas.pydata.org/pandas-docs/stable/reference/api/pandas.DataFrame.merge.html}{\texttt{merge()}},
  where we match on the value of the column \texttt{Decade}. Note that
  the argument to \texttt{merge()} must be a \texttt{DataFrame}, so we
  first have to convert our indicator data.
\item
  Finally, we keep only those observations which have a flag that is
  \texttt{True}.
\end{enumerate}

The rest of the exercise is straightforward as it just repeats what we
have done previously. You can create the bar chart directly with pandas
as follows:

    \begin{tcolorbox}[breakable, size=fbox, boxrule=1pt, pad at break*=1mm,colback=cellbackground, colframe=cellborder]
\prompt{In}{incolor}{40}{\boxspacing}
\begin{Verbatim}[commandchars=\\\{\}]
\PY{n}{df\PYZus{}avg}\PY{o}{.}\PY{n}{plot}\PY{o}{.}\PY{n}{bar}\PY{p}{(}\PY{n}{y}\PY{o}{=}\PY{p}{[}\PY{l+s+s1}{\PYZsq{}}\PY{l+s+s1}{GDP\PYZus{}growth}\PY{l+s+s1}{\PYZsq{}}\PY{p}{,} \PY{l+s+s1}{\PYZsq{}}\PY{l+s+s1}{Inflation}\PY{l+s+s1}{\PYZsq{}}\PY{p}{,} \PY{l+s+s1}{\PYZsq{}}\PY{l+s+s1}{UNRATE}\PY{l+s+s1}{\PYZsq{}}\PY{p}{]}\PY{p}{)}
\end{Verbatim}
\end{tcolorbox}

            \begin{tcolorbox}[breakable, size=fbox, boxrule=.5pt, pad at break*=1mm, opacityfill=0]
\prompt{Out}{outcolor}{40}{\boxspacing}
\begin{Verbatim}[commandchars=\\\{\}]
<AxesSubplot:xlabel='Decade'>
\end{Verbatim}
\end{tcolorbox}
        
    \begin{center}
    \adjustimage{max size={0.9\linewidth}{0.9\paperheight}}{unit8_files/unit8_90_1.pdf}
    \end{center}
    { \hspace*{\fill} \\}
    
    \hypertarget{solution-for-exercise-3}{%
\subsubsection{Solution for exercise 3}\label{solution-for-exercise-3}}

We first read in the CSV file, specifying
\texttt{\textquotesingle{};\textquotesingle{}} as the field separator:

    \begin{tcolorbox}[breakable, size=fbox, boxrule=1pt, pad at break*=1mm,colback=cellbackground, colframe=cellborder]
\prompt{In}{incolor}{41}{\boxspacing}
\begin{Verbatim}[commandchars=\\\{\}]
\PY{k+kn}{import} \PY{n+nn}{pandas} \PY{k}{as} \PY{n+nn}{pd}
\PY{k+kn}{import} \PY{n+nn}{numpy} \PY{k}{as} \PY{n+nn}{np}

\PY{c+c1}{\PYZsh{} Load CSV file}
\PY{n}{filepath} \PY{o}{=} \PY{l+s+s1}{\PYZsq{}}\PY{l+s+s1}{../data/universities.csv}\PY{l+s+s1}{\PYZsq{}}
\PY{n}{df} \PY{o}{=} \PY{n}{pd}\PY{o}{.}\PY{n}{read\PYZus{}csv}\PY{p}{(}\PY{n}{filepath}\PY{p}{,} \PY{n}{sep}\PY{o}{=}\PY{l+s+s1}{\PYZsq{}}\PY{l+s+s1}{;}\PY{l+s+s1}{\PYZsq{}}\PY{p}{)}
\end{Verbatim}
\end{tcolorbox}

    For the first task we use \texttt{apply()} to create a new
\texttt{Series} object for each ratio of interest.

We compute the ratios for each institution which will result in NaNs if
either the numerator of denominator is missing. We thus use
\texttt{np.nanmean()} to compute averages, ignoring any NaNs.

Finally, we combine all \texttt{Series} into a \texttt{DataFrame}. We do
this by specifying the data passed to \texttt{DataFrame()} as a
dictionary, since then we can specify the column names as keys.

    \begin{tcolorbox}[breakable, size=fbox, boxrule=1pt, pad at break*=1mm,colback=cellbackground, colframe=cellborder]
\prompt{In}{incolor}{42}{\boxspacing}
\begin{Verbatim}[commandchars=\\\{\}]
\PY{c+c1}{\PYZsh{} Variant 1}
\PY{c+c1}{\PYZsh{} Compute means using apply()}

\PY{n}{grp} \PY{o}{=} \PY{n}{df}\PY{o}{.}\PY{n}{groupby}\PY{p}{(}\PY{p}{[}\PY{l+s+s1}{\PYZsq{}}\PY{l+s+s1}{Russell}\PY{l+s+s1}{\PYZsq{}}\PY{p}{]}\PY{p}{)}

\PY{c+c1}{\PYZsh{} Create Series objects with the desired means}
\PY{n}{staff} \PY{o}{=} \PY{n}{grp}\PY{o}{.}\PY{n}{apply}\PY{p}{(}\PY{k}{lambda} \PY{n}{x}\PY{p}{:} \PY{n}{np}\PY{o}{.}\PY{n}{nanmean}\PY{p}{(}\PY{n}{x}\PY{p}{[}\PY{l+s+s1}{\PYZsq{}}\PY{l+s+s1}{Staff}\PY{l+s+s1}{\PYZsq{}}\PY{p}{]} \PY{o}{/} \PY{n}{x}\PY{p}{[}\PY{l+s+s1}{\PYZsq{}}\PY{l+s+s1}{Students}\PY{l+s+s1}{\PYZsq{}}\PY{p}{]}\PY{p}{)}\PY{p}{)}
\PY{n}{admin} \PY{o}{=} \PY{n}{grp}\PY{o}{.}\PY{n}{apply}\PY{p}{(}\PY{k}{lambda} \PY{n}{x}\PY{p}{:} \PY{n}{np}\PY{o}{.}\PY{n}{nanmean}\PY{p}{(}\PY{n}{x}\PY{p}{[}\PY{l+s+s1}{\PYZsq{}}\PY{l+s+s1}{Admin}\PY{l+s+s1}{\PYZsq{}}\PY{p}{]} \PY{o}{/} \PY{n}{x}\PY{p}{[}\PY{l+s+s1}{\PYZsq{}}\PY{l+s+s1}{Students}\PY{l+s+s1}{\PYZsq{}}\PY{p}{]}\PY{p}{)}\PY{p}{)}
\PY{c+c1}{\PYZsh{} Budget in millions of pounds}
\PY{n}{budget} \PY{o}{=} \PY{n}{grp}\PY{o}{.}\PY{n}{apply}\PY{p}{(}\PY{k}{lambda} \PY{n}{x}\PY{p}{:} \PY{n}{np}\PY{o}{.}\PY{n}{nanmean}\PY{p}{(}\PY{n}{x}\PY{p}{[}\PY{l+s+s1}{\PYZsq{}}\PY{l+s+s1}{Budget}\PY{l+s+s1}{\PYZsq{}}\PY{p}{]} \PY{o}{/} \PY{n}{x}\PY{p}{[}\PY{l+s+s1}{\PYZsq{}}\PY{l+s+s1}{Students}\PY{l+s+s1}{\PYZsq{}}\PY{p}{]}\PY{p}{)}\PY{p}{)}
\PY{c+c1}{\PYZsh{} Convert to pounds}
\PY{n}{budget} \PY{o}{*}\PY{o}{=} \PY{l+m+mf}{1.0e6}
\PY{c+c1}{\PYZsh{} Count number of institutions in each group.}
\PY{c+c1}{\PYZsh{} We can use the \PYZsq{}Russell\PYZsq{} column for this because it has no}
\PY{c+c1}{\PYZsh{} missing data.}
\PY{n}{count} \PY{o}{=} \PY{n}{grp}\PY{p}{[}\PY{l+s+s1}{\PYZsq{}}\PY{l+s+s1}{Russell}\PY{l+s+s1}{\PYZsq{}}\PY{p}{]}\PY{o}{.}\PY{n}{count}\PY{p}{(}\PY{p}{)}

\PY{c+c1}{\PYZsh{} Create a new DataFrame. Each column is a Series object.}
\PY{n}{df\PYZus{}all} \PY{o}{=} \PY{n}{pd}\PY{o}{.}\PY{n}{DataFrame}\PY{p}{(}\PY{p}{\PYZob{}}\PY{l+s+s1}{\PYZsq{}}\PY{l+s+s1}{Staff\PYZus{}Student}\PY{l+s+s1}{\PYZsq{}}\PY{p}{:} \PY{n}{staff}\PY{p}{,}
                       \PY{l+s+s1}{\PYZsq{}}\PY{l+s+s1}{Admin\PYZus{}Student}\PY{l+s+s1}{\PYZsq{}}\PY{p}{:} \PY{n}{admin}\PY{p}{,}
                       \PY{l+s+s1}{\PYZsq{}}\PY{l+s+s1}{Budget\PYZus{}Student}\PY{l+s+s1}{\PYZsq{}}\PY{p}{:} \PY{n}{budget}\PY{p}{,}
                       \PY{l+s+s1}{\PYZsq{}}\PY{l+s+s1}{Count}\PY{l+s+s1}{\PYZsq{}}\PY{p}{:} \PY{n}{count}\PY{p}{\PYZcb{}}\PY{p}{)}

\PY{n}{df\PYZus{}all}
\end{Verbatim}
\end{tcolorbox}

            \begin{tcolorbox}[breakable, size=fbox, boxrule=.5pt, pad at break*=1mm, opacityfill=0]
\prompt{Out}{outcolor}{42}{\boxspacing}
\begin{Verbatim}[commandchars=\\\{\}]
         Staff\_Student  Admin\_Student  Budget\_Student  Count
Russell
0             0.096219       0.147762    16847.834366      6
1             0.155131       0.169079    35406.453649     17
\end{Verbatim}
\end{tcolorbox}
        
    For the second task, we first insert additional columns which contain
the ratios of interest for each university.

We then drop all unused columns, group by the \texttt{Russell} indicator
and compute the means by directly calling \texttt{mean()} on the group
object.

    \begin{tcolorbox}[breakable, size=fbox, boxrule=1pt, pad at break*=1mm,colback=cellbackground, colframe=cellborder]
\prompt{In}{incolor}{43}{\boxspacing}
\begin{Verbatim}[commandchars=\\\{\}]
\PY{c+c1}{\PYZsh{} Variant 2:}
\PY{c+c1}{\PYZsh{} Compute ratios first, apply aggregation later}

\PY{c+c1}{\PYZsh{} Create new variables directly in original DataFrame}
\PY{n}{df}\PY{p}{[}\PY{l+s+s1}{\PYZsq{}}\PY{l+s+s1}{Staff\PYZus{}Student}\PY{l+s+s1}{\PYZsq{}}\PY{p}{]} \PY{o}{=} \PY{n}{df}\PY{p}{[}\PY{l+s+s1}{\PYZsq{}}\PY{l+s+s1}{Staff}\PY{l+s+s1}{\PYZsq{}}\PY{p}{]} \PY{o}{/} \PY{n}{df}\PY{p}{[}\PY{l+s+s1}{\PYZsq{}}\PY{l+s+s1}{Students}\PY{l+s+s1}{\PYZsq{}}\PY{p}{]}
\PY{n}{df}\PY{p}{[}\PY{l+s+s1}{\PYZsq{}}\PY{l+s+s1}{Admin\PYZus{}Student}\PY{l+s+s1}{\PYZsq{}}\PY{p}{]} \PY{o}{=} \PY{n}{df}\PY{p}{[}\PY{l+s+s1}{\PYZsq{}}\PY{l+s+s1}{Admin}\PY{l+s+s1}{\PYZsq{}}\PY{p}{]} \PY{o}{/} \PY{n}{df}\PY{p}{[}\PY{l+s+s1}{\PYZsq{}}\PY{l+s+s1}{Students}\PY{l+s+s1}{\PYZsq{}}\PY{p}{]}
\PY{c+c1}{\PYZsh{} Budget in pounds (original Budget is in million pounds)}
\PY{n}{df}\PY{p}{[}\PY{l+s+s1}{\PYZsq{}}\PY{l+s+s1}{Budget\PYZus{}Student}\PY{l+s+s1}{\PYZsq{}}\PY{p}{]} \PY{o}{=} \PY{n}{df}\PY{p}{[}\PY{l+s+s1}{\PYZsq{}}\PY{l+s+s1}{Budget}\PY{l+s+s1}{\PYZsq{}}\PY{p}{]} \PY{o}{/} \PY{n}{df}\PY{p}{[}\PY{l+s+s1}{\PYZsq{}}\PY{l+s+s1}{Students}\PY{l+s+s1}{\PYZsq{}}\PY{p}{]} \PY{o}{*} \PY{l+m+mf}{1.0e6}

\PY{c+c1}{\PYZsh{} Keep only newly constructed ratios}
\PY{n}{columns\PYZus{}keep} \PY{o}{=} \PY{p}{[}\PY{n}{name} \PY{k}{for} \PY{n}{name} \PY{o+ow}{in} \PY{n}{df}\PY{o}{.}\PY{n}{columns}
                \PY{k}{if} \PY{n}{name}\PY{o}{.}\PY{n}{endswith}\PY{p}{(}\PY{l+s+s1}{\PYZsq{}}\PY{l+s+s1}{\PYZus{}Student}\PY{l+s+s1}{\PYZsq{}}\PY{p}{)}\PY{p}{]}
\PY{c+c1}{\PYZsh{} Also keep Russell indicator}
\PY{n}{columns\PYZus{}keep} \PY{o}{+}\PY{o}{=} \PY{p}{[}\PY{l+s+s1}{\PYZsq{}}\PY{l+s+s1}{Russell}\PY{l+s+s1}{\PYZsq{}}\PY{p}{]}
\PY{n}{df} \PY{o}{=} \PY{n}{df}\PY{p}{[}\PY{n}{columns\PYZus{}keep}\PY{p}{]}\PY{o}{.}\PY{n}{copy}\PY{p}{(}\PY{p}{)}

\PY{c+c1}{\PYZsh{} Aggregate by Russell indicator}
\PY{n}{grp} \PY{o}{=} \PY{n}{df}\PY{o}{.}\PY{n}{groupby}\PY{p}{(}\PY{p}{[}\PY{l+s+s1}{\PYZsq{}}\PY{l+s+s1}{Russell}\PY{l+s+s1}{\PYZsq{}}\PY{p}{]}\PY{p}{)}
\PY{c+c1}{\PYZsh{} Count number of institutions in each group.}
\PY{c+c1}{\PYZsh{} We can use the \PYZsq{}Russell\PYZsq{} column for this because it has no}
\PY{c+c1}{\PYZsh{} missing data.}
\PY{n}{count} \PY{o}{=} \PY{n}{grp}\PY{p}{[}\PY{l+s+s1}{\PYZsq{}}\PY{l+s+s1}{Russell}\PY{l+s+s1}{\PYZsq{}}\PY{p}{]}\PY{o}{.}\PY{n}{count}\PY{p}{(}\PY{p}{)}

\PY{n}{df\PYZus{}all} \PY{o}{=} \PY{n}{grp}\PY{o}{.}\PY{n}{mean}\PY{p}{(}\PY{p}{)}
\PY{c+c1}{\PYZsh{} Add counter}
\PY{n}{df\PYZus{}all}\PY{p}{[}\PY{l+s+s1}{\PYZsq{}}\PY{l+s+s1}{Count}\PY{l+s+s1}{\PYZsq{}}\PY{p}{]} \PY{o}{=} \PY{n}{count}

\PY{n}{df\PYZus{}all}
\end{Verbatim}
\end{tcolorbox}

            \begin{tcolorbox}[breakable, size=fbox, boxrule=.5pt, pad at break*=1mm, opacityfill=0]
\prompt{Out}{outcolor}{43}{\boxspacing}
\begin{Verbatim}[commandchars=\\\{\}]
         Staff\_Student  Admin\_Student  Budget\_Student  Count
Russell
0             0.096219       0.147762    16847.834366      6
1             0.155131       0.169079    35406.453649     17
\end{Verbatim}
\end{tcolorbox}
        
    We plot the results using pandas's \texttt{bar()} function. Since the
data is of vastly different magnitudes, we specify \texttt{sharey=False}
so that each panel will have its own scaling on the \(y\)-axis.

    \begin{tcolorbox}[breakable, size=fbox, boxrule=1pt, pad at break*=1mm,colback=cellbackground, colframe=cellborder]
\prompt{In}{incolor}{44}{\boxspacing}
\begin{Verbatim}[commandchars=\\\{\}]
\PY{c+c1}{\PYZsh{} Plot results as bar charts, one panel for each variable}

\PY{c+c1}{\PYZsh{} Pretty titles}
\PY{n}{title} \PY{o}{=} \PY{p}{[}\PY{l+s+s1}{\PYZsq{}}\PY{l+s+s1}{Staff/Student}\PY{l+s+s1}{\PYZsq{}}\PY{p}{,} \PY{l+s+s1}{\PYZsq{}}\PY{l+s+s1}{Admin/Student}\PY{l+s+s1}{\PYZsq{}}\PY{p}{,} \PY{l+s+s1}{\PYZsq{}}\PY{l+s+s1}{Budget/Student}\PY{l+s+s1}{\PYZsq{}}\PY{p}{,} \PY{l+s+s1}{\PYZsq{}}\PY{l+s+s1}{Number of Univ.}\PY{l+s+s1}{\PYZsq{}}\PY{p}{]}
\PY{c+c1}{\PYZsh{} Create bar chart using pandas\PYZsq{}s bar() function}
\PY{n}{df\PYZus{}all}\PY{o}{.}\PY{n}{plot}\PY{o}{.}\PY{n}{bar}\PY{p}{(}\PY{n}{sharey}\PY{o}{=}\PY{k+kc}{False}\PY{p}{,} \PY{n}{subplots}\PY{o}{=}\PY{k+kc}{True}\PY{p}{,} \PY{n}{layout}\PY{o}{=}\PY{p}{(}\PY{l+m+mi}{2}\PY{p}{,} \PY{l+m+mi}{2}\PY{p}{)}\PY{p}{,} \PY{n}{legend}\PY{o}{=}\PY{k+kc}{False}\PY{p}{,}
                \PY{n}{title}\PY{o}{=}\PY{n}{title}\PY{p}{)}
\end{Verbatim}
\end{tcolorbox}

            \begin{tcolorbox}[breakable, size=fbox, boxrule=.5pt, pad at break*=1mm, opacityfill=0]
\prompt{Out}{outcolor}{44}{\boxspacing}
\begin{Verbatim}[commandchars=\\\{\}]
array([[<AxesSubplot:title=\{'center':'Staff/Student'\}, xlabel='Russell'>,
        <AxesSubplot:title=\{'center':'Admin/Student'\}, xlabel='Russell'>],
       [<AxesSubplot:title=\{'center':'Budget/Student'\}, xlabel='Russell'>,
        <AxesSubplot:title=\{'center':'Number of Univ.'\}, xlabel='Russell'>]],
      dtype=object)
\end{Verbatim}
\end{tcolorbox}
        
    \begin{center}
    \adjustimage{max size={0.9\linewidth}{0.9\paperheight}}{unit8_files/unit8_98_1.pdf}
    \end{center}
    { \hspace*{\fill} \\}
    
    \hypertarget{solution-for-exercise-4}{%
\subsubsection{Solution for exercise 4}\label{solution-for-exercise-4}}

    We create an indicator variable called \texttt{Pre1800} which is set to
\texttt{True} whenever the founding year in column \texttt{Founded} is
lower than 1800.

We then group the data by \texttt{Country} and \texttt{Pre1800} and
count the number of universities in each group using \texttt{count()}.

    \begin{tcolorbox}[breakable, size=fbox, boxrule=1pt, pad at break*=1mm,colback=cellbackground, colframe=cellborder]
\prompt{In}{incolor}{45}{\boxspacing}
\begin{Verbatim}[commandchars=\\\{\}]
\PY{k+kn}{import} \PY{n+nn}{pandas} \PY{k}{as} \PY{n+nn}{pd}
\PY{k+kn}{import} \PY{n+nn}{matplotlib}\PY{n+nn}{.}\PY{n+nn}{pyplot} \PY{k}{as} \PY{n+nn}{plt}

\PY{c+c1}{\PYZsh{} Load CSV file}
\PY{n}{filepath} \PY{o}{=} \PY{l+s+s1}{\PYZsq{}}\PY{l+s+s1}{../data/universities.csv}\PY{l+s+s1}{\PYZsq{}}
\PY{n}{df} \PY{o}{=} \PY{n}{pd}\PY{o}{.}\PY{n}{read\PYZus{}csv}\PY{p}{(}\PY{n}{filepath}\PY{p}{,} \PY{n}{sep}\PY{o}{=}\PY{l+s+s1}{\PYZsq{}}\PY{l+s+s1}{;}\PY{l+s+s1}{\PYZsq{}}\PY{p}{)}

\PY{c+c1}{\PYZsh{} Create mask for founding period}
\PY{n}{df}\PY{p}{[}\PY{l+s+s1}{\PYZsq{}}\PY{l+s+s1}{Pre1800}\PY{l+s+s1}{\PYZsq{}}\PY{p}{]} \PY{o}{=} \PY{p}{(}\PY{n}{df}\PY{p}{[}\PY{l+s+s1}{\PYZsq{}}\PY{l+s+s1}{Founded}\PY{l+s+s1}{\PYZsq{}}\PY{p}{]} \PY{o}{\PYZlt{}} \PY{l+m+mi}{1800}\PY{p}{)}

\PY{c+c1}{\PYZsh{} Create group by country and founding period;}
\PY{n}{grp} \PY{o}{=} \PY{n}{df}\PY{o}{.}\PY{n}{groupby}\PY{p}{(}\PY{p}{[}\PY{l+s+s1}{\PYZsq{}}\PY{l+s+s1}{Country}\PY{l+s+s1}{\PYZsq{}}\PY{p}{,} \PY{l+s+s1}{\PYZsq{}}\PY{l+s+s1}{Pre1800}\PY{l+s+s1}{\PYZsq{}}\PY{p}{]}\PY{p}{)}

\PY{c+c1}{\PYZsh{} Number of universities by country and founding period.}
\PY{c+c1}{\PYZsh{} Since we are grouping by two attributes, which will create a}
\PY{c+c1}{\PYZsh{} Series with a multi\PYZhy{}level (hierarchical) index}
\PY{n}{count} \PY{o}{=} \PY{n}{grp}\PY{p}{[}\PY{l+s+s1}{\PYZsq{}}\PY{l+s+s1}{Institution}\PY{l+s+s1}{\PYZsq{}}\PY{p}{]}\PY{o}{.}\PY{n}{count}\PY{p}{(}\PY{p}{)}

\PY{n}{count}
\end{Verbatim}
\end{tcolorbox}

            \begin{tcolorbox}[breakable, size=fbox, boxrule=.5pt, pad at break*=1mm, opacityfill=0]
\prompt{Out}{outcolor}{45}{\boxspacing}
\begin{Verbatim}[commandchars=\\\{\}]
Country           Pre1800
England           False      8
                  True       5
Northern Ireland  False      1
Scotland          False      3
                  True       4
Wales             False      2
Name: Institution, dtype: int64
\end{Verbatim}
\end{tcolorbox}
        
    The resulting \texttt{Series} only contains values for those
combinations that are actually present in the data. For example, the
combination \texttt{(Wales,\ True)} does not show up because there are
no Welsh universities founded before 1800 in our sample. We will have to
``complete'' the data and add zero entries in all such cases.

First, we create a \texttt{DataFrame} with countries in rows and the
number of universities for the pre- and post-1800 periods in columns. To
accomplish this, we need to pivot the second row index using the
\href{https://pandas.pydata.org/pandas-docs/stable/reference/api/pandas.DataFrame.unstack.html}{\texttt{unstack()}}
method. The \texttt{level=-1} argument tells it to use the last row
index, and \texttt{fill\_value=0} will assign zeros to all elements that
were not present in the initial \texttt{DataFrame}, such as the
combination \texttt{(Wales,\ True)}.

    \begin{tcolorbox}[breakable, size=fbox, boxrule=1pt, pad at break*=1mm,colback=cellbackground, colframe=cellborder]
\prompt{In}{incolor}{46}{\boxspacing}
\begin{Verbatim}[commandchars=\\\{\}]
\PY{c+c1}{\PYZsh{} DataFrame with countries in rows, Pre\PYZhy{}1800 indicator in columns}

\PY{c+c1}{\PYZsh{} Pivot inner index level to create separate columns for True/False}
\PY{c+c1}{\PYZsh{} values of Pre1800 indicator}
\PY{n}{df\PYZus{}count} \PY{o}{=} \PY{n}{count}\PY{o}{.}\PY{n}{unstack}\PY{p}{(}\PY{n}{level}\PY{o}{=}\PY{o}{\PYZhy{}}\PY{l+m+mi}{1}\PY{p}{,} \PY{n}{fill\PYZus{}value}\PY{o}{=}\PY{l+m+mi}{0}\PY{p}{)}

\PY{c+c1}{\PYZsh{} Set name of column index to something pretty: this will}
\PY{c+c1}{\PYZsh{} be used as the legend title}
\PY{n}{df\PYZus{}count}\PY{o}{.}\PY{n}{columns}\PY{o}{.}\PY{n}{rename}\PY{p}{(}\PY{l+s+s1}{\PYZsq{}}\PY{l+s+s1}{Founding year}\PY{l+s+s1}{\PYZsq{}}\PY{p}{,} \PY{n}{inplace}\PY{o}{=}\PY{k+kc}{True}\PY{p}{)}
\PY{c+c1}{\PYZsh{} Rename columns to get pretty labels in legend}
\PY{n}{df\PYZus{}count}\PY{o}{.}\PY{n}{rename}\PY{p}{(}\PY{n}{columns}\PY{o}{=}\PY{p}{\PYZob{}}\PY{k+kc}{True}\PY{p}{:} \PY{l+s+s1}{\PYZsq{}}\PY{l+s+s1}{Before 1800}\PY{l+s+s1}{\PYZsq{}}\PY{p}{,} \PY{k+kc}{False}\PY{p}{:} \PY{l+s+s1}{\PYZsq{}}\PY{l+s+s1}{After 1800}\PY{l+s+s1}{\PYZsq{}}\PY{p}{\PYZcb{}}\PY{p}{,}
                \PY{n}{inplace}\PY{o}{=}\PY{k+kc}{True}\PY{p}{)}

\PY{n}{df\PYZus{}count}
\end{Verbatim}
\end{tcolorbox}

            \begin{tcolorbox}[breakable, size=fbox, boxrule=.5pt, pad at break*=1mm, opacityfill=0]
\prompt{Out}{outcolor}{46}{\boxspacing}
\begin{Verbatim}[commandchars=\\\{\}]
Founding year     After 1800  Before 1800
Country
England                    8            5
Northern Ireland           1            0
Scotland                   3            4
Wales                      2            0
\end{Verbatim}
\end{tcolorbox}
        
    Whenever we use pandas's built-in plotting functions, these use index
names and labels to automatically label the graph. We therefore first
have to assign these objects ``pretty'' names.

We can then generate the bar chart as follows:

    \begin{tcolorbox}[breakable, size=fbox, boxrule=1pt, pad at break*=1mm,colback=cellbackground, colframe=cellborder]
\prompt{In}{incolor}{47}{\boxspacing}
\begin{Verbatim}[commandchars=\\\{\}]
\PY{c+c1}{\PYZsh{} Create bar chart by country}
\PY{n}{title} \PY{o}{=} \PY{l+s+s1}{\PYZsq{}}\PY{l+s+s1}{Number of universities by founding year}\PY{l+s+s1}{\PYZsq{}}
\PY{c+c1}{\PYZsh{} pass rot=0 to undo the rotation of x\PYZhy{}tick labels}
\PY{c+c1}{\PYZsh{} which pandas applies by default}
\PY{n}{df\PYZus{}count}\PY{o}{.}\PY{n}{plot}\PY{o}{.}\PY{n}{bar}\PY{p}{(}\PY{n}{xlabel}\PY{o}{=}\PY{l+s+s1}{\PYZsq{}}\PY{l+s+s1}{Country}\PY{l+s+s1}{\PYZsq{}}\PY{p}{,} \PY{n}{rot}\PY{o}{=}\PY{l+m+mi}{0}\PY{p}{,} \PY{n}{title}\PY{o}{=}\PY{n}{title}\PY{p}{)}
\end{Verbatim}
\end{tcolorbox}

            \begin{tcolorbox}[breakable, size=fbox, boxrule=.5pt, pad at break*=1mm, opacityfill=0]
\prompt{Out}{outcolor}{47}{\boxspacing}
\begin{Verbatim}[commandchars=\\\{\}]
<AxesSubplot:title=\{'center':'Number of universities by founding year'\},
xlabel='Country'>
\end{Verbatim}
\end{tcolorbox}
        
    \begin{center}
    \adjustimage{max size={0.9\linewidth}{0.9\paperheight}}{unit8_files/unit8_105_1.pdf}
    \end{center}
    { \hspace*{\fill} \\}
    
    Note how the legend title is automatically set to the column index name
and the legend labels use the column index labels.

We create the second \texttt{DataFrame} with the founding period in rows
and country names in columns in exactly the same way, but now call
\texttt{unstack(level=0)} so that the first index level will be pivoted.

    \begin{tcolorbox}[breakable, size=fbox, boxrule=1pt, pad at break*=1mm,colback=cellbackground, colframe=cellborder]
\prompt{In}{incolor}{48}{\boxspacing}
\begin{Verbatim}[commandchars=\\\{\}]
\PY{c+c1}{\PYZsh{} Pivot first row index level to create separate columns for each country}
\PY{n}{df\PYZus{}count} \PY{o}{=} \PY{n}{count}\PY{o}{.}\PY{n}{unstack}\PY{p}{(}\PY{n}{level}\PY{o}{=}\PY{l+m+mi}{0}\PY{p}{,} \PY{n}{fill\PYZus{}value}\PY{o}{=}\PY{l+m+mi}{0}\PY{p}{)}

\PY{c+c1}{\PYZsh{} Set index name to something pretty}
\PY{n}{df\PYZus{}count}\PY{o}{.}\PY{n}{index}\PY{o}{.}\PY{n}{rename}\PY{p}{(}\PY{l+s+s1}{\PYZsq{}}\PY{l+s+s1}{Founding year}\PY{l+s+s1}{\PYZsq{}}\PY{p}{,} \PY{n}{inplace}\PY{o}{=}\PY{k+kc}{True}\PY{p}{)}
\PY{c+c1}{\PYZsh{} Rename index labels to get pretty text in legend}
\PY{n}{df\PYZus{}count}\PY{o}{.}\PY{n}{rename}\PY{p}{(}\PY{n}{index}\PY{o}{=}\PY{p}{\PYZob{}}\PY{k+kc}{True}\PY{p}{:} \PY{l+s+s1}{\PYZsq{}}\PY{l+s+s1}{Before 1800}\PY{l+s+s1}{\PYZsq{}}\PY{p}{,} \PY{k+kc}{False}\PY{p}{:} \PY{l+s+s1}{\PYZsq{}}\PY{l+s+s1}{After 1800}\PY{l+s+s1}{\PYZsq{}}\PY{p}{\PYZcb{}}\PY{p}{,}
                \PY{n}{inplace}\PY{o}{=}\PY{k+kc}{True}\PY{p}{)}

\PY{n}{df\PYZus{}count}
\end{Verbatim}
\end{tcolorbox}

            \begin{tcolorbox}[breakable, size=fbox, boxrule=.5pt, pad at break*=1mm, opacityfill=0]
\prompt{Out}{outcolor}{48}{\boxspacing}
\begin{Verbatim}[commandchars=\\\{\}]
Country        England  Northern Ireland  Scotland  Wales
Founding year
After 1800           8                 1         3      2
Before 1800          5                 0         4      0
\end{Verbatim}
\end{tcolorbox}
        
    \begin{tcolorbox}[breakable, size=fbox, boxrule=1pt, pad at break*=1mm,colback=cellbackground, colframe=cellborder]
\prompt{In}{incolor}{49}{\boxspacing}
\begin{Verbatim}[commandchars=\\\{\}]
\PY{c+c1}{\PYZsh{} Create bar chart by founding year}
\PY{c+c1}{\PYZsh{} pass rot=0 to undo the rotation of x\PYZhy{}tick labels}
\PY{c+c1}{\PYZsh{} which pandas applies by default}
\PY{n}{df\PYZus{}count}\PY{o}{.}\PY{n}{plot}\PY{o}{.}\PY{n}{bar}\PY{p}{(}\PY{n}{rot}\PY{o}{=}\PY{l+m+mi}{0}\PY{p}{,} \PY{n}{title}\PY{o}{=}\PY{n}{title}\PY{p}{)}
\end{Verbatim}
\end{tcolorbox}

            \begin{tcolorbox}[breakable, size=fbox, boxrule=.5pt, pad at break*=1mm, opacityfill=0]
\prompt{Out}{outcolor}{49}{\boxspacing}
\begin{Verbatim}[commandchars=\\\{\}]
<AxesSubplot:title=\{'center':'Number of universities by founding year'\},
xlabel='Founding year'>
\end{Verbatim}
\end{tcolorbox}
        
    \begin{center}
    \adjustimage{max size={0.9\linewidth}{0.9\paperheight}}{unit8_files/unit8_108_1.pdf}
    \end{center}
    { \hspace*{\fill} \\}
    
    \hypertarget{solution-for-exercise-5}{%
\subsubsection{Solution for exercise 5}\label{solution-for-exercise-5}}

This exercise is quite involved, so we will discuss it in parts. First,
we write the function that will be called by \texttt{apply()} to process
sub-sets of the data which belong to a single decade:

    \begin{tcolorbox}[breakable, size=fbox, boxrule=1pt, pad at break*=1mm,colback=cellbackground, colframe=cellborder]
\prompt{In}{incolor}{50}{\boxspacing}
\begin{Verbatim}[commandchars=\\\{\}]
\PY{k}{def} \PY{n+nf}{regress\PYZus{}okun}\PY{p}{(}\PY{n}{x}\PY{p}{)}\PY{p}{:}
    \PY{c+c1}{\PYZsh{} x is a DataFrame, restricted to rows for the current decade}

    \PY{c+c1}{\PYZsh{} Extract dependent and regressor variables}
    \PY{n}{outcome} \PY{o}{=} \PY{n}{x}\PY{p}{[}\PY{l+s+s1}{\PYZsq{}}\PY{l+s+s1}{unempl\PYZus{}gap}\PY{l+s+s1}{\PYZsq{}}\PY{p}{]}\PY{o}{.}\PY{n}{to\PYZus{}numpy}\PY{p}{(}\PY{p}{)}
    \PY{n}{GDP\PYZus{}gap} \PY{o}{=} \PY{n}{x}\PY{p}{[}\PY{l+s+s1}{\PYZsq{}}\PY{l+s+s1}{GDP\PYZus{}gap}\PY{l+s+s1}{\PYZsq{}}\PY{p}{]}\PY{o}{.}\PY{n}{to\PYZus{}numpy}\PY{p}{(}\PY{p}{)}

    \PY{c+c1}{\PYZsh{} Regressor matrix including intercept}
    \PY{n}{regr} \PY{o}{=} \PY{n}{np}\PY{o}{.}\PY{n}{ones}\PY{p}{(}\PY{p}{(}\PY{n+nb}{len}\PY{p}{(}\PY{n}{GDP\PYZus{}gap}\PY{p}{)}\PY{p}{,} \PY{l+m+mi}{2}\PY{p}{)}\PY{p}{)}
    \PY{c+c1}{\PYZsh{} overwrite second column with output gap}
    \PY{n}{regr}\PY{p}{[}\PY{p}{:}\PY{p}{,}\PY{l+m+mi}{1}\PY{p}{]} \PY{o}{=} \PY{n}{GDP\PYZus{}gap}

    \PY{c+c1}{\PYZsh{} Solve least\PYZhy{}squares problem (pass rcond=None to avoid a warning)}
    \PY{n}{coefs}\PY{p}{,} \PY{o}{*}\PY{n}{rest} \PY{o}{=} \PY{n}{np}\PY{o}{.}\PY{n}{linalg}\PY{o}{.}\PY{n}{lstsq}\PY{p}{(}\PY{n}{regr}\PY{p}{,} \PY{n}{outcome}\PY{p}{,} \PY{n}{rcond}\PY{o}{=}\PY{k+kc}{None}\PY{p}{)}

    \PY{c+c1}{\PYZsh{} Construct DataFrame which will be returned to apply()}
    \PY{c+c1}{\PYZsh{} Convert data to 1 x 2 matrix}
    \PY{n}{data} \PY{o}{=} \PY{n}{coefs}\PY{p}{[}\PY{k+kc}{None}\PY{p}{]}
    \PY{n}{columns} \PY{o}{=} \PY{p}{[}\PY{l+s+s1}{\PYZsq{}}\PY{l+s+s1}{Const}\PY{l+s+s1}{\PYZsq{}}\PY{p}{,} \PY{l+s+s1}{\PYZsq{}}\PY{l+s+s1}{GDP\PYZus{}gap}\PY{l+s+s1}{\PYZsq{}}\PY{p}{]}
    \PY{n}{df\PYZus{}out} \PY{o}{=} \PY{n}{pd}\PY{o}{.}\PY{n}{DataFrame}\PY{p}{(}\PY{n}{data}\PY{p}{,} \PY{n}{columns}\PY{o}{=}\PY{n}{columns}\PY{p}{)}

    \PY{k}{return} \PY{n}{df\PYZus{}out}
\end{Verbatim}
\end{tcolorbox}

    This function is passed in a single argument which is a
\texttt{DataFrame} restricted to the sub-sample that is currently being
processed.

\begin{itemize}
\item
  Our task is to perform the required calculations and to return the
  result as a \texttt{DataFrame}. \texttt{apply()} then glues together
  all decade-specific DataFrames to form the result of the operation.
\item
  We first extract the relevant variables as NumPy arrays, and we create
  a regressor matrix which has ones in the first column. This column
  represents the intercept.
\item
  We invoke \texttt{lstsq()} to run the regression. \texttt{lstsq()}
  returns several arguments which we mop up in the tuple \texttt{*rest}
  since we are only interested in the regression coefficients.

  Note that we wouldn't be using \texttt{lstsq()} to run OLS on a
  regular basis, but it's sufficient for this use case.
\item
  Finally, we build the \texttt{DataFrame} to be returned by this
  function. It has only one row (since we ran only one regression) and
  two columns, one for each regression coefficient.
\end{itemize}

This was the hard part. We now need to perform some standard
manipulations to prepare the data:

\begin{enumerate}
\def\labelenumi{\arabic{enumi}.}
\tightlist
\item
  We construct the output gap (in percent), which we store in the column
  \texttt{GDP\_gap}.
\item
  We construct the cyclical unemployment rate and store it in the column
  \texttt{unempl\_gap}.
\item
  We determine the decade each observation belongs to using the same
  code as in previous exercises.
\item
  We then drop all unused variables from the \texttt{DataFrame} and also
  all observations which contain missing values.
\end{enumerate}

Lastly, we can call \texttt{apply()} to run the regression for each
decade.

    \begin{tcolorbox}[breakable, size=fbox, boxrule=1pt, pad at break*=1mm,colback=cellbackground, colframe=cellborder]
\prompt{In}{incolor}{51}{\boxspacing}
\begin{Verbatim}[commandchars=\\\{\}]
\PY{k+kn}{import} \PY{n+nn}{pandas} \PY{k}{as} \PY{n+nn}{pd}
\PY{k+kn}{import} \PY{n+nn}{numpy} \PY{k}{as} \PY{n+nn}{np}
\PY{k+kn}{import} \PY{n+nn}{matplotlib}\PY{n+nn}{.}\PY{n+nn}{pyplot} \PY{k}{as} \PY{n+nn}{plt}

\PY{c+c1}{\PYZsh{} Load CSV file}
\PY{n}{filepath} \PY{o}{=} \PY{l+s+s1}{\PYZsq{}}\PY{l+s+s1}{../data/FRED\PYZus{}QTR.csv}\PY{l+s+s1}{\PYZsq{}}
\PY{n}{df} \PY{o}{=} \PY{n}{pd}\PY{o}{.}\PY{n}{read\PYZus{}csv}\PY{p}{(}\PY{n}{filepath}\PY{p}{,} \PY{n}{sep}\PY{o}{=}\PY{l+s+s1}{\PYZsq{}}\PY{l+s+s1}{,}\PY{l+s+s1}{\PYZsq{}}\PY{p}{)}

\PY{c+c1}{\PYZsh{} Generate output gap (in percent)}
\PY{n}{df}\PY{p}{[}\PY{l+s+s1}{\PYZsq{}}\PY{l+s+s1}{GDP\PYZus{}gap}\PY{l+s+s1}{\PYZsq{}}\PY{p}{]} \PY{o}{=} \PY{p}{(}\PY{n}{df}\PY{p}{[}\PY{l+s+s1}{\PYZsq{}}\PY{l+s+s1}{GDP}\PY{l+s+s1}{\PYZsq{}}\PY{p}{]} \PY{o}{\PYZhy{}} \PY{n}{df}\PY{p}{[}\PY{l+s+s1}{\PYZsq{}}\PY{l+s+s1}{GDPPOT}\PY{l+s+s1}{\PYZsq{}}\PY{p}{]}\PY{p}{)} \PY{o}{/} \PY{n}{df}\PY{p}{[}\PY{l+s+s1}{\PYZsq{}}\PY{l+s+s1}{GDPPOT}\PY{l+s+s1}{\PYZsq{}}\PY{p}{]} \PY{o}{*} \PY{l+m+mf}{100.0}

\PY{c+c1}{\PYZsh{} Generate deviations of unempl. rate from natural unempl. rate}
\PY{n}{df}\PY{p}{[}\PY{l+s+s1}{\PYZsq{}}\PY{l+s+s1}{unempl\PYZus{}gap}\PY{l+s+s1}{\PYZsq{}}\PY{p}{]} \PY{o}{=} \PY{n}{df}\PY{p}{[}\PY{l+s+s1}{\PYZsq{}}\PY{l+s+s1}{UNRATE}\PY{l+s+s1}{\PYZsq{}}\PY{p}{]} \PY{o}{\PYZhy{}} \PY{n}{df}\PY{p}{[}\PY{l+s+s1}{\PYZsq{}}\PY{l+s+s1}{NROU}\PY{l+s+s1}{\PYZsq{}}\PY{p}{]}

\PY{c+c1}{\PYZsh{} Assign decade using // to truncate division to}
\PY{c+c1}{\PYZsh{} integer part. So we have 194x // 10 = 194 for any x.}
\PY{n}{df}\PY{p}{[}\PY{l+s+s1}{\PYZsq{}}\PY{l+s+s1}{Decade}\PY{l+s+s1}{\PYZsq{}}\PY{p}{]} \PY{o}{=} \PY{p}{(}\PY{n}{df}\PY{p}{[}\PY{l+s+s1}{\PYZsq{}}\PY{l+s+s1}{Year}\PY{l+s+s1}{\PYZsq{}}\PY{p}{]} \PY{o}{/}\PY{o}{/} \PY{l+m+mi}{10}\PY{p}{)} \PY{o}{*} \PY{l+m+mi}{10}

\PY{c+c1}{\PYZsh{} Keep only variables of interest}
\PY{n}{df} \PY{o}{=} \PY{n}{df}\PY{p}{[}\PY{p}{[}\PY{l+s+s1}{\PYZsq{}}\PY{l+s+s1}{Decade}\PY{l+s+s1}{\PYZsq{}}\PY{p}{,} \PY{l+s+s1}{\PYZsq{}}\PY{l+s+s1}{GDP\PYZus{}gap}\PY{l+s+s1}{\PYZsq{}}\PY{p}{,} \PY{l+s+s1}{\PYZsq{}}\PY{l+s+s1}{unempl\PYZus{}gap}\PY{l+s+s1}{\PYZsq{}}\PY{p}{]}\PY{p}{]}
\PY{c+c1}{\PYZsh{} Drop rows with any missing obs.}
\PY{n}{df} \PY{o}{=} \PY{n}{df}\PY{o}{.}\PY{n}{dropna}\PY{p}{(}\PY{n}{axis}\PY{o}{=}\PY{l+m+mi}{0}\PY{p}{)}

\PY{c+c1}{\PYZsh{} Group by decade}
\PY{n}{grp} \PY{o}{=} \PY{n}{df}\PY{o}{.}\PY{n}{groupby}\PY{p}{(}\PY{p}{[}\PY{l+s+s1}{\PYZsq{}}\PY{l+s+s1}{Decade}\PY{l+s+s1}{\PYZsq{}}\PY{p}{]}\PY{p}{)}

\PY{c+c1}{\PYZsh{} Apply regression routine to sub\PYZhy{}set of data for each decade}
\PY{n}{df\PYZus{}reg} \PY{o}{=} \PY{n}{grp}\PY{o}{.}\PY{n}{apply}\PY{p}{(}\PY{n}{regress\PYZus{}okun}\PY{p}{)}
\PY{c+c1}{\PYZsh{} Get rid of second row index introduced by apply()}
\PY{n}{df\PYZus{}reg} \PY{o}{=} \PY{n}{df\PYZus{}reg}\PY{o}{.}\PY{n}{reset\PYZus{}index}\PY{p}{(}\PY{n}{level}\PY{o}{=}\PY{o}{\PYZhy{}}\PY{l+m+mi}{1}\PY{p}{,} \PY{n}{drop}\PY{o}{=}\PY{k+kc}{True}\PY{p}{)}

\PY{c+c1}{\PYZsh{} Display intercept and slope coefficients}
\PY{c+c1}{\PYZsh{} estimated for each decade.}
\PY{n}{df\PYZus{}reg}
\end{Verbatim}
\end{tcolorbox}

            \begin{tcolorbox}[breakable, size=fbox, boxrule=.5pt, pad at break*=1mm, opacityfill=0]
\prompt{Out}{outcolor}{51}{\boxspacing}
\begin{Verbatim}[commandchars=\\\{\}]
           Const   GDP\_gap
Decade
1940   -0.259986 -0.567257
1950   -0.277104 -0.494637
1960   -0.331665 -0.467206
1970   -0.032063 -0.398751
1980   -0.178001 -0.666688
1990   -0.102465 -0.489427
2000   -0.355138 -0.723567
2010   -0.279333 -0.983768
\end{Verbatim}
\end{tcolorbox}
        
    The following code creates 8 panels of scatter plots showing the raw
data and overlays a regression line for each decade.

The code is somewhat more involved than usual because we have 9 panels
but only 8 sets of data to be plotted, and we want to add axes labels
only for those panels that are on the left and lower boundaries.

    \begin{tcolorbox}[breakable, size=fbox, boxrule=1pt, pad at break*=1mm,colback=cellbackground, colframe=cellborder]
\prompt{In}{incolor}{52}{\boxspacing}
\begin{Verbatim}[commandchars=\\\{\}]
\PY{c+c1}{\PYZsh{} Number of plots (= number of decades)}
\PY{n}{Nplots} \PY{o}{=} \PY{n+nb}{len}\PY{p}{(}\PY{n}{df\PYZus{}reg}\PY{p}{)}

\PY{c+c1}{\PYZsh{} Fix number of columns, determine rows as needed}
\PY{n}{ncol} \PY{o}{=} \PY{l+m+mi}{3}
\PY{n}{nrow} \PY{o}{=} \PY{n+nb}{int}\PY{p}{(}\PY{n}{np}\PY{o}{.}\PY{n}{ceil}\PY{p}{(}\PY{n}{Nplots} \PY{o}{/} \PY{n}{ncol}\PY{p}{)}\PY{p}{)}

\PY{n}{fig}\PY{p}{,} \PY{n}{axes} \PY{o}{=} \PY{n}{plt}\PY{o}{.}\PY{n}{subplots}\PY{p}{(}\PY{n}{nrow}\PY{p}{,} \PY{n}{ncol}\PY{p}{,} \PY{n}{sharey}\PY{o}{=}\PY{k+kc}{True}\PY{p}{,} \PY{n}{sharex}\PY{o}{=}\PY{k+kc}{True}\PY{p}{,} 
                         \PY{n}{figsize}\PY{o}{=}\PY{p}{(}\PY{l+m+mi}{8}\PY{p}{,} \PY{l+m+mi}{8}\PY{p}{)}\PY{p}{)}

\PY{k}{for} \PY{n}{i}\PY{p}{,} \PY{n}{ax} \PY{o+ow}{in} \PY{n+nb}{enumerate}\PY{p}{(}\PY{n}{axes}\PY{o}{.}\PY{n}{flatten}\PY{p}{(}\PY{p}{)}\PY{p}{)}\PY{p}{:}

    \PY{c+c1}{\PYZsh{} skip if we are out of data (we have 9 panels, but only 8 decades)}
    \PY{k}{if} \PY{n}{i} \PY{o}{\PYZgt{}}\PY{o}{=} \PY{n}{Nplots}\PY{p}{:}
        \PY{c+c1}{\PYZsh{} Turn off frame, axes, etc.}
        \PY{n}{ax}\PY{o}{.}\PY{n}{get\PYZus{}xaxis}\PY{p}{(}\PY{p}{)}\PY{o}{.}\PY{n}{set\PYZus{}visible}\PY{p}{(}\PY{k+kc}{False}\PY{p}{)}
        \PY{n}{ax}\PY{o}{.}\PY{n}{get\PYZus{}yaxis}\PY{p}{(}\PY{p}{)}\PY{o}{.}\PY{n}{set\PYZus{}visible}\PY{p}{(}\PY{k+kc}{False}\PY{p}{)}
        \PY{n}{ax}\PY{o}{.}\PY{n}{set\PYZus{}frame\PYZus{}on}\PY{p}{(}\PY{k+kc}{False}\PY{p}{)}
        \PY{k}{break}

    \PY{c+c1}{\PYZsh{} decade in current iteration}
    \PY{n}{decade} \PY{o}{=} \PY{n}{df\PYZus{}reg}\PY{o}{.}\PY{n}{index}\PY{o}{.}\PY{n}{values}\PY{p}{[}\PY{n}{i}\PY{p}{]}
    \PY{c+c1}{\PYZsh{} restrict DataFrame to decade\PYZhy{}specific data}
    \PY{n}{dfi} \PY{o}{=} \PY{n}{df}\PY{o}{.}\PY{n}{loc}\PY{p}{[}\PY{n}{df}\PY{p}{[}\PY{l+s+s1}{\PYZsq{}}\PY{l+s+s1}{Decade}\PY{l+s+s1}{\PYZsq{}}\PY{p}{]} \PY{o}{==} \PY{n}{decade}\PY{p}{]}
    \PY{c+c1}{\PYZsh{} Scatter plot of raw data}
    \PY{n}{ax}\PY{o}{.}\PY{n}{scatter}\PY{p}{(}\PY{n}{dfi}\PY{p}{[}\PY{l+s+s1}{\PYZsq{}}\PY{l+s+s1}{GDP\PYZus{}gap}\PY{l+s+s1}{\PYZsq{}}\PY{p}{]}\PY{p}{,} \PY{n}{dfi}\PY{p}{[}\PY{l+s+s1}{\PYZsq{}}\PY{l+s+s1}{unempl\PYZus{}gap}\PY{l+s+s1}{\PYZsq{}}\PY{p}{]}\PY{p}{,} \PY{n}{color}\PY{o}{=}\PY{l+s+s1}{\PYZsq{}}\PY{l+s+s1}{steelblue}\PY{l+s+s1}{\PYZsq{}}\PY{p}{,}
               \PY{n}{alpha}\PY{o}{=}\PY{l+m+mf}{0.7}\PY{p}{,} \PY{n}{label}\PY{o}{=}\PY{l+s+s1}{\PYZsq{}}\PY{l+s+s1}{Raw data}\PY{l+s+s1}{\PYZsq{}}\PY{p}{)}
    \PY{c+c1}{\PYZsh{} Extract regression coefficients}
    \PY{n}{const} \PY{o}{=} \PY{n}{df\PYZus{}reg}\PY{o}{.}\PY{n}{loc}\PY{p}{[}\PY{n}{decade}\PY{p}{,} \PY{l+s+s1}{\PYZsq{}}\PY{l+s+s1}{Const}\PY{l+s+s1}{\PYZsq{}}\PY{p}{]}
    \PY{n}{slope} \PY{o}{=} \PY{n}{df\PYZus{}reg}\PY{o}{.}\PY{n}{loc}\PY{p}{[}\PY{n}{decade}\PY{p}{,} \PY{l+s+s1}{\PYZsq{}}\PY{l+s+s1}{GDP\PYZus{}gap}\PY{l+s+s1}{\PYZsq{}}\PY{p}{]}

    \PY{c+c1}{\PYZsh{} plot regression line:}
    \PY{c+c1}{\PYZsh{} We need to provide two points to define the line to be plotted.}
    \PY{n}{ax}\PY{o}{.}\PY{n}{axline}\PY{p}{(}\PY{p}{(}\PY{l+m+mf}{0.0}\PY{p}{,} \PY{n}{const}\PY{p}{)}\PY{p}{,} \PY{p}{(}\PY{l+m+mf}{1.0}\PY{p}{,} \PY{n}{const}\PY{o}{+}\PY{n}{slope}\PY{p}{)}\PY{p}{,} \PY{n}{color}\PY{o}{=}\PY{l+s+s1}{\PYZsq{}}\PY{l+s+s1}{red}\PY{l+s+s1}{\PYZsq{}}\PY{p}{,} 
              \PY{n}{lw}\PY{o}{=}\PY{l+m+mf}{2.0}\PY{p}{,} \PY{n}{label}\PY{o}{=}\PY{l+s+s1}{\PYZsq{}}\PY{l+s+s1}{Regression line}\PY{l+s+s1}{\PYZsq{}}\PY{p}{)}

    \PY{c+c1}{\PYZsh{} Add label containing the current decade}
    \PY{n}{ax}\PY{o}{.}\PY{n}{text}\PY{p}{(}\PY{l+m+mf}{0.95}\PY{p}{,} \PY{l+m+mf}{0.95}\PY{p}{,} \PY{l+s+sa}{f}\PY{l+s+s2}{\PYZdq{}}\PY{l+s+si}{\PYZob{}}\PY{n}{decade}\PY{l+s+si}{\PYZcb{}}\PY{l+s+s2}{\PYZsq{}}\PY{l+s+s2}{s}\PY{l+s+s2}{\PYZdq{}}\PY{p}{,} \PY{n}{transform}\PY{o}{=}\PY{n}{ax}\PY{o}{.}\PY{n}{transAxes}\PY{p}{,}
            \PY{n}{va}\PY{o}{=}\PY{l+s+s1}{\PYZsq{}}\PY{l+s+s1}{top}\PY{l+s+s1}{\PYZsq{}}\PY{p}{,} \PY{n}{ha}\PY{o}{=}\PY{l+s+s1}{\PYZsq{}}\PY{l+s+s1}{right}\PY{l+s+s1}{\PYZsq{}}\PY{p}{)}

    \PY{c+c1}{\PYZsh{} Add legend in the first panel only}
    \PY{k}{if} \PY{n}{i} \PY{o}{==} \PY{l+m+mi}{0}\PY{p}{:}
        \PY{n}{ax}\PY{o}{.}\PY{n}{legend}\PY{p}{(}\PY{n}{loc}\PY{o}{=}\PY{l+s+s1}{\PYZsq{}}\PY{l+s+s1}{lower left}\PY{l+s+s1}{\PYZsq{}}\PY{p}{,} \PY{n}{frameon}\PY{o}{=}\PY{k+kc}{False}\PY{p}{)}

    \PY{c+c1}{\PYZsh{} Add x\PYZhy{} and y\PYZhy{}labels, but only for those panels}
    \PY{c+c1}{\PYZsh{} that are on the left/lower boundary of the figure}
    \PY{k}{if} \PY{n}{i} \PY{o}{\PYZgt{}}\PY{o}{=} \PY{n}{nrow} \PY{o}{*} \PY{p}{(}\PY{n}{ncol} \PY{o}{\PYZhy{}} \PY{l+m+mi}{1}\PY{p}{)}\PY{p}{:}
        \PY{n}{ax}\PY{o}{.}\PY{n}{set\PYZus{}xlabel}\PY{p}{(}\PY{l+s+s1}{\PYZsq{}}\PY{l+s+s1}{Output gap (}\PY{l+s+s1}{\PYZpc{}}\PY{l+s+s1}{)}\PY{l+s+s1}{\PYZsq{}}\PY{p}{)}
    \PY{k}{if} \PY{p}{(}\PY{n}{i} \PY{o}{\PYZpc{}} \PY{l+m+mi}{3}\PY{p}{)} \PY{o}{==} \PY{l+m+mi}{0}\PY{p}{:}
        \PY{n}{ax}\PY{o}{.}\PY{n}{set\PYZus{}ylabel}\PY{p}{(}\PY{l+s+s1}{\PYZsq{}}\PY{l+s+s1}{Cycl. unempl. rate (}\PY{l+s+s1}{\PYZpc{}}\PY{l+s+s1}{\PYZhy{}points)}\PY{l+s+s1}{\PYZsq{}}\PY{p}{)}

\PY{n}{fig}\PY{o}{.}\PY{n}{suptitle}\PY{p}{(}\PY{l+s+s2}{\PYZdq{}}\PY{l+s+s2}{Okun}\PY{l+s+s2}{\PYZsq{}}\PY{l+s+s2}{s law}\PY{l+s+s2}{\PYZdq{}}\PY{p}{)}
\end{Verbatim}
\end{tcolorbox}

            \begin{tcolorbox}[breakable, size=fbox, boxrule=.5pt, pad at break*=1mm, opacityfill=0]
\prompt{Out}{outcolor}{52}{\boxspacing}
\begin{Verbatim}[commandchars=\\\{\}]
Text(0.5, 0.98, "Okun's law")
\end{Verbatim}
\end{tcolorbox}
        
    \begin{center}
    \adjustimage{max size={0.9\linewidth}{0.9\paperheight}}{unit8_files/unit8_114_1.pdf}
    \end{center}
    { \hspace*{\fill} \\}
    

    % Add a bibliography block to the postdoc
    
    
    
\end{document}
