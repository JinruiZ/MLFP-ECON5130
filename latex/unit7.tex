\documentclass[10pt]{scrartcl}

    % included at the top of the generated TeX file

\usepackage{tgcursor}

\usepackage[utf8]{inputenc}

\KOMAoptions{parskip=half*}
\KOMAoptions{paper=a4,twoside=false}
\KOMAoptions{numbers=noendperiod}

\addtokomafont{disposition}{\rmfamily}
\setcounter{tocdepth}{\subsectiontocdepth}

    \usepackage[breakable]{tcolorbox}
    \usepackage{parskip} % Stop auto-indenting (to mimic markdown behaviour)
    
    \usepackage{iftex}
    \ifPDFTeX
    	\usepackage[T1]{fontenc}
    	\usepackage{mathpazo}
    \else
    	\usepackage{fontspec}
    \fi

    % Basic figure setup, for now with no caption control since it's done
    % automatically by Pandoc (which extracts ![](path) syntax from Markdown).
    \usepackage{graphicx}
    % Maintain compatibility with old templates. Remove in nbconvert 6.0
    \let\Oldincludegraphics\includegraphics
    % Ensure that by default, figures have no caption (until we provide a
    % proper Figure object with a Caption API and a way to capture that
    % in the conversion process - todo).
    \usepackage{caption}
    \DeclareCaptionFormat{nocaption}{}
    \captionsetup{format=nocaption,aboveskip=0pt,belowskip=0pt}

    \usepackage{float}
    \floatplacement{figure}{H} % forces figures to be placed at the correct location
    \usepackage{xcolor} % Allow colors to be defined
    \usepackage{enumerate} % Needed for markdown enumerations to work
    \usepackage{geometry} % Used to adjust the document margins
    \usepackage{amsmath} % Equations
    \usepackage{amssymb} % Equations
    \usepackage{textcomp} % defines textquotesingle
    % Hack from http://tex.stackexchange.com/a/47451/13684:
    \AtBeginDocument{%
        \def\PYZsq{\textquotesingle}% Upright quotes in Pygmentized code
    }
    \usepackage{upquote} % Upright quotes for verbatim code
    \usepackage{eurosym} % defines \euro
    \usepackage[mathletters]{ucs} % Extended unicode (utf-8) support
    \usepackage{fancyvrb} % verbatim replacement that allows latex
    \usepackage{grffile} % extends the file name processing of package graphics 
                         % to support a larger range
    \makeatletter % fix for old versions of grffile with XeLaTeX
    \@ifpackagelater{grffile}{2019/11/01}
    {
      % Do nothing on new versions
    }
    {
      \def\Gread@@xetex#1{%
        \IfFileExists{"\Gin@base".bb}%
        {\Gread@eps{\Gin@base.bb}}%
        {\Gread@@xetex@aux#1}%
      }
    }
    \makeatother
    \usepackage[Export]{adjustbox} % Used to constrain images to a maximum size
    \adjustboxset{max size={0.9\linewidth}{0.9\paperheight}}

    % The hyperref package gives us a pdf with properly built
    % internal navigation ('pdf bookmarks' for the table of contents,
    % internal cross-reference links, web links for URLs, etc.)
    \usepackage{hyperref}
    % The default LaTeX title has an obnoxious amount of whitespace. By default,
    % titling removes some of it. It also provides customization options.
    \usepackage{titling}
    \usepackage{longtable} % longtable support required by pandoc >1.10
    \usepackage{booktabs}  % table support for pandoc > 1.12.2
    \usepackage[inline]{enumitem} % IRkernel/repr support (it uses the enumerate* environment)
    \usepackage[normalem]{ulem} % ulem is needed to support strikethroughs (\sout)
                                % normalem makes italics be italics, not underlines
    \usepackage{mathrsfs}
    

    
    % Colors for the hyperref package
    \definecolor{urlcolor}{rgb}{0,.145,.698}
    \definecolor{linkcolor}{rgb}{.71,0.21,0.01}
    \definecolor{citecolor}{rgb}{.12,.54,.11}

    % ANSI colors
    \definecolor{ansi-black}{HTML}{3E424D}
    \definecolor{ansi-black-intense}{HTML}{282C36}
    \definecolor{ansi-red}{HTML}{E75C58}
    \definecolor{ansi-red-intense}{HTML}{B22B31}
    \definecolor{ansi-green}{HTML}{00A250}
    \definecolor{ansi-green-intense}{HTML}{007427}
    \definecolor{ansi-yellow}{HTML}{DDB62B}
    \definecolor{ansi-yellow-intense}{HTML}{B27D12}
    \definecolor{ansi-blue}{HTML}{208FFB}
    \definecolor{ansi-blue-intense}{HTML}{0065CA}
    \definecolor{ansi-magenta}{HTML}{D160C4}
    \definecolor{ansi-magenta-intense}{HTML}{A03196}
    \definecolor{ansi-cyan}{HTML}{60C6C8}
    \definecolor{ansi-cyan-intense}{HTML}{258F8F}
    \definecolor{ansi-white}{HTML}{C5C1B4}
    \definecolor{ansi-white-intense}{HTML}{A1A6B2}
    \definecolor{ansi-default-inverse-fg}{HTML}{FFFFFF}
    \definecolor{ansi-default-inverse-bg}{HTML}{000000}

    % common color for the border for error outputs.
    \definecolor{outerrorbackground}{HTML}{FFDFDF}

    % commands and environments needed by pandoc snippets
    % extracted from the output of `pandoc -s`
    \providecommand{\tightlist}{%
      \setlength{\itemsep}{0pt}\setlength{\parskip}{0pt}}
    \DefineVerbatimEnvironment{Highlighting}{Verbatim}{commandchars=\\\{\}}
    % Add ',fontsize=\small' for more characters per line
    \newenvironment{Shaded}{}{}
    \newcommand{\KeywordTok}[1]{\textcolor[rgb]{0.00,0.44,0.13}{\textbf{{#1}}}}
    \newcommand{\DataTypeTok}[1]{\textcolor[rgb]{0.56,0.13,0.00}{{#1}}}
    \newcommand{\DecValTok}[1]{\textcolor[rgb]{0.25,0.63,0.44}{{#1}}}
    \newcommand{\BaseNTok}[1]{\textcolor[rgb]{0.25,0.63,0.44}{{#1}}}
    \newcommand{\FloatTok}[1]{\textcolor[rgb]{0.25,0.63,0.44}{{#1}}}
    \newcommand{\CharTok}[1]{\textcolor[rgb]{0.25,0.44,0.63}{{#1}}}
    \newcommand{\StringTok}[1]{\textcolor[rgb]{0.25,0.44,0.63}{{#1}}}
    \newcommand{\CommentTok}[1]{\textcolor[rgb]{0.38,0.63,0.69}{\textit{{#1}}}}
    \newcommand{\OtherTok}[1]{\textcolor[rgb]{0.00,0.44,0.13}{{#1}}}
    \newcommand{\AlertTok}[1]{\textcolor[rgb]{1.00,0.00,0.00}{\textbf{{#1}}}}
    \newcommand{\FunctionTok}[1]{\textcolor[rgb]{0.02,0.16,0.49}{{#1}}}
    \newcommand{\RegionMarkerTok}[1]{{#1}}
    \newcommand{\ErrorTok}[1]{\textcolor[rgb]{1.00,0.00,0.00}{\textbf{{#1}}}}
    \newcommand{\NormalTok}[1]{{#1}}
    
    % Additional commands for more recent versions of Pandoc
    \newcommand{\ConstantTok}[1]{\textcolor[rgb]{0.53,0.00,0.00}{{#1}}}
    \newcommand{\SpecialCharTok}[1]{\textcolor[rgb]{0.25,0.44,0.63}{{#1}}}
    \newcommand{\VerbatimStringTok}[1]{\textcolor[rgb]{0.25,0.44,0.63}{{#1}}}
    \newcommand{\SpecialStringTok}[1]{\textcolor[rgb]{0.73,0.40,0.53}{{#1}}}
    \newcommand{\ImportTok}[1]{{#1}}
    \newcommand{\DocumentationTok}[1]{\textcolor[rgb]{0.73,0.13,0.13}{\textit{{#1}}}}
    \newcommand{\AnnotationTok}[1]{\textcolor[rgb]{0.38,0.63,0.69}{\textbf{\textit{{#1}}}}}
    \newcommand{\CommentVarTok}[1]{\textcolor[rgb]{0.38,0.63,0.69}{\textbf{\textit{{#1}}}}}
    \newcommand{\VariableTok}[1]{\textcolor[rgb]{0.10,0.09,0.49}{{#1}}}
    \newcommand{\ControlFlowTok}[1]{\textcolor[rgb]{0.00,0.44,0.13}{\textbf{{#1}}}}
    \newcommand{\OperatorTok}[1]{\textcolor[rgb]{0.40,0.40,0.40}{{#1}}}
    \newcommand{\BuiltInTok}[1]{{#1}}
    \newcommand{\ExtensionTok}[1]{{#1}}
    \newcommand{\PreprocessorTok}[1]{\textcolor[rgb]{0.74,0.48,0.00}{{#1}}}
    \newcommand{\AttributeTok}[1]{\textcolor[rgb]{0.49,0.56,0.16}{{#1}}}
    \newcommand{\InformationTok}[1]{\textcolor[rgb]{0.38,0.63,0.69}{\textbf{\textit{{#1}}}}}
    \newcommand{\WarningTok}[1]{\textcolor[rgb]{0.38,0.63,0.69}{\textbf{\textit{{#1}}}}}
    
    
    % Define a nice break command that doesn't care if a line doesn't already
    % exist.
    \def\br{\hspace*{\fill} \\* }
    % Math Jax compatibility definitions
    \def\gt{>}
    \def\lt{<}
    \let\Oldtex\TeX
    \let\Oldlatex\LaTeX
    \renewcommand{\TeX}{\textrm{\Oldtex}}
    \renewcommand{\LaTeX}{\textrm{\Oldlatex}}
    % Document parameters
    % Document title
    \newcommand*{\mytitle}{Unit 7: Random number generation and statistics}

    % Included at the bottom of the preamble

\usepackage{microtype}


\title{\mytitle}
\author{Richard Foltyn}
\hypersetup{pdfauthor={Richard Foltyn}, pdftitle={\mytitle}}

% Remove horizontal rules in a very hackish way
\renewcommand{\rule}[2]{}

\RequirePackage{xspace}


\newcommand*{\eg}{e.g.\@\xspace}
\newcommand*{\Eg}{E.g.\@\xspace}
\newcommand*{\etc}{etc.\@\xspace}
\newcommand*{\ie}{i.e.\@\xspace}
\newcommand*{\vs}{vs.\@\xspace}
\newcommand*{\viz}{viz.\@\xspace}
\newcommand*{\US}{U.S.\@\xspace}

    
    
    
    
    
% Pygments definitions
\makeatletter
\def\PY@reset{\let\PY@it=\relax \let\PY@bf=\relax%
    \let\PY@ul=\relax \let\PY@tc=\relax%
    \let\PY@bc=\relax \let\PY@ff=\relax}
\def\PY@tok#1{\csname PY@tok@#1\endcsname}
\def\PY@toks#1+{\ifx\relax#1\empty\else%
    \PY@tok{#1}\expandafter\PY@toks\fi}
\def\PY@do#1{\PY@bc{\PY@tc{\PY@ul{%
    \PY@it{\PY@bf{\PY@ff{#1}}}}}}}
\def\PY#1#2{\PY@reset\PY@toks#1+\relax+\PY@do{#2}}

\expandafter\def\csname PY@tok@w\endcsname{\def\PY@tc##1{\textcolor[rgb]{0.73,0.73,0.73}{##1}}}
\expandafter\def\csname PY@tok@c\endcsname{\let\PY@it=\textit\def\PY@tc##1{\textcolor[rgb]{0.25,0.50,0.50}{##1}}}
\expandafter\def\csname PY@tok@cp\endcsname{\def\PY@tc##1{\textcolor[rgb]{0.74,0.48,0.00}{##1}}}
\expandafter\def\csname PY@tok@k\endcsname{\let\PY@bf=\textbf\def\PY@tc##1{\textcolor[rgb]{0.00,0.50,0.00}{##1}}}
\expandafter\def\csname PY@tok@kp\endcsname{\def\PY@tc##1{\textcolor[rgb]{0.00,0.50,0.00}{##1}}}
\expandafter\def\csname PY@tok@kt\endcsname{\def\PY@tc##1{\textcolor[rgb]{0.69,0.00,0.25}{##1}}}
\expandafter\def\csname PY@tok@o\endcsname{\def\PY@tc##1{\textcolor[rgb]{0.40,0.40,0.40}{##1}}}
\expandafter\def\csname PY@tok@ow\endcsname{\let\PY@bf=\textbf\def\PY@tc##1{\textcolor[rgb]{0.67,0.13,1.00}{##1}}}
\expandafter\def\csname PY@tok@nb\endcsname{\def\PY@tc##1{\textcolor[rgb]{0.00,0.50,0.00}{##1}}}
\expandafter\def\csname PY@tok@nf\endcsname{\def\PY@tc##1{\textcolor[rgb]{0.00,0.00,1.00}{##1}}}
\expandafter\def\csname PY@tok@nc\endcsname{\let\PY@bf=\textbf\def\PY@tc##1{\textcolor[rgb]{0.00,0.00,1.00}{##1}}}
\expandafter\def\csname PY@tok@nn\endcsname{\let\PY@bf=\textbf\def\PY@tc##1{\textcolor[rgb]{0.00,0.00,1.00}{##1}}}
\expandafter\def\csname PY@tok@ne\endcsname{\let\PY@bf=\textbf\def\PY@tc##1{\textcolor[rgb]{0.82,0.25,0.23}{##1}}}
\expandafter\def\csname PY@tok@nv\endcsname{\def\PY@tc##1{\textcolor[rgb]{0.10,0.09,0.49}{##1}}}
\expandafter\def\csname PY@tok@no\endcsname{\def\PY@tc##1{\textcolor[rgb]{0.53,0.00,0.00}{##1}}}
\expandafter\def\csname PY@tok@nl\endcsname{\def\PY@tc##1{\textcolor[rgb]{0.63,0.63,0.00}{##1}}}
\expandafter\def\csname PY@tok@ni\endcsname{\let\PY@bf=\textbf\def\PY@tc##1{\textcolor[rgb]{0.60,0.60,0.60}{##1}}}
\expandafter\def\csname PY@tok@na\endcsname{\def\PY@tc##1{\textcolor[rgb]{0.49,0.56,0.16}{##1}}}
\expandafter\def\csname PY@tok@nt\endcsname{\let\PY@bf=\textbf\def\PY@tc##1{\textcolor[rgb]{0.00,0.50,0.00}{##1}}}
\expandafter\def\csname PY@tok@nd\endcsname{\def\PY@tc##1{\textcolor[rgb]{0.67,0.13,1.00}{##1}}}
\expandafter\def\csname PY@tok@s\endcsname{\def\PY@tc##1{\textcolor[rgb]{0.73,0.13,0.13}{##1}}}
\expandafter\def\csname PY@tok@sd\endcsname{\let\PY@it=\textit\def\PY@tc##1{\textcolor[rgb]{0.73,0.13,0.13}{##1}}}
\expandafter\def\csname PY@tok@si\endcsname{\let\PY@bf=\textbf\def\PY@tc##1{\textcolor[rgb]{0.73,0.40,0.53}{##1}}}
\expandafter\def\csname PY@tok@se\endcsname{\let\PY@bf=\textbf\def\PY@tc##1{\textcolor[rgb]{0.73,0.40,0.13}{##1}}}
\expandafter\def\csname PY@tok@sr\endcsname{\def\PY@tc##1{\textcolor[rgb]{0.73,0.40,0.53}{##1}}}
\expandafter\def\csname PY@tok@ss\endcsname{\def\PY@tc##1{\textcolor[rgb]{0.10,0.09,0.49}{##1}}}
\expandafter\def\csname PY@tok@sx\endcsname{\def\PY@tc##1{\textcolor[rgb]{0.00,0.50,0.00}{##1}}}
\expandafter\def\csname PY@tok@m\endcsname{\def\PY@tc##1{\textcolor[rgb]{0.40,0.40,0.40}{##1}}}
\expandafter\def\csname PY@tok@gh\endcsname{\let\PY@bf=\textbf\def\PY@tc##1{\textcolor[rgb]{0.00,0.00,0.50}{##1}}}
\expandafter\def\csname PY@tok@gu\endcsname{\let\PY@bf=\textbf\def\PY@tc##1{\textcolor[rgb]{0.50,0.00,0.50}{##1}}}
\expandafter\def\csname PY@tok@gd\endcsname{\def\PY@tc##1{\textcolor[rgb]{0.63,0.00,0.00}{##1}}}
\expandafter\def\csname PY@tok@gi\endcsname{\def\PY@tc##1{\textcolor[rgb]{0.00,0.63,0.00}{##1}}}
\expandafter\def\csname PY@tok@gr\endcsname{\def\PY@tc##1{\textcolor[rgb]{1.00,0.00,0.00}{##1}}}
\expandafter\def\csname PY@tok@ge\endcsname{\let\PY@it=\textit}
\expandafter\def\csname PY@tok@gs\endcsname{\let\PY@bf=\textbf}
\expandafter\def\csname PY@tok@gp\endcsname{\let\PY@bf=\textbf\def\PY@tc##1{\textcolor[rgb]{0.00,0.00,0.50}{##1}}}
\expandafter\def\csname PY@tok@go\endcsname{\def\PY@tc##1{\textcolor[rgb]{0.53,0.53,0.53}{##1}}}
\expandafter\def\csname PY@tok@gt\endcsname{\def\PY@tc##1{\textcolor[rgb]{0.00,0.27,0.87}{##1}}}
\expandafter\def\csname PY@tok@err\endcsname{\def\PY@bc##1{\setlength{\fboxsep}{0pt}\fcolorbox[rgb]{1.00,0.00,0.00}{1,1,1}{\strut ##1}}}
\expandafter\def\csname PY@tok@kc\endcsname{\let\PY@bf=\textbf\def\PY@tc##1{\textcolor[rgb]{0.00,0.50,0.00}{##1}}}
\expandafter\def\csname PY@tok@kd\endcsname{\let\PY@bf=\textbf\def\PY@tc##1{\textcolor[rgb]{0.00,0.50,0.00}{##1}}}
\expandafter\def\csname PY@tok@kn\endcsname{\let\PY@bf=\textbf\def\PY@tc##1{\textcolor[rgb]{0.00,0.50,0.00}{##1}}}
\expandafter\def\csname PY@tok@kr\endcsname{\let\PY@bf=\textbf\def\PY@tc##1{\textcolor[rgb]{0.00,0.50,0.00}{##1}}}
\expandafter\def\csname PY@tok@bp\endcsname{\def\PY@tc##1{\textcolor[rgb]{0.00,0.50,0.00}{##1}}}
\expandafter\def\csname PY@tok@fm\endcsname{\def\PY@tc##1{\textcolor[rgb]{0.00,0.00,1.00}{##1}}}
\expandafter\def\csname PY@tok@vc\endcsname{\def\PY@tc##1{\textcolor[rgb]{0.10,0.09,0.49}{##1}}}
\expandafter\def\csname PY@tok@vg\endcsname{\def\PY@tc##1{\textcolor[rgb]{0.10,0.09,0.49}{##1}}}
\expandafter\def\csname PY@tok@vi\endcsname{\def\PY@tc##1{\textcolor[rgb]{0.10,0.09,0.49}{##1}}}
\expandafter\def\csname PY@tok@vm\endcsname{\def\PY@tc##1{\textcolor[rgb]{0.10,0.09,0.49}{##1}}}
\expandafter\def\csname PY@tok@sa\endcsname{\def\PY@tc##1{\textcolor[rgb]{0.73,0.13,0.13}{##1}}}
\expandafter\def\csname PY@tok@sb\endcsname{\def\PY@tc##1{\textcolor[rgb]{0.73,0.13,0.13}{##1}}}
\expandafter\def\csname PY@tok@sc\endcsname{\def\PY@tc##1{\textcolor[rgb]{0.73,0.13,0.13}{##1}}}
\expandafter\def\csname PY@tok@dl\endcsname{\def\PY@tc##1{\textcolor[rgb]{0.73,0.13,0.13}{##1}}}
\expandafter\def\csname PY@tok@s2\endcsname{\def\PY@tc##1{\textcolor[rgb]{0.73,0.13,0.13}{##1}}}
\expandafter\def\csname PY@tok@sh\endcsname{\def\PY@tc##1{\textcolor[rgb]{0.73,0.13,0.13}{##1}}}
\expandafter\def\csname PY@tok@s1\endcsname{\def\PY@tc##1{\textcolor[rgb]{0.73,0.13,0.13}{##1}}}
\expandafter\def\csname PY@tok@mb\endcsname{\def\PY@tc##1{\textcolor[rgb]{0.40,0.40,0.40}{##1}}}
\expandafter\def\csname PY@tok@mf\endcsname{\def\PY@tc##1{\textcolor[rgb]{0.40,0.40,0.40}{##1}}}
\expandafter\def\csname PY@tok@mh\endcsname{\def\PY@tc##1{\textcolor[rgb]{0.40,0.40,0.40}{##1}}}
\expandafter\def\csname PY@tok@mi\endcsname{\def\PY@tc##1{\textcolor[rgb]{0.40,0.40,0.40}{##1}}}
\expandafter\def\csname PY@tok@il\endcsname{\def\PY@tc##1{\textcolor[rgb]{0.40,0.40,0.40}{##1}}}
\expandafter\def\csname PY@tok@mo\endcsname{\def\PY@tc##1{\textcolor[rgb]{0.40,0.40,0.40}{##1}}}
\expandafter\def\csname PY@tok@ch\endcsname{\let\PY@it=\textit\def\PY@tc##1{\textcolor[rgb]{0.25,0.50,0.50}{##1}}}
\expandafter\def\csname PY@tok@cm\endcsname{\let\PY@it=\textit\def\PY@tc##1{\textcolor[rgb]{0.25,0.50,0.50}{##1}}}
\expandafter\def\csname PY@tok@cpf\endcsname{\let\PY@it=\textit\def\PY@tc##1{\textcolor[rgb]{0.25,0.50,0.50}{##1}}}
\expandafter\def\csname PY@tok@c1\endcsname{\let\PY@it=\textit\def\PY@tc##1{\textcolor[rgb]{0.25,0.50,0.50}{##1}}}
\expandafter\def\csname PY@tok@cs\endcsname{\let\PY@it=\textit\def\PY@tc##1{\textcolor[rgb]{0.25,0.50,0.50}{##1}}}

\def\PYZbs{\char`\\}
\def\PYZus{\char`\_}
\def\PYZob{\char`\{}
\def\PYZcb{\char`\}}
\def\PYZca{\char`\^}
\def\PYZam{\char`\&}
\def\PYZlt{\char`\<}
\def\PYZgt{\char`\>}
\def\PYZsh{\char`\#}
\def\PYZpc{\char`\%}
\def\PYZdl{\char`\$}
\def\PYZhy{\char`\-}
\def\PYZsq{\char`\'}
\def\PYZdq{\char`\"}
\def\PYZti{\char`\~}
% for compatibility with earlier versions
\def\PYZat{@}
\def\PYZlb{[}
\def\PYZrb{]}
\makeatother


    % For linebreaks inside Verbatim environment from package fancyvrb. 
    \makeatletter
        \newbox\Wrappedcontinuationbox 
        \newbox\Wrappedvisiblespacebox 
        \newcommand*\Wrappedvisiblespace {\textcolor{red}{\textvisiblespace}} 
        \newcommand*\Wrappedcontinuationsymbol {\textcolor{red}{\llap{\tiny$\m@th\hookrightarrow$}}} 
        \newcommand*\Wrappedcontinuationindent {3ex } 
        \newcommand*\Wrappedafterbreak {\kern\Wrappedcontinuationindent\copy\Wrappedcontinuationbox} 
        % Take advantage of the already applied Pygments mark-up to insert 
        % potential linebreaks for TeX processing. 
        %        {, <, #, %, $, ' and ": go to next line. 
        %        _, }, ^, &, >, - and ~: stay at end of broken line. 
        % Use of \textquotesingle for straight quote. 
        \newcommand*\Wrappedbreaksatspecials {% 
            \def\PYGZus{\discretionary{\char`\_}{\Wrappedafterbreak}{\char`\_}}% 
            \def\PYGZob{\discretionary{}{\Wrappedafterbreak\char`\{}{\char`\{}}% 
            \def\PYGZcb{\discretionary{\char`\}}{\Wrappedafterbreak}{\char`\}}}% 
            \def\PYGZca{\discretionary{\char`\^}{\Wrappedafterbreak}{\char`\^}}% 
            \def\PYGZam{\discretionary{\char`\&}{\Wrappedafterbreak}{\char`\&}}% 
            \def\PYGZlt{\discretionary{}{\Wrappedafterbreak\char`\<}{\char`\<}}% 
            \def\PYGZgt{\discretionary{\char`\>}{\Wrappedafterbreak}{\char`\>}}% 
            \def\PYGZsh{\discretionary{}{\Wrappedafterbreak\char`\#}{\char`\#}}% 
            \def\PYGZpc{\discretionary{}{\Wrappedafterbreak\char`\%}{\char`\%}}% 
            \def\PYGZdl{\discretionary{}{\Wrappedafterbreak\char`\$}{\char`\$}}% 
            \def\PYGZhy{\discretionary{\char`\-}{\Wrappedafterbreak}{\char`\-}}% 
            \def\PYGZsq{\discretionary{}{\Wrappedafterbreak\textquotesingle}{\textquotesingle}}% 
            \def\PYGZdq{\discretionary{}{\Wrappedafterbreak\char`\"}{\char`\"}}% 
            \def\PYGZti{\discretionary{\char`\~}{\Wrappedafterbreak}{\char`\~}}% 
        } 
        % Some characters . , ; ? ! / are not pygmentized. 
        % This macro makes them "active" and they will insert potential linebreaks 
        \newcommand*\Wrappedbreaksatpunct {% 
            \lccode`\~`\.\lowercase{\def~}{\discretionary{\hbox{\char`\.}}{\Wrappedafterbreak}{\hbox{\char`\.}}}% 
            \lccode`\~`\,\lowercase{\def~}{\discretionary{\hbox{\char`\,}}{\Wrappedafterbreak}{\hbox{\char`\,}}}% 
            \lccode`\~`\;\lowercase{\def~}{\discretionary{\hbox{\char`\;}}{\Wrappedafterbreak}{\hbox{\char`\;}}}% 
            \lccode`\~`\:\lowercase{\def~}{\discretionary{\hbox{\char`\:}}{\Wrappedafterbreak}{\hbox{\char`\:}}}% 
            \lccode`\~`\?\lowercase{\def~}{\discretionary{\hbox{\char`\?}}{\Wrappedafterbreak}{\hbox{\char`\?}}}% 
            \lccode`\~`\!\lowercase{\def~}{\discretionary{\hbox{\char`\!}}{\Wrappedafterbreak}{\hbox{\char`\!}}}% 
            \lccode`\~`\/\lowercase{\def~}{\discretionary{\hbox{\char`\/}}{\Wrappedafterbreak}{\hbox{\char`\/}}}% 
            \catcode`\.\active
            \catcode`\,\active 
            \catcode`\;\active
            \catcode`\:\active
            \catcode`\?\active
            \catcode`\!\active
            \catcode`\/\active 
            \lccode`\~`\~ 	
        }
    \makeatother

    \let\OriginalVerbatim=\Verbatim
    \makeatletter
    \renewcommand{\Verbatim}[1][1]{%
        %\parskip\z@skip
        \sbox\Wrappedcontinuationbox {\Wrappedcontinuationsymbol}%
        \sbox\Wrappedvisiblespacebox {\FV@SetupFont\Wrappedvisiblespace}%
        \def\FancyVerbFormatLine ##1{\hsize\linewidth
            \vtop{\raggedright\hyphenpenalty\z@\exhyphenpenalty\z@
                \doublehyphendemerits\z@\finalhyphendemerits\z@
                \strut ##1\strut}%
        }%
        % If the linebreak is at a space, the latter will be displayed as visible
        % space at end of first line, and a continuation symbol starts next line.
        % Stretch/shrink are however usually zero for typewriter font.
        \def\FV@Space {%
            \nobreak\hskip\z@ plus\fontdimen3\font minus\fontdimen4\font
            \discretionary{\copy\Wrappedvisiblespacebox}{\Wrappedafterbreak}
            {\kern\fontdimen2\font}%
        }%
        
        % Allow breaks at special characters using \PYG... macros.
        \Wrappedbreaksatspecials
        % Breaks at punctuation characters . , ; ? ! and / need catcode=\active 	
        \OriginalVerbatim[#1,fontsize=\small,codes*=\Wrappedbreaksatpunct]%
    }
    \makeatother

    % Exact colors from NB
    \definecolor{incolor}{HTML}{303F9F}
    \definecolor{outcolor}{HTML}{D84315}
    \definecolor{cellborder}{HTML}{CFCFCF}
    \definecolor{cellbackground}{HTML}{F7F7F7}
    
    % prompt
    \makeatletter
    \newcommand{\boxspacing}{\kern\kvtcb@left@rule\kern\kvtcb@boxsep}
    \makeatother
    \newcommand{\prompt}[4]{
        {\ttfamily\llap{{\color{#2}[#3]:\hspace{3pt}#4}}\vspace{-\baselineskip}}
    }
    

    
    % Prevent overflowing lines due to hard-to-break entities
    \sloppy 
    % Setup hyperref package
    \hypersetup{
      breaklinks=true,  % so long urls are correctly broken across lines
      colorlinks=true,
      urlcolor=urlcolor,
      linkcolor=linkcolor,
      citecolor=citecolor,
      }
    % Slightly bigger margins than the latex defaults
    
    \geometry{verbose,tmargin=1in,bmargin=1in,lmargin=1in,rmargin=1in}
    
    

\begin{document}
    
    \maketitle
    \tableofcontents

    

    
    \hypertarget{random-number-generation-and-statistics}{%
\section{Random number generation and
statistics}\label{random-number-generation-and-statistics}}

In this unit, we examine how to generate random numbers for various
probability distributions in NumPy. Additionally, we take a look at
SciPy's \texttt{stats} package which implements PDFs and other functions
for numerous probability distributions.

\hypertarget{random-number-generators}{%
\subsection{Random number generators}\label{random-number-generators}}

Currently, there are several ways to draw random numbers in Python:

\begin{enumerate}
\def\labelenumi{\arabic{enumi}.}
\item
  The \emph{new} programming interface implemented in NumPy, introduced
  in version 1.17 (the current version as of this writing is 1.19)
  {[}\href{https://numpy.org/doc/stable/reference/random/generator.html}{official
  documentation}{]}.
\item
  The \emph{legacy} programming interface implemented in NumPy
  {[}\href{https://numpy.org/doc/stable/reference/random/legacy.html}{official
  documentation}{]}.

  While these functions have been superseded by the new implementation,
  they continue to work. If you are familiar with the legacy interface,
  you can read about what has changed in the new interface
  \href{https://numpy.org/doc/stable/reference/random/new-or-different.html}{here}.
\item
  The Python standard library itself also includes random number
  generators in the \texttt{random} module
  {[}\href{https://docs.python.org/dev/library/random.html\#random.random}{official
  documentation}{]}.

  We won't be using this implementation at all, since for our purposes
  \texttt{numpy.random} is preferable as it supports NumPy arrays.
\end{enumerate}

The programming interface for generating random numbers in NumPy changed
substantially in release 1.17. We discuss the new interface in this unit
since it offers several advantages, including faster algorithms for some
distributions. Moreover, one would expect the legacy interface to be
removed at some point in the future. However, most examples you will
find in textbooks and on the internet are likely to use the old variant.

\textbf{A note on random-number generation}

Computers usually cannot draw truly random numbers, so we often talk
about \emph{pseudo-random number generators} (PRNG). Given an initial
seed, these PRNGs will always produce the same sequence of ``random''
numbers, at least if run on the same machine, using the same underlying
algorithm, etc. For scientific purposes this is actually desirable as it
allows us to create reproducible results. For simplicity, we will
nevertheless be using the terms ``random number'' and ``random number
generator'' (RNG), omitting the ``pseudo'' prefix.

\hypertarget{simple-random-data-generation}{%
\subsubsection{Simple random data
generation}\label{simple-random-data-generation}}

Let's begin with the most simple case, which uses the \texttt{random()}
function to draw numbers that are uniformly distributed on the half-open
interval \([0.0, 1.0)\).

    \begin{tcolorbox}[breakable, size=fbox, boxrule=1pt, pad at break*=1mm,colback=cellbackground, colframe=cellborder]
\prompt{In}{incolor}{1}{\boxspacing}
\begin{Verbatim}[commandchars=\\\{\}]
\PY{k+kn}{from} \PY{n+nn}{numpy}\PY{n+nn}{.}\PY{n+nn}{random} \PY{k+kn}{import} \PY{n}{default\PYZus{}rng}
\PY{n}{rng} \PY{o}{=} \PY{n}{default\PYZus{}rng}\PY{p}{(}\PY{p}{)}         \PY{c+c1}{\PYZsh{} obtain default RNG implementation}
\PY{n}{rng}\PY{o}{.}\PY{n}{random}\PY{p}{(}\PY{l+m+mi}{5}\PY{p}{)}               \PY{c+c1}{\PYZsh{} return array of 5 random numbers}
\end{Verbatim}
\end{tcolorbox}

            \begin{tcolorbox}[breakable, size=fbox, boxrule=.5pt, pad at break*=1mm, opacityfill=0]
\prompt{Out}{outcolor}{1}{\boxspacing}
\begin{Verbatim}[commandchars=\\\{\}]
array([0.98442984, 0.69473815, 0.14118774, 0.84393217, 0.65855017])
\end{Verbatim}
\end{tcolorbox}
        
    Calling \texttt{random()} this way will return a different set of
numbers each time (this might, for example, depend on the system time).
To obtain the same draw each time, we can pass an initial \emph{seed}
when creating an instance of the RNG like this:

    \begin{tcolorbox}[breakable, size=fbox, boxrule=1pt, pad at break*=1mm,colback=cellbackground, colframe=cellborder]
\prompt{In}{incolor}{2}{\boxspacing}
\begin{Verbatim}[commandchars=\\\{\}]
\PY{n}{seed} \PY{o}{=} \PY{l+m+mi}{123}
\PY{n}{rng} \PY{o}{=} \PY{n}{default\PYZus{}rng}\PY{p}{(}\PY{n}{seed}\PY{p}{)}     \PY{c+c1}{\PYZsh{} obtain default RNG implementation,}
                            \PY{c+c1}{\PYZsh{} initialise seed}
\PY{n}{rng}\PY{o}{.}\PY{n}{random}\PY{p}{(}\PY{l+m+mi}{5}\PY{p}{)}               \PY{c+c1}{\PYZsh{} return array of 5 random numbers}
\end{Verbatim}
\end{tcolorbox}

            \begin{tcolorbox}[breakable, size=fbox, boxrule=.5pt, pad at break*=1mm, opacityfill=0]
\prompt{Out}{outcolor}{2}{\boxspacing}
\begin{Verbatim}[commandchars=\\\{\}]
array([0.68235186, 0.05382102, 0.22035987, 0.18437181, 0.1759059 ])
\end{Verbatim}
\end{tcolorbox}
        
    The \texttt{seed} argument needs to be an integer or an array of
integers. This way, each call gives the same numbers, as can easily be
illustrated with a loop:

    \begin{tcolorbox}[breakable, size=fbox, boxrule=1pt, pad at break*=1mm,colback=cellbackground, colframe=cellborder]
\prompt{In}{incolor}{3}{\boxspacing}
\begin{Verbatim}[commandchars=\\\{\}]
\PY{n}{seed} \PY{o}{=} \PY{l+m+mi}{123}
\PY{k}{for} \PY{n}{i} \PY{o+ow}{in} \PY{n+nb}{range}\PY{p}{(}\PY{l+m+mi}{5}\PY{p}{)}\PY{p}{:}
    \PY{n}{rng} \PY{o}{=} \PY{n}{default\PYZus{}rng}\PY{p}{(}\PY{n}{seed}\PY{p}{)}
    \PY{n+nb}{print}\PY{p}{(}\PY{n}{rng}\PY{o}{.}\PY{n}{random}\PY{p}{(}\PY{l+m+mi}{5}\PY{p}{)}\PY{p}{)}
\end{Verbatim}
\end{tcolorbox}

    \begin{Verbatim}[commandchars=\\\{\}]
[0.68235186 0.05382102 0.22035987 0.18437181 0.1759059 ]
[0.68235186 0.05382102 0.22035987 0.18437181 0.1759059 ]
[0.68235186 0.05382102 0.22035987 0.18437181 0.1759059 ]
[0.68235186 0.05382102 0.22035987 0.18437181 0.1759059 ]
[0.68235186 0.05382102 0.22035987 0.18437181 0.1759059 ]
    \end{Verbatim}

    You can remove the \texttt{seed} to verify that the set of number will
then differ in each iteration.

Alternatively, we might want to draw random integers by calling
\texttt{integers()}, which returns numbers from a ``discrete uniform''
distribution on a given interval
{[}\href{https://numpy.org/doc/stable/reference/random/generated/numpy.random.Generator.integers.html}{documentation}{]}:

    \begin{tcolorbox}[breakable, size=fbox, boxrule=1pt, pad at break*=1mm,colback=cellbackground, colframe=cellborder]
\prompt{In}{incolor}{4}{\boxspacing}
\begin{Verbatim}[commandchars=\\\{\}]
\PY{n}{rng}\PY{o}{.}\PY{n}{integers}\PY{p}{(}\PY{l+m+mi}{2}\PY{p}{,} \PY{n}{size}\PY{o}{=}\PY{l+m+mi}{5}\PY{p}{)}         \PY{c+c1}{\PYZsh{} vector of 5 integers from set \PYZob{}0, 1\PYZcb{}}
                                \PY{c+c1}{\PYZsh{} here we specify only the (non\PYZhy{}inclusive)}
                                \PY{c+c1}{\PYZsh{} upper bound 2}
\end{Verbatim}
\end{tcolorbox}

            \begin{tcolorbox}[breakable, size=fbox, boxrule=.5pt, pad at break*=1mm, opacityfill=0]
\prompt{Out}{outcolor}{4}{\boxspacing}
\begin{Verbatim}[commandchars=\\\{\}]
array([0, 1, 0, 1, 0])
\end{Verbatim}
\end{tcolorbox}
        
    \begin{tcolorbox}[breakable, size=fbox, boxrule=1pt, pad at break*=1mm,colback=cellbackground, colframe=cellborder]
\prompt{In}{incolor}{5}{\boxspacing}
\begin{Verbatim}[commandchars=\\\{\}]
\PY{n}{rng}\PY{o}{.}\PY{n}{integers}\PY{p}{(}\PY{l+m+mi}{1}\PY{p}{,} \PY{l+m+mi}{10}\PY{p}{,} \PY{n}{size}\PY{o}{=}\PY{l+m+mi}{5}\PY{p}{)}     \PY{c+c1}{\PYZsh{} specify lower and upper bound}
\end{Verbatim}
\end{tcolorbox}

            \begin{tcolorbox}[breakable, size=fbox, boxrule=.5pt, pad at break*=1mm, opacityfill=0]
\prompt{Out}{outcolor}{5}{\boxspacing}
\begin{Verbatim}[commandchars=\\\{\}]
array([3, 8, 8, 8, 9])
\end{Verbatim}
\end{tcolorbox}
        
    \begin{tcolorbox}[breakable, size=fbox, boxrule=1pt, pad at break*=1mm,colback=cellbackground, colframe=cellborder]
\prompt{In}{incolor}{6}{\boxspacing}
\begin{Verbatim}[commandchars=\\\{\}]
\PY{n}{rng}\PY{o}{.}\PY{n}{integers}\PY{p}{(}\PY{l+m+mi}{1}\PY{p}{,} \PY{l+m+mi}{10}\PY{p}{,} \PY{n}{size}\PY{o}{=}\PY{l+m+mi}{5}\PY{p}{,} \PY{n}{endpoint}\PY{o}{=}\PY{k+kc}{True}\PY{p}{)}      \PY{c+c1}{\PYZsh{} include upper bound}
\end{Verbatim}
\end{tcolorbox}

            \begin{tcolorbox}[breakable, size=fbox, boxrule=.5pt, pad at break*=1mm, opacityfill=0]
\prompt{Out}{outcolor}{6}{\boxspacing}
\begin{Verbatim}[commandchars=\\\{\}]
array([1, 6, 3, 3, 3])
\end{Verbatim}
\end{tcolorbox}
        
    We can create higher-order arrays by passing a list or tuple as the
\texttt{size} argument:

    \begin{tcolorbox}[breakable, size=fbox, boxrule=1pt, pad at break*=1mm,colback=cellbackground, colframe=cellborder]
\prompt{In}{incolor}{7}{\boxspacing}
\begin{Verbatim}[commandchars=\\\{\}]
\PY{n}{rng}\PY{o}{.}\PY{n}{random}\PY{p}{(}\PY{n}{size}\PY{o}{=}\PY{p}{[}\PY{l+m+mi}{2}\PY{p}{,} \PY{l+m+mi}{5}\PY{p}{]}\PY{p}{)}             \PY{c+c1}{\PYZsh{} Create 2x5 array of floats}
                                    \PY{c+c1}{\PYZsh{} on [0.0, 1.0)}
\end{Verbatim}
\end{tcolorbox}

            \begin{tcolorbox}[breakable, size=fbox, boxrule=.5pt, pad at break*=1mm, opacityfill=0]
\prompt{Out}{outcolor}{7}{\boxspacing}
\begin{Verbatim}[commandchars=\\\{\}]
array([[0.21376296, 0.74146705, 0.6299402 , 0.92740726, 0.23190819],
       [0.79912513, 0.51816504, 0.23155562, 0.16590399, 0.49778897]])
\end{Verbatim}
\end{tcolorbox}
        
    \begin{tcolorbox}[breakable, size=fbox, boxrule=1pt, pad at break*=1mm,colback=cellbackground, colframe=cellborder]
\prompt{In}{incolor}{8}{\boxspacing}
\begin{Verbatim}[commandchars=\\\{\}]
\PY{n}{rng}\PY{o}{.}\PY{n}{integers}\PY{p}{(}\PY{l+m+mi}{2}\PY{p}{,} \PY{n}{size}\PY{o}{=}\PY{p}{[}\PY{l+m+mi}{2}\PY{p}{,}\PY{l+m+mi}{3}\PY{p}{,}\PY{l+m+mi}{4}\PY{p}{]}\PY{p}{)}       \PY{c+c1}{\PYZsh{} Create 2x3x4 array of integers \PYZob{}0,1\PYZcb{}}
\end{Verbatim}
\end{tcolorbox}

            \begin{tcolorbox}[breakable, size=fbox, boxrule=.5pt, pad at break*=1mm, opacityfill=0]
\prompt{Out}{outcolor}{8}{\boxspacing}
\begin{Verbatim}[commandchars=\\\{\}]
array([[[1, 0, 1, 0],
        [0, 0, 0, 0],
        [0, 0, 1, 0]],

       [[1, 0, 1, 1],
        [1, 1, 1, 0],
        [0, 0, 1, 0]]])
\end{Verbatim}
\end{tcolorbox}
        
    \textbf{Legacy interface}

For completeness, let's look how you would accomplish the same using the
\emph{legacy} NumPy interface:

    \begin{tcolorbox}[breakable, size=fbox, boxrule=1pt, pad at break*=1mm,colback=cellbackground, colframe=cellborder]
\prompt{In}{incolor}{9}{\boxspacing}
\begin{Verbatim}[commandchars=\\\{\}]
\PY{k+kn}{from} \PY{n+nn}{numpy}\PY{n+nn}{.}\PY{n+nn}{random} \PY{k+kn}{import} \PY{n}{random\PYZus{}sample}\PY{p}{,} \PY{n}{randint}\PY{p}{,} \PY{n}{seed}
\PY{n}{seed}\PY{p}{(}\PY{l+m+mi}{123}\PY{p}{)}
\PY{n}{random\PYZus{}sample}\PY{p}{(}\PY{l+m+mi}{5}\PY{p}{)}
\end{Verbatim}
\end{tcolorbox}

            \begin{tcolorbox}[breakable, size=fbox, boxrule=.5pt, pad at break*=1mm, opacityfill=0]
\prompt{Out}{outcolor}{9}{\boxspacing}
\begin{Verbatim}[commandchars=\\\{\}]
array([0.69646919, 0.28613933, 0.22685145, 0.55131477, 0.71946897])
\end{Verbatim}
\end{tcolorbox}
        
    \begin{tcolorbox}[breakable, size=fbox, boxrule=1pt, pad at break*=1mm,colback=cellbackground, colframe=cellborder]
\prompt{In}{incolor}{10}{\boxspacing}
\begin{Verbatim}[commandchars=\\\{\}]
\PY{n}{randint}\PY{p}{(}\PY{l+m+mi}{2}\PY{p}{,} \PY{n}{size}\PY{o}{=}\PY{l+m+mi}{5}\PY{p}{)}      \PY{c+c1}{\PYZsh{} draw random integers from \PYZob{}0,1\PYZcb{}}
\end{Verbatim}
\end{tcolorbox}

            \begin{tcolorbox}[breakable, size=fbox, boxrule=.5pt, pad at break*=1mm, opacityfill=0]
\prompt{Out}{outcolor}{10}{\boxspacing}
\begin{Verbatim}[commandchars=\\\{\}]
array([1, 1, 0, 1, 0])
\end{Verbatim}
\end{tcolorbox}
        
    The legacy interface defines global functions \texttt{seed},
\texttt{random\_sample}, etc. within the \texttt{numpy.random} module,
which are implicitly associated with a global RNG object. This implicit
association has been removed in the new programming model and you now
have to obtain an RNG instance explicitly, for example by using the
\texttt{default\_rng()} function, as demonstrated above.

    \hypertarget{drawing-random-numbers-from-distributions}{%
\subsubsection{Drawing random numbers from
distributions}\label{drawing-random-numbers-from-distributions}}

Often we want to draw random numbers from a specific distribution, such
as the normal or log-normal distributions. The RNGs in
\texttt{numpy.random} support a multitude of distributions, including:

\begin{itemize}
\tightlist
\item
  \texttt{binomial()}
\item
  \texttt{exponential()}
\item
  \texttt{normal()}
\item
  \texttt{lognormal()}
\item
  \texttt{multivariate\_normal()}
\item
  \texttt{uniform()}
\end{itemize}

and many others. For a complete list, see the
\href{https://numpy.org/doc/stable/reference/random/generator.html\#distributions}{official
documentation}.

We will illustrate the use of these functions for the normal and
multivariate normal distributions. For example, you can draw from the
normal distribution with mean \(\mu=1.0\) and standard deviation
\(\sigma=0.5\) as follows:

    \begin{tcolorbox}[breakable, size=fbox, boxrule=1pt, pad at break*=1mm,colback=cellbackground, colframe=cellborder]
\prompt{In}{incolor}{11}{\boxspacing}
\begin{Verbatim}[commandchars=\\\{\}]
\PY{k+kn}{from} \PY{n+nn}{numpy}\PY{n+nn}{.}\PY{n+nn}{random} \PY{k+kn}{import} \PY{n}{default\PYZus{}rng}
\PY{n}{rng} \PY{o}{=} \PY{n}{default\PYZus{}rng}\PY{p}{(}\PY{l+m+mi}{123}\PY{p}{)}

\PY{c+c1}{\PYZsh{} location and scale parameters of normal distribution}
\PY{n}{mu} \PY{o}{=} \PY{l+m+mf}{1.0}
\PY{n}{sigma} \PY{o}{=} \PY{l+m+mf}{0.5}

\PY{c+c1}{\PYZsh{} Draw 10000 normal numbers}
\PY{n}{x} \PY{o}{=} \PY{n}{rng}\PY{o}{.}\PY{n}{normal}\PY{p}{(}\PY{n}{loc}\PY{o}{=}\PY{n}{mu}\PY{p}{,} \PY{n}{scale}\PY{o}{=}\PY{n}{sigma}\PY{p}{,} \PY{n}{size}\PY{o}{=}\PY{l+m+mi}{10000}\PY{p}{)}    \PY{c+c1}{\PYZsh{} mean and std. are passed as}
                                                   \PY{c+c1}{\PYZsh{} loc and scale arguments}

\PY{c+c1}{\PYZsh{} plot the results}
\PY{k+kn}{import} \PY{n+nn}{matplotlib}\PY{n+nn}{.}\PY{n+nn}{pyplot} \PY{k}{as} \PY{n+nn}{plt}
\PY{n}{plt}\PY{o}{.}\PY{n}{hist}\PY{p}{(}\PY{n}{x}\PY{p}{,} \PY{n}{bins}\PY{o}{=}\PY{l+m+mi}{50}\PY{p}{,} \PY{n}{linewidth}\PY{o}{=}\PY{l+m+mf}{0.5}\PY{p}{,} \PY{n}{edgecolor}\PY{o}{=}\PY{l+s+s1}{\PYZsq{}}\PY{l+s+s1}{white}\PY{l+s+s1}{\PYZsq{}}\PY{p}{)}
\PY{n}{plt}\PY{o}{.}\PY{n}{xlabel}\PY{p}{(}\PY{l+s+sa}{r}\PY{l+s+s1}{\PYZsq{}}\PY{l+s+s1}{Realized random number}\PY{l+s+s1}{\PYZsq{}}\PY{p}{)}
\PY{n}{plt}\PY{o}{.}\PY{n}{ylabel}\PY{p}{(}\PY{l+s+s1}{\PYZsq{}}\PY{l+s+s1}{Bin size}\PY{l+s+s1}{\PYZsq{}}\PY{p}{)}
\PY{n}{plt}\PY{o}{.}\PY{n}{title}\PY{p}{(}\PY{l+s+s1}{\PYZsq{}}\PY{l+s+s1}{Histogram of normal draws}\PY{l+s+s1}{\PYZsq{}}\PY{p}{)}
\end{Verbatim}
\end{tcolorbox}

            \begin{tcolorbox}[breakable, size=fbox, boxrule=.5pt, pad at break*=1mm, opacityfill=0]
\prompt{Out}{outcolor}{11}{\boxspacing}
\begin{Verbatim}[commandchars=\\\{\}]
Text(0.5, 1.0, 'Histogram of normal draws')
\end{Verbatim}
\end{tcolorbox}
        
    \begin{center}
    \adjustimage{max size={0.9\linewidth}{0.9\paperheight}}{unit7_files/unit7_18_1.pdf}
    \end{center}
    { \hspace*{\fill} \\}
    
    To draw from the multivariate normal, we need to specify a vector of
means \(\mu\) and the variance-covariance matrix \(\Sigma\), which we
set to

\[\mu = \left[\begin{array}{c} 0 \\ 1\end{array}\right], \qquad \Sigma=\left[\begin{array}{cc}\sigma_1^2 & \rho \sigma_1\sigma_2 \\ \rho\sigma_1\sigma_2 & \sigma_2^2\end{array}\right]\]

with \(\sigma_1 = 0.5\), \(\sigma_2 = 1.0\) and \(\rho = 0.5\).

    \begin{tcolorbox}[breakable, size=fbox, boxrule=1pt, pad at break*=1mm,colback=cellbackground, colframe=cellborder]
\prompt{In}{incolor}{12}{\boxspacing}
\begin{Verbatim}[commandchars=\\\{\}]
\PY{k+kn}{import} \PY{n+nn}{numpy} \PY{k}{as} \PY{n+nn}{np}
\PY{k+kn}{from} \PY{n+nn}{numpy}\PY{n+nn}{.}\PY{n+nn}{random} \PY{k+kn}{import} \PY{n}{default\PYZus{}rng}
\PY{k+kn}{import} \PY{n+nn}{matplotlib}\PY{n+nn}{.}\PY{n+nn}{pyplot} \PY{k}{as} \PY{n+nn}{plt}

\PY{n}{rng} \PY{o}{=} \PY{n}{default\PYZus{}rng}\PY{p}{(}\PY{l+m+mi}{123}\PY{p}{)}

\PY{n}{mu} \PY{o}{=} \PY{n}{np}\PY{o}{.}\PY{n}{array}\PY{p}{(}\PY{p}{(}\PY{l+m+mf}{0.0}\PY{p}{,} \PY{l+m+mf}{1.0}\PY{p}{)}\PY{p}{)}       \PY{c+c1}{\PYZsh{} vector of means}
\PY{n}{sigma1} \PY{o}{=} \PY{l+m+mf}{0.5}                    \PY{c+c1}{\PYZsh{} Std. dev. of first dimension}
\PY{n}{sigma2} \PY{o}{=} \PY{l+m+mf}{1.0}                    \PY{c+c1}{\PYZsh{} Std. dev. of second dimension}
\PY{n}{rho} \PY{o}{=} \PY{l+m+mf}{0.5}                       \PY{c+c1}{\PYZsh{} Correlation coefficient}

\PY{c+c1}{\PYZsh{} Create variance\PYZhy{}covariance matrix}
\PY{n}{vcv} \PY{o}{=} \PY{n}{np}\PY{o}{.}\PY{n}{array}\PY{p}{(}\PY{p}{[}\PY{p}{[}\PY{n}{sigma1}\PY{o}{*}\PY{o}{*}\PY{l+m+mf}{2.0}\PY{p}{,} \PY{n}{rho}\PY{o}{*}\PY{n}{sigma1}\PY{o}{*}\PY{n}{sigma2}\PY{p}{]}\PY{p}{,}
                \PY{p}{[}\PY{n}{rho}\PY{o}{*}\PY{n}{sigma1}\PY{o}{*}\PY{n}{sigma2}\PY{p}{,} \PY{n}{sigma2}\PY{o}{*}\PY{o}{*}\PY{l+m+mf}{2.0}\PY{p}{]}\PY{p}{]}\PY{p}{)}

\PY{c+c1}{\PYZsh{} Draw MVN random numbers}
\PY{n}{x} \PY{o}{=} \PY{n}{rng}\PY{o}{.}\PY{n}{multivariate\PYZus{}normal}\PY{p}{(}\PY{n}{mean}\PY{o}{=}\PY{n}{mu}\PY{p}{,} \PY{n}{cov}\PY{o}{=}\PY{n}{vcv}\PY{p}{,} \PY{n}{size}\PY{o}{=}\PY{l+m+mi}{500}\PY{p}{)}

\PY{c+c1}{\PYZsh{} Scatter plot of sample}
\PY{n}{plt}\PY{o}{.}\PY{n}{scatter}\PY{p}{(}\PY{n}{x}\PY{p}{[}\PY{p}{:}\PY{p}{,} \PY{l+m+mi}{0}\PY{p}{]}\PY{p}{,} \PY{n}{x}\PY{p}{[}\PY{p}{:}\PY{p}{,} \PY{l+m+mi}{1}\PY{p}{]}\PY{p}{)}
\PY{n}{plt}\PY{o}{.}\PY{n}{xlabel}\PY{p}{(}\PY{l+s+sa}{r}\PY{l+s+s1}{\PYZsq{}}\PY{l+s+s1}{\PYZdl{}x\PYZus{}1\PYZdl{}}\PY{l+s+s1}{\PYZsq{}}\PY{p}{)}
\PY{n}{plt}\PY{o}{.}\PY{n}{ylabel}\PY{p}{(}\PY{l+s+sa}{r}\PY{l+s+s1}{\PYZsq{}}\PY{l+s+s1}{\PYZdl{}x\PYZus{}2\PYZdl{}}\PY{l+s+s1}{\PYZsq{}}\PY{p}{)}
\PY{n}{plt}\PY{o}{.}\PY{n}{title}\PY{p}{(}\PY{l+s+s1}{\PYZsq{}}\PY{l+s+s1}{Draws from bivariate normal distribution}\PY{l+s+s1}{\PYZsq{}}\PY{p}{)}
\end{Verbatim}
\end{tcolorbox}

            \begin{tcolorbox}[breakable, size=fbox, boxrule=.5pt, pad at break*=1mm, opacityfill=0]
\prompt{Out}{outcolor}{12}{\boxspacing}
\begin{Verbatim}[commandchars=\\\{\}]
Text(0.5, 1.0, 'Draws from bivariate normal distribution')
\end{Verbatim}
\end{tcolorbox}
        
    \begin{center}
    \adjustimage{max size={0.9\linewidth}{0.9\paperheight}}{unit7_files/unit7_20_1.pdf}
    \end{center}
    { \hspace*{\fill} \\}
    
    \begin{center}\rule{0.5\linewidth}{0.5pt}\end{center}

\hypertarget{more-functions-for-probability-distributions}{%
\subsection{More functions for probability
distributions}\label{more-functions-for-probability-distributions}}

NumPy itself only implements distribution-specific RNGs. Frequently, we
want to evaluate probability density functions (PDFs), cumulative
distribution functions (CDFs) or compute some moments such as the mean
of a random variable following some distribution. The SciPy project
implements these functions for a wide range of discrete and continuous
univariate distributions as well as for a few multivariate ones in the
\texttt{scipy.stats} package.

The most useful functions include:

\begin{itemize}
\tightlist
\item
  \texttt{pdf()}: probability density function
\item
  \texttt{cdf()}: cumulative distribution function
\item
  \texttt{ppf()}: percent point function (inverse of \texttt{cdf})
\item
  \texttt{moment()}: non-central moment of some order \(n\)
\item
  \texttt{expect()}: expected value of a function (of one argument) with
  respect to the distribution
\end{itemize}

The parameters that need to be passed to these functions are
distribution dependent. See the
\href{https://docs.scipy.org/doc/scipy/reference/stats.html}{official
documentation} for details.

\emph{Examples:}

We can overlay the histogram of normal draws with the actual normal PDF
as follows:

    \begin{tcolorbox}[breakable, size=fbox, boxrule=1pt, pad at break*=1mm,colback=cellbackground, colframe=cellborder]
\prompt{In}{incolor}{13}{\boxspacing}
\begin{Verbatim}[commandchars=\\\{\}]
\PY{k+kn}{from} \PY{n+nn}{numpy}\PY{n+nn}{.}\PY{n+nn}{random} \PY{k+kn}{import} \PY{n}{default\PYZus{}rng}
\PY{k+kn}{from} \PY{n+nn}{scipy}\PY{n+nn}{.}\PY{n+nn}{stats} \PY{k+kn}{import} \PY{n}{norm}                \PY{c+c1}{\PYZsh{} import normal distribution}
\PY{k+kn}{import} \PY{n+nn}{matplotlib}\PY{n+nn}{.}\PY{n+nn}{pyplot} \PY{k}{as} \PY{n+nn}{plt}
\PY{n}{rng} \PY{o}{=} \PY{n}{default\PYZus{}rng}\PY{p}{(}\PY{l+m+mi}{123}\PY{p}{)}

\PY{c+c1}{\PYZsh{} location and scale parameters of normal distribution}
\PY{n}{mu} \PY{o}{=} \PY{l+m+mf}{1.0}
\PY{n}{sigma} \PY{o}{=} \PY{l+m+mf}{0.5}

\PY{c+c1}{\PYZsh{} Draw 10000 normal numbers}
\PY{n}{x} \PY{o}{=} \PY{n}{rng}\PY{o}{.}\PY{n}{normal}\PY{p}{(}\PY{n}{loc}\PY{o}{=}\PY{n}{mu}\PY{p}{,} \PY{n}{scale}\PY{o}{=}\PY{n}{sigma}\PY{p}{,} \PY{n}{size}\PY{o}{=}\PY{l+m+mi}{10000}\PY{p}{)}    \PY{c+c1}{\PYZsh{} mean and std. are passed as}
                                                   \PY{c+c1}{\PYZsh{} loc and scale arguments}

\PY{c+c1}{\PYZsh{} plot histogram}
\PY{n}{plt}\PY{o}{.}\PY{n}{hist}\PY{p}{(}\PY{n}{x}\PY{p}{,} \PY{n}{bins}\PY{o}{=}\PY{l+m+mi}{50}\PY{p}{,} \PY{n}{density}\PY{o}{=}\PY{k+kc}{True}\PY{p}{,} \PY{n}{linewidth}\PY{o}{=}\PY{l+m+mf}{0.5}\PY{p}{,} \PY{n}{edgecolor}\PY{o}{=}\PY{l+s+s1}{\PYZsq{}}\PY{l+s+s1}{white}\PY{l+s+s1}{\PYZsq{}}\PY{p}{,}
         \PY{n}{label}\PY{o}{=}\PY{l+s+s1}{\PYZsq{}}\PY{l+s+s1}{Histogram}\PY{l+s+s1}{\PYZsq{}}\PY{p}{)}

\PY{c+c1}{\PYZsh{} Create x\PYZhy{}values for PDF plot, using mean +/\PYZhy{} 3 std.}
\PY{n}{xvalues} \PY{o}{=} \PY{n}{np}\PY{o}{.}\PY{n}{linspace}\PY{p}{(}\PY{n}{mu} \PY{o}{\PYZhy{}} \PY{l+m+mi}{3}\PY{o}{*}\PY{n}{sigma}\PY{p}{,} \PY{n}{mu} \PY{o}{+} \PY{l+m+mi}{3}\PY{o}{*}\PY{n}{sigma}\PY{p}{,} \PY{l+m+mi}{100}\PY{p}{)}
\PY{c+c1}{\PYZsh{} Compute PDF of normal distr. at given x\PYZhy{}values}
\PY{n}{pdf} \PY{o}{=} \PY{n}{norm}\PY{o}{.}\PY{n}{pdf}\PY{p}{(}\PY{n}{xvalues}\PY{p}{,} \PY{n}{loc}\PY{o}{=}\PY{n}{mu}\PY{p}{,} \PY{n}{scale}\PY{o}{=}\PY{n}{sigma}\PY{p}{)}
\PY{c+c1}{\PYZsh{} Plot PDF}
\PY{n}{plt}\PY{o}{.}\PY{n}{plot}\PY{p}{(}\PY{n}{xvalues}\PY{p}{,} \PY{n}{pdf}\PY{p}{,} \PY{n}{linewidth}\PY{o}{=}\PY{l+m+mf}{2.0}\PY{p}{,} \PY{n}{color}\PY{o}{=}\PY{l+s+s1}{\PYZsq{}}\PY{l+s+s1}{red}\PY{l+s+s1}{\PYZsq{}}\PY{p}{,} \PY{n}{label}\PY{o}{=}\PY{l+s+s1}{\PYZsq{}}\PY{l+s+s1}{PDF}\PY{l+s+s1}{\PYZsq{}}\PY{p}{)}
\PY{n}{plt}\PY{o}{.}\PY{n}{xlabel}\PY{p}{(}\PY{l+s+sa}{r}\PY{l+s+s1}{\PYZsq{}}\PY{l+s+s1}{Realized random number}\PY{l+s+s1}{\PYZsq{}}\PY{p}{)}
\PY{n}{plt}\PY{o}{.}\PY{n}{ylabel}\PY{p}{(}\PY{l+s+s1}{\PYZsq{}}\PY{l+s+s1}{Density}\PY{l+s+s1}{\PYZsq{}}\PY{p}{)}
\PY{n}{plt}\PY{o}{.}\PY{n}{title}\PY{p}{(}\PY{l+s+s1}{\PYZsq{}}\PY{l+s+s1}{Histogram of normal draws}\PY{l+s+s1}{\PYZsq{}}\PY{p}{)}
\PY{n}{plt}\PY{o}{.}\PY{n}{legend}\PY{p}{(}\PY{p}{)}
\end{Verbatim}
\end{tcolorbox}

            \begin{tcolorbox}[breakable, size=fbox, boxrule=.5pt, pad at break*=1mm, opacityfill=0]
\prompt{Out}{outcolor}{13}{\boxspacing}
\begin{Verbatim}[commandchars=\\\{\}]
<matplotlib.legend.Legend at 0x7f33c88d3190>
\end{Verbatim}
\end{tcolorbox}
        
    \begin{center}
    \adjustimage{max size={0.9\linewidth}{0.9\paperheight}}{unit7_files/unit7_22_1.pdf}
    \end{center}
    { \hspace*{\fill} \\}
    
    In the above example we pass \texttt{density=True} to Matplotlib's
\texttt{hist()} plotting function so that the result is rescaled to be
comparable to the actual PDF.

Sometimes we need to compute the expectation of a function \(g(x)\) with
respect to a given distribution with PDF \(f(x)\) on some interval
\((a,b)\):

\[E[g(x)] = \int_a^b g(x) f(x) dx\]

For example, we might want to know the mean of a \emph{truncated} normal
with parameters \(\mu=0\), \(\sigma=1.0\) with support on
\((-\infty,0)\), ie.

\[E[x| x \leq 0] = \int_{-\infty}^0 x \frac{f(x)}{F(0)}dx\]

where \(f(x)\) and \(F(x)\) are the PDF and CDF of the standard normal.
We can compute it as follows:

    \begin{tcolorbox}[breakable, size=fbox, boxrule=1pt, pad at break*=1mm,colback=cellbackground, colframe=cellborder]
\prompt{In}{incolor}{14}{\boxspacing}
\begin{Verbatim}[commandchars=\\\{\}]
\PY{k+kn}{from} \PY{n+nn}{scipy}\PY{n+nn}{.}\PY{n+nn}{stats} \PY{k+kn}{import} \PY{n}{norm}
\PY{k+kn}{import} \PY{n+nn}{numpy} \PY{k}{as} \PY{n+nn}{np}

\PY{n}{lb} \PY{o}{=} \PY{o}{\PYZhy{}}\PY{n}{np}\PY{o}{.}\PY{n}{inf}            \PY{c+c1}{\PYZsh{} integration lower bound}
\PY{n}{ub} \PY{o}{=} \PY{l+m+mf}{0.0}                \PY{c+c1}{\PYZsh{} integration upper bound}

\PY{n}{mu} \PY{o}{=} \PY{l+m+mf}{0.0}                \PY{c+c1}{\PYZsh{} mean of the (untruncated) normal}
\PY{n}{sigma} \PY{o}{=} \PY{l+m+mf}{1.0}             \PY{c+c1}{\PYZsh{} std. dev. of the (untruncated) normal}

\PY{n}{cdf0} \PY{o}{=} \PY{n}{norm}\PY{o}{.}\PY{n}{cdf}\PY{p}{(}\PY{l+m+mf}{0.0}\PY{p}{,} \PY{n}{loc}\PY{o}{=}\PY{n}{mu}\PY{p}{,} \PY{n}{scale}\PY{o}{=}\PY{n}{sigma}\PY{p}{)}       \PY{c+c1}{\PYZsh{} CDF at 0}

\PY{c+c1}{\PYZsh{} Compute conditional expected value}
\PY{n}{Ex} \PY{o}{=} \PY{n}{norm}\PY{o}{.}\PY{n}{expect}\PY{p}{(}\PY{k}{lambda} \PY{n}{x}\PY{p}{:} \PY{n}{x}\PY{o}{/}\PY{n}{cdf0}\PY{p}{,} \PY{n}{loc}\PY{o}{=}\PY{n}{mu}\PY{p}{,} \PY{n}{scale}\PY{o}{=}\PY{n}{sigma}\PY{p}{,} \PY{n}{lb}\PY{o}{=}\PY{n}{lb}\PY{p}{,} \PY{n}{ub}\PY{o}{=}\PY{n}{ub}\PY{p}{)}
\PY{n}{Ex}                      \PY{c+c1}{\PYZsh{} print conditional expectation}
\end{Verbatim}
\end{tcolorbox}

            \begin{tcolorbox}[breakable, size=fbox, boxrule=.5pt, pad at break*=1mm, opacityfill=0]
\prompt{Out}{outcolor}{14}{\boxspacing}
\begin{Verbatim}[commandchars=\\\{\}]
-0.7978845608028651
\end{Verbatim}
\end{tcolorbox}
        
    Here we define the function to be integrated as
\(g(x) = \frac{x}{F(0)}\), and we pass it to \texttt{expect()} as a
lambda expression.

    \begin{center}\rule{0.5\linewidth}{0.5pt}\end{center}

\hypertarget{statistics-functions}{%
\subsection{Statistics functions}\label{statistics-functions}}

In the previous section we examined functions associated with specific
distributions. Additionally, there are numerous routines to process
\emph{sample} data which are spread across NumPy and SciPy.

In NumPy, the most useful routines include:

\begin{itemize}
\tightlist
\item
  \texttt{mean()}: sample mean
\item
  \texttt{std()}, \texttt{var()}: sample standard deviation and variance
\item
  \texttt{percentile()}, \texttt{quantile()}: percentiles or quantiles
  of a given array
\item
  \texttt{corrcoef()}: Pearson correlation coefficient
\item
  \texttt{cov()}: sample variance-covariance matrix
\item
  \texttt{histogram()}: histogram of data. This only bins the data, as
  opposed to Matplotlib's \texttt{hist()} which plots it.
\end{itemize}

You can find the full list of routines in the
\href{https://numpy.org/doc/stable/reference/routines.statistics.html}{official
documentation}.

On top of that, the \texttt{scipy.stats} package contains functions to
compute all sorts of descriptive statistics and statistical hypothesis
tests. Many of these routines are too specific to be listed here, so
have a look at the
\href{https://docs.scipy.org/doc/scipy/reference/stats.html\#summary-statistics}{official
documentation} if you need to perform statistical analysis of your
sample data.

\emph{Examples:}

To compute the pairwise correlations of a sample drawn from a
multivariate normal distribution we proceed as follows:

    \begin{tcolorbox}[breakable, size=fbox, boxrule=1pt, pad at break*=1mm,colback=cellbackground, colframe=cellborder]
\prompt{In}{incolor}{15}{\boxspacing}
\begin{Verbatim}[commandchars=\\\{\}]
\PY{k+kn}{import} \PY{n+nn}{numpy} \PY{k}{as} \PY{n+nn}{np}
\PY{k+kn}{from} \PY{n+nn}{numpy}\PY{n+nn}{.}\PY{n+nn}{random} \PY{k+kn}{import} \PY{n}{default\PYZus{}rng}

\PY{n}{rng} \PY{o}{=} \PY{n}{default\PYZus{}rng}\PY{p}{(}\PY{l+m+mi}{123}\PY{p}{)}

\PY{c+c1}{\PYZsh{} vector of multivariate normal means}
\PY{n}{mu} \PY{o}{=} \PY{n}{np}\PY{o}{.}\PY{n}{array}\PY{p}{(}\PY{p}{[}\PY{o}{\PYZhy{}}\PY{l+m+mf}{1.0}\PY{p}{,} \PY{l+m+mf}{1.0}\PY{p}{]}\PY{p}{)}

\PY{n}{sigma1} \PY{o}{=} \PY{l+m+mf}{0.5}                    \PY{c+c1}{\PYZsh{} Std. dev. of first dimension}
\PY{n}{sigma2} \PY{o}{=} \PY{l+m+mf}{1.0}                    \PY{c+c1}{\PYZsh{} Std. dev. of second dimension}
\PY{n}{rho} \PY{o}{=} \PY{l+m+mf}{0.5}                       \PY{c+c1}{\PYZsh{} Correlation coefficient}

\PY{c+c1}{\PYZsh{} variance\PYZhy{}covariance matrix}
\PY{n}{vcv} \PY{o}{=} \PY{n}{np}\PY{o}{.}\PY{n}{array}\PY{p}{(}\PY{p}{[}\PY{p}{[}\PY{n}{sigma1}\PY{o}{*}\PY{o}{*}\PY{l+m+mf}{2.0}\PY{p}{,} \PY{n}{rho}\PY{o}{*}\PY{n}{sigma1}\PY{o}{*}\PY{n}{sigma2}\PY{p}{]}\PY{p}{,}
                \PY{p}{[}\PY{n}{rho}\PY{o}{*}\PY{n}{sigma1}\PY{o}{*}\PY{n}{sigma2}\PY{p}{,} \PY{n}{sigma2}\PY{o}{*}\PY{o}{*}\PY{l+m+mf}{2.0}\PY{p}{]}\PY{p}{]}\PY{p}{)}

\PY{c+c1}{\PYZsh{} Draw some multivariate normal random numbers}
\PY{n}{x} \PY{o}{=} \PY{n}{rng}\PY{o}{.}\PY{n}{multivariate\PYZus{}normal}\PY{p}{(}\PY{n}{mean}\PY{o}{=}\PY{n}{mu}\PY{p}{,} \PY{n}{cov}\PY{o}{=}\PY{n}{vcv}\PY{p}{,} \PY{n}{size}\PY{o}{=}\PY{l+m+mi}{1000}\PY{p}{)}

\PY{c+c1}{\PYZsh{} Compute correlation coefficient}
\PY{n}{np}\PY{o}{.}\PY{n}{corrcoef}\PY{p}{(}\PY{n}{x}\PY{o}{.}\PY{n}{T}\PY{p}{)}        \PY{c+c1}{\PYZsh{} expects each row to contain one variable}
\end{Verbatim}
\end{tcolorbox}

            \begin{tcolorbox}[breakable, size=fbox, boxrule=.5pt, pad at break*=1mm, opacityfill=0]
\prompt{Out}{outcolor}{15}{\boxspacing}
\begin{Verbatim}[commandchars=\\\{\}]
array([[1.        , 0.51768322],
       [0.51768322, 1.        ]])
\end{Verbatim}
\end{tcolorbox}
        
    Depending on the sample size, the correlation coefficient reported in
the off-diagonal elements might or might not be close to the \(\rho\)
used to draw the random data.

In the next example, we demonstrate how to compute some descriptive
statistics for a sample drawn from a 3-dimensional multivariate normal
distribution:

    \begin{tcolorbox}[breakable, size=fbox, boxrule=1pt, pad at break*=1mm,colback=cellbackground, colframe=cellborder]
\prompt{In}{incolor}{16}{\boxspacing}
\begin{Verbatim}[commandchars=\\\{\}]
\PY{k+kn}{import} \PY{n+nn}{scipy}\PY{n+nn}{.}\PY{n+nn}{stats}
\PY{k+kn}{import} \PY{n+nn}{numpy} \PY{k}{as} \PY{n+nn}{np}
\PY{k+kn}{from} \PY{n+nn}{numpy}\PY{n+nn}{.}\PY{n+nn}{random} \PY{k+kn}{import} \PY{n}{default\PYZus{}rng}

\PY{n}{rng} \PY{o}{=} \PY{n}{default\PYZus{}rng}\PY{p}{(}\PY{l+m+mi}{123}\PY{p}{)}

\PY{c+c1}{\PYZsh{} vector of multivariate normal means}
\PY{n}{mu} \PY{o}{=} \PY{n}{np}\PY{o}{.}\PY{n}{array}\PY{p}{(}\PY{p}{[}\PY{o}{\PYZhy{}}\PY{l+m+mf}{1.0}\PY{p}{,} \PY{l+m+mf}{0.0}\PY{p}{,} \PY{l+m+mf}{1.0}\PY{p}{]}\PY{p}{)}

\PY{c+c1}{\PYZsh{} variance\PYZhy{}covariance matrix}
\PY{n}{vcv} \PY{o}{=} \PY{n}{np}\PY{o}{.}\PY{n}{array}\PY{p}{(}\PY{p}{[}\PY{p}{[}\PY{l+m+mf}{1.0}\PY{p}{,} \PY{l+m+mf}{0.5}\PY{p}{,} \PY{l+m+mf}{0.2}\PY{p}{]}\PY{p}{,}
                \PY{p}{[}\PY{l+m+mf}{0.5}\PY{p}{,} \PY{l+m+mf}{2.0}\PY{p}{,} \PY{l+m+mf}{0.7}\PY{p}{]}\PY{p}{,}
                \PY{p}{[}\PY{l+m+mf}{0.2}\PY{p}{,} \PY{l+m+mf}{0.7}\PY{p}{,} \PY{l+m+mf}{0.5}\PY{p}{]}\PY{p}{]}\PY{p}{)}

\PY{c+c1}{\PYZsh{} Draw some multivariate normal random numbers}
\PY{n}{x} \PY{o}{=} \PY{n}{rng}\PY{o}{.}\PY{n}{multivariate\PYZus{}normal}\PY{p}{(}\PY{n}{mean}\PY{o}{=}\PY{n}{mu}\PY{p}{,} \PY{n}{cov}\PY{o}{=}\PY{n}{vcv}\PY{p}{,} \PY{n}{size}\PY{o}{=}\PY{l+m+mi}{100}\PY{p}{)}

\PY{c+c1}{\PYZsh{} Compute some descriptive statistics}
\PY{n}{nobs}\PY{p}{,} \PY{n}{minmax}\PY{p}{,} \PY{n}{mean}\PY{p}{,} \PY{n}{variance}\PY{p}{,} \PY{n}{skewness}\PY{p}{,} \PY{n}{kurtosis} \PY{o}{=} \PY{n}{scipy}\PY{o}{.}\PY{n}{stats}\PY{o}{.}\PY{n}{describe}\PY{p}{(}\PY{n}{x}\PY{p}{)}
\PY{n}{mean}        \PY{c+c1}{\PYZsh{} array of means}
\end{Verbatim}
\end{tcolorbox}

            \begin{tcolorbox}[breakable, size=fbox, boxrule=.5pt, pad at break*=1mm, opacityfill=0]
\prompt{Out}{outcolor}{16}{\boxspacing}
\begin{Verbatim}[commandchars=\\\{\}]
array([-0.98486214, -0.0719401 ,  0.99084898])
\end{Verbatim}
\end{tcolorbox}
        
    \begin{tcolorbox}[breakable, size=fbox, boxrule=1pt, pad at break*=1mm,colback=cellbackground, colframe=cellborder]
\prompt{In}{incolor}{17}{\boxspacing}
\begin{Verbatim}[commandchars=\\\{\}]
\PY{n}{variance}    \PY{c+c1}{\PYZsh{} array of variances}
\end{Verbatim}
\end{tcolorbox}

            \begin{tcolorbox}[breakable, size=fbox, boxrule=.5pt, pad at break*=1mm, opacityfill=0]
\prompt{Out}{outcolor}{17}{\boxspacing}
\begin{Verbatim}[commandchars=\\\{\}]
array([0.80017787, 1.96834418, 0.37118602])
\end{Verbatim}
\end{tcolorbox}
        
    To illustrate how to use one of the many tests implemented in
\texttt{scipy.stats}, we compute the
\href{https://en.wikipedia.org/wiki/Jarque\%E2\%80\%93Bera_test}{Jarque-Bera
test} statistic using \texttt{jarque\_bera()}. This is a goodness-of-fit
test to assess whether a sample has zero skewness and excess kurtosis
and could thus be normally distributed.

    \begin{tcolorbox}[breakable, size=fbox, boxrule=1pt, pad at break*=1mm,colback=cellbackground, colframe=cellborder]
\prompt{In}{incolor}{18}{\boxspacing}
\begin{Verbatim}[commandchars=\\\{\}]
\PY{k+kn}{from} \PY{n+nn}{scipy}\PY{n+nn}{.}\PY{n+nn}{stats} \PY{k+kn}{import} \PY{n}{jarque\PYZus{}bera}
\PY{k+kn}{from} \PY{n+nn}{numpy}\PY{n+nn}{.}\PY{n+nn}{random} \PY{k+kn}{import} \PY{n}{default\PYZus{}rng}

\PY{n}{rng} \PY{o}{=} \PY{n}{default\PYZus{}rng}\PY{p}{(}\PY{l+m+mi}{123}\PY{p}{)}

\PY{c+c1}{\PYZsh{} Draw from univariate normal}
\PY{n}{x} \PY{o}{=} \PY{n}{rng}\PY{o}{.}\PY{n}{normal}\PY{p}{(}\PY{n}{loc}\PY{o}{=}\PY{l+m+mf}{1.0}\PY{p}{,} \PY{n}{scale}\PY{o}{=}\PY{l+m+mf}{2.0}\PY{p}{,} \PY{n}{size}\PY{o}{=}\PY{l+m+mi}{10000}\PY{p}{)}

\PY{c+c1}{\PYZsh{} Compute Jarque\PYZhy{}Bera test statistic}
\PY{n}{jb\PYZus{}stat}\PY{p}{,} \PY{n}{pvalue} \PY{o}{=} \PY{n}{jarque\PYZus{}bera}\PY{p}{(}\PY{n}{x}\PY{p}{)}
\PY{n+nb}{print}\PY{p}{(}\PY{l+s+sa}{f}\PY{l+s+s1}{\PYZsq{}}\PY{l+s+s1}{Test statistic: }\PY{l+s+si}{\PYZob{}}\PY{n}{jb\PYZus{}stat}\PY{l+s+si}{:}\PY{l+s+s1}{.3f}\PY{l+s+si}{\PYZcb{}}\PY{l+s+s1}{, p\PYZhy{}value: }\PY{l+s+si}{\PYZob{}}\PY{n}{pvalue}\PY{l+s+si}{:}\PY{l+s+s1}{.3f}\PY{l+s+si}{\PYZcb{}}\PY{l+s+s1}{\PYZsq{}}\PY{p}{)}
\end{Verbatim}
\end{tcolorbox}

    \begin{Verbatim}[commandchars=\\\{\}]
Test statistic: 3.472, p-value: 0.176
    \end{Verbatim}

    With a p-value of about 0.17 we cannot reject the null hypothesis of
zero skewness and zero excess kurtosis.

    \begin{center}\rule{0.5\linewidth}{0.5pt}\end{center}

\hypertarget{exercises}{%
\section{Exercises}\label{exercises}}

TBA

    \begin{center}\rule{0.5\linewidth}{0.5pt}\end{center}

\hypertarget{solutions}{%
\section{Solutions}\label{solutions}}

TBA


    % Add a bibliography block to the postdoc
    
    
    
\end{document}
