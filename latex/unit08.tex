\documentclass{scrartcl}

  % included at the top of the generated TeX file

\usepackage{tgcursor}

\usepackage[utf8]{inputenc}

\KOMAoptions{parskip=half*}
\KOMAoptions{paper=a4,twoside=false}
\KOMAoptions{numbers=noendperiod}

\addtokomafont{disposition}{\rmfamily}
\setcounter{tocdepth}{\subsectiontocdepth}

    \usepackage[breakable]{tcolorbox}
    \usepackage{parskip} % Stop auto-indenting (to mimic markdown behaviour)
    

    % Basic figure setup, for now with no caption control since it's done
    % automatically by Pandoc (which extracts ![](path) syntax from Markdown).
    \usepackage{graphicx}
    % Maintain compatibility with old templates. Remove in nbconvert 6.0
    \let\Oldincludegraphics\includegraphics
    % Ensure that by default, figures have no caption (until we provide a
    % proper Figure object with a Caption API and a way to capture that
    % in the conversion process - todo).
    \usepackage{caption}
    \DeclareCaptionFormat{nocaption}{}
    \captionsetup{format=nocaption,aboveskip=0pt,belowskip=0pt}

    \usepackage{float}
    \floatplacement{figure}{H} % forces figures to be placed at the correct location
    \usepackage{xcolor} % Allow colors to be defined
    \usepackage{enumerate} % Needed for markdown enumerations to work
    \usepackage{geometry} % Used to adjust the document margins
    \usepackage{amsmath} % Equations
    \usepackage{amssymb} % Equations
    \usepackage{textcomp} % defines textquotesingle
    % Hack from http://tex.stackexchange.com/a/47451/13684:
    \AtBeginDocument{%
        \def\PYZsq{\textquotesingle}% Upright quotes in Pygmentized code
    }
    \usepackage{upquote} % Upright quotes for verbatim code
    \usepackage{eurosym} % defines \euro

    \usepackage{iftex}
    \ifPDFTeX
        \usepackage[T1]{fontenc}
        \IfFileExists{alphabeta.sty}{
              \usepackage{alphabeta}
          }{
              \usepackage[mathletters]{ucs}
              \usepackage[utf8]{inputenc}
          }
    \else
        \usepackage{fontspec}
        \usepackage{unicode-math}
    \fi

    \usepackage{fancyvrb} % verbatim replacement that allows latex
    \usepackage{grffile} % extends the file name processing of package graphics 
                         % to support a larger range
    \makeatletter % fix for old versions of grffile with XeLaTeX
    \@ifpackagelater{grffile}{2019/11/01}
    {
      % Do nothing on new versions
    }
    {
      \def\Gread@@xetex#1{%
        \IfFileExists{"\Gin@base".bb}%
        {\Gread@eps{\Gin@base.bb}}%
        {\Gread@@xetex@aux#1}%
      }
    }
    \makeatother
    \usepackage[Export]{adjustbox} % Used to constrain images to a maximum size
    \adjustboxset{max size={0.9\linewidth}{0.9\paperheight}}

    % The hyperref package gives us a pdf with properly built
    % internal navigation ('pdf bookmarks' for the table of contents,
    % internal cross-reference links, web links for URLs, etc.)
    \usepackage{hyperref}
    % The default LaTeX title has an obnoxious amount of whitespace. By default,
    % titling removes some of it. It also provides customization options.
    \usepackage{longtable} % longtable support required by pandoc >1.10
    \usepackage{booktabs}  % table support for pandoc > 1.12.2
    \usepackage{array}     % table support for pandoc >= 2.11.3
    \usepackage{calc}      % table minipage width calculation for pandoc >= 2.11.1
    \usepackage[inline]{enumitem} % IRkernel/repr support (it uses the enumerate* environment)
    \usepackage[normalem]{ulem} % ulem is needed to support strikethroughs (\sout)
                                % normalem makes italics be italics, not underlines
    \usepackage{mathrsfs}
    

    
    % Colors for the hyperref package
    \definecolor{urlcolor}{rgb}{0,.145,.698}
    \definecolor{linkcolor}{rgb}{.71,0.21,0.01}
    \definecolor{citecolor}{rgb}{.12,.54,.11}

    % ANSI colors
    \definecolor{ansi-black}{HTML}{3E424D}
    \definecolor{ansi-black-intense}{HTML}{282C36}
    \definecolor{ansi-red}{HTML}{E75C58}
    \definecolor{ansi-red-intense}{HTML}{B22B31}
    \definecolor{ansi-green}{HTML}{00A250}
    \definecolor{ansi-green-intense}{HTML}{007427}
    \definecolor{ansi-yellow}{HTML}{DDB62B}
    \definecolor{ansi-yellow-intense}{HTML}{B27D12}
    \definecolor{ansi-blue}{HTML}{208FFB}
    \definecolor{ansi-blue-intense}{HTML}{0065CA}
    \definecolor{ansi-magenta}{HTML}{D160C4}
    \definecolor{ansi-magenta-intense}{HTML}{A03196}
    \definecolor{ansi-cyan}{HTML}{60C6C8}
    \definecolor{ansi-cyan-intense}{HTML}{258F8F}
    \definecolor{ansi-white}{HTML}{C5C1B4}
    \definecolor{ansi-white-intense}{HTML}{A1A6B2}
    \definecolor{ansi-default-inverse-fg}{HTML}{FFFFFF}
    \definecolor{ansi-default-inverse-bg}{HTML}{000000}

    % common color for the border for error outputs.
    \definecolor{outerrorbackground}{HTML}{FFDFDF}

    % commands and environments needed by pandoc snippets
    % extracted from the output of `pandoc -s`
    \providecommand{\tightlist}{%
      \setlength{\itemsep}{0pt}\setlength{\parskip}{0pt}}
    \DefineVerbatimEnvironment{Highlighting}{Verbatim}{commandchars=\\\{\}}
    % Add ',fontsize=\small' for more characters per line
    \newenvironment{Shaded}{}{}
    \newcommand{\KeywordTok}[1]{\textcolor[rgb]{0.00,0.44,0.13}{\textbf{{#1}}}}
    \newcommand{\DataTypeTok}[1]{\textcolor[rgb]{0.56,0.13,0.00}{{#1}}}
    \newcommand{\DecValTok}[1]{\textcolor[rgb]{0.25,0.63,0.44}{{#1}}}
    \newcommand{\BaseNTok}[1]{\textcolor[rgb]{0.25,0.63,0.44}{{#1}}}
    \newcommand{\FloatTok}[1]{\textcolor[rgb]{0.25,0.63,0.44}{{#1}}}
    \newcommand{\CharTok}[1]{\textcolor[rgb]{0.25,0.44,0.63}{{#1}}}
    \newcommand{\StringTok}[1]{\textcolor[rgb]{0.25,0.44,0.63}{{#1}}}
    \newcommand{\CommentTok}[1]{\textcolor[rgb]{0.38,0.63,0.69}{\textit{{#1}}}}
    \newcommand{\OtherTok}[1]{\textcolor[rgb]{0.00,0.44,0.13}{{#1}}}
    \newcommand{\AlertTok}[1]{\textcolor[rgb]{1.00,0.00,0.00}{\textbf{{#1}}}}
    \newcommand{\FunctionTok}[1]{\textcolor[rgb]{0.02,0.16,0.49}{{#1}}}
    \newcommand{\RegionMarkerTok}[1]{{#1}}
    \newcommand{\ErrorTok}[1]{\textcolor[rgb]{1.00,0.00,0.00}{\textbf{{#1}}}}
    \newcommand{\NormalTok}[1]{{#1}}
    
    % Additional commands for more recent versions of Pandoc
    \newcommand{\ConstantTok}[1]{\textcolor[rgb]{0.53,0.00,0.00}{{#1}}}
    \newcommand{\SpecialCharTok}[1]{\textcolor[rgb]{0.25,0.44,0.63}{{#1}}}
    \newcommand{\VerbatimStringTok}[1]{\textcolor[rgb]{0.25,0.44,0.63}{{#1}}}
    \newcommand{\SpecialStringTok}[1]{\textcolor[rgb]{0.73,0.40,0.53}{{#1}}}
    \newcommand{\ImportTok}[1]{{#1}}
    \newcommand{\DocumentationTok}[1]{\textcolor[rgb]{0.73,0.13,0.13}{\textit{{#1}}}}
    \newcommand{\AnnotationTok}[1]{\textcolor[rgb]{0.38,0.63,0.69}{\textbf{\textit{{#1}}}}}
    \newcommand{\CommentVarTok}[1]{\textcolor[rgb]{0.38,0.63,0.69}{\textbf{\textit{{#1}}}}}
    \newcommand{\VariableTok}[1]{\textcolor[rgb]{0.10,0.09,0.49}{{#1}}}
    \newcommand{\ControlFlowTok}[1]{\textcolor[rgb]{0.00,0.44,0.13}{\textbf{{#1}}}}
    \newcommand{\OperatorTok}[1]{\textcolor[rgb]{0.40,0.40,0.40}{{#1}}}
    \newcommand{\BuiltInTok}[1]{{#1}}
    \newcommand{\ExtensionTok}[1]{{#1}}
    \newcommand{\PreprocessorTok}[1]{\textcolor[rgb]{0.74,0.48,0.00}{{#1}}}
    \newcommand{\AttributeTok}[1]{\textcolor[rgb]{0.49,0.56,0.16}{{#1}}}
    \newcommand{\InformationTok}[1]{\textcolor[rgb]{0.38,0.63,0.69}{\textbf{\textit{{#1}}}}}
    \newcommand{\WarningTok}[1]{\textcolor[rgb]{0.38,0.63,0.69}{\textbf{\textit{{#1}}}}}
    
    
    % Define a nice break command that doesn't care if a line doesn't already
    % exist.
    \def\br{\hspace*{\fill} \\* }
    % Math Jax compatibility definitions
    \def\gt{>}
    \def\lt{<}
    \let\Oldtex\TeX
    \let\Oldlatex\LaTeX
    \renewcommand{\TeX}{\textrm{\Oldtex}}
    \renewcommand{\LaTeX}{\textrm{\Oldlatex}}
    % Document parameters
    % Document title
    \newcommand*{\mytitle}{Unit 8: Data input and output}

    % Included at the bottom of the preamble

\usepackage{microtype}


\title{\mytitle}
\author{Richard Foltyn}
\hypersetup{pdfauthor={Richard Foltyn}, pdftitle={\mytitle}}

% Remove horizontal rules in a very hackish way
\renewcommand{\rule}[2]{}

\RequirePackage{xspace}


\newcommand*{\eg}{e.g.\@\xspace}
\newcommand*{\Eg}{E.g.\@\xspace}
\newcommand*{\etc}{etc.\@\xspace}
\newcommand*{\ie}{i.e.\@\xspace}
\newcommand*{\vs}{vs.\@\xspace}
\newcommand*{\viz}{viz.\@\xspace}
\newcommand*{\US}{U.S.\@\xspace}

    
    
    
    
    
% Pygments definitions
\makeatletter
\def\PY@reset{\let\PY@it=\relax \let\PY@bf=\relax%
    \let\PY@ul=\relax \let\PY@tc=\relax%
    \let\PY@bc=\relax \let\PY@ff=\relax}
\def\PY@tok#1{\csname PY@tok@#1\endcsname}
\def\PY@toks#1+{\ifx\relax#1\empty\else%
    \PY@tok{#1}\expandafter\PY@toks\fi}
\def\PY@do#1{\PY@bc{\PY@tc{\PY@ul{%
    \PY@it{\PY@bf{\PY@ff{#1}}}}}}}
\def\PY#1#2{\PY@reset\PY@toks#1+\relax+\PY@do{#2}}

\@namedef{PY@tok@w}{\def\PY@tc##1{\textcolor[rgb]{0.73,0.73,0.73}{##1}}}
\@namedef{PY@tok@c}{\let\PY@it=\textit\def\PY@tc##1{\textcolor[rgb]{0.24,0.48,0.48}{##1}}}
\@namedef{PY@tok@cp}{\def\PY@tc##1{\textcolor[rgb]{0.61,0.40,0.00}{##1}}}
\@namedef{PY@tok@k}{\let\PY@bf=\textbf\def\PY@tc##1{\textcolor[rgb]{0.00,0.50,0.00}{##1}}}
\@namedef{PY@tok@kp}{\def\PY@tc##1{\textcolor[rgb]{0.00,0.50,0.00}{##1}}}
\@namedef{PY@tok@kt}{\def\PY@tc##1{\textcolor[rgb]{0.69,0.00,0.25}{##1}}}
\@namedef{PY@tok@o}{\def\PY@tc##1{\textcolor[rgb]{0.40,0.40,0.40}{##1}}}
\@namedef{PY@tok@ow}{\let\PY@bf=\textbf\def\PY@tc##1{\textcolor[rgb]{0.67,0.13,1.00}{##1}}}
\@namedef{PY@tok@nb}{\def\PY@tc##1{\textcolor[rgb]{0.00,0.50,0.00}{##1}}}
\@namedef{PY@tok@nf}{\def\PY@tc##1{\textcolor[rgb]{0.00,0.00,1.00}{##1}}}
\@namedef{PY@tok@nc}{\let\PY@bf=\textbf\def\PY@tc##1{\textcolor[rgb]{0.00,0.00,1.00}{##1}}}
\@namedef{PY@tok@nn}{\let\PY@bf=\textbf\def\PY@tc##1{\textcolor[rgb]{0.00,0.00,1.00}{##1}}}
\@namedef{PY@tok@ne}{\let\PY@bf=\textbf\def\PY@tc##1{\textcolor[rgb]{0.80,0.25,0.22}{##1}}}
\@namedef{PY@tok@nv}{\def\PY@tc##1{\textcolor[rgb]{0.10,0.09,0.49}{##1}}}
\@namedef{PY@tok@no}{\def\PY@tc##1{\textcolor[rgb]{0.53,0.00,0.00}{##1}}}
\@namedef{PY@tok@nl}{\def\PY@tc##1{\textcolor[rgb]{0.46,0.46,0.00}{##1}}}
\@namedef{PY@tok@ni}{\let\PY@bf=\textbf\def\PY@tc##1{\textcolor[rgb]{0.44,0.44,0.44}{##1}}}
\@namedef{PY@tok@na}{\def\PY@tc##1{\textcolor[rgb]{0.41,0.47,0.13}{##1}}}
\@namedef{PY@tok@nt}{\let\PY@bf=\textbf\def\PY@tc##1{\textcolor[rgb]{0.00,0.50,0.00}{##1}}}
\@namedef{PY@tok@nd}{\def\PY@tc##1{\textcolor[rgb]{0.67,0.13,1.00}{##1}}}
\@namedef{PY@tok@s}{\def\PY@tc##1{\textcolor[rgb]{0.73,0.13,0.13}{##1}}}
\@namedef{PY@tok@sd}{\let\PY@it=\textit\def\PY@tc##1{\textcolor[rgb]{0.73,0.13,0.13}{##1}}}
\@namedef{PY@tok@si}{\let\PY@bf=\textbf\def\PY@tc##1{\textcolor[rgb]{0.64,0.35,0.47}{##1}}}
\@namedef{PY@tok@se}{\let\PY@bf=\textbf\def\PY@tc##1{\textcolor[rgb]{0.67,0.36,0.12}{##1}}}
\@namedef{PY@tok@sr}{\def\PY@tc##1{\textcolor[rgb]{0.64,0.35,0.47}{##1}}}
\@namedef{PY@tok@ss}{\def\PY@tc##1{\textcolor[rgb]{0.10,0.09,0.49}{##1}}}
\@namedef{PY@tok@sx}{\def\PY@tc##1{\textcolor[rgb]{0.00,0.50,0.00}{##1}}}
\@namedef{PY@tok@m}{\def\PY@tc##1{\textcolor[rgb]{0.40,0.40,0.40}{##1}}}
\@namedef{PY@tok@gh}{\let\PY@bf=\textbf\def\PY@tc##1{\textcolor[rgb]{0.00,0.00,0.50}{##1}}}
\@namedef{PY@tok@gu}{\let\PY@bf=\textbf\def\PY@tc##1{\textcolor[rgb]{0.50,0.00,0.50}{##1}}}
\@namedef{PY@tok@gd}{\def\PY@tc##1{\textcolor[rgb]{0.63,0.00,0.00}{##1}}}
\@namedef{PY@tok@gi}{\def\PY@tc##1{\textcolor[rgb]{0.00,0.52,0.00}{##1}}}
\@namedef{PY@tok@gr}{\def\PY@tc##1{\textcolor[rgb]{0.89,0.00,0.00}{##1}}}
\@namedef{PY@tok@ge}{\let\PY@it=\textit}
\@namedef{PY@tok@gs}{\let\PY@bf=\textbf}
\@namedef{PY@tok@gp}{\let\PY@bf=\textbf\def\PY@tc##1{\textcolor[rgb]{0.00,0.00,0.50}{##1}}}
\@namedef{PY@tok@go}{\def\PY@tc##1{\textcolor[rgb]{0.44,0.44,0.44}{##1}}}
\@namedef{PY@tok@gt}{\def\PY@tc##1{\textcolor[rgb]{0.00,0.27,0.87}{##1}}}
\@namedef{PY@tok@err}{\def\PY@bc##1{{\setlength{\fboxsep}{\string -\fboxrule}\fcolorbox[rgb]{1.00,0.00,0.00}{1,1,1}{\strut ##1}}}}
\@namedef{PY@tok@kc}{\let\PY@bf=\textbf\def\PY@tc##1{\textcolor[rgb]{0.00,0.50,0.00}{##1}}}
\@namedef{PY@tok@kd}{\let\PY@bf=\textbf\def\PY@tc##1{\textcolor[rgb]{0.00,0.50,0.00}{##1}}}
\@namedef{PY@tok@kn}{\let\PY@bf=\textbf\def\PY@tc##1{\textcolor[rgb]{0.00,0.50,0.00}{##1}}}
\@namedef{PY@tok@kr}{\let\PY@bf=\textbf\def\PY@tc##1{\textcolor[rgb]{0.00,0.50,0.00}{##1}}}
\@namedef{PY@tok@bp}{\def\PY@tc##1{\textcolor[rgb]{0.00,0.50,0.00}{##1}}}
\@namedef{PY@tok@fm}{\def\PY@tc##1{\textcolor[rgb]{0.00,0.00,1.00}{##1}}}
\@namedef{PY@tok@vc}{\def\PY@tc##1{\textcolor[rgb]{0.10,0.09,0.49}{##1}}}
\@namedef{PY@tok@vg}{\def\PY@tc##1{\textcolor[rgb]{0.10,0.09,0.49}{##1}}}
\@namedef{PY@tok@vi}{\def\PY@tc##1{\textcolor[rgb]{0.10,0.09,0.49}{##1}}}
\@namedef{PY@tok@vm}{\def\PY@tc##1{\textcolor[rgb]{0.10,0.09,0.49}{##1}}}
\@namedef{PY@tok@sa}{\def\PY@tc##1{\textcolor[rgb]{0.73,0.13,0.13}{##1}}}
\@namedef{PY@tok@sb}{\def\PY@tc##1{\textcolor[rgb]{0.73,0.13,0.13}{##1}}}
\@namedef{PY@tok@sc}{\def\PY@tc##1{\textcolor[rgb]{0.73,0.13,0.13}{##1}}}
\@namedef{PY@tok@dl}{\def\PY@tc##1{\textcolor[rgb]{0.73,0.13,0.13}{##1}}}
\@namedef{PY@tok@s2}{\def\PY@tc##1{\textcolor[rgb]{0.73,0.13,0.13}{##1}}}
\@namedef{PY@tok@sh}{\def\PY@tc##1{\textcolor[rgb]{0.73,0.13,0.13}{##1}}}
\@namedef{PY@tok@s1}{\def\PY@tc##1{\textcolor[rgb]{0.73,0.13,0.13}{##1}}}
\@namedef{PY@tok@mb}{\def\PY@tc##1{\textcolor[rgb]{0.40,0.40,0.40}{##1}}}
\@namedef{PY@tok@mf}{\def\PY@tc##1{\textcolor[rgb]{0.40,0.40,0.40}{##1}}}
\@namedef{PY@tok@mh}{\def\PY@tc##1{\textcolor[rgb]{0.40,0.40,0.40}{##1}}}
\@namedef{PY@tok@mi}{\def\PY@tc##1{\textcolor[rgb]{0.40,0.40,0.40}{##1}}}
\@namedef{PY@tok@il}{\def\PY@tc##1{\textcolor[rgb]{0.40,0.40,0.40}{##1}}}
\@namedef{PY@tok@mo}{\def\PY@tc##1{\textcolor[rgb]{0.40,0.40,0.40}{##1}}}
\@namedef{PY@tok@ch}{\let\PY@it=\textit\def\PY@tc##1{\textcolor[rgb]{0.24,0.48,0.48}{##1}}}
\@namedef{PY@tok@cm}{\let\PY@it=\textit\def\PY@tc##1{\textcolor[rgb]{0.24,0.48,0.48}{##1}}}
\@namedef{PY@tok@cpf}{\let\PY@it=\textit\def\PY@tc##1{\textcolor[rgb]{0.24,0.48,0.48}{##1}}}
\@namedef{PY@tok@c1}{\let\PY@it=\textit\def\PY@tc##1{\textcolor[rgb]{0.24,0.48,0.48}{##1}}}
\@namedef{PY@tok@cs}{\let\PY@it=\textit\def\PY@tc##1{\textcolor[rgb]{0.24,0.48,0.48}{##1}}}

\def\PYZbs{\char`\\}
\def\PYZus{\char`\_}
\def\PYZob{\char`\{}
\def\PYZcb{\char`\}}
\def\PYZca{\char`\^}
\def\PYZam{\char`\&}
\def\PYZlt{\char`\<}
\def\PYZgt{\char`\>}
\def\PYZsh{\char`\#}
\def\PYZpc{\char`\%}
\def\PYZdl{\char`\$}
\def\PYZhy{\char`\-}
\def\PYZsq{\char`\'}
\def\PYZdq{\char`\"}
\def\PYZti{\char`\~}
% for compatibility with earlier versions
\def\PYZat{@}
\def\PYZlb{[}
\def\PYZrb{]}
\makeatother


    % For linebreaks inside Verbatim environment from package fancyvrb. 
    \makeatletter
        \newbox\Wrappedcontinuationbox 
        \newbox\Wrappedvisiblespacebox 
        \newcommand*\Wrappedvisiblespace {\textcolor{red}{\textvisiblespace}} 
        \newcommand*\Wrappedcontinuationsymbol {\textcolor{red}{\llap{\tiny$\m@th\hookrightarrow$}}} 
        \newcommand*\Wrappedcontinuationindent {3ex } 
        \newcommand*\Wrappedafterbreak {\kern\Wrappedcontinuationindent\copy\Wrappedcontinuationbox} 
        % Take advantage of the already applied Pygments mark-up to insert 
        % potential linebreaks for TeX processing. 
        %        {, <, #, %, $, ' and ": go to next line. 
        %        _, }, ^, &, >, - and ~: stay at end of broken line. 
        % Use of \textquotesingle for straight quote. 
        \newcommand*\Wrappedbreaksatspecials {% 
            \def\PYGZus{\discretionary{\char`\_}{\Wrappedafterbreak}{\char`\_}}% 
            \def\PYGZob{\discretionary{}{\Wrappedafterbreak\char`\{}{\char`\{}}% 
            \def\PYGZcb{\discretionary{\char`\}}{\Wrappedafterbreak}{\char`\}}}% 
            \def\PYGZca{\discretionary{\char`\^}{\Wrappedafterbreak}{\char`\^}}% 
            \def\PYGZam{\discretionary{\char`\&}{\Wrappedafterbreak}{\char`\&}}% 
            \def\PYGZlt{\discretionary{}{\Wrappedafterbreak\char`\<}{\char`\<}}% 
            \def\PYGZgt{\discretionary{\char`\>}{\Wrappedafterbreak}{\char`\>}}% 
            \def\PYGZsh{\discretionary{}{\Wrappedafterbreak\char`\#}{\char`\#}}% 
            \def\PYGZpc{\discretionary{}{\Wrappedafterbreak\char`\%}{\char`\%}}% 
            \def\PYGZdl{\discretionary{}{\Wrappedafterbreak\char`\$}{\char`\$}}% 
            \def\PYGZhy{\discretionary{\char`\-}{\Wrappedafterbreak}{\char`\-}}% 
            \def\PYGZsq{\discretionary{}{\Wrappedafterbreak\textquotesingle}{\textquotesingle}}% 
            \def\PYGZdq{\discretionary{}{\Wrappedafterbreak\char`\"}{\char`\"}}% 
            \def\PYGZti{\discretionary{\char`\~}{\Wrappedafterbreak}{\char`\~}}% 
        } 
        % Some characters . , ; ? ! / are not pygmentized. 
        % This macro makes them "active" and they will insert potential linebreaks 
        \newcommand*\Wrappedbreaksatpunct {% 
            \lccode`\~`\.\lowercase{\def~}{\discretionary{\hbox{\char`\.}}{\Wrappedafterbreak}{\hbox{\char`\.}}}% 
            \lccode`\~`\,\lowercase{\def~}{\discretionary{\hbox{\char`\,}}{\Wrappedafterbreak}{\hbox{\char`\,}}}% 
            \lccode`\~`\;\lowercase{\def~}{\discretionary{\hbox{\char`\;}}{\Wrappedafterbreak}{\hbox{\char`\;}}}% 
            \lccode`\~`\:\lowercase{\def~}{\discretionary{\hbox{\char`\:}}{\Wrappedafterbreak}{\hbox{\char`\:}}}% 
            \lccode`\~`\?\lowercase{\def~}{\discretionary{\hbox{\char`\?}}{\Wrappedafterbreak}{\hbox{\char`\?}}}% 
            \lccode`\~`\!\lowercase{\def~}{\discretionary{\hbox{\char`\!}}{\Wrappedafterbreak}{\hbox{\char`\!}}}% 
            \lccode`\~`\/\lowercase{\def~}{\discretionary{\hbox{\char`\/}}{\Wrappedafterbreak}{\hbox{\char`\/}}}% 
            \catcode`\.\active
            \catcode`\,\active 
            \catcode`\;\active
            \catcode`\:\active
            \catcode`\?\active
            \catcode`\!\active
            \catcode`\/\active 
            \lccode`\~`\~ 	
        }
    \makeatother

    \let\OriginalVerbatim=\Verbatim
    \makeatletter
    \renewcommand{\Verbatim}[1][1]{%
        %\parskip\z@skip
        \sbox\Wrappedcontinuationbox {\Wrappedcontinuationsymbol}%
        \sbox\Wrappedvisiblespacebox {\FV@SetupFont\Wrappedvisiblespace}%
        \def\FancyVerbFormatLine ##1{\hsize\linewidth
            \vtop{\raggedright\hyphenpenalty\z@\exhyphenpenalty\z@
                \doublehyphendemerits\z@\finalhyphendemerits\z@
                \strut ##1\strut}%
        }%
        % If the linebreak is at a space, the latter will be displayed as visible
        % space at end of first line, and a continuation symbol starts next line.
        % Stretch/shrink are however usually zero for typewriter font.
        \def\FV@Space {%
            \nobreak\hskip\z@ plus\fontdimen3\font minus\fontdimen4\font
            \discretionary{\copy\Wrappedvisiblespacebox}{\Wrappedafterbreak}
            {\kern\fontdimen2\font}%
        }%
        
        % Allow breaks at special characters using \PYG... macros.
        \Wrappedbreaksatspecials
        % Breaks at punctuation characters . , ; ? ! and / need catcode=\active 	
        \OriginalVerbatim[#1,fontsize=\small,codes*=\Wrappedbreaksatpunct]%
    }
    \makeatother

    % Exact colors from NB
    \definecolor{incolor}{HTML}{303F9F}
    \definecolor{outcolor}{HTML}{D84315}
    \definecolor{cellborder}{HTML}{CFCFCF}
    \definecolor{cellbackground}{HTML}{F7F7F7}
    
    % prompt
    \makeatletter
    \newcommand{\boxspacing}{\kern\kvtcb@left@rule\kern\kvtcb@boxsep}
    \makeatother
    \newcommand{\prompt}[4]{
        {\ttfamily\llap{{\color{#2}[#3]:\hspace{3pt}#4}}\vspace{-\baselineskip}}
    }
    

    
    % Prevent overflowing lines due to hard-to-break entities
    \sloppy 
    % Setup hyperref package
    \hypersetup{
      breaklinks=true,  % so long urls are correctly broken across lines
      colorlinks=true,
      urlcolor=urlcolor,
      linkcolor=linkcolor,
      citecolor=citecolor,
      }
    % Slightly bigger margins than the latex defaults
    
    \geometry{verbose,tmargin=1in,bmargin=1in,lmargin=1in,rmargin=1in}
    
    

\begin{document}
    
    \maketitle
    \tableofcontents
    
    

    
    \hypertarget{data-input-and-output}{%
\section{Data input and output}\label{data-input-and-output}}

In this unit we discuss input and output, or I/O for short. We focus
exclusively on I/O routines used to load and store data from files that
are relevant for numerical computation and data analysis.

    \hypertarget{io-with-numpy}{%
\subsection{I/O with NumPy}\label{io-with-numpy}}

\hypertarget{loading-text-data}{%
\subsubsection{Loading text data}\label{loading-text-data}}

We have already encountered the most basic, and probably most frequently
used NumPy I/O routine, \texttt{np.loadtxt()}. We often use files that
store data as text files containing character-separated values (CSV)
since virtually any application supports this data format. The most
important I/O functions to process text data are:

\begin{itemize}
\tightlist
\item
  \href{https://numpy.org/doc/stable/reference/generated/numpy.loadtxt.html}{\texttt{np.loadtxt()}}:
  load data from a text file.
\item
  \href{https://numpy.org/doc/stable/reference/generated/numpy.genfromtxt.html}{\texttt{np.genfromtxt()}}:
  load data from a text file and handle missing data.
\item
  \href{https://numpy.org/doc/stable/reference/generated/numpy.savetxt.html}{\texttt{np.savetxt()}}:
  save a NumPy array to a text file.
\end{itemize}

There are a few other I/O functions in NumPy, for example to write
arrays as raw binary data. We won't cover them here, but you can find
them in the
\href{https://numpy.org/doc/stable/reference/routines.io.html}{official
documentation}.

Imagine we have the following tabular data from
\href{https://fred.stlouisfed.org/}{FRED}, where the first two rows look
as follows:

\begin{longtable}[]{@{}llll@{}}
\toprule
Year & GDP & CPI & UNRATE\tabularnewline
\midrule
\endhead
1948 & 2118.5 & 24.0 & 3.8\tabularnewline
1949 & 2106.6 & 23.8 & 6.0\tabularnewline
\bottomrule
\end{longtable}

To load this CSV file as a NumPy array, we use \texttt{loadtxt()}:

    \begin{tcolorbox}[breakable, size=fbox, boxrule=1pt, pad at break*=1mm,colback=cellbackground, colframe=cellborder]
\prompt{In}{incolor}{1}{\boxspacing}
\begin{Verbatim}[commandchars=\\\{\}]
\PY{k+kn}{import} \PY{n+nn}{numpy} \PY{k}{as} \PY{n+nn}{np}

\PY{c+c1}{\PYZsh{} relative path to CSV file}
\PY{n}{file} \PY{o}{=} \PY{l+s+s1}{\PYZsq{}}\PY{l+s+s1}{../data/FRED.csv}\PY{l+s+s1}{\PYZsq{}}

\PY{c+c1}{\PYZsh{} load CSV}
\PY{n}{data} \PY{o}{=} \PY{n}{np}\PY{o}{.}\PY{n}{loadtxt}\PY{p}{(}\PY{n}{file}\PY{p}{,} \PY{n}{skiprows}\PY{o}{=}\PY{l+m+mi}{1}\PY{p}{,} \PY{n}{delimiter}\PY{o}{=}\PY{l+s+s1}{\PYZsq{}}\PY{l+s+s1}{,}\PY{l+s+s1}{\PYZsq{}}\PY{p}{)}
\PY{n}{data}\PY{p}{[}\PY{p}{:}\PY{l+m+mi}{2}\PY{p}{]}        \PY{c+c1}{\PYZsh{} Display first two rows}
\end{Verbatim}
\end{tcolorbox}

            \begin{tcolorbox}[breakable, size=fbox, boxrule=.5pt, pad at break*=1mm, opacityfill=0]
\prompt{Out}{outcolor}{1}{\boxspacing}
\begin{Verbatim}[commandchars=\\\{\}]
array([[1948. , 2118.5,   24. ,    3.8],
       [1949. , 2106.6,   23.8,    6. ]])
\end{Verbatim}
\end{tcolorbox}
        
    The default settings will in many cases be appropriate to load whatever
CSV file we might have. However, we'll occasionally want to specify the
following arguments to override the defaults:

\begin{itemize}
\tightlist
\item
  \texttt{delimiter}: Character used to separate individual fields
  (default: space).
\item
  \texttt{skiprows=n}: Skip the first \texttt{n} rows. For example, if
  the CSV file contains a header with variable names,
  \texttt{skiprows=1} needs to be specified as NumPy by default cannot
  process these names.
\item
  \texttt{dtype}: Enforce a particular data type for the resulting
  array.
\item
  \texttt{encoding}: Set the character encoding of the input data. This
  is usually not needed, but can be required to import data with
  non-latin characters that are not encoded using Unicode.
\end{itemize}

While \texttt{loadtxt()} is simple to use, it quickly reaches its limits
with more complex data sets. For example, when we try to load our sample
of universities with \texttt{loadtxt()}, we get the following error:

    \begin{tcolorbox}[breakable, size=fbox, boxrule=1pt, pad at break*=1mm,colback=cellbackground, colframe=cellborder]
\prompt{In}{incolor}{2}{\boxspacing}
\begin{Verbatim}[commandchars=\\\{\}]
\PY{k+kn}{import} \PY{n+nn}{numpy} \PY{k}{as} \PY{n+nn}{np}

\PY{n}{file} \PY{o}{=} \PY{l+s+s1}{\PYZsq{}}\PY{l+s+s1}{../data/universities.csv}\PY{l+s+s1}{\PYZsq{}}

\PY{c+c1}{\PYZsh{} Try to load CSV data that contains strings}
\PY{c+c1}{\PYZsh{} This will result in an error!}
\PY{n}{data} \PY{o}{=} \PY{n}{np}\PY{o}{.}\PY{n}{loadtxt}\PY{p}{(}\PY{n}{file}\PY{p}{,} \PY{n}{delimiter}\PY{o}{=}\PY{l+s+s1}{\PYZsq{}}\PY{l+s+s1}{;}\PY{l+s+s1}{\PYZsq{}}\PY{p}{,} \PY{n}{skiprows}\PY{o}{=}\PY{l+m+mi}{1}\PY{p}{)}
\end{Verbatim}
\end{tcolorbox}

    \begin{Verbatim}[commandchars=\\\{\}, frame=single, framerule=2mm, rulecolor=\color{outerrorbackground}]
\textcolor{ansi-red}{ValueError}\textcolor{ansi-red}{:} could not convert string to float: '"University of Glasgow"'

    \end{Verbatim}

    This code fails for two reasons:

\begin{enumerate}
\def\labelenumi{\arabic{enumi}.}
\tightlist
\item
  The file contains strings and floats, and \texttt{loadtxt()} by
  default cannot load mixed data.
\item
  There are missing values (empty fields), which \texttt{loadtxt()}
  cannot handle either.
\end{enumerate}

We can address the first issue by creating a so-called
\href{https://numpy.org/doc/stable/user/basics.rec.html}{structured
array}, \ie an array that contains fields with mixed data. This is
accomplished by constructing a special \texttt{dtype} object that
specifies the field names and their data types:

    \begin{tcolorbox}[breakable, size=fbox, boxrule=1pt, pad at break*=1mm,colback=cellbackground, colframe=cellborder]
\prompt{In}{incolor}{3}{\boxspacing}
\begin{Verbatim}[commandchars=\\\{\}]
\PY{c+c1}{\PYZsh{} Define names and data types for fields in CSV file}
\PY{c+c1}{\PYZsh{}   Data types are defined using two tokens:}
\PY{c+c1}{\PYZsh{}       1.  The main data type (U: unicode string, f: float, i: integer)}
\PY{c+c1}{\PYZsh{}       2.  The precision or field width}
\PY{n}{dtypes} \PY{o}{=} \PY{n}{np}\PY{o}{.}\PY{n}{dtype}\PY{p}{(}\PY{p}{[}\PY{p}{(}\PY{l+s+s1}{\PYZsq{}}\PY{l+s+s1}{Institution}\PY{l+s+s1}{\PYZsq{}}\PY{p}{,} \PY{l+s+s1}{\PYZsq{}}\PY{l+s+s1}{U30}\PY{l+s+s1}{\PYZsq{}}\PY{p}{)}\PY{p}{,}      \PY{c+c1}{\PYZsh{} unicode string of length 30}
                   \PY{p}{(}\PY{l+s+s1}{\PYZsq{}}\PY{l+s+s1}{Country}\PY{l+s+s1}{\PYZsq{}}\PY{p}{,} \PY{l+s+s1}{\PYZsq{}}\PY{l+s+s1}{U20}\PY{l+s+s1}{\PYZsq{}}\PY{p}{)}\PY{p}{,}          \PY{c+c1}{\PYZsh{} unicode string of length 20}
                   \PY{p}{(}\PY{l+s+s1}{\PYZsq{}}\PY{l+s+s1}{Founded}\PY{l+s+s1}{\PYZsq{}}\PY{p}{,} \PY{l+s+s1}{\PYZsq{}}\PY{l+s+s1}{i4}\PY{l+s+s1}{\PYZsq{}}\PY{p}{)}\PY{p}{,}           \PY{c+c1}{\PYZsh{} integer, 4 bytes}
                   \PY{p}{(}\PY{l+s+s1}{\PYZsq{}}\PY{l+s+s1}{Students}\PY{l+s+s1}{\PYZsq{}}\PY{p}{,} \PY{l+s+s1}{\PYZsq{}}\PY{l+s+s1}{i4}\PY{l+s+s1}{\PYZsq{}}\PY{p}{)}\PY{p}{,}
                   \PY{p}{(}\PY{l+s+s1}{\PYZsq{}}\PY{l+s+s1}{Staff}\PY{l+s+s1}{\PYZsq{}}\PY{p}{,} \PY{l+s+s1}{\PYZsq{}}\PY{l+s+s1}{i4}\PY{l+s+s1}{\PYZsq{}}\PY{p}{)}\PY{p}{,}
                   \PY{p}{(}\PY{l+s+s1}{\PYZsq{}}\PY{l+s+s1}{Admin}\PY{l+s+s1}{\PYZsq{}}\PY{p}{,} \PY{l+s+s1}{\PYZsq{}}\PY{l+s+s1}{i4}\PY{l+s+s1}{\PYZsq{}}\PY{p}{)}\PY{p}{,}
                   \PY{p}{(}\PY{l+s+s1}{\PYZsq{}}\PY{l+s+s1}{Budget}\PY{l+s+s1}{\PYZsq{}}\PY{p}{,} \PY{l+s+s1}{\PYZsq{}}\PY{l+s+s1}{f8}\PY{l+s+s1}{\PYZsq{}}\PY{p}{)}\PY{p}{,}            \PY{c+c1}{\PYZsh{} float, 8 bytes}
                   \PY{p}{(}\PY{l+s+s1}{\PYZsq{}}\PY{l+s+s1}{Russell}\PY{l+s+s1}{\PYZsq{}}\PY{p}{,} \PY{l+s+s1}{\PYZsq{}}\PY{l+s+s1}{i1}\PY{l+s+s1}{\PYZsq{}}\PY{p}{)}\PY{p}{]}\PY{p}{)}          \PY{c+c1}{\PYZsh{} integer, 1 byte}

\PY{n}{data} \PY{o}{=} \PY{n}{np}\PY{o}{.}\PY{n}{loadtxt}\PY{p}{(}\PY{n}{file}\PY{p}{,} \PY{n}{delimiter}\PY{o}{=}\PY{l+s+s1}{\PYZsq{}}\PY{l+s+s1}{;}\PY{l+s+s1}{\PYZsq{}}\PY{p}{,} \PY{n}{skiprows}\PY{o}{=}\PY{l+m+mi}{1}\PY{p}{,} \PY{n}{dtype}\PY{o}{=}\PY{n}{dtypes}\PY{p}{)}
\end{Verbatim}
\end{tcolorbox}

    \begin{Verbatim}[commandchars=\\\{\}, frame=single, framerule=2mm, rulecolor=\color{outerrorbackground}]
\textcolor{ansi-red}{ValueError}\textcolor{ansi-red}{:} could not convert string to float: ''

    \end{Verbatim}

    However, this still fails because the of a few missing values.

We can get around this by using \texttt{genfromtxt()}, which is more
flexible and can also deal with missing data:

    \begin{tcolorbox}[breakable, size=fbox, boxrule=1pt, pad at break*=1mm,colback=cellbackground, colframe=cellborder]
\prompt{In}{incolor}{4}{\boxspacing}
\begin{Verbatim}[commandchars=\\\{\}]
\PY{c+c1}{\PYZsh{} load data using genfromtxt()}
\PY{c+c1}{\PYZsh{} We still need to specify the dtype defined above!}
\PY{n}{data} \PY{o}{=} \PY{n}{np}\PY{o}{.}\PY{n}{genfromtxt}\PY{p}{(}\PY{n}{file}\PY{p}{,} \PY{n}{delimiter}\PY{o}{=}\PY{l+s+s1}{\PYZsq{}}\PY{l+s+s1}{;}\PY{l+s+s1}{\PYZsq{}}\PY{p}{,} \PY{n}{dtype}\PY{o}{=}\PY{n}{dtypes}\PY{p}{,} \PY{n}{encoding}\PY{o}{=}\PY{l+s+s1}{\PYZsq{}}\PY{l+s+s1}{utf8}\PY{l+s+s1}{\PYZsq{}}\PY{p}{,}
                     \PY{n}{skip\PYZus{}header}\PY{o}{=}\PY{l+m+mi}{1}\PY{p}{)}

\PY{c+c1}{\PYZsh{} Determine rows with missing data:}
\PY{c+c1}{\PYZsh{}   \PYZhy{} missing integers are coded as \PYZhy{}1}
\PY{c+c1}{\PYZsh{}   \PYZhy{} missing floats are coded as np.nan}
\PY{n}{missing} \PY{o}{=} \PY{p}{(}\PY{n}{data}\PY{p}{[}\PY{l+s+s1}{\PYZsq{}}\PY{l+s+s1}{Staff}\PY{l+s+s1}{\PYZsq{}}\PY{p}{]} \PY{o}{\PYZlt{}} \PY{l+m+mi}{0}\PY{p}{)} \PY{o}{|} \PY{p}{(}\PY{n}{data}\PY{p}{[}\PY{l+s+s1}{\PYZsq{}}\PY{l+s+s1}{Admin}\PY{l+s+s1}{\PYZsq{}}\PY{p}{]} \PY{o}{\PYZlt{}} \PY{l+m+mi}{0}\PY{p}{)} \PY{o}{|} \PY{n}{np}\PY{o}{.}\PY{n}{isnan}\PY{p}{(}\PY{n}{data}\PY{p}{[}\PY{l+s+s1}{\PYZsq{}}\PY{l+s+s1}{Budget}\PY{l+s+s1}{\PYZsq{}}\PY{p}{]}\PY{p}{)}

\PY{c+c1}{\PYZsh{} print rows with missing values}
\PY{n}{data}\PY{p}{[}\PY{n}{missing}\PY{p}{]}
\end{Verbatim}
\end{tcolorbox}

            \begin{tcolorbox}[breakable, size=fbox, boxrule=.5pt, pad at break*=1mm, opacityfill=0]
\prompt{Out}{outcolor}{4}{\boxspacing}
\begin{Verbatim}[commandchars=\\\{\}]
array([('"University of Strathclyde"', '"Scotland"', 1964, 22640,   -1, 3200,
304.4, 0),
       ('"University of Oxford"', '"England"', 1096, 24515, 7000,   -1, 2450. ,
1),
       ('"University of Manchester"', '"England"', 2004, 40250, 3849,   -1,
1095.4, 1),
       ('"University of Birmingham"', '"England"', 1825, 35445, 4020,   -1,
673.8, 1),
       ('"University of Nottingham"', '"England"', 1798, 30798, 3495,   -1,
656.5, 1),
       ('"University of Stirling"', '"Scotland"', 1967,  9548,   -1, 1872,
113.3, 0),
       ('"Swansea University"', '"Wales"', 1920, 20620,   -1, 3290,    nan, 0)],
      dtype=[('Institution', '<U30'), ('Country', '<U20'), ('Founded', '<i4'),
('Students', '<i4'), ('Staff', '<i4'), ('Admin', '<i4'), ('Budget', '<f8'),
('Russell', 'i1')])
\end{Verbatim}
\end{tcolorbox}
        
    While the CSV file can now be processed without errors, you see that
NumPy does not remove the double quotes around strings such as the
university names. Instead of trying to fix this, it is advisable to just
use pandas to load this kind of data which handles all these problems
automatically. We examine this alternative below.

    \hypertarget{saving-data-to-text-files}{%
\subsubsection{Saving data to text
files}\label{saving-data-to-text-files}}

To save a NumPy array to a CSV file, there is a logical counterpart to
\texttt{np.loadtxt()} which is called \texttt{np.savetxt()}.

    \begin{tcolorbox}[breakable, size=fbox, boxrule=1pt, pad at break*=1mm,colback=cellbackground, colframe=cellborder]
\prompt{In}{incolor}{5}{\boxspacing}
\begin{Verbatim}[commandchars=\\\{\}]
\PY{k+kn}{import} \PY{n+nn}{numpy} \PY{k}{as} \PY{n+nn}{np}
\PY{k+kn}{import} \PY{n+nn}{os}\PY{n+nn}{.}\PY{n+nn}{path}
\PY{k+kn}{import} \PY{n+nn}{tempfile}

\PY{c+c1}{\PYZsh{} Generate some random data on [0,1)}
\PY{n}{data} \PY{o}{=} \PY{n}{np}\PY{o}{.}\PY{n}{random}\PY{o}{.}\PY{n}{default\PYZus{}rng}\PY{p}{(}\PY{l+m+mi}{123}\PY{p}{)}\PY{o}{.}\PY{n}{random}\PY{p}{(}\PY{n}{size}\PY{o}{=}\PY{p}{(}\PY{l+m+mi}{10}\PY{p}{,} \PY{l+m+mi}{5}\PY{p}{)}\PY{p}{)}

\PY{c+c1}{\PYZsh{} create temporary directory}
\PY{n}{d} \PY{o}{=} \PY{n}{tempfile}\PY{o}{.}\PY{n}{TemporaryDirectory}\PY{p}{(}\PY{p}{)}

\PY{c+c1}{\PYZsh{} path to CSV file}
\PY{n}{file} \PY{o}{=} \PY{n}{os}\PY{o}{.}\PY{n}{path}\PY{o}{.}\PY{n}{join}\PY{p}{(}\PY{n}{d}\PY{o}{.}\PY{n}{name}\PY{p}{,} \PY{l+s+s1}{\PYZsq{}}\PY{l+s+s1}{data.csv}\PY{l+s+s1}{\PYZsq{}}\PY{p}{)}

\PY{c+c1}{\PYZsh{} Print destination file \PYZhy{} this will be different each time}
\PY{n+nb}{print}\PY{p}{(}\PY{l+s+sa}{f}\PY{l+s+s1}{\PYZsq{}}\PY{l+s+s1}{Saving CSV file to }\PY{l+s+si}{\PYZob{}}\PY{n}{file}\PY{l+s+si}{\PYZcb{}}\PY{l+s+s1}{\PYZsq{}}\PY{p}{)}

\PY{c+c1}{\PYZsh{} Write NumPy array to CSV file. The fmt argument specifies}
\PY{c+c1}{\PYZsh{} that data should be saved as floating\PYZhy{}point using a}
\PY{c+c1}{\PYZsh{} field width of 8 characters and 5 decimal digits.}
\PY{n}{np}\PY{o}{.}\PY{n}{savetxt}\PY{p}{(}\PY{n}{file}\PY{p}{,} \PY{n}{data}\PY{p}{,} \PY{n}{delimiter}\PY{o}{=}\PY{l+s+s1}{\PYZsq{}}\PY{l+s+s1}{;}\PY{l+s+s1}{\PYZsq{}}\PY{p}{,} \PY{n}{fmt}\PY{o}{=}\PY{l+s+s1}{\PYZsq{}}\PY{l+s+si}{\PYZpc{}8.5f}\PY{l+s+s1}{\PYZsq{}}\PY{p}{)}
\end{Verbatim}
\end{tcolorbox}

    \begin{Verbatim}[commandchars=\\\{\}]
Saving CSV file to /tmp/tmp8o8baj8c/data.csv
    \end{Verbatim}

    The above code creates a \(10 \times 5\) matrix of random floats and
stores these in the file \texttt{data.csv} using 5 significant digits.

We store the destination file in a temporary directory which we create
as follows:

\begin{itemize}
\tightlist
\item
  Because we cannot know in advance on which system this code is run
  (\eg the operating system and directory layout), we cannot hard-code
  a file path.
\item
  Moreover, we do not know whether the code is run with write
  permissions in any particular folder.
\item
  We work around this issue by asking the Python runtime to create a
  writeable temporary directory \emph{for the system where the code is
  being run}.
\item
  We use the routines in the
  \href{https://docs.python.org/3/library/tempfile.html}{\texttt{tempfile}}
  module to create this temporary directory.
\end{itemize}

Of course, on your own computer you do not need to use a temporary
directory, but can instead use any directory where your user has write
permissions. For example, on Windows you could use something along the
lines of

\begin{verbatim}
file = 'C:/Users/Path/to/file.txt'
np.savetxt(file, data, delimiter=';', fmt='%8.5f')
\end{verbatim}

You can even use relative paths. To store a file in the current working
directory it is sufficient to just pass the file name:

\begin{verbatim}
file = 'file.txt'
np.savetxt(file, data, delimiter=';', fmt='%8.5f')
\end{verbatim}


\hypertarget{io-with-pandas}{%
\subsection{I/O with pandas}\label{io-with-pandas}}

Pandas's I/O routines are more powerful than those implemented in NumPy:

\begin{itemize}
\tightlist
\item
  They support reading and writing numerous file formats.
\item
  They support heterogeneous data without having to specify the data
  type in advance.
\item
  They gracefully handle missing values.
\end{itemize}

For these reasons, it is often preferable to directly use pandas to
process data instead of NumPy.

The most important routines are:

\begin{itemize}
\tightlist
\item
  \href{https://pandas.pydata.org/pandas-docs/stable/reference/api/pandas.read_csv.html}{\texttt{read\_csv()}},
  \href{https://pandas.pydata.org/pandas-docs/stable/reference/api/pandas.DataFrame.to_csv.html}{\texttt{to\_csv()}}:
  Read or write CSV text files
\item
  \href{https://pandas.pydata.org/pandas-docs/stable/reference/api/pandas.read_fwf.html}{\texttt{read\_fwf()}}:
  Read data with fixed field widths, \ie text data that does not use
  delimiters to separate fields.
\item
  \href{https://pandas.pydata.org/pandas-docs/stable/reference/api/pandas.read_excel.html}{\texttt{read\_excel()}},
  \href{https://pandas.pydata.org/pandas-docs/stable/reference/api/pandas.DataFrame.to_excel.html}{\texttt{to\_excel()}}:
  Read or write Excel spreadsheets
\item
  \href{https://pandas.pydata.org/pandas-docs/stable/reference/api/pandas.read_stata.html}{\texttt{read\_stata()}},
  \href{https://pandas.pydata.org/pandas-docs/stable/reference/api/pandas.DataFrame.to_stata.html}{\texttt{to\_stata()}}:
  Read or write Stata's \texttt{.dta} files.
\end{itemize}

For a complete list of I/O routines, see the
\href{https://pandas.pydata.org/pandas-docs/stable/user_guide/io.html}{official
documentation}.

To illustrate, we repeat the above examples using pandas's
\texttt{read\_csv()}. Since the FRED data contains only floating-point
data, the result is very similar to reading in a NumPy array.

    \begin{tcolorbox}[breakable, size=fbox, boxrule=1pt, pad at break*=1mm,colback=cellbackground, colframe=cellborder]
\prompt{In}{incolor}{6}{\boxspacing}
\begin{Verbatim}[commandchars=\\\{\}]
\PY{k+kn}{import} \PY{n+nn}{pandas} \PY{k}{as} \PY{n+nn}{pd}

\PY{c+c1}{\PYZsh{} relative path to CSV file}
\PY{n}{file} \PY{o}{=} \PY{l+s+s1}{\PYZsq{}}\PY{l+s+s1}{../data/FRED.csv}\PY{l+s+s1}{\PYZsq{}}

\PY{n}{df} \PY{o}{=} \PY{n}{pd}\PY{o}{.}\PY{n}{read\PYZus{}csv}\PY{p}{(}\PY{n}{file}\PY{p}{,} \PY{n}{sep}\PY{o}{=}\PY{l+s+s1}{\PYZsq{}}\PY{l+s+s1}{,}\PY{l+s+s1}{\PYZsq{}}\PY{p}{)}
\PY{n}{df}\PY{o}{.}\PY{n}{head}\PY{p}{(}\PY{l+m+mi}{2}\PY{p}{)}          \PY{c+c1}{\PYZsh{} Display the first 2 rows of data}
\end{Verbatim}
\end{tcolorbox}

            \begin{tcolorbox}[breakable, size=fbox, boxrule=.5pt, pad at break*=1mm, opacityfill=0]
\prompt{Out}{outcolor}{6}{\boxspacing}
\begin{Verbatim}[commandchars=\\\{\}]
   Year     GDP   CPI  UNRATE
0  1948  2118.5  24.0     3.8
1  1949  2106.6  23.8     6.0
\end{Verbatim}
\end{tcolorbox}
        
    The difference between NumPy and pandas become obvious when we try to
load our university data: this works out of the box, without the need to
specify any data types or to handle missing values:

    \begin{tcolorbox}[breakable, size=fbox, boxrule=1pt, pad at break*=1mm,colback=cellbackground, colframe=cellborder]
\prompt{In}{incolor}{7}{\boxspacing}
\begin{Verbatim}[commandchars=\\\{\}]
\PY{k+kn}{import} \PY{n+nn}{pandas} \PY{k}{as} \PY{n+nn}{pd}

\PY{c+c1}{\PYZsh{} relative path to CSV file}
\PY{n}{file} \PY{o}{=} \PY{l+s+s1}{\PYZsq{}}\PY{l+s+s1}{../data/universities.csv}\PY{l+s+s1}{\PYZsq{}}

\PY{n}{df} \PY{o}{=} \PY{n}{pd}\PY{o}{.}\PY{n}{read\PYZus{}csv}\PY{p}{(}\PY{n}{file}\PY{p}{,} \PY{n}{sep}\PY{o}{=}\PY{l+s+s1}{\PYZsq{}}\PY{l+s+s1}{;}\PY{l+s+s1}{\PYZsq{}}\PY{p}{)}
\PY{n}{df}\PY{o}{.}\PY{n}{tail}\PY{p}{(}\PY{l+m+mi}{3}\PY{p}{)}      \PY{c+c1}{\PYZsh{} show last 3 rows}
\end{Verbatim}
\end{tcolorbox}

            \begin{tcolorbox}[breakable, size=fbox, boxrule=.5pt, pad at break*=1mm, opacityfill=0]
\prompt{Out}{outcolor}{7}{\boxspacing}
\begin{Verbatim}[commandchars=\\\{\}]
                   Institution           Country  Founded  Students   Staff  \textbackslash{}
20      University of Stirling          Scotland     1967      9548     NaN
21  Queen’s University Belfast  Northern Ireland     1810     18438  2414.0
22          Swansea University             Wales     1920     20620     NaN

     Admin  Budget  Russell
20  1872.0   113.3        0
21  1489.0   369.2        1
22  3290.0     NaN        0
\end{Verbatim}
\end{tcolorbox}
        
    Note that missing values are correctly converted to \texttt{np.nan} and
the double quotes surrounding strings are automatically removed!

Unlike NumPy, pandas can also process other popular data formats such as
MS Excel files (or OpenDocument spreadsheets):

    \begin{tcolorbox}[breakable, size=fbox, boxrule=1pt, pad at break*=1mm,colback=cellbackground, colframe=cellborder]
\prompt{In}{incolor}{8}{\boxspacing}
\begin{Verbatim}[commandchars=\\\{\}]
\PY{k+kn}{import} \PY{n+nn}{pandas} \PY{k}{as} \PY{n+nn}{pd}

\PY{c+c1}{\PYZsh{} Excel file containing university data}
\PY{n}{file} \PY{o}{=} \PY{l+s+s1}{\PYZsq{}}\PY{l+s+s1}{../data/universities.xlsx}\PY{l+s+s1}{\PYZsq{}}

\PY{n}{df} \PY{o}{=} \PY{n}{pd}\PY{o}{.}\PY{n}{read\PYZus{}excel}\PY{p}{(}\PY{n}{file}\PY{p}{,} \PY{n}{sheet\PYZus{}name}\PY{o}{=}\PY{l+s+s1}{\PYZsq{}}\PY{l+s+s1}{universities}\PY{l+s+s1}{\PYZsq{}}\PY{p}{)}
\PY{n}{df}\PY{o}{.}\PY{n}{head}\PY{p}{(}\PY{l+m+mi}{3}\PY{p}{)}
\end{Verbatim}
\end{tcolorbox}

            \begin{tcolorbox}[breakable, size=fbox, boxrule=.5pt, pad at break*=1mm, opacityfill=0]
\prompt{Out}{outcolor}{8}{\boxspacing}
\begin{Verbatim}[commandchars=\\\{\}]
                Institution   Country  Founded  Students   Staff   Admin  \textbackslash{}
0     University of Glasgow  Scotland     1451     30805  2942.0  4003.0
1   University of Edinburgh  Scotland     1583     34275  4589.0  6107.0
2  University of St Andrews  Scotland     1413      8984  1137.0  1576.0

   Budget  Russell
0   626.5        1
1  1102.0        1
2   251.2        0
\end{Verbatim}
\end{tcolorbox}
        
    The routine \texttt{read\_excel()} takes the argument
\texttt{sheet\_name} to specify the sheet that should be read.

\begin{itemize}
\tightlist
\item
  Note that the Python package
  \href{https://pypi.org/project/openpyxl/}{\texttt{openpyxl}} needs to
  be installed in order to read files from Excel 2003 and above.
\item
  To read older Excel files (\texttt{.xls}), you need the package
  \href{https://pypi.org/project/xlrd/}{\texttt{xlrd}}.
\end{itemize}

Finally, we often encounter text files with fixed field widths, since
this is a commonly used format in older applications (for example,
fixed-width files are easy to create in Fortran). To illustrate, the
fixed-width variant of our FRED data looks like this:

\begin{verbatim}
 Year GDP    CPI  UNRATE
 1948 2118.5   24     3.8
 1949 2106.6 23.8       6
 1950 2289.5 24.1     5.2
 1951 2473.8   26     3.3
 1952 2574.9 26.6       3
\end{verbatim}

You see that the column \texttt{Year} occupies the first 5 characters,
the \texttt{GDP} column the next 7 characters, and so on. To read such
files, the width (\ie the number of characters) has to be explicitly
specified:

    \begin{tcolorbox}[breakable, size=fbox, boxrule=1pt, pad at break*=1mm,colback=cellbackground, colframe=cellborder]
\prompt{In}{incolor}{9}{\boxspacing}
\begin{Verbatim}[commandchars=\\\{\}]
\PY{k+kn}{import} \PY{n+nn}{pandas} \PY{k}{as} \PY{n+nn}{pd}

\PY{c+c1}{\PYZsh{} File name of FRED data, stored as fixed\PYZhy{}width text}
\PY{n}{file} \PY{o}{=} \PY{l+s+s1}{\PYZsq{}}\PY{l+s+s1}{../data/FRED\PYZhy{}fixed.csv}\PY{l+s+s1}{\PYZsq{}}

\PY{c+c1}{\PYZsh{} field widths are passed as list to read\PYZus{}fwf()}
\PY{n}{df} \PY{o}{=} \PY{n}{pd}\PY{o}{.}\PY{n}{read\PYZus{}fwf}\PY{p}{(}\PY{n}{file}\PY{p}{,} \PY{n}{widths}\PY{o}{=}\PY{p}{[}\PY{l+m+mi}{5}\PY{p}{,} \PY{l+m+mi}{7}\PY{p}{,} \PY{l+m+mi}{5}\PY{p}{,} \PY{l+m+mi}{8}\PY{p}{]}\PY{p}{)}
\PY{n}{df}\PY{o}{.}\PY{n}{head}\PY{p}{(}\PY{l+m+mi}{3}\PY{p}{)}
\end{Verbatim}
\end{tcolorbox}

            \begin{tcolorbox}[breakable, size=fbox, boxrule=.5pt, pad at break*=1mm, opacityfill=0]
\prompt{Out}{outcolor}{9}{\boxspacing}
\begin{Verbatim}[commandchars=\\\{\}]
   Year     GDP   CPI  UNRATE
0  1948  2118.5  24.0     3.8
1  1949  2106.6  23.8     6.0
2  1950  2289.5  24.1     5.2
\end{Verbatim}
\end{tcolorbox}
        
    Here the \texttt{widths} argument accepts a list that contains the
number of characters to be used for each field.


\hypertarget{pickling}{%
\subsection{Pickling}\label{pickling}}

A wholly different approach to data I/O is taken by Python's built-in
\href{https://docs.python.org/3/library/pickle.html}{\texttt{pickle}}
module. Almost any Python object can be dumped into a binary file and
read back using \texttt{pickle.dump()} and \texttt{pickle.load()}.

The big advantage over other methods is that hierarchies of objects are
automatically supported. For example, we can pickle a list containing a
\texttt{tuple}, a string and a NumPy array:

    \begin{tcolorbox}[breakable, size=fbox, boxrule=1pt, pad at break*=1mm,colback=cellbackground, colframe=cellborder]
\prompt{In}{incolor}{10}{\boxspacing}
\begin{Verbatim}[commandchars=\\\{\}]
\PY{k+kn}{import} \PY{n+nn}{numpy} \PY{k}{as} \PY{n+nn}{np}
\PY{k+kn}{import} \PY{n+nn}{pickle}
\PY{k+kn}{import} \PY{n+nn}{tempfile}
\PY{k+kn}{import} \PY{n+nn}{os}\PY{n+nn}{.}\PY{n+nn}{path}

\PY{c+c1}{\PYZsh{} Generate 2d array of integers}
\PY{n}{arr} \PY{o}{=} \PY{n}{np}\PY{o}{.}\PY{n}{arange}\PY{p}{(}\PY{l+m+mi}{10}\PY{p}{)}\PY{o}{.}\PY{n}{reshape}\PY{p}{(}\PY{p}{(}\PY{l+m+mi}{2}\PY{p}{,} \PY{o}{\PYZhy{}}\PY{l+m+mi}{1}\PY{p}{)}\PY{p}{)}
\PY{n}{tpl} \PY{o}{=} \PY{p}{(}\PY{l+m+mi}{1}\PY{p}{,} \PY{l+m+mi}{2}\PY{p}{,} \PY{l+m+mi}{3}\PY{p}{)}
\PY{n}{text} \PY{o}{=} \PY{l+s+s1}{\PYZsq{}}\PY{l+s+s1}{Pickle is very powerful!}\PY{l+s+s1}{\PYZsq{}}

\PY{c+c1}{\PYZsh{} data: several nested containers and strings}
\PY{n}{data} \PY{o}{=} \PY{p}{[}\PY{n}{tpl}\PY{p}{,} \PY{n}{text}\PY{p}{,} \PY{n}{arr}\PY{p}{]}

\PY{c+c1}{\PYZsh{} create temporary directory}
\PY{n}{d} \PY{o}{=} \PY{n}{tempfile}\PY{o}{.}\PY{n}{TemporaryDirectory}\PY{p}{(}\PY{p}{)}
\PY{c+c1}{\PYZsh{} Binary destination file}
\PY{n}{file} \PY{o}{=} \PY{n}{os}\PY{o}{.}\PY{n}{path}\PY{o}{.}\PY{n}{join}\PY{p}{(}\PY{n}{d}\PY{o}{.}\PY{n}{name}\PY{p}{,} \PY{l+s+s1}{\PYZsq{}}\PY{l+s+s1}{data.bin}\PY{l+s+s1}{\PYZsq{}}\PY{p}{)}

\PY{c+c1}{\PYZsh{} print destination file path}
\PY{n+nb}{print}\PY{p}{(}\PY{l+s+sa}{f}\PY{l+s+s1}{\PYZsq{}}\PY{l+s+s1}{Pickled data written to }\PY{l+s+si}{\PYZob{}}\PY{n}{file}\PY{l+s+si}{\PYZcb{}}\PY{l+s+s1}{\PYZsq{}}\PY{p}{)}

\PY{k}{with} \PY{n+nb}{open}\PY{p}{(}\PY{n}{file}\PY{p}{,} \PY{l+s+s1}{\PYZsq{}}\PY{l+s+s1}{wb}\PY{l+s+s1}{\PYZsq{}}\PY{p}{)} \PY{k}{as} \PY{n}{f}\PY{p}{:}
    \PY{n}{pickle}\PY{o}{.}\PY{n}{dump}\PY{p}{(}\PY{n}{data}\PY{p}{,} \PY{n}{f}\PY{p}{)}
\end{Verbatim}
\end{tcolorbox}

    \begin{Verbatim}[commandchars=\\\{\}]
Pickled data written to /tmp/tmpko3y7jsf/data.bin
    \end{Verbatim}

    We can then read back the data as follows:

    \begin{tcolorbox}[breakable, size=fbox, boxrule=1pt, pad at break*=1mm,colback=cellbackground, colframe=cellborder]
\prompt{In}{incolor}{11}{\boxspacing}
\begin{Verbatim}[commandchars=\\\{\}]
\PY{c+c1}{\PYZsh{} load pickle data from above}
\PY{k}{with} \PY{n+nb}{open}\PY{p}{(}\PY{n}{file}\PY{p}{,} \PY{l+s+s1}{\PYZsq{}}\PY{l+s+s1}{rb}\PY{l+s+s1}{\PYZsq{}}\PY{p}{)} \PY{k}{as} \PY{n}{f}\PY{p}{:}
    \PY{n}{data} \PY{o}{=} \PY{n}{pickle}\PY{o}{.}\PY{n}{load}\PY{p}{(}\PY{n}{f}\PY{p}{)}

\PY{c+c1}{\PYZsh{} expand data into its components}
\PY{n}{tpl}\PY{p}{,} \PY{n}{text}\PY{p}{,} \PY{n}{arr} \PY{o}{=} \PY{n}{data}
\PY{n}{arr}         \PY{c+c1}{\PYZsh{} prints previously generated 2d array}
\end{Verbatim}
\end{tcolorbox}

            \begin{tcolorbox}[breakable, size=fbox, boxrule=.5pt, pad at break*=1mm, opacityfill=0]
\prompt{Out}{outcolor}{11}{\boxspacing}
\begin{Verbatim}[commandchars=\\\{\}]
array([[0, 1, 2, 3, 4],
       [5, 6, 7, 8, 9]])
\end{Verbatim}
\end{tcolorbox}
        
    The above example introduces a few concepts we have not encountered so
far:

\begin{enumerate}
\def\labelenumi{\arabic{enumi}.}
\item
  The built-in function
  \href{https://docs.python.org/3/library/functions.html\#open}{\texttt{open()}}
  is used to open files for reading or writing.

  \begin{itemize}
  \tightlist
  \item
    The second argument indicates whether a file should be read-only,
    \texttt{r}, or writeable, \texttt{w}.
  \item
    The \texttt{b} sets the file mode to \emph{binary}, \ie its
    contents are \emph{not} human-readable text.
  \end{itemize}
\item
  We usually access files using a so-called \emph{context manager}. A
  context manager is created via the \texttt{with} statement.

  A big advantage of using a context manager is that the file resource
  made available as \texttt{f} in the block following \texttt{with} is
  automatically cleaned up as soon as the block exits. This is
  particularly important when writing data.
\end{enumerate}

So why not always use \texttt{pickle} to load and store data?

\begin{enumerate}
\def\labelenumi{\arabic{enumi}.}
\tightlist
\item
  Pickling is Python-specific and no other application can process
  pickled data.
\item
  The pickle protocol can change in a newer version of Python, and you
  might not be able to read back your old pickled objects.
\item
  Even worse, because projects such as NumPy and pandas implement their
  own pickling routines, you might not even be able to unpickle old
  DataFrames when you upgrade to a newer pandas version!
\item
  \texttt{pickle} is not secure: It is possible to construct binary data
  that will execute arbitrary code when unpickling, so you don't want to
  unpickle data from untrusted sources.
\item
  Some objects cannot be pickled automatically. For example, this
  applies to any classes defined with Numba or Cython, unless special
  care is taken to implement the pickle protocol.
\end{enumerate}

\texttt{pickle} is great for internal use when you do not need to
exchange data with others and have complete control over your computing
environment (\ie you can enforce a specific version of Python and the
libraries you are using). For anything else, you should avoid it.


    % Add a bibliography block to the postdoc
    
    
    
\end{document}
